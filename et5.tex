% Created 2023-08-03 Thu 12:41
\documentclass[9pt, b5paper]{article}
\usepackage{xeCJK}
\usepackage[T1]{fontenc}
\usepackage{bera}
\usepackage[scaled]{beraserif}
\usepackage[scaled]{berasans}
\usepackage[scaled]{beramono}
\usepackage[cache=false]{minted}
\usepackage{xltxtra}
\usepackage{graphicx}
\usepackage{xcolor}
\usepackage{multirow}
\usepackage{multicol}
\usepackage{float}
\usepackage{textcomp}
\usepackage{algorithm}
\usepackage{algorithmic}
\usepackage{latexsym}
\usepackage{natbib}
\usepackage{geometry}
\geometry{left=1.2cm,right=1.2cm,top=1.5cm,bottom=1.2cm}
\usepackage[xetex,colorlinks=true,CJKbookmarks=true,linkcolor=blue,urlcolor=blue,menucolor=blue]{hyperref}
\newminted{common-lisp}{fontsize=\footnotesize} 
\author{deepwaterooo}
\date{\today}
\title{ET 框架学习笔记(五)--网络交互相关与Actor 机制}
\hypersetup{
  pdfkeywords={},
  pdfsubject={},
  pdfcreator={Emacs 28.2 (Org mode 8.2.7c)}}
\begin{document}

\maketitle
\tableofcontents

\section{StartConfigComponent: 找【各种服】的起始初始化地址}
\label{sec-1}
\begin{itemize}
\item 现在,先特殊重点理解:一台【服务端】物理机起来,N 核 N 进程里【主线程 Process 场景】的启动过程。
\item 【服务端、各服务器的配置、启动初始化】:是这个模块想要总结的内容。这个模块,因为框架重构里所接入的【路由器系统】的整合(感觉起来,就是通过网络,一台台服务端的服务器起来,一台台起来的服务器都向某个路由服,如同各客户端实时向位置服更新客户端的位置信息般,各小服专职服都向路由服上班打卡?要把这些看明白),让活宝妹理解起这个模块来显得相对困难,大概明天上午一上午的时间,都会花在这个模块上。
\end{itemize}
\subsection{OptionAttribute: 系统里的标签属性。}
\label{sec-1-1}
\begin{itemize}
\item 那么就是,命令行相关的。也就是ET 框架里所封装的,可以命令行,发布命令,启动【服务端】时的配置相关。【快看懂了吧。爱表哥,爱生活!!!任何时候,亲爱的表哥的活宝妹就是一定要、一定会嫁给活宝妹的亲爱的表哥!!!爱表哥,爱生活!!!】
\begin{minted}[fontsize=\scriptsize,linenos=false]{csharp}
namespace CommandLine {
    // 【CommandLine】:那么就是,命令行相关的。也就是ET 框架里所封装的,可以命令行,发布命令,启动【服务端】时的配置相关
    [AttributeUsage(AttributeTargets.Property, AllowMultiple = false, Inherited = true)]
    public sealed class OptionAttribute : BaseAttribute {
        public OptionAttribute();
        public OptionAttribute(string longName);
        public OptionAttribute(char shortName);
        public OptionAttribute(char shortName, string longName);
        public string LongName { get; }
        public string ShortName { get; }
        public string SetName { get; set; }
        public char Separator { get; set; }
        public string Group { get; set; }
    }
}
\end{minted}
\end{itemize}
\subsection{Options 单例类:}
\label{sec-1-2}
\begin{itemize}
\item 不知道这个单例类,是什么时候生成,什么情况下值会发生变化?跟命令行相关的话,当命令行启动【服务端】时,会启动这个【单例类】吗?网上搜下。
\begin{minted}[fontsize=\scriptsize,linenos=false]{csharp}
public class Options: Singleton<Options> { // 这个【单例类】,确实还没能看懂。单例类,不是组件添加形式。把【OptionAttribute】标签看懂
    [Option("AppType", Required = false, Default = AppType.Server, HelpText = "AppType enum")]
    public AppType AppType { get; set; }
    [Option("StartConfig", Required = false, Default = "StartConfig/Localhost")]
    public string StartConfig { get; set; }
    [Option("Process", Required = false, Default = 1)]
    public int Process { get; set; }

    [Option("Develop", Required = false, Default = 0, HelpText = "develop mode, 0正式 1开发 2压测")]
    public int Develop { get; set; }
    [Option("LogLevel", Required = false, Default = 2)]
    public int LogLevel { get; set; }

    [Option("Console", Required = false, Default = 0)]
    public int Console { get; set; }
    // 进程启动是否创建该进程的scenes
    [Option("CreateScenes", Required = false, Default = 1)]
    public int CreateScenes { get; set; }
}
\end{minted}
\end{itemize}

\subsection{模块里所用到的几个。NET 里的接口, 以及自定义的框架底层辅助体系类等}
\label{sec-1-3}
\subsubsection{ISupportInitialize: 【初始化】的支持接口,就是提供了【初始化之前】【初始化之后】的回调,两个API}
\label{sec-1-3-1}
\begin{minted}[fontsize=\scriptsize,linenos=false]{csharp}
namespace System.ComponentModel {
    public interface ISupportInitialize {
        void BeginInit();
        void EndInit();
    }
}
\end{minted}
\subsubsection{IInvoke: 抽象类会在事件系统 EventSystem.cs 中被用到}
\label{sec-1-3-2}
\begin{minted}[fontsize=\scriptsize,linenos=false]{csharp}
public interface IInvoke {
    Type Type { get; }
}
public abstract class AInvokeHandler<A>: IInvoke where A: struct {
    public Type Type {
        get {
            return typeof (A);
        }
    }
    public abstract void Handle(A a);
}
public abstract class AInvokeHandler<A, T>: IInvoke where A: struct {
    public Type Type {
        get {
            return typeof (A);
        }
    }
    public abstract T Handle(A a);
}
\end{minted}
\subsubsection{ISingleton 单例类接口:框架最底层,有狠多必要的单例类包装,统一实现这个单例接口,就是抽象提纯到框架最底层封装}
\label{sec-1-3-3}
\begin{minted}[fontsize=\scriptsize,linenos=false]{csharp}
public interface ISingleton: IDisposable {
    void Register();
    void Destroy();
    bool IsDisposed();
}
public abstract class Singleton<T>: ISingleton where T: Singleton<T>, new() {
    private bool isDisposed;
    [StaticField]
    private static T instance;
    public static T Instance {
        get {
            return instance;
        }
    }
    void ISingleton.Register() {
        if (instance != null) 
            throw new Exception($"singleton register twice! {typeof (T).Name}");
        instance = (T)this;
    }
    void ISingleton.Destroy() {
        if (this.isDisposed) 
            return;
        this.isDisposed = true;
        instance.Dispose();
        instance = null;
    }
    bool ISingleton.IsDisposed() {
        return this.isDisposed;
    }
    public virtual void Dispose() {
    }
}
\end{minted}
\subsubsection{IMerge: 在Proto 相关的地方,某些类如StartProcessConfig.cs 会实现这个接口,进程中以消息的形式传递这部分原理也要弄懂}
\label{sec-1-3-4}
\begin{itemize}
\item 这个接口,框架里定义了,主要用来帮助实现【动态路由】的。动态路由:网络中的路由器彼此之间互相通信,传递各自的路由信息,利用收到的路由信息来『自动合并』更新和维护自己路由表的过程。【动态路由特点】:自动化程度高,减少管理任务,错误率较低,但是占用网络资源。
\item 它定义了一个合并接口。因为这模块类中的诸多 Protobuf 相关的标签,活宝妹想,它们应该是可以以消息的形式进程间传递的。
\item 那么如果服务端的配置可以以消息的形式进程间传递,它合并时,谁与谁,如何合并的?感觉狠复杂的样子,要解一解。。。它是用在【动态路由系统】的模块。当一个路由器自动每 10 分钟周期性去扫描周围是否存在路由器邻居的时候,会自动合并。用行话说是,动态路由是网络中路由器之间互相通信,传递路由信息,利用收到的路由信息更新路由表的过程。这里【更新路由表】,说的就是当扫到了周围存在的路由器邻居,就更新自己当前路由器的路由表Info 成员变量。
\item 它能实时的适应网络结构的变化。如果路由更新信息表明网络发生了变化,路由选择软件就会重新计算路由,并发出新的路由更新信息。这些信息通过各个网络,引起各路由器重新启动其路由算法,并更新各自的路由表以动态的反映网络拓扑的变化。
\item 因为关于进程间消息自动合并?的这一块儿不懂,可以去找一下,什么情况下会调用这个合并?
\end{itemize}
\begin{minted}[fontsize=\scriptsize,linenos=false]{csharp}
public interface IMerge {
    void Merge(object o);
}
\end{minted}
\subsection{ProtoObject: 继承自上面的系统接口,定义必要的回调抽象API}
\label{sec-1-4}
\begin{minted}[fontsize=\scriptsize,linenos=false]{csharp}
public abstract class ProtoObject: Object, ISupportInitialize {
    public object Clone() { // 【进程间可传递的消息】:为什么这里的复制过程,是先序列化,再反序列化?
        // 不明白:消息明明就是反序列化好的,为什么再来一遍:序列化、反序列化(虽然这个再一遍的过程是 ProtoBuf 里的序列化与反序列化方法)?
        // 翻到Protobuf 里的反序列化方法,去查看:ET 框架的封装里,
            // 在底层内存流上的反序列化方法时(ProtobufHelper.Deserialize()),会调用 ISupportInitialize 的EndInit()回调,序列化后可做的事的回调
            // 序列化前的回调,是哪里调用的?BeginInit() 回调在框架里,只有在MongoHelper.cs 的Json 序列化前,会调用;ProtoBuf 序列化前好像跳过了这个回调
            // 就是提供了两个接口:调用与不调用,还是分不同的序列化工具
        byte[] bytes = SerializeHelper.Serialize(this);
        return SerializeHelper.Deserialize(this.GetType(), bytes, 0, bytes.Length);
    }
    public virtual void BeginInit() {
    }
    public virtual void EndInit() {
    }
    public virtual void AfterEndInit() { // 这个回调,与上一个 EndInit() 区别是?
    }
}
\end{minted}
\subsection{ConfigLoader.cs: 【服务端】是理解接下来部分的基础。【客户端】有不同逻辑。所以要把两边的都看一下}
\label{sec-1-5}
\begin{itemize}
\item 这个类名奇怪的地方是:它明明是定义了两个Invoke 标签事件的触发回调逻辑,为什么它的名字叫的是ConfigLoader? 感觉是扫描程序域里所有的【Config】标签一样。。。
\item 【任何时候,亲爱的表哥的活宝妹就是一定要嫁给亲爱的表哥!!!爱表哥,爱生活!!!】
\item 这个文件的GetAllConfigBytes 类中的回调:会去事件系统拿程序域里所有标记【Config】标签的类型,并根据这些标签类型是否为四大单例类之一来确认读取配置的位置。就是四个单例管理类的配置位置会相对特殊一点儿。
\end{itemize}
\begin{minted}[fontsize=\scriptsize,linenos=false]{csharp}
[Invoke] // 激活系: 这个激活系是同属ET 强大的事件系统的一个标签和回调逻辑,处理两种类型: GetAllConfigBytes 和 GetOneConfigBytes
public class GetAllConfigBytes: AInvokeHandler<ConfigComponent.GetAllConfigBytes, Dictionary<Type, byte[]>> {
    public override Dictionary<Type, byte[]> Handle(ConfigComponent.GetAllConfigBytes args) {
        Dictionary<Type, byte[]> output = new Dictionary<Type, byte[]>();
        List<string> startConfigs = new List<string>() {
            "StartMachineConfigCategory",  // 涉及底层配置的几个单例类,为什么这四个单例类类型重要: Machine, Process 进程、Scene 场景, Zone 区
            "StartProcessConfigCategory", 
            "StartSceneConfigCategory", 
            "StartZoneConfigCategory",
        };
// 类型:这里,扫的是所有【Invoke】标签(好像不对),还是说如【Invoke(TimerInvokeType.ActorMessegaeSenderChecker)】之类的Invoke 标签的类型属性?去看一下方法定义
        HashSet<Type> configTypes = EventSystem.Instance.GetTypes(typeof (ConfigAttribute)); // 【Config】标签:返回程序域里所有的【Config】标签类型
        foreach (Type configType in configTypes) {
            string configFilePath;
            if (startConfigs.Contains(configType.Name)) { // 【单例管理类型】:有特异性的配置路径
                configFilePath = $"../Config/Excel/s/{Options.Instance.StartConfig}/{configType.Name}.bytes";    
            } else { // 其它:人海里的路人甲,读下配置就扔掉
                configFilePath = $"../Config/Excel/s/{configType.Name}.bytes";
            }
            output[configType] = File.ReadAllBytes(configFilePath);
        }
        return output;
    }
}
[Invoke]
public class GetOneConfigBytes: AInvokeHandler<ConfigComponent.GetOneConfigBytes, byte[]> {
    public override byte[] Handle(ConfigComponent.GetOneConfigBytes args) {
        // 【Invoke 回调逻辑】:从框架特定位置,读取特定属性条款的配置,返回字节数组
        byte[] configBytes = File.ReadAllBytes($"../Config/{args.ConfigName}.bytes");
        return configBytes;
    }
}
\end{minted}
\subsection{ConfigLoader:【客户端】}
\label{sec-1-6}
\begin{itemize}
\item 【客户端】与【服务端】不同的是,客户端需要区分当前的运行,是在编辑器模式下,还是真正运行在客户端设备(PC 平台)。编辑器模式下,如服务端,去特定的位置去读配置文件;而真正的客户端,就需要从热更新资源服务器(斗地主参考项目中,仍是有个其它语言的最小最精致热更新资源包专职服务器的,ET7 里好像没有了,而是放在一个特定的文件夹下?)服务端来下载配置资源包,读取资源包里的配置内容,并字典管理,
\end{itemize}
\begin{minted}[fontsize=\scriptsize,linenos=false]{csharp}
[Invoke]
public class GetAllConfigBytes: AInvokeHandler<ConfigComponent.GetAllConfigBytes, Dictionary<Type, byte[]>> {
    public override Dictionary<Type, byte[]> Handle(ConfigComponent.GetAllConfigBytes args) {
        Dictionary<Type, byte[]> output = new Dictionary<Type, byte[]>();
        HashSet<Type> configTypes = EventSystem.Instance.GetTypes(typeof (ConfigAttribute));
        if (Define.IsEditor) { // 【编辑器模式下】:
            string ct = "cs";
            GlobalConfig globalConfig = Resources.Load<GlobalConfig>("GlobalConfig"); // 加载全局模式:这里没有看懂
            CodeMode codeMode = globalConfig.CodeMode;
            switch (codeMode) {
                case CodeMode.Client:
                    ct = "c";
                    break;
                case CodeMode.Server:
                    ct = "s";
                    break;
                case CodeMode.ClientServer:
                    ct = "cs";
                    break;
                default:
                    throw new ArgumentOutOfRangeException();
            }
            List<string> startConfigs = new List<string>() {
                "StartMachineConfigCategory", 
                "StartProcessConfigCategory", 
                "StartSceneConfigCategory", 
                "StartZoneConfigCategory",
            };
            foreach (Type configType in configTypes) {
                string configFilePath;
                if (startConfigs.Contains(configType.Name)) {
                    configFilePath = $"../Config/Excel/{ct}/{Options.Instance.StartConfig}/{configType.Name}.bytes";    
                } else {
                    configFilePath = $"../Config/Excel/{ct}/{configType.Name}.bytes";
                }
                output[configType] = File.ReadAllBytes(configFilePath);
            }
        } else {
            using (Root.Instance.Scene.AddComponent<ResourcesComponent>()) { // <<<<<<<<<<<<<<<<<<<< 
                const string configBundleName = "config.unity3d";
                ResourcesComponent.Instance.LoadBundle(configBundleName);

                foreach (Type configType in configTypes) {
                    TextAsset v = ResourcesComponent.Instance.GetAsset(configBundleName, configType.Name) as TextAsset;
                    output[configType] = v.bytes;
                }
            }
        }
        return output;
    }
}
[Invoke]
public class GetOneConfigBytes: AInvokeHandler<ConfigComponent.GetOneConfigBytes, byte[]> {
    public override byte[] Handle(ConfigComponent.GetOneConfigBytes args) {
        // TextAsset v = ResourcesComponent.Instance.GetAsset("config.unity3d", configName) as TextAsset;
        // return v.bytes;
        throw new NotImplementedException("client cant use LoadOneConfig");
    }
}
\end{minted}
\subsection{ConfigComponent 组件:单例类。底层组件,负责服务端配置相关管理?}
\label{sec-1-7}
\begin{itemize}
\item 这个底层组件的内部,涉及ET 标签事件系统的扫描【Config】标签,并Invoke 相关(服务端的配置与启动?)这里花点儿时间,再进去把ET 事件系统中各小服服务端根据(excel? 等) 配置文件来加载和启动服务端(或是服务端的必要配置)的原理弄懂
\item 框架事件系统里,有对各种不同标签的处理逻辑。Invoke 同理。程序域加载时,它扫描和管理框架里的所有必要相关标签,同Invoke 标签同样有字典(套字典)纪录管理不同参数类型(args)的字典,字典里不同类型(type) 的激活处理器。对于特定的参数类型,type 类型,如果能够找到激活处理器,就会触发调用此激活回调,来作相应的处理。
\end{itemize}
\begin{minted}[fontsize=\scriptsize,linenos=false]{csharp}
public T Invoke<A, T>(int type, A args) where A: struct {
    // 先试着去拿,框架里这个【特定 args 类型】的所有标签申明过的 invokeHandlers
    if (!this.allInvokes.TryGetValue(typeof(A), out var invokeHandlers)) {
        throw new Exception($"Invoke error: {typeof(A).Name}");
    }
    // 再试着去拿,【特定类型 type】的 invokeHandler 处理器
    if (!invokeHandlers.TryGetValue(type, out var invokeHandler)) {
        throw new Exception($"Invoke error: {typeof(A).Name} {type}");
    }
    var aInvokeHandler = invokeHandler as AInvokeHandler<A, T>;
    if (aInvokeHandler == null) {
        throw new Exception($"Invoke error, not AInvokeHandler: {typeof(T).Name} {type}");
    }
    return aInvokeHandler.Handle(args); // 调用【Invoke】标签的相应处理回调逻辑
}
public void Invoke<A>(A args) where A: struct {
    Invoke(0, args);
}
public T Invoke<A, T>(A args) where A: struct {
    return Invoke<A, T>(0, args);
}
\end{minted}
\begin{itemize}
\item 框架最底层的封装原理如此。这里,更多的是需要去找当前配置系,激活处理器的具体实现逻辑(在ConfigLoader.cs 文件里,两个回调类类型),来理解这个初始化加载模块。
\item 感觉今天上午把目前看到的这些,读得还算比较透彻。【亲爱的表哥,活宝妹一定要嫁的亲爱的表哥!!!任何时候,亲爱的表哥的活宝妹就是一定要嫁给亲爱的表哥!!爱表哥,爱生活!!!】
\end{itemize}
\begin{minted}[fontsize=\scriptsize,linenos=false]{csharp}
// Config组件会扫描所有的有【Config】标签的配置,加载进来:它借助了两套加载系统,加载一个配置,与加载所有配置。而配置仍是通过【Config】标签来标记配置类型
public class ConfigComponent: Singleton<ConfigComponent> {
    public struct GetAllConfigBytes {  }
    public struct GetOneConfigBytes {
        public string ConfigName;// 只是用一个字符串来区分不同配置 
    }
    private readonly Dictionary<Type, ISingleton> allConfig = new Dictionary<Type, ISingleton>();
    public override void Dispose() {
        foreach (var kv in this.allConfig) {
            kv.Value.Destroy();
        }
    }
    public object LoadOneConfig(Type configType) {
        this.allConfig.TryGetValue(configType, out ISingleton oneConfig);// oneConfig:这里算是自定义变量的【申明与赋值】?
        if (oneConfig != null) {
            oneConfig.Destroy();
        } 
        // 跟进Invoke: 去看一下框架里事件系统,找到具体的激活回调逻辑定义类:ConfigLoader.cs, 去查看里面对 GetOneConfigBytes 类型的激活触发逻辑
        byte[] oneConfigBytes = EventSystem.Instance.Invoke<GetOneConfigBytes, byte[]>(new GetOneConfigBytes() {ConfigName = configType.FullName});
        object category = SerializeHelper.Deserialize(configType, oneConfigBytes, 0, oneConfigBytes.Length);
        ISingleton singleton = category as ISingleton;
        singleton.Register(); // 【单例类初始化】:如果已经初始化过,会抛异常;单例类只初始化一次
        this.allConfig[configType] = singleton; // 底层:管理类单例类,不同类型,各有一个。框架里就有上面看过的四大单例类
        return category;
    }
    public void Load() { // 【加载】:系统加载,程序域加载 
        this.allConfig.Clear(); // 清空
        // 【原理】:借助框架强大事件系统,扫描域里【Invoke|()】标签(2 种);根据参数类型,调用触发激活逻辑,到服务端特定路径特定文件中去读取所有相关配置,并返回字典
        Dictionary<Type, byte[]> configBytes = EventSystem.Instance.Invoke<GetAllConfigBytes, Dictionary<Type, byte[]>>(new GetAllConfigBytes());
        foreach (Type type in configBytes.Keys) {
            byte[] oneConfigBytes = configBytes[type];
            this.LoadOneInThread(type, oneConfigBytes);
        }
    }
    public async ETTask LoadAsync() { // 哪里会调用这个方法?Entry.cs 服务端起来的时候,会调用此底层组件,加载各单例管理类。细看一下这里服务端启动初始化逻辑
        this.allConfig.Clear();
        Dictionary<Type, byte[]> configBytes = EventSystem.Instance.Invoke<GetAllConfigBytes, Dictionary<Type, byte[]>>(new GetAllConfigBytes());
        using ListComponent<Task> listTasks = ListComponent<Task>.Create();
        foreach (Type type in configBytes.Keys) {
            byte[] oneConfigBytes = configBytes[type];
// 四大单例管理类(Machine,Process,Scene,Zone):每个单例类,开一个任务线路去完成?好像是这样的。
// 不明白为什么必须管理那四个,多不同场景可以位于同一进程,一台机器可以多核多进程?区区区。。。不明白
            Task task = Task.Run(() => LoadOneInThread(type, oneConfigBytes)); 
            listTasks.Add(task);
        }
        await Task.WhenAll(listTasks.ToArray());
    }

    private void LoadOneInThread(Type configType, byte[] oneConfigBytes) {
        object category = SerializeHelper.Deserialize(configType, oneConfigBytes, 0, oneConfigBytes.Length);
        lock (this) {
            ISingleton singleton = category as ISingleton;
            singleton.Register(); // 注册单例类:就是启动初始化一个单例类吧,框架里 Invoke 配置相关,有四大单例类
            this.allConfig[configType] = singleton;
        }
    }
}
\end{minted}
\subsection{ConfigSingleton<T>: ProtoObject, ISingleton}
\label{sec-1-8}
\begin{itemize}
\item 【配置单例泛型类】:实现ISingleton 接口,适用于各种不同类型的单例类管理(生成Register, 销毁Destroy, 以及加载完成后的回调管理)。
\end{itemize}
\begin{minted}[fontsize=\scriptsize,linenos=false]{java}
public abstract class ConfigSingleton<T>: ProtoObject, ISingleton where T: ConfigSingleton<T>, new() {
        [StaticField]
        private static T instance;
        public static T Instance {
            get {
                return instance ??= ConfigComponent.Instance.LoadOneConfig(typeof (T)) as T;
            }
        }
        void ISingleton.Register() {
            if (instance != null) {
                throw new Exception($"singleton register twice! {typeof (T).Name}");
            }
            instance = (T)this;
        }
        void ISingleton.Destroy() {
            T t = instance;
            instance = null;
            t.Dispose();
        }
        bool ISingleton.IsDisposed() {
            throw new NotImplementedException();
        }
        public override void AfterEndInit() { } // <<<<<<<<<<<<<<<<<<<< 一个回调接口API
        public virtual void Dispose() { }
    }
\end{minted}
\subsection{StartMachineConfig: 抓四大单例管理类中的一个来读一下}
\label{sec-1-9}
\begin{itemize}
\item 同样的命名空间,同一个文件,完全相同的类型,没弄明白的是,它为什么会在框架里出现两遍?是叫 partial-class, 可是这样的原理、两个文件的区别,以及用途,是在哪里是什么?
\item 这个单例类型只存在于【服务端】。但是ET 框架里,双端框架有多种不同运行模式。客户端可以作为独立客户端来运行,也可以作为双端模式运行(就是内自带一个服务端)。这里的服务端就同理,可是作为独立服务端,只作服务端,也可以作为客户端在双端运行模式中,客户端自身所携带的服务端来运行。所以,框架里它出现了两次。
\item 另一个问题是:这个类是 Generated (/Users/hhj/pubFrameWorks/ET/Unity/Assets/Scripts/Codes/Model/Generate/ClientServer/Config/StartMachineConfig.cs), 是框架自动生成的类,没有看懂。为什么框架会生成这个类?
\item 今天大概就只能读到这里了,剩下的明天上午再读。。。
\item 独立的服务端,框架生成的文件?作为客户端双端运行模式下的服务端:框架生成的文件?
\item Proto 相关的标签,各种各样的标签,看得懂的标签还好,不懂的Proto 标签看得。。。
\end{itemize}
\begin{minted}[fontsize=\scriptsize,linenos=false]{csharp}
[ProtoContract]
[Config]
public partial class StartMachineConfigCategory : ConfigSingleton<StartMachineConfigCategory>, IMerge { // 实现了这个合并接口
    [ProtoIgnore]
    [BsonIgnore]
    private Dictionary<int, StartMachineConfig> dict = new Dictionary<int, StartMachineConfig>();
    [BsonElement]
    [ProtoMember(1)]
    private List<StartMachineConfig> list = new List<StartMachineConfig>();
    public void Merge(object o) { // 实现接口里申明的方法
        StartMachineConfigCategory s = o as StartMachineConfigCategory;
        this.list.AddRange(s.list); // 这里就可以是,进程间可传递的消息,的自动合并
    }
    [ProtoAfterDeserialization]        
    public void ProtoEndInit() {
        foreach (StartMachineConfig config in list) {
            config.AfterEndInit();
            this.dict.Add(config.Id, config);
        }
        this.list.Clear();
        this.AfterEndInit();
    }
    public StartMachineConfig Get(int id) {
        this.dict.TryGetValue(id, out StartMachineConfig item);
        if (item == null) 
            throw new Exception($"配置找不到,配置表名: {nameof (StartMachineConfig)},配置id: {id}");
        return item;
    }
    public bool Contain(int id) {
        return this.dict.ContainsKey(id);
    }
    public Dictionary<int, StartMachineConfig> GetAll() {
        return this.dict;
    }
    public StartMachineConfig GetOne() {
        if (this.dict == null || this.dict.Count <= 0) 
            return null;
        return this.dict.Values.GetEnumerator().Current;
    }
}
[ProtoContract]
public partial class StartMachineConfig: ProtoObject, IConfig {
    [ProtoMember(1)]
    public int Id { get; set; }
    [ProtoMember(2)]
    public string InnerIP { get; set; }
    [ProtoMember(3)]
    public string OuterIP { get; set; }
    [ProtoMember(4)]
    public string WatcherPort { get; set; }
}
\end{minted}
\begin{itemize}
\item 也没有看出这两个文件有任何的区别,只是任何一个具备服务端功能的项目(.csproj)都还是需要这个文件而已。
\item 下面的文件就不放了,因为四大单例类(Machine, Process, Scene, Zone)还各不同,只抓一个只代表四分之一。。。得一个一个去分析。
\end{itemize}
\subsection{StartProcessConfig: 【任何时候,亲爱的表哥的活宝妹就是一定要嫁给亲爱的表哥!!!爱表哥,爱生活!!!】}
\label{sec-1-10}
\begin{itemize}
\item 按现有的理解,Machine 是一个相对大的单位;一个Machine 可以多核多进程多Process; 一个核一个进程一个Process 可以多线程多任务管理,一个Process 里可以并存多个不同的 SceneType 【并存多个相同或不同功能的小服:登录服,网关服,房间服。。】;Zone 区,还不懂算是什么意思
\item 与上面的Machine 不同的是,Process 真正涉及了Partial 的概念。同上一样,存在于【服务端】。可是因为 config 部分类的存在,框架里有四个文件。这里要把 partial 的原因弄明白.
\item 就是两个文件,分别存在于 Config 文件夹,与ConfigPartial 文件夹,不明白是为什么
\item 这里,把一个版本的源码先贴这里,改天再看
\end{itemize}
\begin{minted}[fontsize=\scriptsize,linenos=false]{csharp}
[ProtoContract]
[Config]
public partial class StartProcessConfigCategory : ConfigSingleton<StartProcessConfigCategory>, IMerge {
    [ProtoIgnore]
    [BsonIgnore]
    private Dictionary<int, StartProcessConfig> dict = new Dictionary<int, StartProcessConfig>();
    [BsonElement]
    [ProtoMember(1)]
    private List<StartProcessConfig> list = new List<StartProcessConfig>();
    public void Merge(object o) {
        StartProcessConfigCategory s = o as StartProcessConfigCategory;
        this.list.AddRange(s.list);
    }
    [ProtoAfterDeserialization]        
    public void ProtoEndInit() {
        foreach (StartProcessConfig config in list) {
            config.AfterEndInit();
            this.dict.Add(config.Id, config);
        }
        this.list.Clear();
        this.AfterEndInit();
    }
    public StartProcessConfig Get(int id) {
        this.dict.TryGetValue(id, out StartProcessConfig item);
        if (item == null) {
            throw new Exception($"配置找不到,配置表名: {nameof (StartProcessConfig)},配置id: {id}");
        }
        return item;
    }
    public bool Contain(int id) {
        return this.dict.ContainsKey(id);
    }
    public Dictionary<int, StartProcessConfig> GetAll() {
        return this.dict;
    }
    public StartProcessConfig GetOne() {
        if (this.dict == null || this.dict.Count <= 0) {
            return null;
        }
        return this.dict.Values.GetEnumerator().Current;
    }
}
[ProtoContract]
public partial class StartProcessConfig: ProtoObject, IConfig {
    [ProtoMember(1)]
    public int Id { get; set; }
    [ProtoMember(2)]
    public int MachineId { get; set; }
    [ProtoMember(3)]
    public int InnerPort { get; set; }
}
\end{minted}
\subsection{StartSceneConfig: ISupportInitialize 【各种服-配置,场景配置】}
\label{sec-1-11}
\begin{minted}[fontsize=\scriptsize,linenos=false]{csharp}
public partial class StartSceneConfig: ISupportInitialize {
    public long InstanceId;
    public SceneType Type; // 场景类型

    public StartProcessConfig StartProcessConfig {
        get {
            return StartProcessConfigCategory.Instance.Get(this.Process);
        }
    }
    public StartZoneConfig StartZoneConfig {
        get {
            return StartZoneConfigCategory.Instance.Get(this.Zone);
        }
    }
    // 内网地址外网端口,通过防火墙映射端口过来
    private IPEndPoint innerIPOutPort;
    public IPEndPoint InnerIPOutPort {
        get {
            if (innerIPOutPort == null) {
                this.innerIPOutPort = NetworkHelper.ToIPEndPoint($"{this.StartProcessConfig.InnerIP}:{this.OuterPort}");
            }
            return this.innerIPOutPort;
        }
    }
    // 外网地址外网端口
    private IPEndPoint outerIPPort;
    public IPEndPoint OuterIPPort {
        get {
            if (this.outerIPPort == null) {
                this.outerIPPort = NetworkHelper.ToIPEndPoint($"{this.StartProcessConfig.OuterIP}:{this.OuterPort}");
            }
            return this.outerIPPort;
        }
    }
    public override void AfterEndInit() {
        this.Type = EnumHelper.FromString<SceneType>(this.SceneType);
        InstanceIdStruct instanceIdStruct = new InstanceIdStruct(this.Process, (uint) this.Id);
        this.InstanceId = instanceIdStruct.ToLong();
    }
}
\end{minted}
\subsection{StartSceneConfigCategory : 【Matchs!】ConfigSingleton<StartSceneConfigCategory>, IMerge}
\label{sec-1-12}
\begin{itemize}
\item 为什么这个类,会是写了两遍呢?有什么不同?跟前面类似,存在于任何具备服务端功能的模块。【服务端】【双端】
\item 读里面的登录服,会知道它是如何管理登录服的(就是后面的例子,当它要拿登录服的地址的时候),它们是区服,就是分各个小区管理。如果集群是这个样子,大概匹配服也就是一样分小区管理了。
\item 那么这个配置管理里,因为我要用匹配服与地图服,也要对至少是匹配服进行管理。那么,我在申请匹配的时候,网关服才能拿到匹配服的地址。
\item 只在【服务端】存在。但是在双端模式、与服务端模式下,每种端有两个文件来定义这个类。。一个在【ProtoContract】里,可能可以进程间消息传递?一个在 ConfigPartial 文件夹里
\item 这里的部分类 partial-class 仍然是没弄明白。什么情况下使用哪个类,不同部分类的实现原理。
\item 【重构】:因为我现在还比较喜欢使用Unity 下自带的双端模式,可是暂时只改【双端模式 ClientServer】下的文件,另一个专职服务端可能晚点儿再补上去。不用昨天晚上一样每个文件都改。
\item 不知道下面的源码,属于端的两种模式、部分类的两个文件,四个中的哪一个?
\end{itemize}
\begin{minted}[fontsize=\scriptsize,linenos=false]{csharp}
// 配置文件处理,或是服务器启动相关类,以前都没仔细读过
public partial class StartSceneConfigCategory {
    public MultiMap<int, StartSceneConfig> Gates = new MultiMap<int, StartSceneConfig>();
    public MultiMap<int, StartSceneConfig> ProcessScenes = new MultiMap<int, StartSceneConfig>();
    public Dictionary<long, Dictionary<string, StartSceneConfig>> ClientScenesByName = new Dictionary<long, Dictionary<string, StartSceneConfig>>();
    public StartSceneConfig LocationConfig;
    public List<StartSceneConfig> Realms = new List<StartSceneConfig>();
    public List<StartSceneConfig> Matchs = new List<StartSceneConfig>(); // <<<<<<<<<<<<<<<<<<<< 添加管理
    public List<StartSceneConfig> Routers = new List<StartSceneConfig>();
    public List<StartSceneConfig> Robots = new List<StartSceneConfig>();
    public StartSceneConfig BenchmarkServer;

    public List<StartSceneConfig> GetByProcess(int process) {
        return this.ProcessScenes[process];
    }
    public StartSceneConfig GetBySceneName(int zone, string name) {
        return this.ClientScenesByName[zone][name];
    }
    public override void AfterEndInit() {
        foreach (StartSceneConfig startSceneConfig in this.GetAll().Values) {
            this.ProcessScenes.Add(startSceneConfig.Process, startSceneConfig);
                
            if (!this.ClientScenesByName.ContainsKey(startSceneConfig.Zone)) {
                this.ClientScenesByName.Add(startSceneConfig.Zone, new Dictionary<string, StartSceneConfig>());
            }
            this.ClientScenesByName[startSceneConfig.Zone].Add(startSceneConfig.Name, startSceneConfig);
                
            switch (startSceneConfig.Type) {
            case SceneType.Realm:
                this.Realms.Add(startSceneConfig);
                break;
            case SceneType.Gate:
                this.Gates.Add(startSceneConfig.Zone, startSceneConfig);
                break;
            case SceneType.Match:                  // <<<<<<<<<<<<<<<<<<<< 自己加的
                this.Matchs.Add(startSceneConfig); // <<<<<<<<<<<<<<<<<<<< 
                break;
            case SceneType.Location:
                this.LocationConfig = startSceneConfig;
                break;
            case SceneType.Robot:
                this.Robots.Add(startSceneConfig);
                break;
            case SceneType.Router:
                this.Routers.Add(startSceneConfig);
                break;
            case SceneType.BenchmarkServer:
                this.BenchmarkServer = startSceneConfig;
                break;
            }
        }
    }
}
\end{minted}
\section{Actor 消息相关:跟上个章节Net 相关一起总结,两个都不太清楚。放一起总结,希望都能够理解清楚}
\label{sec-2}
\begin{itemize}
\item 跨进程【发送消息】与【返回消息】的过程,总感觉无法完整地看通一遍。这个是狠久前的总结,还是修改更新下。等亲爱的表哥的活宝妹搬进新住处后,会改完所有的编译错误,会需要把这个重构游戏写完整。
\item ET中,正常的网络消息需要建立一个session链接来发送,这类消息对应的proto需要由IMessage,IResponse,IRequest来修饰。(这是最常规,感觉最容易理解的)
\item 另外还有一种消息机制,称为 \textbf{【Actor机制】} ,挂载了MailBoxComponent的实体会成为一个actor. 而向Actor发送消息可以根据实体的instanceId来发送,不需要自己建立session链接,这类消息在proto中会打上IActorRequest, IActorResponse, IActorMessage的注释,标识为Actor消息。这种机制极大简化了服务器间向Actor发送消息的逻辑,使得实体间通信更加灵活方便。
\item 上面的,自己去想明白,挂载了MailBoxComponent的组件实体,知道对方实体的 instanceId, 背后的封装原理,仍然是对方实体 instanceId 之类的生成得比较聪明,自带自家进程 id, 让MailBoxCompoent 能够方便拿到发向收消息的进程?忘记了,好像是这样的。就是本质上仍是第一种,但封装得狠受用户弱弱程序员方便实用。。。
\item 但有的时候实体需要在服务器间传递(这一块儿还没有涉入,可以简单理解为玩家 me 从加州地图,重入到了亲爱的表哥身边的地图,不嫁给亲爱的表哥就永远不再离开。 me 大概可以理解为从一个地图服搬家转移重入到了另一个地图服, me 所属的进程可能已经变了),每次传递都会实例化一个新的,其instanceId也会变,但实体的id始终不会变,所以为了应对实体传递的问题,增加了proto需要修饰为IActorLocationRequest, IActorLocationResponse, IActorLocationMessage的消息【这一块儿仍不懂,改天再捡】,它可以根据实体Id来发送消息,不受实体在服务器间传递的影响,很好的解决了上面的问题。
\end{itemize}
\subsection{ActorMessageSender: 知道对方的instanceId,使用这个类发actor消息}
\label{sec-2-1}
\begin{itemize}
\item Tcs 成员变量:精华在这里:因为内部自带一个IActorResponse 的异步任务成员变量,可以帮助实现异步消息的自动回复
\item 正是因为内部成员自带一个异步任务,所以会多一个成员变量,就是标记是否要抛异常。这是异步任务成员变量带来的
\begin{minted}[fontsize=\scriptsize,linenos=false]{csharp}
public readonly struct ActorMessageSender {
    public long ActorId { get; }
    public long CreateTime { get; } // 最近接收或者发送消息的时间
    public IActorRequest Request { get; }      // 结构体,也自动封装了,发送的消息
    public bool NeedException { get; }         // 这上下三行:就帮助实现,返回消息的自动回复的结构包装
    public ETTask<IActorResponse> Tcs { get; } // <<<<<<<<<<<<<<<<<<<< 精华在这里:因为内部自带一个IActorResponse 的异步任务成员变量,可以帮助实现异步消息的自动回复
    public ActorMessageSender(long actorId, IActorRequest iActorRequest, ETTask<IActorResponse> tcs, bool needException) { // tv ... 
        this.ActorId = actorId;
        this.Request = iActorRequest;
        this.CreateTime = TimeHelper.ServerNow();
        this.Tcs = tcs;
        this.NeedException = needException;
    }
}
\end{minted}
\end{itemize}
\subsection{ActorMessageSenderComponent: 这个组件里有个计时器自动计时的超时时段、特定超时类型的超时时长成员变量,背后有套计时器管理组件,自动检测消息的发送超时。}
\label{sec-2-2}
\begin{itemize}
\item 超时时间:这个组件有计时器自动计时和超时激活的逻辑,这里定义了这个组件类型的超时时长,在ActorMessageSenderComponentSystem.cs 文件的 \textbf{【Invoke(TimerInvokeType.ActorMessageSenderChecker)】} 标注的ActorMessageSenderChecker 里会用到,检测超时与否
\item \textbf{【组件里消息自动超时Timer 的计时器机制】} :
\begin{itemize}
\item long TimeoutCheckTimer 是个重复闹钟
\item \textbf{【TimerComponent】} :是框架里的单例类,那么应该是,框架里所有的 Timer 定时计时器,应该是由这个单例管理类统一管理。那么这个组件应该能够负责相关逻辑。
\begin{minted}[fontsize=\scriptsize,linenos=false]{csharp}
[ComponentOf(typeof(Scene))]
public class ActorMessageSenderComponent: Entity, IAwake, IDestroy {
// 超时时间:这个组件有计时器自动计时和超时激活的逻辑,这里定义了这个组件类型的超时时长,在【Invoke(TimerInvokeType.ActorMessageSenderChecker)】标注的ActorMessageSenderChecker 里会用到,检测超时与否
    public const long TIMEOUT_TIME = 40 * 1000;
    public static ActorMessageSenderComponent Instance { get; set; }
    public int RpcId;
    public readonly SortedDictionary<int, ActorMessageSender> requestCallback = new SortedDictionary<int, ActorMessageSender>();
// 这个 long: 是重复闹钟的闹钟实例ID, 用来区分任何其它闹钟的
    public long TimeoutCheckTimer; 
    public List<int> TimeoutActorMessageSenders = new List<int>(); // 这桢更新里:待发送给的(接收者rpcId)接收者链表
}
\end{minted}
\end{itemize}
\end{itemize}
\subsection{ActorMessageSenderComponentSystem: 这个类底层封装比较多,功能模块因为是服务器端不太敦悉,多看几遍}
\label{sec-2-3}
\begin{itemize}
\item 这个类,可以看见ET7 框架更为系统化、消息机制的更为往底层或说更进一步的封装,就是今天下午看见的,以前的 handle() 或是 run() 方法,或回调实例 Action<T> reply, 现在的封装里,这些什么创建回复实例之类的,全部封装到了管理器或是帮助类
\item 如果发向同一个进程,则直接处理,不需要通过网络层。内网组件处理内网消息:这个分支可以再跟一下源码,理解一下重构的事件机制流程
\item 这个生成系,前半部分的计时器消息超时检测,看懂了;后半部分,还没看懂连能。今天上午能连多少连多少
\item 后半部分:是消息发送组件的相对底层逻辑。上层逻辑连通内外网消息,消息处理器,和读到消息发布事件后的触发调用等几个类。要把它们的连通流通原理弄懂。
\begin{minted}[fontsize=\scriptsize,linenos=false]{csharp}
[FriendOf(typeof(ActorMessageSenderComponent))]
public static class ActorMessageSenderComponentSystem {
    // 它自带个计时器,就是说,当服务器繁忙处理不过来,它就极有可能会自动超时,若是超时了,就返回个超时消息回去发送者告知一下,必要时它可以重发。而不超时,就正常基本流程处理了.那么,它就是一个服务端超负载下的自动减压逻辑
    [Invoke(TimerInvokeType.ActorMessageSenderChecker)] // 另一个新标签,激活系: 它标记说,这个激活系类,是 XXX 类型;紧跟着,就定义这个 XXX 类型的激活系类
    public class ActorMessageSenderChecker: ATimer<ActorMessageSenderComponent> {
        protected override void Run(ActorMessageSenderComponent self) { // 申明方法的接口是:ATimer<T> 抽象实现类,它实现了 AInvokeHandler<TimerCallback>
            try {
                self.Check(); // 调用组件自己的方法
             } catch (Exception e) {
                Log.Error($"move timer error: {self.Id}\n{e}");
            }
        }
    }
    [ObjectSystem]
    public class ActorMessageSenderComponentAwakeSystem: AwakeSystem<ActorMessageSenderComponent> {
// 【组件重复闹钟的设置】:实现组件内,消息的自动计时,超时触发Invoke 标签,调用相关逻辑来检测超时消息
        protected override void Awake(ActorMessageSenderComponent self) {
            ActorMessageSenderComponent.Instance = self;
// 这个重复闹钟,是消息自动计时超时过滤器的上下文连接桥梁
// 它注册的回调 TimerInvokeType.ActorMessageSenderChecker, 会每个消息超时的时候,都会回来调用 checker 的 Run()==>Check() 方法
// 应该是重复闹钟每秒重复一次,就每秒检查一次,调用一次Check() 方法来检查超时?是过滤器会给服务器减压;但这里的自动检测会把压分在各消息发送组件服务器上
// 这个重复间隔 1 秒钟的时间间隔,它计 1 秒钟开始,重复的逻辑是重复闹钟处理
            self.TimeoutCheckTimer = TimerComponent.Instance.NewRepeatedTimer(1000, TimerInvokeType.ActorMessageSenderChecker, self);
        }
    }//...
// Run() 方法:通过同步异常到ETTask, 通过ETTask 封装的抛异常方式抛出两类异常并返回;和对正常非异常返回消息,同步结果到ETTask, ETTask() 用触发调用注册过的非空回调
// 传进来的参数:是一个IActorResponse 实例,是有最小预处理(初始化了最基本成员变量:异常类型)、【写了个半好】的结果(异常)。结果还没同步到异步任务,待写;返回消息,待发送
    private static void Run(ActorMessageSender self, IActorResponse response) { 
        // 对于每个超时了的消息:超时错误码都是:ErrorCore.ERR_ActorTimeout, 所以会从发送消息超时异常里抛出异常,不用发送错误码【消息】回去,是抛异常
        if (response.Error == ErrorCore.ERR_ActorTimeout) { // 写:发送消息超时异常。因为同步到异步任务 ETTask 里,所以异步任务模块 ETTask会自动抛出异常
            self.Tcs.SetException(new Exception($"Rpc error: request, 注意Actor消息超时,请注意查看是否死锁或者没有reply: actorId: {self.ActorId} {self.Request}, response: {response}"));
            return;
        }
// 这个Run() 方法,并不是只有 Check() 【发送消息超时异常】一个方法调用。什么情况下的调用,会走到下面的分支?文件尾,有正常消息同步结果到ETTask 的调用 
// ActorMessageSenderComponent 一个组件,一次只执行一个(返回)消息发送任务,成员变量永远只管当前任务,
// 也是因为Actor 机制是并行的,一个使者一次只能发一个消息 ...
// 【组件管理器的执行频率, Run() 方法的调用频率】:要是消息太多,发不完怎么办呢?去搜索下面调用 Run() 方法的正常结果消息的调用处理频率。。。
        if (self.NeedException && ErrorCore.IsRpcNeedThrowException(response.Error)) { // 若是有异常(判断条件:消息要抛异常否?是否真有异常?),就先抛异常
            self.Tcs.SetException(new Exception($"Rpc error: actorId: {self.ActorId} request: {self.Request}, response: {response}"));
            return;
        }
        self.Tcs.SetResult(response); // 【写结果】:将【写了个半好】的消息,写进同步到异步任务的结果里;把异步任务的状态设置为完成;并触发必要的非空回调到发送者
        // 上面【异步任务 ETTask.SetResult()】,会调用注册过的一个回调,所以ETTask 封装,设置结果这一步,会自动触发调用注册过的一个回调(如果没有设置回调,因为空,就不会调用)
        // ETTask.SetResult() 异步任务写结果了,非空回调是会调用。非空回调是什么,是把返回消息发回去吗?不是。因为有独立的发送逻辑。
        // 再去想 IMHandler: 它是消息处理器。问题就变成是,当返回消息写好了,写好了一个完整的可以发送、待发送的消息,谁来处理的?有某个更底层的封装会调用这个类的发送逻辑。去把这个更底层的封装找出来,就是框架封装里,调用这个生成类Send() 方法的地方。
        // 这个服,这个自带计时器减压装配装置自带的消息处理器逻辑会处理?不是这个。减压装置,有发送消息超时,只触发最小检测,并抛发送消息超时异常给发送者告知,不写任何结果消息 
    }
    private static void Check(this ActorMessageSenderComponent self) {
        long timeNow = TimeHelper.ServerNow();
        foreach ((int key, ActorMessageSender value) in self.requestCallback) {
            // 因为是顺序发送的,所以,检测到第一个不超时的就退出
            // 超时触发的激活逻辑:是有至少一个超时的消息,才会【激活触发检测】;而检测到第一个不超时的,就退出下面的循环。
            if (timeNow < value.CreateTime + ActorMessageSenderComponent.TIMEOUT_TIME) 
                break;
            self.TimeoutActorMessageSenders.Add(key);
        }
// 超时触发的激活逻辑:是有至少一个超时的消息,才会【激活触发检测】;而检测到第一个不超时的,就退出上面的循环。
// 检测到第一个不超时的,理论上说,一旦有一个超时消息就会触发超时检测,但实际使用上,可能存在当检测逻辑被触发走到这里,实际中存在两个或是再多一点儿的超时消息?
        foreach (int rpcId in self.TimeoutActorMessageSenders) { // 一一遍历【超时了的消息】 :
            ActorMessageSender actorMessageSender = self.requestCallback[rpcId];
            self.requestCallback.Remove(rpcId);
            try { // ActorHelper.CreateResponse() 框架系统性的封装:也是通过对消息的发送类型与对应的回复类型的管理,使用帮助类,自动根据类型统一创建回复消息的实例
                // 对于每个超时了的消息:超时错误码都是:ErrorCore.ERR_ActorTimeout. 也就是,是个异常消息的回复消息实例生成帮助类
                IActorResponse response = ActorHelper.CreateResponse(actorMessageSender.Request, ErrorCore.ERR_ActorTimeout);
                Run(actorMessageSender, response); // 猜测:方法逻辑是,把回复消息发送给对应的接收消息的 rpcId
            } catch (Exception e) {
                Log.Error(e.ToString());
            }
        }
        self.TimeoutActorMessageSenders.Clear();
    }

    public static void Send(this ActorMessageSenderComponent self, long actorId, IMessage message) { // 发消息:这个方法,发所有类型的消息,最基接口
        if (actorId == 0) 
            throw new Exception($"actor id is 0: {message}");
        ProcessActorId processActorId = new(actorId);
        // 这里做了优化,如果发向同一个进程,则直接处理,不需要通过网络层
        if (processActorId.Process == Options.Instance.Process) { // 没看懂:这里怎么就说,消息是发向同一进程的了?
            NetInnerComponent.Instance.HandleMessage(actorId, message); // 原理清楚:本进程消息,直接交由本进程内网组件处理
            return;
        }
        Session session = NetInnerComponent.Instance.Get(processActorId.Process); // 非本进程消息,去走网络层
        session.Send(processActorId.ActorId, message);
    }
    public static int GetRpcId(this ActorMessageSenderComponent self) {
        return ++self.RpcId;
    }
    // 这个方法:只对当前进程的发送要求IActorResponse 的消息,封装自家进程的 rpcId, 也就是标明本进程发的消息,来自其它进程的返回消息,到时发到本进程。是特殊使用
    public static async ETTask<IActorResponse> Call(
        this ActorMessageSenderComponent self,
        long actorId,
        IActorRequest request,
        bool needException = true
        ) {
        request.RpcId = self.GetRpcId(); // 封装本进程的 rpcId 
        if (actorId == 0) throw new Exception($"actor id is 0: {request}");
        return await self.Call(actorId, request.RpcId, request, needException);
    }
    // 【艰森诲涩难懂!!】是更底层的实现细节,它封装帮助实现ET7 里消息超时自动过滤抛异常、返回消息的底层封装自动回复、封装了异步任务和必要成员变量来实现这些辅助过滤器等功能 
    public static async ETTask<IActorResponse> Call( // 跨进程发请求消息(要求回复):返回跨进程异步调用结果。是 await 关键字调用,用在异步方法里
        this ActorMessageSenderComponent self,
        long actorId,
        int rpcId,
        IActorRequest iActorRequest,
        bool needException = true
        ) {
        if (actorId == 0) 
            throw new Exception($"actor id is 0: {iActorRequest}");
// 对象池里:取一个异步任务。用这个异步作务实例,去创建下面的消息发送器实例。这里的 IActorResponse T 应该只是一个索引。因为前面看见系统扫描标签系创建返回实例,套到这个索引
        var tcs = ETTask<IActorResponse>.Create(true);
        // 下面,封装好消息发送器,交由消息发送组件管理;交由其管理,就自带消息发送计时超时过滤机制,实现服务器超负荷时的自动分压减压处理。一旦超时自动报废。。。
        self.requestCallback.Add(rpcId, new ActorMessageSender(actorId, iActorRequest, tcs, needException)); 
        self.Send(actorId, iActorRequest); // 把请求消息发出去:所有消息,都调用这个 
        long beginTime = TimeHelper.ServerFrameTime();
// 自己想一下的话:异步消息发出去,某个服会处理,有返回消息的话,这个服处理后会返回一个返回消息。
// 那么下面一行,不是等待创建 Create() 异步任务(同步方法狠快),而是等待这个处理发送消息的服,处理并返回来返回消息(是说,那个服,把处理结果同步到异步任务)
// 不是等异步任务的创建完成(同步方法狠快),实际是等处理发送消息的服,处理完并写好返回消息,同步到异步任务。
// 那个ETTask 里的回调 callback,是怎么回调的?这里Tcs 没有设置任何回调。ETTask 里所谓回调,是执行异步状态机的下一步,没有实际应用层面的回调意义
// 或说把返回消息的内容填好,【应该还没发回到消息发送者???】返回消息填好了,ETTask 异步任务的结果同步到位了,底层会自动发回来
// 【异步任务结果是怎么回来的?】是前面看过的IMHandler 的底层封装(AMRpcHandler 的抽象逻辑里)发送回来的。ET7 IMHandler 不是重构实现了返回消息的自动发送回复给发送者吗?再去看一遍。
        IActorResponse response = await tcs;  // 等待消息处理服处理完,写好同步好结果到异步任务、异步任务执行完成,状态为 Succeed
        long endTime = TimeHelper.ServerFrameTime();
        long costTime = endTime - beginTime;
        if (costTime > 200) 
            Log.Warning($"actor rpc time > 200: {costTime} {iActorRequest}");
        return response; // 返回:异步网络调用的结果
    }
// 【组件管理器的执行频率, Run() 方法的调用频率】:要是消息太多,发不完怎么办呢?去搜索下面调用 Run() 方法的正常结果消息的调用处理频率。。。
// 【ActorHandleHelper 帮助类】:老是调用这里的方法,要去查那个文件。【本质:内网消息处理器的处理逻辑,一旦是返回消息,就会调用 ActorHandleHelper, 会调用这个方法来处理返回消息】        
// 下面方法:处理IActorResponse 消息,也就是,发回复消息给收消息的人XX, 那么谁发,怎么发,就是这个方法的定义
    // 当是处理【同一进程的消息】:拿到的消息发送器就是当前组件自己,那么只要把结果同步到当前组件的Tcs 异步任务结果里,异步任务结果就会自动触发调用注册过的回调。全部流程结束
    public static void HandleIActorResponse(this ActorMessageSenderComponent self, IActorResponse response) {
        ActorMessageSender actorMessageSender;
// 下面取、实例化 ActorMessageSender 来看,感觉收消息的 rpcId, 与消息发送者 ActorMessageSender 成一一对应关系。上面的Call() 方法里,创建实例化消息发送者就是这么创始垢 
        if (!self.requestCallback.TryGetValue(response.RpcId, out actorMessageSender)) // 这里取不到,是说,这个返回消息的发送已经被处理了?
            return;
        self.requestCallback.Remove(response.RpcId); // 这个有序字典,就成为实时更新:随时添加,随时删除
        Run(actorMessageSender, response); // <<<<<<<<<<<<<<<<<<<< 
    }
}
\end{minted}
\item 几个类弄懂: ActorHandleHelper, 以及再上面的,NetInnerComponentOnReadEvent 事件发布等,上层调用的几座桥连通了,才算把整个流程弄懂了。
\item 现在不懂的就变成为:更为底层的,Session 会话框,socket 层的机制。但是因为它们更为底层,亲爱的表哥的活宝妹,现在把有限的精力投入支理解这个框架,适配自己的游戏比较重要。其它不太重要,或是更为底层的,改天有必要的时候再捡再看。【爱表哥,爱生活!!!任何时候,活宝妹就是一定要嫁给亲爱的表哥!!!爱表哥,爱生活!!!】
\end{itemize}
\subsection{LocationProxyComponent: 【位置代理组件】:为什么称它为代理?}
\label{sec-2-4}
\begin{itemize}
\item 就是有个启动类管理 StartSceneConfigCategory 类,它会分门别类地管理一些什么网关、注册登录服,地址服之类的东西。然后从这个里面拿位置服务器地址?大概意思是这样。写得可能不对。今天剩下一点儿时间,再稍微看一下
\item 感觉先前、上面仍然是写得不伦不类。总之,位置相关组件就是管理框架里各种可收发消息的实例,他们所在的(场景?位置?服务器地址?)相关位置信息。【亲爱的表哥的活宝妹就是一定要嫁给亲爱的表哥!!活宝妹只是在等:亲爱的表哥同活宝妹的一纸结婚证。活宝妹若是还没能嫁给亲爱的表哥,活宝妹就永远守候在亲爱的表哥的身边!!爱表哥,爱生活!!!】
\begin{minted}[fontsize=\scriptsize,linenos=false]{csharp}
[ComponentOf(typeof(Scene))]
public class LocationProxyComponent: Entity, IAwake, IDestroy {
    [StaticField]
    public static LocationProxyComponent Instance;
}
\end{minted}
\end{itemize}
\subsection{LocationProxyComponentSystem:}
\label{sec-2-5}
\begin{itemize}
\item 为什么要加那堆什么也没曾看懂的源码在那里?
\end{itemize}
\begin{minted}[fontsize=\scriptsize,linenos=false]{csharp}
// [ObjectSystem] awake() etc
\end{minted}
\subsection{:一个添加位置信息的请求消息处理类,示例}
\label{sec-2-6}

\subsection{ActorLocationSender: 知道对方的Id,使用这个类发actor消息}
\label{sec-2-7}
\begin{minted}[fontsize=\scriptsize,linenos=false]{csharp}
[ChildOf(typeof(ActorLocationSenderComponent))]
public class ActorLocationSender: Entity, IAwake, IDestroy {
    public long ActorId;
    public long LastSendOrRecvTime; // 最近接收或者发送消息的时间
    public int Error;
}
\end{minted}
\subsection{ActorLocationSenderComponent: 位置发送组件}
\label{sec-2-8}
\begin{minted}[fontsize=\scriptsize,linenos=false]{csharp}
 [ComponentOf(typeof(Scene))]
 public class ActorLocationSenderComponent: Entity, IAwake, IDestroy {
     public const long TIMEOUT_TIME = 60 * 1000;
     public static ActorLocationSenderComponent Instance { get; set; }
     public long CheckTimer;
 }
\end{minted}
\subsection{ActorLocationSenderComponentSystem: 这个类,也要明天上午再看一下}
\label{sec-2-9}
\begin{minted}[fontsize=\scriptsize,linenos=false]{csharp}
[Invoke(TimerInvokeType.ActorLocationSenderChecker)]
public class ActorLocationSenderChecker: ATimer<ActorLocationSenderComponent> {
    protected override void Run(ActorLocationSenderComponent self) {
        try {
            self.Check();
        }
        catch (Exception e) {
            Log.Error($"move timer error: {self.Id}\n{e}");
        }
    }
}
// [ObjectSystem] // ...
[FriendOf(typeof(ActorLocationSenderComponent))]
[FriendOf(typeof(ActorLocationSender))]
public static class ActorLocationSenderComponentSystem {
    public static void Check(this ActorLocationSenderComponent self) {
        using (ListComponent<long> list = ListComponent<long>.Create()) {
            long timeNow = TimeHelper.ServerNow();
            foreach ((long key, Entity value) in self.Children) {
                ActorLocationSender actorLocationMessageSender = (ActorLocationSender) value;
                if (timeNow > actorLocationMessageSender.LastSendOrRecvTime + ActorLocationSenderComponent.TIMEOUT_TIME) 
                    list.Add(key);
            }
            foreach (long id in list) {
                self.Remove(id);
            }
        }
    }
    private static ActorLocationSender GetOrCreate(this ActorLocationSenderComponent self, long id) {
        if (id == 0) 
            throw new Exception($"actor id is 0");
        if (self.Children.TryGetValue(id, out Entity actorLocationSender)) {
            return (ActorLocationSender) actorLocationSender;
        }
        actorLocationSender = self.AddChildWithId<ActorLocationSender>(id);
        return (ActorLocationSender) actorLocationSender;
    }
    private static void Remove(this ActorLocationSenderComponent self, long id) {
        if (!self.Children.TryGetValue(id, out Entity actorMessageSender)) 
            return;
        actorMessageSender.Dispose();
    }
    public static void Send(this ActorLocationSenderComponent self, long entityId, IActorRequest message) {
        self.Call(entityId, message).Coroutine();
    }
    public static async ETTask<IActorResponse> Call(this ActorLocationSenderComponent self, long entityId, IActorRequest iActorRequest) {
        ActorLocationSender actorLocationSender = self.GetOrCreate(entityId);
        // 先序列化好
        int rpcId = ActorMessageSenderComponent.Instance.GetRpcId();
        iActorRequest.RpcId = rpcId;
        long actorLocationSenderInstanceId = actorLocationSender.InstanceId;
        using (await CoroutineLockComponent.Instance.Wait(CoroutineLockType.ActorLocationSender, entityId)) {
            if (actorLocationSender.InstanceId != actorLocationSenderInstanceId) 
                throw new RpcException(ErrorCore.ERR_ActorTimeout, $"{iActorRequest}");
            // 队列中没处理的消息返回跟上个消息一样的报错
            if (actorLocationSender.Error == ErrorCore.ERR_NotFoundActor) 
                return ActorHelper.CreateResponse(iActorRequest, actorLocationSender.Error);
            try {
                return await self.CallInner(actorLocationSender, rpcId, iActorRequest);
            }
            catch (RpcException) {
                self.Remove(actorLocationSender.Id);
                throw;
            }
            catch (Exception e) {
                self.Remove(actorLocationSender.Id);
                throw new Exception($"{iActorRequest}", e);
            }
        }
    }
    private static async ETTask<IActorResponse> CallInner(this ActorLocationSenderComponent self, ActorLocationSender actorLocationSender, int rpcId, IActorRequest iActorRequest) {
        int failTimes = 0;
        long instanceId = actorLocationSender.InstanceId;
        actorLocationSender.LastSendOrRecvTime = TimeHelper.ServerNow();
        while (true) {
            if (actorLocationSender.ActorId == 0) {
                actorLocationSender.ActorId = await LocationProxyComponent.Instance.Get(actorLocationSender.Id);
                if (actorLocationSender.InstanceId != instanceId) 
                    throw new RpcException(ErrorCore.ERR_ActorLocationSenderTimeout2, $"{iActorRequest}");
            }
            if (actorLocationSender.ActorId == 0) {
                actorLocationSender.Error = ErrorCore.ERR_NotFoundActor;
                return ActorHelper.CreateResponse(iActorRequest, ErrorCore.ERR_NotFoundActor);
            }
            IActorResponse response = await ActorMessageSenderComponent.Instance.Call(actorLocationSender.ActorId, rpcId, iActorRequest, false);
            if (actorLocationSender.InstanceId != instanceId) 
                throw new RpcException(ErrorCore.ERR_ActorLocationSenderTimeout3, $"{iActorRequest}");
            switch (response.Error) {
                case ErrorCore.ERR_NotFoundActor: {
                    // 如果没找到Actor,重试
                    ++failTimes;
                    if (failTimes > 20) {
                        Log.Debug($"actor send message fail, actorid: {actorLocationSender.Id}");
                        actorLocationSender.Error = ErrorCore.ERR_NotFoundActor;
                        // 这里不能删除actor,要让后面等待发送的消息也返回ERR_NotFoundActor,直到超时删除
                        return response;
                    }
                    // 等待0.5s再发送
                    await TimerComponent.Instance.WaitAsync(500);
                    if (actorLocationSender.InstanceId != instanceId)
                        throw new RpcException(ErrorCore.ERR_ActorLocationSenderTimeout4, $"{iActorRequest}");
                    actorLocationSender.ActorId = 0;
                    continue;
                }
                case ErrorCore.ERR_ActorTimeout: 
                    throw new RpcException(response.Error, $"{iActorRequest}");
            }
            if (ErrorCore.IsRpcNeedThrowException(response.Error)) {
                throw new RpcException(response.Error, $"Message: {response.Message} Request: {iActorRequest}");
            }
            return response;
        }
    }
}
\end{minted}
\subsection{ActorHelper: 帮助创建IActorResponse 回复消息。狠简单}
\label{sec-2-10}
\begin{minted}[fontsize=\scriptsize,linenos=false]{csharp}
public static class ActorHelper {
    public static IActorResponse CreateResponse(IActorRequest iActorRequest, int error) {
        Type responseType = OpcodeTypeComponent.Instance.GetResponseType(iActorRequest.GetType());
        IActorResponse response = (IActorResponse)Activator.CreateInstance(responseType);
        response.Error = error;
        response.RpcId = iActorRequest.RpcId;
        return response;
    }
}
\end{minted}
\subsection{Actor 消息处理器:基本原理}
\label{sec-2-11}
\begin{itemize}
\item 消息到达MailboxComponent,MailboxComponent是有类型的,不同的类型邮箱可以做不同的处理。目前有两种邮箱类型GateSession跟MessageDispatcher。
\begin{itemize}
\item GateSession邮箱在收到消息的时候会立即转发给客户端。Actor 消息是指来自于服务端的消息(一定是来自于服务端的消息?Actor 一定是进程间,来自于其它服务端的?)。网关服是小区下所有用户的接收消息的代理。所以,网关服一旦收到服务端的返回消息,作为小区下所有用户的代理,就直接转发相应用户。【亲爱的表哥,永远是活宝妹的代理!任何时候,亲爱的表哥的活宝妹就是一定要嫁给亲爱的表哥!!爱表哥,爱生活!!!】
\item MessageDispatcher类型会再次对Actor消息进行分发到具体的Handler处理,默认的MailboxComponent类型是MessageDispatcher。
\end{itemize}
\end{itemize}
\subsection{MailboxType}
\label{sec-2-12}
\begin{minted}[fontsize=\scriptsize,linenos=false]{csharp}
public enum MailboxType {
    MessageDispatcher, // 消息分发器
    UnOrderMessageDispatcher,// 无序分发
    GateSession,// 网关?
}
\end{minted}

\subsection{ActorMessageDispatcherInfo | ActorMessageDispatcherComponent: 【消息分发器组件】}
\label{sec-2-13}
\begin{minted}[fontsize=\scriptsize,linenos=false]{csharp}
public class ActorMessageDispatcherInfo {
    public SceneType SceneType { get; }
    public IMActorHandler IMActorHandler { get; }
    public ActorMessageDispatcherInfo(SceneType sceneType, IMActorHandler imActorHandler) {
        this.SceneType = sceneType;
        this.IMActorHandler = imActorHandler;
    }
}
// Actor消息分发组件
[ComponentOf(typeof(Scene))] // 场景的子组件
public class ActorMessageDispatcherComponent: Entity, IAwake, IDestroy, ILoad {
    [StaticField]
    public static ActorMessageDispatcherComponent Instance; // 全局单例吗?好像是,只在【服务端】添加了这个组件
    // 下面的字典:去看下,同一类型,什么情况下会有一个链表的不同消息分发处理器?
    public readonly Dictionary<Type, List<ActorMessageDispatcherInfo>> ActorMessageHandlers = new();
}
\end{minted}
\begin{itemize}
\item 添加全局单例组件的地方是在:
\end{itemize}
\begin{minted}[fontsize=\scriptsize,linenos=false]{csharp}
[Event(SceneType.Process)]
public class EntryEvent2_InitServer: AEvent<ET.EventType.EntryEvent2> {
    protected override async ETTask Run(Scene scene, ET.EventType.EntryEvent2 args) {
        // 发送普通actor消息
        Root.Instance.Scene.AddComponent<ActorMessageSenderComponent>(); // 【服务端】几个组件:现在这个组件,最熟悉
        // 自已添加:【数据库管理类组件】
        Root.Instance.Scene.AddComponent<DBManagerComponent>(); // 【服务端】几个组件:现在这个组件,最熟悉
        // 发送location actor消息
        Root.Instance.Scene.AddComponent<ActorLocationSenderComponent>();
        // 访问location server的组件
        Root.Instance.Scene.AddComponent<LocationProxyComponent>();
        Root.Instance.Scene.AddComponent<ActorMessageDispatcherComponent>();
        Root.Instance.Scene.AddComponent<ServerSceneManagerComponent>();
        Root.Instance.Scene.AddComponent<RobotCaseComponent>();
        Root.Instance.Scene.AddComponent<NavmeshComponent>();
        // 【添加组件】:这里,还可以再添加一些游戏必要【根组件】,如果可以在服务器启动的时候添加的话。会影响服务器启动性能
// ....
}
\end{minted}
\subsection{ActorMessageDispatcherComponentHelper: 帮助类}
\label{sec-2-14}
\begin{itemize}
\item Actor消息分发组件:对于管理器里的,对同一发送消息类型,不同场景下不同处理器的链表管理,多看几遍
\item 这里,对于同一发送消息类型, 是会、是可能存在【从不同的场景类型中返回,带不同的消息处理器】 以致于必须得链表管理同一发送消息类型的不同可能处理情况。
\begin{minted}[fontsize=\scriptsize,linenos=false]{csharp}
[FriendOf(typeof(ActorMessageDispatcherComponent))] // Actor消息分发组件:对于管理器里的,对同一发送消息类型,不同场景下不同处理器的链表管理,多看几遍
public static class ActorMessageDispatcherComponentHelper {// Awake() Load() Destroy() 省略掉了
    private static void Load(this ActorMessageDispatcherComponent self) { // 加载:程序域回载的时候
        self.ActorMessageHandlers.Clear(); // 清空字典 
        var types = EventSystem.Instance.GetTypes(typeof (ActorMessageHandlerAttribute)); // 扫描程序域里的特定消息处理器标签 
        foreach (Type type in types) {
            object obj = Activator.CreateInstance(type); // 加载时:框架封装,自动创建【消息处理器】实例
            IMActorHandler imHandler = obj as IMActorHandler;
            if (imHandler == null) {
                throw new Exception($"message handler not inherit IMActorHandler abstract class: {obj.GetType().FullName}");
            }
            object[] attrs = type.GetCustomAttributes(typeof(ActorMessageHandlerAttribute), false);
            foreach (object attr in attrs) {
                ActorMessageHandlerAttribute actorMessageHandlerAttribute = attr as ActorMessageHandlerAttribute;
                Type messageType = imHandler.GetRequestType(); // 因为消息处理接口的封装:可以拿到发送类型
                Type handleResponseType = imHandler.GetResponseType();// 因为消息处理接口的封装:可以拿到返回消息的类型
                if (handleResponseType != null) {
                    Type responseType = OpcodeTypeComponent.Instance.GetResponseType(messageType);
                    if (handleResponseType != responseType) {
                        throw new Exception($"message handler response type error: {messageType.FullName}");
                    }
                }
                // 将必要的消息【发送类型】【返回类型】存起来,统一管理,备用
                // 这里,对于同一发送消息类型, 是会、是可能存在【从不同的场景类型中返回,带不同的消息处理器】 以致于必须得链表管理
                // 这里,感觉因为想不到、从概念上也地无法理解,可能会存在的适应情况、上下文场景,所以这里的链表管理同一发送消息类型,理解起来还有点儿困难
                ActorMessageDispatcherInfo actorMessageDispatcherInfo = new(actorMessageHandlerAttribute.SceneType, imHandler);
                self.RegisterHandler(messageType, actorMessageDispatcherInfo); // 存在本管理组件,所管理的字典里
            }
        }
    }
    private static void RegisterHandler(this ActorMessageDispatcherComponent self, Type type, ActorMessageDispatcherInfo handler) {
        // 这里,对于同一发送消息类型, 是会、是可能存在【从不同的场景类型中返回,带不同的消息处理器】 以致于必须得链表管理
        // 这里,感觉因为想不到、从概念上也地无法理解,可能会存在的适应情况、上下文场景,所以这里的链表管理同一发送消息类型,理解起来还有点儿困难
        if (!self.ActorMessageHandlers.ContainsKey(type)) 
            self.ActorMessageHandlers.Add(type, new List<ActorMessageDispatcherInfo>());
        self.ActorMessageHandlers[type].Add(handler);
    }
    public static async ETTask Handle(this ActorMessageDispatcherComponent self, Entity entity, int fromProcess, object message) {
        List<ActorMessageDispatcherInfo> list;
        if (!self.ActorMessageHandlers.TryGetValue(message.GetType(), out list)) // 根据消息的发送类型,来取所有可能的处理器包装链表 
            throw new Exception($"not found message handler: {message}");
        SceneType sceneType = entity.DomainScene().SceneType; // 定位:当前消息的场景类型
        foreach (ActorMessageDispatcherInfo actorMessageDispatcherInfo in list) { // 遍历:这个发送消息类型,所有存在注册过的消息处理器封装
            if (actorMessageDispatcherInfo.SceneType != sceneType)  // 场景不符就跳过
                continue;
            // 定位:是当前特定场景下的消息处理器,那么,就调用这个处理器,要它去干事。【爱表哥,爱生活!!!任何时候,活宝妹就是一定要嫁给亲爱的表哥!!!】
            await actorMessageDispatcherInfo.IMActorHandler.Handle(entity, fromProcess, message);   
        }
    }
}
\end{minted}
\end{itemize}
\subsection{ActorMessageHandlerAttribute 标签系: 去找几个典型标签看看}
\label{sec-2-15}
\begin{minted}[fontsize=\scriptsize,linenos=false]{csharp}
public class ActorMessageHandlerAttribute: BaseAttribute {
    public SceneType SceneType { get; }
    public ActorMessageHandlerAttribute(SceneType sceneType) {
        this.SceneType = sceneType;
    }
}
\end{minted}
\subsection{[ActorMessageHandler(SceneType.Gate)] 标签使用举例:}
\label{sec-2-16}
\begin{itemize}
\item 是以前框架中或是参考项目中的例子。标签使用申明说,这是【网关服】上的一个Actor 消息处理器定义类。
\item 框架中这个标签的例子还有很多。这里是随便抓一个出来。
\begin{minted}[fontsize=\scriptsize,linenos=false]{csharp}
[ActorMessageHandler(SceneType.Gate)]
public class Actor_MatchSucess_NttHandler : AMActorHandler<User, Actor_MatchSucess_Ntt> {
    protected override void Run(User user, Actor_MatchSucess_Ntt message) {
        user.IsMatching = false;
        user.ActorID = message.GamerID;
        Log.Info($"玩家{user.UserID}匹配成功");
    }
}
\end{minted}
\end{itemize}
\subsection{MailBoxComponent: 挂上这个组件表示该Entity是一个Actor,接收的消息将会队列处理}
\label{sec-2-17}
\begin{minted}[fontsize=\scriptsize,linenos=false]{csharp}
// 挂上这个组件表示该Entity是一个Actor,接收的消息将会队列处理
[ComponentOf]
public class MailBoxComponent: Entity, IAwake, IAwake<MailboxType> {
    // Mailbox的类型
    public MailboxType MailboxType { get; set; }
}
\end{minted}
\subsection{【服务端】ActorHandleHelper 帮助类:连接上下层的中间层桥梁}
\label{sec-2-18}
\begin{itemize}
\item 读了ActorMessageSenderComponentSystem.cs 的具体的消息内容处理、发送,以及计时器消息的超时自动抛超时错误码过滤等底层逻辑处理,
\item 读上下面的顶层的 NetInnerComponentOnReadEvent.cs 的顶层某个某些服,读到消息后的消息处理逻辑
\item 知道,当前帮助类,就是衔接上面的两条顶层调用,与底层具体处理逻辑的桥,把框架上中下层连接连通起来。
\item 分析这个类,应该可以理解底层不同逻辑方法的前后调用关系,消息处理的逻辑模块先后顺序,以及必要的可能的调用频率,或调用上下文情境等。明天上午再看一下
\item 是谁调用这个帮助类? \textbf{IMHandler类的某些继承类} 。我目前仍只总结和清楚了两个抽象继承类,但还不曾熟悉任何实现子类,要去弄那些,顺便把位置相关的也弄懂了
\item 上面 \textbf{【ActorMessageSenderComponentSystem.cs】的使用情境} :有个 \textbf{【服务端热更新的帮助】类MessageHelper.cs}, 发Actor 消息,与ActorLocation 位置消息,也会都是调用 ActorMessageSenderComponentSystem.cs 里定义的底层逻辑。 
\begin{minted}[fontsize=\scriptsize,linenos=false]{csharp}
public static class ActorHandleHelper {
    public static void Reply(int fromProcess, IActorResponse response) {
        if (fromProcess == Options.Instance.Process) { // 返回消息是同一个进程:没明白,这里为什么就断定是同一进程的消息了?直接处理
            // NetInnerComponent.Instance.HandleMessage(realActorId, response); // 等同于直接调用下面这句【我自己暂时放回来的】
            ActorMessageSenderComponent.Instance.HandleIActorResponse(response); // 【没读懂:】同一个进程内的消息,不走网络层,直接处理。什么情况下会是发给同一个进程的?ET7 重构后,同一进程下可能会有不同的先前小服:Realm 注册登录服,Gate 服等;如果不同的SceneType.Map-etc 先前场景小服只要在同一进程,就可以不走网络层吗?
            return;
        }
        // 【不同进程的消息处理:】走网络层,就是调用会话框来发出消息
        Session replySession = NetInnerComponent.Instance.Get(fromProcess); // 从内网组件单例中去拿会话框:不同进程消息,一定走网络,通过会话框把返回消息发回去
        replySession.Send(response);
    }
    public static void HandleIActorResponse(IActorResponse response) {
        ActorMessageSenderComponent.Instance.HandleIActorResponse(response);
    }
    // 分发actor消息
    [EnableAccessEntiyChild]
    public static async ETTask HandleIActorRequest(long actorId, IActorRequest iActorRequest) {
        InstanceIdStruct instanceIdStruct = new(actorId);
        int fromProcess = instanceIdStruct.Process;
        instanceIdStruct.Process = Options.Instance.Process;
        long realActorId = instanceIdStruct.ToLong();
        Entity entity = Root.Instance.Get(realActorId);
        if (entity == null) {
            IActorResponse response = ActorHelper.CreateResponse(iActorRequest, ErrorCore.ERR_NotFoundActor);
            Reply(fromProcess, response);
            return;
        }
        MailBoxComponent mailBoxComponent = entity.GetComponent<MailBoxComponent>();
        if (mailBoxComponent == null) {
            Log.Warning($"actor not found mailbox: {entity.GetType().Name} {realActorId} {iActorRequest}");
            IActorResponse response = ActorHelper.CreateResponse(iActorRequest, ErrorCore.ERR_NotFoundActor);
            Reply(fromProcess, response);
            return;
        }
        switch (mailBoxComponent.MailboxType) {
            case MailboxType.MessageDispatcher: {
                using (await CoroutineLockComponent.Instance.Wait(CoroutineLockType.Mailbox, realActorId)) {
                    if (entity.InstanceId != realActorId) {
                        IActorResponse response = ActorHelper.CreateResponse(iActorRequest, ErrorCore.ERR_NotFoundActor);
                        Reply(fromProcess, response);
                        break;
                    } // 调用管理器组件的处理方法 
                    await ActorMessageDispatcherComponent.Instance.Handle(entity, fromProcess, iActorRequest);
                }
                break;
            }
            case MailboxType.UnOrderMessageDispatcher: {
                await ActorMessageDispatcherComponent.Instance.Handle(entity, fromProcess, iActorRequest);
                break;
            }
            case MailboxType.GateSession:
            default:
                throw new Exception($"no mailboxtype: {mailBoxComponent.MailboxType} {iActorRequest}");
        }
    }
    // 分发actor消息
    [EnableAccessEntiyChild]
    public static async ETTask HandleIActorMessage(long actorId, IActorMessage iActorMessage) {
        InstanceIdStruct instanceIdStruct = new(actorId);
        int fromProcess = instanceIdStruct.Process;
        instanceIdStruct.Process = Options.Instance.Process;
        long realActorId = instanceIdStruct.ToLong();
        Entity entity = Root.Instance.Get(realActorId);
        if (entity == null) {
            Log.Error($"not found actor: {realActorId} {iActorMessage}");
            return;
        }
        MailBoxComponent mailBoxComponent = entity.GetComponent<MailBoxComponent>();
        if (mailBoxComponent == null) {
            Log.Error($"actor not found mailbox: {entity.GetType().Name} {realActorId} {iActorMessage}");
            return;
        }
        switch (mailBoxComponent.MailboxType) {
            case MailboxType.MessageDispatcher: {
                using (await CoroutineLockComponent.Instance.Wait(CoroutineLockType.Mailbox, realActorId)) {
                    if (entity.InstanceId != realActorId) 
                        break;
                    await ActorMessageDispatcherComponent.Instance.Handle(entity, fromProcess, iActorMessage);
                }
                break;
            }
            case MailboxType.UnOrderMessageDispatcher: {
                await ActorMessageDispatcherComponent.Instance.Handle(entity, fromProcess, iActorMessage);
                break;
            }
            case MailboxType.GateSession: {
                if (entity is Session gateSession) 
                    // 发送给客户端
                    gateSession.Send(iActorMessage);
                break;
            }
            default:
                throw new Exception($"no mailboxtype: {mailBoxComponent.MailboxType} {iActorMessage}");
        }
    }
}
\end{minted}
\end{itemize}
\subsection{NetInnerComponentOnReadEvent:}
\label{sec-2-19}
\begin{itemize}
\item 框架相对顶层的:某个某些服,读到消息后,发布读到消息事件后,触发的消息处理逻辑
\item 这个,应该是服务端发布读事件后,触发的订阅者处理读到消息的回调逻辑:分消息类型,进行不同的处理
\end{itemize}
\begin{minted}[fontsize=\scriptsize,linenos=false]{csharp}
// 这个,应该是服务端发布读事件后,触发的订阅者处理读到消息的回调逻辑:分消息类型,进行不同的处理
[Event(SceneType.Process)]
public class NetInnerComponentOnReadEvent: AEvent<NetInnerComponentOnRead> {
    protected override async ETTask Run(Scene scene, NetInnerComponentOnRead args) {
        try {
            long actorId = args.ActorId;
            object message = args.Message;
            // 收到actor消息,放入actor队列
            switch (message) { // 分不同的消息类型,借助 ActorHandleHelper 帮助类,对消息进行处理。既处理【请求消息】,也处理【返回消息】,还【普通消息】
                case IActorResponse iActorResponse: {
                    ActorHandleHelper.HandleIActorResponse(iActorResponse);
                    break;
                }
                case IActorRequest iActorRequest: {
                    await ActorHandleHelper.HandleIActorRequest(actorId, iActorRequest);
                    break;
                }
                case IActorMessage iActorMessage: {
                    await ActorHandleHelper.HandleIActorMessage(actorId, iActorMessage);
                    break;
                }
            }
        }
        catch (Exception e) {
            Log.Error($"InnerMessageDispatcher error: {args.Message.GetType().Name}\n{e}");
        }
        await ETTask.CompletedTask;
    }
}
\end{minted}

\section{Net 网络交互相关:【服务端+客户端】只是稍微改装成事件机制。模块没理解透、总结不全,还需要借助总结,和改掉所有编译错误后的运行、以及运行日志,来理解这个流程}
\label{sec-3}
\begin{itemize}
\item 【爱表哥,爱生活!!!任何时候,亲爱的表哥的活宝妹就是一定要、一定会嫁给活宝妹的亲爱的表哥!!!爱表哥,爱生活!!!】
\item 感觉核心逻辑,跨进程发消息,收返回消息,基本都看懂了。更底层的,可是相对高层?的服务之间,【NetThreadComponent组件】等,仍是不懂。
\item 这个模块:感觉就是 \textbf{【模块,自顶向下,异步网络调用的传递方向等,弄不懂;或底层信道上发消息两端的底层回调,不懂!】}
\item 现在还没弄清楚:Server, Client, Inner, 好像没有Outer 了,几个相对模块算是怎么回事?
\item 不管是【网络服务端NetServerComponent】,还是【网络客户端 NetClientComponent】组件,它们都管理无数个与【这个端】建立连接的【会话框】。
\end{itemize}
\subsection{RpcInfo: 【消息的包装体】。内部包装一个Tcs 异步任务,桥接异步结果给调用方。合并入其它小节}
\label{sec-3-1}
\begin{itemize}
\item 结合NetServerComponentOnReadEvent 来读。
\item 在NetServerComponentOnReadEvent 中,IResponse 【返回消息】是会话框上直接返回同步异步任务的异步结果,将【返回消息】异步给调用方。
\end{itemize}
\begin{minted}[fontsize=\scriptsize,linenos=false]{csharp}
public readonly struct RpcInfo { // 【消息】包装体:可以是进程内的。可是它包装的是基类接口,与扩展接口如何区分?
    public readonly IRequest Request;
    public readonly ETTask<IResponse> Tcs;// 这个【异步任务Tcs】是包装的精华桥梁
    public RpcInfo(IRequest request) {
        this.Request = request;
        this.Tcs = ETTask<IResponse>.Create(true);
    }
}
\end{minted}
\subsection{NetThreadComponent:}
\label{sec-3-2}
\begin{minted}[fontsize=\scriptsize,linenos=false]{csharp}
namespace ET {
// 【NetThreadComponent 组件】:网络交互的底层原理不懂。没有生成系,只有一个【NetInnerComponentSystem】。外网组件找不见
// 这个模块:感觉就是模块,自顶向下,异步网络调用的传递方向等,弄不懂;或底层信道上发消息两端的底层回调,不懂!
// 是每个场景【SceneType?】:里都必须有的异步线程组件. 场景 Scene, 与场景类型SceneType
    [ComponentOf(typeof(Scene))] 
    public class NetThreadComponent: Entity, IAwake, ILateUpdate, IDestroy {
        [StaticField]
        public static NetThreadComponent Instance; // 单例
        public Thread thread;
        public bool isStop;
    }
}
\end{minted}
\subsection{NetServerComponent: NetServerComponentOnRead 结构体。}
\label{sec-3-3}
\begin{itemize}
\item 【必须去想】:【服务端】到底是什么?不是每个进程上的什么东西,而是每个【场景Scene】所启动的该场景上的服务类型。不同场景之间的服务类型,可以不同?
\item 证实这一点儿的【半确认:因为这个组件可能是我从参考项目里,我自己搬过来的的】依据,是有【Realm 注册登录服】和【Gate 网关服】添加了这个组件。
\item 不同结构体的封装,是根据需要来的。框架里,有封装过【Session 会话框】的, [rpcId] 的,【Tcs 异步任务】的。看需求。 
\begin{minted}[fontsize=\scriptsize,linenos=false]{java}
public struct NetServerComponentOnRead {
    public Session Session;
    public object Message;
}
[ComponentOf(typeof(Scene))]
public class NetServerComponent: Entity, IAwake<IPEndPoint>, IDestroy {
    public int ServiceId;
}
\end{minted}
\end{itemize}
\subsection{NetServerComponentSystem: 场景上的【服务端】组件,可发布【服务端读到消息事件】}
\label{sec-3-4}
\begin{itemize}
\item 【生成系】重点:它可以 \textbf{发布NetServerComponentOnRead 事件。} 理解这个事件的【发布】与【订阅者回调】的过程,如下:
\begin{itemize}
\item 一个核一个进程上,可能有的【1-N】个场景中,某个场景充当【服务端】发布了该事件。当前核上的这个场景发布事件的触发原因是:主线程回调到了这个场景(网络异步线程)读到消息事件?(这里的主线程,与异步线程,想起来仍奇怪。可是同一核同一进程里,就只能多线程,每个线程当作一个场景了)
\item 【事件的订阅者】:进程上的 NetServerComponentOnReadEvent
\item 进程被【1-N】个不同场景共享,是更底层。这里发出事件,【消息的接收者】,可能在【同一进程其它场景】,也可能在【其它进程】其它场景
\item 这里,【事件发布】到【事件订阅者】的过程,更像是,由某个场景,到【1-N】个可能场景所共享的,更底层的对应核,的过程
\item 【1-N】个可能场景所共享的,更底层的【这一个对应核】:订阅了事件。处理逻辑:是本进程的场景,接收场景去处理;不同进程? rpc 。。。
\end{itemize}
\item 现在,我先把它想成是:一个进程可以有的【1-N】个场景中,每个场景,所启动的服务类型。不同场景,所启动的服务类型,应该可以不同?
\item 当这个场景充当了【服务端】,其它所有与这个【当前场景服务端】建立的会话框的另一端,就都自动当作【客户端】。(感觉这里理解不太透,暂时仍这么想)
\begin{minted}[fontsize=\scriptsize,linenos=false]{java}
[FriendOf(typeof(NetServerComponent))] // 【服务端组件】:负责【服务端】的网络交互部分
public static class NetServerComponentSystem {
    [ObjectSystem]
    public class AwakeSystem: AwakeSystem<NetServerComponent, IPEndPoint> {
        protected override void Awake(NetServerComponent self, IPEndPoint address) {
            // 当一个【场景启动】起来,向NetServices 单例总管,注册三大回调。当向总管注册三回调的时候,它,不是相当于是总管的【客户端】?
            // 更像是,【单线程多进程架构】里,异步网络线程,向主线程,注册三大回调
            self.ServiceId = NetServices.Instance.AddService(new KService(address, ServiceType.Outer));
            NetServices.Instance.RegisterAcceptCallback(self.ServiceId, self.OnAccept); // 三个回调 
            NetServices.Instance.RegisterReadCallback(self.ServiceId, self.OnRead);
            NetServices.Instance.RegisterErrorCallback(self.ServiceId, self.OnError);
        }
    }
    [ObjectSystem]
    public class NetKcpComponentDestroySystem: DestroySystem<NetServerComponent> {
        protected override void Destroy(NetServerComponent self) {
            NetServices.Instance.RemoveService(self.ServiceId);
        }
    }
    private static void OnError(this NetServerComponent self, long channelId, int error) {
        Session session = self.GetChild<Session>(channelId);
        if (session == null) return;
        session.Error = error;
        session.Dispose();
    }
    // 这个channelId是由CreateAcceptChannelId生成的
    private static void OnAccept(this NetServerComponent self, long channelId, IPEndPoint ipEndPoint) {
        // 【创建会话框】:当此【服务端】组件,接受了一个客户端,就建一个与接收的【客户端】的会话框
        Session session = self.AddChildWithId<Session, int>(channelId, self.ServiceId);
        session.RemoteAddress = ipEndPoint; // 【当前会话框】,它的远程是,一个【客户端】的IP 地址
        if (self.DomainScene().SceneType != SceneType.BenchmarkServer) { // 区分:同一功能,【服务端】的处理逻辑,与【客户端】的处理逻辑 
            // 挂上这个组件,5秒就会删除session,所以客户端验证完成要删除这个组件。该组件的作用就是防止外挂一直连接不发消息也不进行权限验证
            // 【客户端】逻辑,客户端验证的地方:C2G_LoginGateHandler: 这个例子,当前自称服务端组件,才更像【客户端】呢
            session.AddComponent<SessionAcceptTimeoutComponent>(); // 上面原标注:【客户端验证】的逻辑
            // 客户端连接,2秒检查一次recv消息,10秒没有消息则断开(与那个,此服务端接收不到心跳包的客户端,的连接)。【活宝妹就是一定要嫁给亲爱的表哥!!!】
            //【自己的理解】:【客户端】有心跳包告知服务端,各客户端的连接状况;【服务端】:同样有服务端此组件来检测说,哪个客户端掉线了?
            session.AddComponent<SessionIdleCheckerComponent>(); // 检查【会话框】是否有效:【30 秒内】至少发送过消息,至少接收过消息,否则视为闲置回收
        }
    }
    // 从这里继续往前倒,去找哪里发布事件, message 是什么类型,什么内容?【这里就是不懂】
    private static void OnRead(this NetServerComponent self, long channelId, long actorId, object message) {
        Session session = self.GetChild<Session>(channelId); // 从当前【服务端】所管理的所有会话框(连接的所有客户端)里,找到对应的 session(客户端 )
        if (session == null) return;
        session.LastRecvTime = TimeHelper.ClientNow();
        OpcodeHelper.LogMsg(self.DomainZone(), message);
        // 【发布事件】:服务端组件读到了消息。
        EventSystem.Instance.Publish(Root.Instance.Scene, new NetServerComponentOnRead() {Session = session, Message = message});
        // 【事件的订阅者】:进程上的 NetServerComponentOnReadEvent
        // 进程被【1-N】个不同场景共享,是更底层。这里发出事件,【消息的接收者】,可能在【同一进程其它场景】,也可能在【其它进程】其它场景
        // 这里,【事件发布】到【事件订阅者】的过程,更像是,由某个场景,到【1-N】个可能场景所共享的,更底层的对应核,的过程
        // 【1-N】个可能场景所共享的,更底层的【这一个对应核】:订阅了事件。处理逻辑:是本进程的场景,接收场景去处理;不同进程? rpc 。。。
    }
}
\end{minted}
\end{itemize}
\subsection{NetServerComponentOnReadEvent: NetServerComponent组件,会发布事件,触发此回调类}
\label{sec-3-5}
\begin{itemize}
\item 框架里什么地方添加了这些【NetServerComponent服务端】的组件?Realm 注册登录服,和网关服。(虽然这两个小服添加了这个服务端组件,但还不知道,是不是自己干的好事儿!!)SceneFactory 类里的。也就是说:这个组件,是有可能,重构后的框架里,是不需要的?都是自己没能把源码管理好,给混的。。
\end{itemize}
\begin{minted}[fontsize=\scriptsize,linenos=false]{csharp}
case SceneType.Realm: // 注册登录服:
    scene.AddComponent<NetServerComponent, IPEndPoint>(startSceneConfig.InnerIPOutPort);
    break;
case SceneType.Gate:
    scene.AddComponent<NetServerComponent, IPEndPoint>(startSceneConfig.InnerIPOutPort);
    scene.AddComponent<PlayerComponent>();
    scene.AddComponent<GateSessionKeyComponent>();
    break;
\end{minted}
\begin{itemize}
\item 如果消息类型是【返回消息】:就【会话框】上,调用会话框的OnResponse() 方法处理。处理逻辑,也就是把(来自同一进程其它场景, 或来自其它进程的【并不能限定只来自于本进程】)返回的【返回消息】内容,同步到封装的Tcs 异步结果里。当异步正常结果写好,框架的异步封装,就自动实现了异步结果、异步回给调用方(逻辑在调用【发送消息】发送过程的方法里)。
\item 这个类里,对于其它消息类型,上次并没能读完整和理解透彻:就是那个【发送位置消息请求】的请求者,与【被索要位置信息】的被请求者,协程锁,锁的是哪个?现在,感觉两个都可以上锁,可是两个都锁、都有必要吗?是的,框架里是两方都上锁的,既锁向位置服要地址的发送者,也锁被要地址小伙伴所在进程,锁在【进程】层面上?
\begin{itemize}
\item 【位置服】里,被请求位置信息的,同时间可以有多个不同进程的索要者,要上锁;
\item 请求消息的发送者,同时间,有什么多进程,同时要它发消息的情况?(这里暂时想不出来)理论是客观存在的。多进程队列安全,就得上锁。意思是说,多个不同进程,都想要入队列要当前 actorId 发送消息,只能按照分配给它们的【独占锁】的先后顺序入队、修改共享队列里的消息内容(这里是添加消息)
\end{itemize}
\item 判断【位置消息】里的 actorId, 是发送者的,还是被请求者的,去找消息发送之前,消息创建的地方。看框架能否找到一个例子。现在就是找不到一个这样的真正发送出去的位置消息的例子
\item 【任何时候,亲爱的表哥的活宝妹就是一定要嫁给亲爱的表哥!!爱表哥,爱生活!!!】
\end{itemize}
\begin{minted}[fontsize=\scriptsize,linenos=false]{csharp}
// 为什么Realm 注册登录服,与Gate 网关服里【服务端】组件发布的事情,会有这个场景的订阅者接收事件?
// 【SceneType.Process】:需要特殊理解,极为特殊的进程场景。它是每个核每个进程必备的一个特殊场景吗?是。Root 单根,首先启动进程场景。为同进程下添加任何其它场景打下座基。
[Event(SceneType.Process)]  // 【进程】场景?:来处理这个服务端组件事件?外网组件添加的地方是在:【Realm 注册登录服】与【网关服】。是自己写错了?
public class NetServerComponentOnReadEvent: AEvent<NetServerComponentOnRead> {
    protected override async ETTask Run(Scene scene, NetServerComponentOnRead args) {
        Session session = args.Session;
        object message = args.Message;
        // 【服务端上,会话框】Session: 收到回复消息,会去处理【会话框】上字典管理的回调,将回调的Tcs 异步结果写好。写好了,即刻异步结果到消息请求方
        if (message is IResponse response) { // 到达本进程的【返回消息】: 本进程上将结果写回去,狠简单
            // 借由Tcs 异步,会话框上会同步【返回消息】的内容到Tcs 异步任务的结果;Tcs 任务结果一旦写好,消息请求方就能收到结果
            session.OnResponse(response); 
            return; 
        } 
        // 根据消息接口判断是不是Actor消息,不同的接口做不同的处理,比如需要转发给Chat Scene,可以做一个IChatMessage接口
        switch (message) { // 【发送消息】+【不要求回复的消息】
            // 【ActorLocationSenderComponent】:先把这一两个组件逻辑给理顺了
            case IActorLocationRequest actorLocationRequest: { // gate session收到actor rpc消息,先向actor 发送rpc请求,再将请求结果返回客户端【原标注】 
                long unitId = session.GetComponent<SessionPlayerComponent>().PlayerId;
                int rpcId = actorLocationRequest.RpcId; // 这里要保存客户端的rpcId 
                long instanceId = session.InstanceId;
                IResponse iResponse = await ActorLocationSenderComponent.Instance.Call(unitId, actorLocationRequest); // 【rpcId】 vs 【unitId】:【被】要位置的两方?
                iResponse.RpcId = rpcId; // 【发送消息】与【返回消息】的 rpcId 是一样的。可是这里的设置,感觉狠奇怪。【位置服】是怎么处理的,这里为什么还得写?
                // session可能已经断开了,所以这里需要判断
                if (session.InstanceId == instanceId) 
                    session.Send(iResponse);
                break;
            }
            case IActorLocationMessage actorLocationMessage: { // 【普通,不要求回复的位置消息】
                long unitId = session.GetComponent<SessionPlayerComponent>().PlayerId;
                ActorLocationSenderComponent.Instance.Send(unitId, actorLocationMessage); // 把这里发送位置消息再看一遍,快速看一遍,总记不住
                break;
            }
            case IActorRequest actorRequest:  // 分发IActorRequest消息,目前没有用到,需要的自己添加 
                break;
            case IActorMessage actorMessage:  // 分发IActorMessage消息,目前没有用到,需要的自己添加 
                break;
        default: { // 非Actor消息的话:应该就是本进程消息,不走网络层,进程内处理
                // 非Actor消息: MessageDispatcherComponent 全局单例吗?是的
            MessageDispatcherComponent.Instance.Handle(session, message); 
                break;
            }
        }
    }
}
\end{minted}
\subsection{Option 单例类:}
\label{sec-3-6}
\begin{itemize}
\item 上面,留了一个不懂的地方:一台物理机上同一个核,同一个进程内,的【多线程多场景管理】里,为什么有一个专用的SceneType.Process. 这个场景如【亲爱的表哥在活宝妹心中的地位一样特殊】,要把这个理解透彻。现在把这个翻一遍。
\item 理解上,在同一进程的多线程管理里,是会区分【主线程】与【异步网络线程】的。这个SceneType.Process 像是【主线程】,需要处理【本核本进程内,多线程管理,主线程与异步线程的同步等逻辑】,也负责处理【多核多进程间,或与其它物理机等的网络交互】等主线程逻辑;而任意(可能受一个进程所可以开辟的多线程数目,硬件限制?)添加的【0-N】个任务线程,充当框架里可以随时再添加的同一进程上的【其它场景 SceneType】。
\item SceneType.Process: 每个核、进程上的【进程场景】
\item OptionAttribute: 命令行的选项标签。这里似乎也看不出什么来。
\end{itemize}
\begin{minted}[fontsize=\scriptsize,linenos=false]{csharp}
public class Options: Singleton<Options> { // 这个【单例类】,确实还没能看懂。单例类,不是组件添加形式。把【OptionAttribute】标签看懂
    [Option("AppType", Required = false, Default = AppType.Server, HelpText = "AppType enum")]
    public AppType AppType { get; set; }
    [Option("StartConfig", Required = false, Default = "StartConfig/Localhost")]
    public string StartConfig { get; set; }
    [Option("Process", Required = false, Default = 1)]
    public int Process { get; set; }

    [Option("Develop", Required = false, Default = 0, HelpText = "develop mode, 0正式 1开发 2压测")]
    public int Develop { get; set; }
    [Option("LogLevel", Required = false, Default = 2)]
    public int LogLevel { get; set; }
    [Option("Console", Required = false, Default = 0)]
    public int Console { get; set; }
    // 进程启动是否创建该进程的scenes
    [Option("CreateScenes", Required = false, Default = 1)]
    public int CreateScenes { get; set; }
}
\end{minted}
\begin{itemize}
\item 框架里第一次调用【Options 单例类】实例的地方,也就是这个【单例类】的初始化的过程,看一下。
\begin{itemize}
\item 第一次调用的地方,是在双端框架的Init 类里。先把这里的源码放一点儿,再去找哪里调双端的Init.cs? Program.cs 里会调这个类的Init.Start() 方法。同一个Init 类找出三个文件来。
\end{itemize}
\item 比如【客户端】起始的时候,命令行,Init 里会去Parse 命令行里传进来的参数,放将命令行的参数配置写入记入到Options 单例类里。后来看见过的,也是用这个单例类里的Process 来判定,比如【返回消息】是否为【本进程消息】等判定进程是否为同一个。
\item 只是上面,进程是一个核。现在能想到的是命令行启动一台物理机N 个核,每个核也是可以命令行单独、以核为单位配制的?【这里不懂,框架里,是如何封装,命令行来启动一个核的?】
\end{itemize}
\begin{minted}[fontsize=\scriptsize,linenos=false]{csharp}
  public static class Init {
      public static void Start() {
          try {    
              AppDomain.CurrentDomain.UnhandledException += (sender, e) => {
                  Log.Error(e.ExceptionObject.ToString());
              };
              // 异步方法全部会回掉到主线程
              Game.AddSingleton<MainThreadSynchronizationContext>();
              // 命令行参数
              Parser.Default.ParseArguments<Options>(System.Environment.GetCommandLineArgs())
                  .WithNotParsed(error => throw new Exception($"命令行格式错误! {error}"))
                  .WithParsed(Game.AddSingleton);

              Game.AddSingleton<TimeInfo>();
              Game.AddSingleton<Logger>().ILog = new NLogger(Options.Instance.AppType.ToString(), Options.Instance.Process, "../Config/NLog/NLog.config");
              Game.AddSingleton<ObjectPool>();
              Game.AddSingleton<IdGenerater>();
              Game.AddSingleton<EventSystem>();
              Game.AddSingleton<TimerComponent>();
              Game.AddSingleton<CoroutineLockComponent>();

              ETTask.ExceptionHandler += Log.Error;
              Log.Console($"{Parser.Default.FormatCommandLine(Options.Instance)}");
              Game.AddSingleton<CodeLoader>().Start();
          } catch (Exception e) {
              Log.Error(e);
          }
      }
\end{minted}
\subsection{NetClientComponent: 【网络客户端】组件:这个,感觉与【服务端】定义申明上看是一样的}
\label{sec-3-7}
\begin{itemize}
\item 先去看:框架里,什么上下文添加了这个组件?【客户端组件】一定是添加在【客户端】。【客户端场景 ClientScene】添加有这个组件。
\item 什么是【客户端】?
\end{itemize}
\begin{minted}[fontsize=\scriptsize,linenos=false]{csharp}
public struct NetClientComponentOnRead {
    public Session Session;
    public object Message;
}
[ComponentOf(typeof(Scene))]
public class NetClientComponent: Entity, IAwake<AddressFamily>, IDestroy {
    public int ServiceId;
}
\end{minted}
\subsection{NetClientComponentSystem: 【服务端】也是类似事件系统的改装}
\label{sec-3-8}
\begin{minted}[fontsize=\scriptsize,linenos=false]{csharp}
[FriendOf(typeof(NetClientComponent))] // 把这个【网络客户端】组件的主要笔记要点,再快速写一遍
public static class NetClientComponentSystem {
    [ObjectSystem]
    public class AwakeSystem: AwakeSystem<NetClientComponent, AddressFamily> {
        protected override void Awake(NetClientComponent self, AddressFamily addressFamily) { // 需要什么样的参数,就传什么样的参数
            self.ServiceId = NetServices.Instance.AddService(new KService(addressFamily, ServiceType.Outer)); // 开启了与这个客户端的网络服务
            NetServices.Instance.RegisterReadCallback(self.ServiceId, self.OnRead); // 注册订阅【读】网络消息事件,应该是从网络服务的服务端订阅
            NetServices.Instance.RegisterErrorCallback(self.ServiceId, self.OnError); // 注册订阅【出错】事件
        }
    }
    [ObjectSystem]
    public class DestroySystem: DestroySystem<NetClientComponent> {
        protected override void Destroy(NetClientComponent self) {
            NetServices.Instance.RemoveService(self.ServiceId); // 直接移除这个网络服务
        }
    }
    private static void OnRead(this NetClientComponent self, long channelId, long actorId, object message) {
        Session session = self.GetChild<Session>(channelId); // 拿:相应的会话框
        if (session == null) { // 空:直接返回
            return;
        }
        session.LastRecvTime = TimeHelper.ClientNow();
        OpcodeHelper.LogMsg(self.DomainZone(), message);
// 发布事件:事件的接收者,应该是【客户端】的Session 层面的进一步读取消息内容(内存流上读消息?),改天再去细看。
        EventSystem.Instance.Publish(Root.Instance.Scene, new NetClientComponentOnRead() {Session = session, Message = message}); 
    }
    private static void OnError(this NetClientComponent self, long channelId, int error) {
        Session session = self.GetChild<Session>(channelId); // 同样,先去拿会话框:因为这些异步网络的消息传递,都是建立在一个个会话框的基础上的
        if (session == null)  // 空:直接返回 
            return;
        session.Error = error;
        session.Dispose();
    }
    public static Session Create(this NetClientComponent self, IPEndPoint realIPEndPoint) {
        long channelId = NetServices.Instance.CreateConnectChannelId();
        Session session = self.AddChildWithId<Session, int>(channelId, self.ServiceId); // 创建必要的会话框,方便交通
        session.RemoteAddress = realIPEndPoint;
        if (self.DomainScene().SceneType != SceneType.Benchmark) {
            session.AddComponent<SessionIdleCheckerComponent>(); // 不知道这个是干什么的,改天再看
        }
        NetServices.Instance.CreateChannel(self.ServiceId, session.Id, realIPEndPoint); // 创建信道
        return session;
    }
    public static Session Create(this NetClientComponent self, IPEndPoint routerIPEndPoint, IPEndPoint realIPEndPoint, uint localConn) {
        long channelId = localConn;
        Session session = self.AddChildWithId<Session, int>(channelId, self.ServiceId);
        session.RemoteAddress = realIPEndPoint;
        if (self.DomainScene().SceneType != SceneType.Benchmark) {
            session.AddComponent<SessionIdleCheckerComponent>();
        }
        NetServices.Instance.CreateChannel(self.ServiceId, session.Id, routerIPEndPoint);
        return session;
    }
}
\end{minted}
\subsection{NetClientComponentOnReadEvent: 【网络客户端】读到消息事件:它,要如何处理读到的消息呢?}
\label{sec-3-9}
\begin{minted}[fontsize=\scriptsize,linenos=false]{csharp}
[Event(SceneType.Process)] // 作用单位:进程【一个核】。一个进程可以有多个不同的场景。
public class NetClientComponentOnReadEvent: AEvent<NetClientComponentOnRead> { // 事件 NetClientComponentOnRead 的发出者是:NetClientComponentSystem
    protected override async ETTask Run(Scene scene, NetClientComponentOnRead args) {
        Session session = args.Session;
        object message = args.Message;
        if (message is IResponse response) {  // 【返回消息】:待同步结果到Tcs
            session.OnResponse(response); // 【会话框】上将【返回消息】写入、同步到Tcs 异步任务的结果中去
            return;
        }
        // 【普通消息或者是Rpc请求消息?】:前面我写得对吗?这里说,【网络客户端组件】读到消息事件,接下来,分配到相应【会话框场景】去处理消息 
        MessageDispatcherComponent.Instance.Handle(session, message);
        await ETTask.CompletedTask;
    }
}
\end{minted}
\subsection{NetInnerComponent: 【服务端】对不同进程的处理组件。是服务器的组件}
\label{sec-3-10}
\begin{itemize}
\item 服务端的内网组件:这个组件,要想一下,同其它组件有什么不同?
\end{itemize}
\begin{minted}[fontsize=\scriptsize,linenos=false]{java}
namespace ET.Server {
    // 【服务器】:对不同进程的一些处理
    public struct ProcessActorId {
        public int Process;
        public long ActorId;
        public ProcessActorId(long actorId) {
            InstanceIdStruct instanceIdStruct = new InstanceIdStruct(actorId);
            this.Process = instanceIdStruct.Process;
            instanceIdStruct.Process = Options.Instance.Process;
            this.ActorId = instanceIdStruct.ToLong();
        }
    }
    // 下面这个结构体:可以用来封装发布内网读事件
    public struct NetInnerComponentOnRead {
        public long ActorId;
        public object Message;
    }
    
    [ComponentOf(typeof(Scene))]
    public class NetInnerComponent: Entity, IAwake<IPEndPoint>, IAwake, IDestroy {
        public int ServiceId;
        
        public NetworkProtocol InnerProtocol = NetworkProtocol.KCP;
        [StaticField]
        public static NetInnerComponent Instance;
    }
}
\end{minted}
\subsection{NetInnerComponentSystem: 生成系}
\label{sec-3-11}
\begin{itemize}
\item 处理内网消息:它发布了一个内网读到消息的事件。那么订阅过它的客户端?相关事件会被触发。去看 NetClientComponentOnReadEvent 类
\begin{minted}[fontsize=\scriptsize,linenos=false]{java}
[FriendOf(typeof(NetInnerComponent))]
public static class NetInnerComponentSystem {
    [ObjectSystem]
    public class NetInnerComponentAwakeSystem: AwakeSystem<NetInnerComponent> {
        protected override void Awake(NetInnerComponent self) {
            NetInnerComponent.Instance = self;
            switch (self.InnerProtocol) {
                case NetworkProtocol.TCP: {
                    self.ServiceId = NetServices.Instance.AddService(new TService(AddressFamily.InterNetwork, ServiceType.Inner));
                    break;
                }
                case NetworkProtocol.KCP: {
                    self.ServiceId = NetServices.Instance.AddService(new KService(AddressFamily.InterNetwork, ServiceType.Inner));
                    break;
                }
            }
            NetServices.Instance.RegisterReadCallback(self.ServiceId, self.OnRead);
            NetServices.Instance.RegisterErrorCallback(self.ServiceId, self.OnError);
        }
    }
    [ObjectSystem]
    public class NetInnerComponentAwake1System: AwakeSystem<NetInnerComponent, IPEndPoint> {
        protected override void Awake(NetInnerComponent self, IPEndPoint address) {
            NetInnerComponent.Instance = self;
            switch (self.InnerProtocol) {
                case NetworkProtocol.TCP: {
                    self.ServiceId = NetServices.Instance.AddService(new TService(address, ServiceType.Inner));
                    break;
                }
                case NetworkProtocol.KCP: {
                    self.ServiceId = NetServices.Instance.AddService(new KService(address, ServiceType.Inner));
                    break;
                }
            }
            NetServices.Instance.RegisterAcceptCallback(self.ServiceId, self.OnAccept);
            NetServices.Instance.RegisterReadCallback(self.ServiceId, self.OnRead);
            NetServices.Instance.RegisterErrorCallback(self.ServiceId, self.OnError);
        }
    }
    [ObjectSystem]
    public class NetInnerComponentDestroySystem: DestroySystem<NetInnerComponent> {
        protected override void Destroy(NetInnerComponent self) {
            NetServices.Instance.RemoveService(self.ServiceId);
        }
    }
    private static void OnRead(this NetInnerComponent self, long channelId, long actorId, object message) {
        Session session = self.GetChild<Session>(channelId);
        if (session == null) 
            return;
        session.LastRecvTime = TimeHelper.ClientFrameTime();
        self.HandleMessage(actorId, message);
    }
// 这里,内网组件,处理内网消息看出,这些都重构成了事件机制,发布根场景内网组件读到消息事件
    public static void HandleMessage(this NetInnerComponent self, long actorId, object message) {
        EventSystem.Instance.Publish(Root.Instance.Scene, new NetInnerComponentOnRead() { ActorId = actorId, Message = message });
    }
    private static void OnError(this NetInnerComponent self, long channelId, int error) {
        Session session = self.GetChild<Session>(channelId);
        if (session == null) {
            return;
        }
        session.Error = error;
        session.Dispose();
    }
    // 这个channelId是由CreateAcceptChannelId生成的
    private static void OnAccept(this NetInnerComponent self, long channelId, IPEndPoint ipEndPoint) {
        Session session = self.AddChildWithId<Session, int>(channelId, self.ServiceId);
        session.RemoteAddress = ipEndPoint;
        // session.AddComponent<SessionIdleCheckerComponent, int, int, int>(NetThreadComponent.checkInteral, NetThreadComponent.recvMaxIdleTime, NetThreadComponent.sendMaxIdleTime);
    }
    private static Session CreateInner(this NetInnerComponent self, long channelId, IPEndPoint ipEndPoint) {
        Session session = self.AddChildWithId<Session, int>(channelId, self.ServiceId);
        session.RemoteAddress = ipEndPoint;
        NetServices.Instance.CreateChannel(self.ServiceId, channelId, ipEndPoint);
        // session.AddComponent<InnerPingComponent>();
        // session.AddComponent<SessionIdleCheckerComponent, int, int, int>(NetThreadComponent.checkInteral, NetThreadComponent.recvMaxIdleTime, NetThreadComponent.sendMaxIdleTime);
        return session;
    }
    // 内网actor session,channelId是进程号。【自己的理解】:这些内网服务器间,或说重构的SceneType 间,有维护着会话框的,比如Realm 注册登录服与Gate 网关服等
    public static Session Get(this NetInnerComponent self, long channelId) {
        Session session = self.GetChild<Session>(channelId);
        if (session != null) { // 有已经创建过,就直接返回
            return session;
        } // 下面,还没创建过,就创建一个会话框
        IPEndPoint ipEndPoint = StartProcessConfigCategory.Instance.Get((int) channelId).InnerIPPort;
        session = self.CreateInner(channelId, ipEndPoint);
        return session;
    }
}
\end{minted}
\end{itemize}
\subsection{MessageDispatcherInfo: 在【MessageDispatcherComponent】中}
\label{sec-3-12}
\subsection{MessageDispatcherComponent: 全局全框架单例:【活宝妹就是一定要嫁给亲爱的表哥!!爱表哥,爱生活!!!】}
\label{sec-3-13}
\begin{minted}[fontsize=\scriptsize,linenos=false]{csharp}
// 总管:对每个场景SceneType,消息分发器
// 这个类,可以简单地理解为:先前的各种服,现在的各种服务端场景,它们所拥有的消息处理器实例的封装。
// 那么默认,每种场景,只有一个消息处理器实体类( 可以去验证这点儿 )
public class MessageDispatcherInfo { 
    public SceneType SceneType { get; }
    public IMHandler IMHandler { get; }
    public MessageDispatcherInfo(SceneType sceneType, IMHandler imHandler) {
        this.SceneType = sceneType;
        this.IMHandler = imHandler;
    }
}
// 消息分发组件
[ComponentOf(typeof(Scene))]
public class MessageDispatcherComponent: Entity, IAwake, IDestroy, ILoad {
// 按下面的字典看,消息分发器,全局单例,是的!【活宝妹就是一定要嫁给亲爱的表哥!!】
    public static MessageDispatcherComponent Instance { get; set; }  // 【全局单例】 
    public readonly Dictionary<ushort, List<MessageDispatcherInfo>> Handlers = new(); // 总管的字典
}P
\end{minted}
\begin{itemize}
\item 这个组件全局单例,添加的地主是在框架服务器启动的时候,公共组件部分的添加。组件的字典,会管理全框架下所有的MessageDispatcherInfo 相产。
\begin{itemize}
\item 来自于文件 EntryEvent1\_InitShare:
\end{itemize}
\end{itemize}
\begin{minted}[fontsize=\scriptsize,linenos=false]{csharp}
// 公用的相关组件的初始化:
[Event(SceneType.Process)]
public class EntryEvent1_InitShare: AEvent<EventType.EntryEvent1> {
    // 【全局单例】组件:
    protected override async ETTask Run(Scene scene, EventType.EntryEvent1 args) {
        Root.Instance.Scene.AddComponent<NetThreadComponent>();
        Root.Instance.Scene.AddComponent<OpcodeTypeComponent>();
        Root.Instance.Scene.AddComponent<MessageDispatcherComponent>(); // <<<<<<<<<<<<<<<<<<<< 
        Root.Instance.Scene.AddComponent<NumericWatcherComponent>();
        Root.Instance.Scene.AddComponent<AIDispatcherComponent>();
        Root.Instance.Scene.AddComponent<ClientSceneManagerComponent>();
        await ETTask.CompletedTask;
    }
}
\end{minted}
\subsection{MessageDispatcherComponentSystem:}
\label{sec-3-14}
\begin{minted}[fontsize=\scriptsize,linenos=false]{csharp}
// 扫描框架里的标签系【MessageHandler(SceneType)】
private static void Load(this MessageDispatcherComponent self) {
    self.Handlers.Clear();
    HashSet<Type> types = EventSystem.Instance.GetTypes(typeof (MessageHandlerAttribute));
    foreach (Type type in types) {
        IMHandler iMHandler = Activator.CreateInstance(type) as IMHandler;
        if (iMHandler == null) {
            Log.Error($"message handle {type.Name} 需要继承 IMHandler");
            continue;
        }
        object[] attrs = type.GetCustomAttributes(typeof(MessageHandlerAttribute), false);
        foreach (object attr in attrs) {
            MessageHandlerAttribute messageHandlerAttribute = attr as MessageHandlerAttribute;
            Type messageType = iMHandler.GetMessageType();
            ushort opcode = NetServices.Instance.GetOpcode(messageType); // 这里相对、理解上的困难是:感觉无法把OpCode 网络操作码与消息类型,从概念上连接起来
            if (opcode == 0) {
                Log.Error($"消息opcode为0: {messageType.Name}");
                continue;
            } // 下面:下面是创建一个包装体,注册备用 
            MessageDispatcherInfo messageDispatcherInfo = new (messageHandlerAttribute.SceneType, iMHandler);
            self.RegisterHandler(opcode, messageDispatcherInfo);
        }
    }
}
private static void RegisterHandler(this MessageDispatcherComponent self, ushort opcode, MessageDispatcherInfo handler) {
    if (!self.Handlers.ContainsKey(opcode)) 
        self.Handlers.Add(opcode, new List<MessageDispatcherInfo>());
    self.Handlers[opcode].Add(handler); // 加入管理体系来管理
}
public static void Handle(this MessageDispatcherComponent self, Session session, object message) {
    List<MessageDispatcherInfo> actions;
    ushort opcode = NetServices.Instance.GetOpcode(message.GetType());
    if (!self.Handlers.TryGetValue(opcode, out actions)) {
        Log.Error($"消息没有处理: {opcode} {message}");
        return;
    }
    // 这里就不明白:它的那些 Domain 什么的
    SceneType sceneType = session.DomainScene().SceneType; // 【会话框】:哈哈哈,这是会话框两端,哪一端的场景呢?感觉像是会话框的什么Domain 场景?
    foreach (MessageDispatcherInfo ev in actions) {
        if (ev.SceneType != sceneType) 
            continue;
        try {
            ev.IMHandler.Handle(session, message); // 处理分派消息:也就是调用IMHandler 接口的方法来处理消息
        } catch (Exception e) {
            Log.Error(e);
        }
    }
}
\end{minted}

\subsection{MessageDispatcherComponentHelper:}
\label{sec-3-15}
\begin{itemize}
\item 【会话框】:哈哈哈,这是会话框两端,哪一端的场景呢?分不清。。。去找出来!客户端?网关服?就是说,这里的消息分发处理,还是没有弄明白的。
\end{itemize}
\begin{minted}[fontsize=\scriptsize,linenos=false]{csharp}
// 消息分发组件
[FriendOf(typeof(MessageDispatcherComponent))]
public static class MessageDispatcherComponentHelper { // Awake() etc...
    private static void Load(this MessageDispatcherComponent self) {
        self.Handlers.Clear();
        HashSet<Type> types = EventSystem.Instance.GetTypes(typeof (MessageHandlerAttribute));
        foreach (Type type in types) {
            IMHandler iMHandler = Activator.CreateInstance(type) as IMHandler;
            if (iMHandler == null) {
                Log.Error($"message handle {type.Name} 需要继承 IMHandler");
                continue;
            }
            object[] attrs = type.GetCustomAttributes(typeof(MessageHandlerAttribute), false);
            foreach (object attr in attrs) {
                MessageHandlerAttribute messageHandlerAttribute = attr as MessageHandlerAttribute;
                Type messageType = iMHandler.GetMessageType();
                ushort opcode = NetServices.Instance.GetOpcode(messageType);
                if (opcode == 0) {
                    Log.Error($"消息opcode为0: {messageType.Name}");
                    continue;
                }
                MessageDispatcherInfo messageDispatcherInfo = new (messageHandlerAttribute.SceneType, iMHandler);
                self.RegisterHandler(opcode, messageDispatcherInfo);
            }
        }
    }
    private static void RegisterHandler(this MessageDispatcherComponent self, ushort opcode, MessageDispatcherInfo handler) {
        if (!self.Handlers.ContainsKey(opcode)) {
            self.Handlers.Add(opcode, new List<MessageDispatcherInfo>());
        }
        self.Handlers[opcode].Add(handler);
    }
    public static void Handle(this MessageDispatcherComponent self, Session session, object message) {
        List<MessageDispatcherInfo> actions;
        ushort opcode = NetServices.Instance.GetOpcode(message.GetType());
        if (!self.Handlers.TryGetValue(opcode, out actions)) {
            Log.Error($"消息没有处理: {opcode} {message}");
            return;
        }
        SceneType sceneType = session.DomainScene().SceneType; // 【会话框】:哈哈哈,这是会话框两端,哪一端的场景呢?分不清。。。去找出来!客户端?网关服?
        foreach (MessageDispatcherInfo ev in actions) {
            if (ev.SceneType != sceneType) 
                continue;
            try {
                ev.IMHandler.Handle(session, message);
            }
            catch (Exception e) {
                Log.Error(e);
            }
        }
    }
}
\end{minted}
\subsection{SessionIdleCheckerComponent: 【会话框】闲置状态管理组件}
\label{sec-3-16}
\begin{itemize}
\item 【会话框】闲置状态管理组件:当服务器太忙,一个会话框闲置太久,有没有什么逻辑会回收闲置会话框来提高服务器性能什么之类的?
\item 框架里ET 命名空间:设置的机制是,任何会话框,超过 30 秒不曾发送和接收过(要30 秒内既发送过也接收到过消息)消息,都算作超时,回收,提到服务器性能。
\begin{minted}[fontsize=\scriptsize,linenos=false]{csharp}
// 【会话框】闲置状态管理组件:当服务器太忙,一个会话框闲置太久,有没有什么逻辑会回收闲置会话框来提高服务器性能什么之类的?
[ComponentOf(typeof(Session))]
public class SessionIdleCheckerComponent: Entity, IAwake, IDestroy {
    public long RepeatedTimer;
}
\end{minted}
\end{itemize}
\subsection{SessionIdleCheckerComponentSystem: SessionIdleChecker 激活类,}
\label{sec-3-17}
\begin{itemize}
\item 这是前面读过的、类似实现原理的超时机制。感觉这个类,现在读起来狠简单。没有门槛。
\begin{minted}[fontsize=\scriptsize,linenos=false]{csharp}
[Invoke(TimerInvokeType.SessionIdleChecker)]
public class SessionIdleChecker: ATimer<SessionIdleCheckerComponent> {
    protected override void Run(SessionIdleCheckerComponent self) {
        try {
            self.Check();
        } catch (Exception e) {
            Log.Error($"move timer error: {self.Id}\n{e}");
        }
    }
}
[ObjectSystem]
public class SessionIdleCheckerComponentAwakeSystem: AwakeSystem<SessionIdleCheckerComponent> {
    protected override void Awake(SessionIdleCheckerComponent self) {
        // 同样设置:【重复闹钟】:任何时候,亲爱的表哥的活宝妹就是一定要嫁给亲爱的表哥!!!
        self.RepeatedTimer = TimerComponent.Instance.NewRepeatedTimer(SessionIdleCheckerComponentSystem.CheckInteral, TimerInvokeType.SessionIdleChecker, self);
    }
}// 。。。
public static class SessionIdleCheckerComponentSystem {
    public const int CheckInteral = 2000; // 每隔 2 秒
    public static void Check(this SessionIdleCheckerComponent self) {
        Session session = self.GetParent<Session>();
        long timeNow = TimeHelper.ClientNow();
        // 常量类定义:会话框最长每个 30 秒;
        // 判断:30 秒内,曾经发送过消息,并且也接收过消息,直接返回;否则,算作【会话框】超时
        if (timeNow - session.LastRecvTime < ConstValue.SessionTimeoutTime && timeNow - session.LastSendTime < ConstValue.SessionTimeoutTime) 
            return;
        Log.Info($"session timeout: {session.Id} {timeNow} {session.LastRecvTime} {session.LastSendTime} {timeNow - session.LastRecvTime} {timeNow - session.LastSendTime}");
        session.Error = ErrorCore.ERR_SessionSendOrRecvTimeout; // 【会话框】超时回收
        session.Dispose();
    }
}
\end{minted}
\end{itemize}
\subsection{MessageHelper: 不知道这个类是作什么用的,使用场景等。过会儿看下}
\label{sec-3-18}
\begin{itemize}
\item 这个类,仍然是桥接,类的各个方法里,所调用的是ActorMessageSenderComponent 里所定义的方法,来实现发送Actor 消息等。
\end{itemize}
\begin{minted}[fontsize=\scriptsize,linenos=false]{csharp}
public static class MessageHelper {
    public static void NoticeUnitAdd(Unit unit, Unit sendUnit) {
        M2C_CreateUnits createUnits = new M2C_CreateUnits() { Units = new List<UnitInfo>() };
        createUnits.Units.Add(UnitHelper.CreateUnitInfo(sendUnit));
        MessageHelper.SendToClient(unit, createUnits);
    }
    public static void NoticeUnitRemove(Unit unit, Unit sendUnit) {
        M2C_RemoveUnits removeUnits = new M2C_RemoveUnits() {Units = new List<long>()};
        removeUnits.Units.Add(sendUnit.Id);
        MessageHelper.SendToClient(unit, removeUnits);
    }
    public static void Broadcast(Unit unit, IActorMessage message) {
        Dictionary<long, AOIEntity> dict = unit.GetBeSeePlayers();
        // 网络底层做了优化,同一个消息不会多次序列化
        foreach (AOIEntity u in dict.Values) {
            ActorMessageSenderComponent.Instance.Send(u.Unit.GetComponent<UnitGateComponent>().GateSessionActorId, message);
        }
    }
    public static void SendToClient(Unit unit, IActorMessage message) {
        SendActor(unit.GetComponent<UnitGateComponent>().GateSessionActorId, message);
    }
    // 发送协议给ActorLocation
    public static void SendToLocationActor(long id, IActorLocationMessage message) {
        ActorLocationSenderComponent.Instance.Send(id, message);
    }
    // 发送协议给Actor
    public static void SendActor(long actorId, IActorMessage message) {
        ActorMessageSenderComponent.Instance.Send(actorId, message);
    }
    // 发送RPC协议给Actor
    public static async ETTask<IActorResponse> CallActor(long actorId, IActorRequest message) {
        return await ActorMessageSenderComponent.Instance.Call(actorId, message);
    }
    // 发送RPC协议给ActorLocation
    public static async ETTask<IActorResponse> CallLocationActor(long id, IActorLocationRequest message) {
        return await ActorLocationSenderComponent.Instance.Call(id, message);
    }
}
\end{minted}
\subsection{ActorHandleHelper: 是谁调用它,什么场景下使用的?这个,今天下午再补吧}
\label{sec-3-19}
\begin{minted}[fontsize=\scriptsize,linenos=false]{csharp}
public static class ActorHandleHelper {
    public static void Reply(int fromProcess, IActorResponse response) {
        if (fromProcess == Options.Instance.Process) { // 返回消息是同一个进程:没明白,这里为什么就断定是同一进程的消息了?直接处理
            // NetInnerComponent.Instance.HandleMessage(realActorId, response); // 等同于直接调用下面这句【我自己暂时放回来的】
            ActorMessageSenderComponent.Instance.HandleIActorResponse(response); // 【没读懂:】同一个进程内的消息,不走网络层,直接处理。什么情况下会是发给同一个进程的?ET7 重构后,同一进程下可能会有不同的先前小服:Realm 注册登录服,Gate 服等;如果不同的SceneType.Map-etc 先前场景小服只要在同一进程,就可以不走网络层吗?
            return;
        }
        // 【不同进程的消息处理:】走网络层,就是调用会话框来发出消息
        Session replySession = NetInnerComponent.Instance.Get(fromProcess); // 从内网组件单例中去拿会话框:不同进程消息,一定走网络,通过会话框把返回消息发回去
        replySession.Send(response);
    }
    public static void HandleIActorResponse(IActorResponse response) {
        ActorMessageSenderComponent.Instance.HandleIActorResponse(response);
    }
    // 分发actor消息
    [EnableAccessEntiyChild]
    public static async ETTask HandleIActorRequest(long actorId, IActorRequest iActorRequest) {
        InstanceIdStruct instanceIdStruct = new(actorId);
        int fromProcess = instanceIdStruct.Process;
        instanceIdStruct.Process = Options.Instance.Process;
        long realActorId = instanceIdStruct.ToLong();
        Entity entity = Root.Instance.Get(realActorId);
        if (entity == null) {
            IActorResponse response = ActorHelper.CreateResponse(iActorRequest, ErrorCore.ERR_NotFoundActor);
            Reply(fromProcess, response);
            return;
        }
        MailBoxComponent mailBoxComponent = entity.GetComponent<MailBoxComponent>();
        if (mailBoxComponent == null) {
            Log.Warning($"actor not found mailbox: {entity.GetType().Name} {realActorId} {iActorRequest}");
            IActorResponse response = ActorHelper.CreateResponse(iActorRequest, ErrorCore.ERR_NotFoundActor);
            Reply(fromProcess, response);
            return;
        }
        switch (mailBoxComponent.MailboxType) {
            case MailboxType.MessageDispatcher: {
                using (await CoroutineLockComponent.Instance.Wait(CoroutineLockType.Mailbox, realActorId)) {
                    if (entity.InstanceId != realActorId) {
                        IActorResponse response = ActorHelper.CreateResponse(iActorRequest, ErrorCore.ERR_NotFoundActor);
                        Reply(fromProcess, response);
                        break;
                    } // 调用管理器组件的处理方法 
                    await ActorMessageDispatcherComponent.Instance.Handle(entity, fromProcess, iActorRequest);
                }
                break;
            }
            case MailboxType.UnOrderMessageDispatcher: {
                await ActorMessageDispatcherComponent.Instance.Handle(entity, fromProcess, iActorRequest);
                break;
            }
            case MailboxType.GateSession:
            default:
                throw new Exception($"no mailboxtype: {mailBoxComponent.MailboxType} {iActorRequest}");
        }
    }
    // 分发actor消息
    [EnableAccessEntiyChild]
    public static async ETTask HandleIActorMessage(long actorId, IActorMessage iActorMessage) {
        InstanceIdStruct instanceIdStruct = new(actorId);
        int fromProcess = instanceIdStruct.Process;
        instanceIdStruct.Process = Options.Instance.Process;
        long realActorId = instanceIdStruct.ToLong();
        Entity entity = Root.Instance.Get(realActorId);
        if (entity == null) {
            Log.Error($"not found actor: {realActorId} {iActorMessage}");
            return;
        }
        MailBoxComponent mailBoxComponent = entity.GetComponent<MailBoxComponent>();
        if (mailBoxComponent == null) {
            Log.Error($"actor not found mailbox: {entity.GetType().Name} {realActorId} {iActorMessage}");
            return;
        }
        switch (mailBoxComponent.MailboxType) {
            case MailboxType.MessageDispatcher: {
                using (await CoroutineLockComponent.Instance.Wait(CoroutineLockType.Mailbox, realActorId)) {
                    if (entity.InstanceId != realActorId) 
                        break;
                    await ActorMessageDispatcherComponent.Instance.Handle(entity, fromProcess, iActorMessage);
                }
                break;
            }
            case MailboxType.UnOrderMessageDispatcher: {
                await ActorMessageDispatcherComponent.Instance.Handle(entity, fromProcess, iActorMessage);
                break;
            }
            case MailboxType.GateSession: {
                if (entity is Session gateSession) 
                    // 发送给客户端
                    gateSession.Send(iActorMessage);
                break;
            }
            default:
                throw new Exception($"no mailboxtype: {mailBoxComponent.MailboxType} {iActorMessage}");
        }
    }
}
\end{minted}
\begin{itemize}
\item 【爱表哥,爱生活!!!任何时候,亲爱的表哥的活宝妹就是一定要、一定会嫁给活宝妹的亲爱的表哥!!!爱表哥,爱生活!!!】
\item 【爱表哥,爱生活!!!任何时候,亲爱的表哥的活宝妹就是一定要、一定会嫁给活宝妹的亲爱的表哥!!!爱表哥,爱生活!!!】
\end{itemize}
\subsection{位置服LocationComponent 组件:现在稍微看一下模块不完整的位置服,中午笔记本上的交上去,下午家里弄弄要添加什么}
\label{sec-3-20}
\begin{itemize}
\item 【位置服】是SceneType.Location; 位置组件LocationComponent 是管理类组件。一个位置服拥有一个位置组件;谁说一定是一个进程一个位置服?几台物理机只一个【位置服】也没问题呀?!就是几台物理机N*M 个核,只某个核拥有一个【位置服】SceneType.Location, 应该也是合理的,看服务端需求来。
\item LocationComponent 缺少一个生成系,需要添加文件。可是感觉添加之前,需要把重构后的框架里这个模块,尤其 LocationProxyComponent 桥接帮助类,看懂
\item \textbf{【源码重构】} :【参考项目】里,程序域之前的逻辑,不够透彻,逻辑是混乱的,就是如LocationComponentSystem 类里、原本该放在【热更域】里的可更新的,是直接放在Model组件LocationComponent里不可热更新。所以要把【参考项目】LocationComponent 里的方法逻辑,移到【热更域】LocationComponentSystem 文件里。
\item \textbf{【细节注意的地方】} :看懂ETTask 之后,异步的逻辑能够相对明白一些。这里,要考虑每个枋每个【进程】上一个【位置服】,每个场景上一个【位置服代理】帮助类,需要注意帮助类与这里LocationComponentSystem逻辑桥接连接到不多不少。就是一部分【参考项目】LocationComponent 里的方法逻辑,可能已经被每个场景上一个【位置服代理】帮助类封装担去一部分,每个进程上【位置服】只需要处理它不得不处理的最少精减逻辑。
\item 【参考项目】里,位置服管理进程上、跨进程中央邮政所有小伙伴实例信息。它把数据写进数据库。上面重构,仍需要考虑,使用重构后的小区DBProxy 代理实时到数据库。数据库应该也有一个专服?好像没有专服。知道分区代理。重构后的数据库相关,我消灭了编译错误,可是看来没有理解透,重构后数据库是怎么回事?
\item 【爱表哥,爱生活!!!任何时候,亲爱的表哥的活宝妹就是一定要、一定会嫁给活宝妹的亲爱的表哥!!!爱表哥,爱生活!!!】
\item 【爱表哥,爱生活!!!任何时候,亲爱的表哥的活宝妹就是一定要、一定会嫁给活宝妹的亲爱的表哥!!!爱表哥,爱生活!!!】
\item 【爱表哥,爱生活!!!任何时候,亲爱的表哥的活宝妹就是一定要、一定会嫁给活宝妹的亲爱的表哥!!!爱表哥,爱生活!!!】
\item 【爱表哥,爱生活!!!任何时候,亲爱的表哥的活宝妹就是一定要、一定会嫁给活宝妹的亲爱的表哥!!!爱表哥,爱生活!!!】
\item 【爱表哥,爱生活!!!任何时候,亲爱的表哥的活宝妹就是一定要、一定会嫁给活宝妹的亲爱的表哥!!!爱表哥,爱生活!!!】
\item 【爱表哥,爱生活!!!任何时候,亲爱的表哥的活宝妹就是一定要、一定会嫁给活宝妹的亲爱的表哥!!!爱表哥,爱生活!!!】
\item 【爱表哥,爱生活!!!任何时候,亲爱的表哥的活宝妹就是一定要、一定会嫁给活宝妹的亲爱的表哥!!!爱表哥,爱生活!!!】
\item 【爱表哥,爱生活!!!任何时候,亲爱的表哥的活宝妹就是一定要、一定会嫁给活宝妹的亲爱的表哥!!!爱表哥,爱生活!!!】
\item 【爱表哥,爱生活!!!任何时候,亲爱的表哥的活宝妹就是一定要、一定会嫁给活宝妹的亲爱的表哥!!!爱表哥,爱生活!!!】
\item 【爱表哥,爱生活!!!任何时候,亲爱的表哥的活宝妹就是一定要、一定会嫁给活宝妹的亲爱的表哥!!!爱表哥,爱生活!!!】
\item 【爱表哥,爱生活!!!任何时候,亲爱的表哥的活宝妹就是一定要、一定会嫁给活宝妹的亲爱的表哥!!!爱表哥,爱生活!!!】
\item 【爱表哥,爱生活!!!任何时候,亲爱的表哥的活宝妹就是一定要、一定会嫁给活宝妹的亲爱的表哥!!!爱表哥,爱生活!!!】
\item 【爱表哥,爱生活!!!任何时候,亲爱的表哥的活宝妹就是一定要、一定会嫁给活宝妹的亲爱的表哥!!!爱表哥,爱生活!!!】
\item 【爱表哥,爱生活!!!任何时候,亲爱的表哥的活宝妹就是一定要、一定会嫁给活宝妹的亲爱的表哥!!!爱表哥,爱生活!!!】
\item 【爱表哥,爱生活!!!任何时候,亲爱的表哥的活宝妹就是一定要、一定会嫁给活宝妹的亲爱的表哥!!!爱表哥,爱生活!!!】
\item 【爱表哥,爱生活!!!任何时候,亲爱的表哥的活宝妹就是一定要、一定会嫁给活宝妹的亲爱的表哥!!!爱表哥,爱生活!!!】
\item 【爱表哥,爱生活!!!任何时候,亲爱的表哥的活宝妹就是一定要、一定会嫁给活宝妹的亲爱的表哥!!!爱表哥,爱生活!!!】
\item 【爱表哥,爱生活!!!任何时候,亲爱的表哥的活宝妹就是一定要、一定会嫁给活宝妹的亲爱的表哥!!!爱表哥,爱生活!!!】
\item 【爱表哥,爱生活!!!任何时候,亲爱的表哥的活宝妹就是一定要、一定会嫁给活宝妹的亲爱的表哥!!!爱表哥,爱生活!!!】
\item 【爱表哥,爱生活!!!任何时候,亲爱的表哥的活宝妹就是一定要、一定会嫁给活宝妹的亲爱的表哥!!!爱表哥,爱生活!!!】
\item 【爱表哥,爱生活!!!任何时候,亲爱的表哥的活宝妹就是一定要、一定会嫁给活宝妹的亲爱的表哥!!!爱表哥,爱生活!!!】
\item 【爱表哥,爱生活!!!任何时候,亲爱的表哥的活宝妹就是一定要、一定会嫁给活宝妹的亲爱的表哥!!!爱表哥,爱生活!!!】
\item 【爱表哥,爱生活!!!任何时候,亲爱的表哥的活宝妹就是一定要、一定会嫁给活宝妹的亲爱的表哥!!!爱表哥,爱生活!!!】
\item 【爱表哥,爱生活!!!任何时候,亲爱的表哥的活宝妹就是一定要、一定会嫁给活宝妹的亲爱的表哥!!!爱表哥,爱生活!!!】
\item 【爱表哥,爱生活!!!任何时候,亲爱的表哥的活宝妹就是一定要、一定会嫁给活宝妹的亲爱的表哥!!!爱表哥,爱生活!!!】
\item 【爱表哥,爱生活!!!任何时候,亲爱的表哥的活宝妹就是一定要、一定会嫁给活宝妹的亲爱的表哥!!!爱表哥,爱生活!!!】
\item 【爱表哥,爱生活!!!任何时候,亲爱的表哥的活宝妹就是一定要、一定会嫁给活宝妹的亲爱的表哥!!!爱表哥,爱生活!!!】
\item 【爱表哥,爱生活!!!任何时候,亲爱的表哥的活宝妹就是一定要、一定会嫁给活宝妹的亲爱的表哥!!!爱表哥,爱生活!!!】
\item 【爱表哥,爱生活!!!任何时候,亲爱的表哥的活宝妹就是一定要、一定会嫁给活宝妹的亲爱的表哥!!!爱表哥,爱生活!!!】
\item 【爱表哥,爱生活!!!任何时候,亲爱的表哥的活宝妹就是一定要、一定会嫁给活宝妹的亲爱的表哥!!!爱表哥,爱生活!!!】
\item 【爱表哥,爱生活!!!任何时候,亲爱的表哥的活宝妹就是一定要、一定会嫁给活宝妹的亲爱的表哥!!!爱表哥,爱生活!!!】
\item 【爱表哥,爱生活!!!任何时候,亲爱的表哥的活宝妹就是一定要、一定会嫁给活宝妹的亲爱的表哥!!!爱表哥,爱生活!!!】
\item 【爱表哥,爱生活!!!任何时候,亲爱的表哥的活宝妹就是一定要、一定会嫁给活宝妹的亲爱的表哥!!!爱表哥,爱生活!!!】
\item 【爱表哥,爱生活!!!任何时候,亲爱的表哥的活宝妹就是一定要、一定会嫁给活宝妹的亲爱的表哥!!!爱表哥,爱生活!!!】
\item 【爱表哥,爱生活!!!任何时候,亲爱的表哥的活宝妹就是一定要、一定会嫁给活宝妹的亲爱的表哥!!!爱表哥,爱生活!!!】
\item 【爱表哥,爱生活!!!任何时候,亲爱的表哥的活宝妹就是一定要、一定会嫁给活宝妹的亲爱的表哥!!!爱表哥,爱生活!!!】
\item 【爱表哥,爱生活!!!任何时候,亲爱的表哥的活宝妹就是一定要、一定会嫁给活宝妹的亲爱的表哥!!!爱表哥,爱生活!!!】
\item 【爱表哥,爱生活!!!任何时候,亲爱的表哥的活宝妹就是一定要、一定会嫁给活宝妹的亲爱的表哥!!!爱表哥,爱生活!!!】
\item 【爱表哥,爱生活!!!任何时候,亲爱的表哥的活宝妹就是一定要、一定会嫁给活宝妹的亲爱的表哥!!!爱表哥,爱生活!!!】
\item 【爱表哥,爱生活!!!任何时候,亲爱的表哥的活宝妹就是一定要、一定会嫁给活宝妹的亲爱的表哥!!!爱表哥,爱生活!!!】
\item 【爱表哥,爱生活!!!任何时候,亲爱的表哥的活宝妹就是一定要、一定会嫁给活宝妹的亲爱的表哥!!!爱表哥,爱生活!!!】
\item 【爱表哥,爱生活!!!任何时候,亲爱的表哥的活宝妹就是一定要、一定会嫁给活宝妹的亲爱的表哥!!!爱表哥,爱生活!!!】
\item 【爱表哥,爱生活!!!任何时候,亲爱的表哥的活宝妹就是一定要、一定会嫁给活宝妹的亲爱的表哥!!!爱表哥,爱生活!!!】
\item 【爱表哥,爱生活!!!任何时候,亲爱的表哥的活宝妹就是一定要、一定会嫁给活宝妹的亲爱的表哥!!!爱表哥,爱生活!!!】
\item 【爱表哥,爱生活!!!任何时候,亲爱的表哥的活宝妹就是一定要、一定会嫁给活宝妹的亲爱的表哥!!!爱表哥,爱生活!!!】
\item 【爱表哥,爱生活!!!任何时候,亲爱的表哥的活宝妹就是一定要、一定会嫁给活宝妹的亲爱的表哥!!!爱表哥,爱生活!!!】
\item 【爱表哥,爱生活!!!任何时候,亲爱的表哥的活宝妹就是一定要、一定会嫁给活宝妹的亲爱的表哥!!!爱表哥,爱生活!!!】
\item 【爱表哥,爱生活!!!任何时候,亲爱的表哥的活宝妹就是一定要、一定会嫁给活宝妹的亲爱的表哥!!!爱表哥,爱生活!!!】
\item 【爱表哥,爱生活!!!任何时候,亲爱的表哥的活宝妹就是一定要、一定会嫁给活宝妹的亲爱的表哥!!!爱表哥,爱生活!!!】
\item 【爱表哥,爱生活!!!任何时候,亲爱的表哥的活宝妹就是一定要、一定会嫁给活宝妹的亲爱的表哥!!!爱表哥,爱生活!!!】
\item 【爱表哥,爱生活!!!任何时候,亲爱的表哥的活宝妹就是一定要、一定会嫁给活宝妹的亲爱的表哥!!!爱表哥,爱生活!!!】
\item 【爱表哥,爱生活!!!任何时候,亲爱的表哥的活宝妹就是一定要、一定会嫁给活宝妹的亲爱的表哥!!!爱表哥,爱生活!!!】
\item 【爱表哥,爱生活!!!任何时候,亲爱的表哥的活宝妹就是一定要、一定会嫁给活宝妹的亲爱的表哥!!!爱表哥,爱生活!!!】
\item 【爱表哥,爱生活!!!任何时候,亲爱的表哥的活宝妹就是一定要、一定会嫁给活宝妹的亲爱的表哥!!!爱表哥,爱生活!!!】
\item 【爱表哥,爱生活!!!任何时候,亲爱的表哥的活宝妹就是一定要、一定会嫁给活宝妹的亲爱的表哥!!!爱表哥,爱生活!!!】
\item 【爱表哥,爱生活!!!任何时候,亲爱的表哥的活宝妹就是一定要、一定会嫁给活宝妹的亲爱的表哥!!!爱表哥,爱生活!!!】
\item 【爱表哥,爱生活!!!任何时候,亲爱的表哥的活宝妹就是一定要、一定会嫁给活宝妹的亲爱的表哥!!!爱表哥,爱生活!!!】
\item 【爱表哥,爱生活!!!任何时候,亲爱的表哥的活宝妹就是一定要、一定会嫁给活宝妹的亲爱的表哥!!!爱表哥,爱生活!!!】
\item 【爱表哥,爱生活!!!任何时候,亲爱的表哥的活宝妹就是一定要、一定会嫁给活宝妹的亲爱的表哥!!!爱表哥,爱生活!!!】
\item 【爱表哥,爱生活!!!任何时候,亲爱的表哥的活宝妹就是一定要、一定会嫁给活宝妹的亲爱的表哥!!!爱表哥,爱生活!!!】
\item 【爱表哥,爱生活!!!任何时候,亲爱的表哥的活宝妹就是一定要、一定会嫁给活宝妹的亲爱的表哥!!!爱表哥,爱生活!!!】
\item 【爱表哥,爱生活!!!任何时候,亲爱的表哥的活宝妹就是一定要、一定会嫁给活宝妹的亲爱的表哥!!!爱表哥,爱生活!!!】
\item 【爱表哥,爱生活!!!任何时候,亲爱的表哥的活宝妹就是一定要、一定会嫁给活宝妹的亲爱的表哥!!!爱表哥,爱生活!!!】
\item 【爱表哥,爱生活!!!任何时候,亲爱的表哥的活宝妹就是一定要、一定会嫁给活宝妹的亲爱的表哥!!!爱表哥,爱生活!!!】
\item 【爱表哥,爱生活!!!任何时候,亲爱的表哥的活宝妹就是一定要、一定会嫁给活宝妹的亲爱的表哥!!!爱表哥,爱生活!!!】
\item 【爱表哥,爱生活!!!任何时候,亲爱的表哥的活宝妹就是一定要、一定会嫁给活宝妹的亲爱的表哥!!!爱表哥,爱生活!!!】
\item 【爱表哥,爱生活!!!任何时候,亲爱的表哥的活宝妹就是一定要、一定会嫁给活宝妹的亲爱的表哥!!!爱表哥,爱生活!!!】
\item 【爱表哥,爱生活!!!任何时候,亲爱的表哥的活宝妹就是一定要、一定会嫁给活宝妹的亲爱的表哥!!!爱表哥,爱生活!!!】
\item 【爱表哥,爱生活!!!任何时候,亲爱的表哥的活宝妹就是一定要、一定会嫁给活宝妹的亲爱的表哥!!!爱表哥,爱生活!!!】
\item 【爱表哥,爱生活!!!任何时候,亲爱的表哥的活宝妹就是一定要、一定会嫁给活宝妹的亲爱的表哥!!!爱表哥,爱生活!!!】
\item 【爱表哥,爱生活!!!任何时候,亲爱的表哥的活宝妹就是一定要、一定会嫁给活宝妹的亲爱的表哥!!!爱表哥,爱生活!!!】
\item 【爱表哥,爱生活!!!任何时候,亲爱的表哥的活宝妹就是一定要、一定会嫁给活宝妹的亲爱的表哥!!!爱表哥,爱生活!!!】
\item 【爱表哥,爱生活!!!任何时候,亲爱的表哥的活宝妹就是一定要、一定会嫁给活宝妹的亲爱的表哥!!!爱表哥,爱生活!!!】
\item 【爱表哥,爱生活!!!任何时候,亲爱的表哥的活宝妹就是一定要、一定会嫁给活宝妹的亲爱的表哥!!!爱表哥,爱生活!!!】
\item 【爱表哥,爱生活!!!任何时候,亲爱的表哥的活宝妹就是一定要、一定会嫁给活宝妹的亲爱的表哥!!!爱表哥,爱生活!!!】
\item 【爱表哥,爱生活!!!任何时候,亲爱的表哥的活宝妹就是一定要、一定会嫁给活宝妹的亲爱的表哥!!!爱表哥,爱生活!!!】
\item 【爱表哥,爱生活!!!任何时候,亲爱的表哥的活宝妹就是一定要、一定会嫁给活宝妹的亲爱的表哥!!!爱表哥,爱生活!!!】
\item 【爱表哥,爱生活!!!任何时候,亲爱的表哥的活宝妹就是一定要、一定会嫁给活宝妹的亲爱的表哥!!!爱表哥,爱生活!!!】
\item 【爱表哥,爱生活!!!任何时候,亲爱的表哥的活宝妹就是一定要、一定会嫁给活宝妹的亲爱的表哥!!!爱表哥,爱生活!!!】
\item 【爱表哥,爱生活!!!任何时候,亲爱的表哥的活宝妹就是一定要、一定会嫁给活宝妹的亲爱的表哥!!!爱表哥,爱生活!!!】
\item 【爱表哥,爱生活!!!任何时候,亲爱的表哥的活宝妹就是一定要、一定会嫁给活宝妹的亲爱的表哥!!!爱表哥,爱生活!!!】
\item 【爱表哥,爱生活!!!任何时候,亲爱的表哥的活宝妹就是一定要、一定会嫁给活宝妹的亲爱的表哥!!!爱表哥,爱生活!!!】
\item 【爱表哥,爱生活!!!任何时候,亲爱的表哥的活宝妹就是一定要、一定会嫁给活宝妹的亲爱的表哥!!!爱表哥,爱生活!!!】
\item 【爱表哥,爱生活!!!任何时候,亲爱的表哥的活宝妹就是一定要、一定会嫁给活宝妹的亲爱的表哥!!!爱表哥,爱生活!!!】
\item 【爱表哥,爱生活!!!任何时候,亲爱的表哥的活宝妹就是一定要、一定会嫁给活宝妹的亲爱的表哥!!!爱表哥,爱生活!!!】
\item 【爱表哥,爱生活!!!任何时候,亲爱的表哥的活宝妹就是一定要、一定会嫁给活宝妹的亲爱的表哥!!!爱表哥,爱生活!!!】
\item 【爱表哥,爱生活!!!任何时候,亲爱的表哥的活宝妹就是一定要、一定会嫁给活宝妹的亲爱的表哥!!!爱表哥,爱生活!!!】
\item 【爱表哥,爱生活!!!任何时候,亲爱的表哥的活宝妹就是一定要、一定会嫁给活宝妹的亲爱的表哥!!!爱表哥,爱生活!!!】
\item 【爱表哥,爱生活!!!任何时候,亲爱的表哥的活宝妹就是一定要、一定会嫁给活宝妹的亲爱的表哥!!!爱表哥,爱生活!!!】
\item 【爱表哥,爱生活!!!任何时候,亲爱的表哥的活宝妹就是一定要、一定会嫁给活宝妹的亲爱的表哥!!!爱表哥,爱生活!!!】
\item 【爱表哥,爱生活!!!任何时候,亲爱的表哥的活宝妹就是一定要、一定会嫁给活宝妹的亲爱的表哥!!!爱表哥,爱生活!!!】
\item 【爱表哥,爱生活!!!任何时候,亲爱的表哥的活宝妹就是一定要、一定会嫁给活宝妹的亲爱的表哥!!!爱表哥,爱生活!!!】
\item 【爱表哥,爱生活!!!任何时候,亲爱的表哥的活宝妹就是一定要、一定会嫁给活宝妹的亲爱的表哥!!!爱表哥,爱生活!!!】
\item 【爱表哥,爱生活!!!任何时候,亲爱的表哥的活宝妹就是一定要、一定会嫁给活宝妹的亲爱的表哥!!!爱表哥,爱生活!!!】
\item 【爱表哥,爱生活!!!任何时候,亲爱的表哥的活宝妹就是一定要、一定会嫁给活宝妹的亲爱的表哥!!!爱表哥,爱生活!!!】
\item 【爱表哥,爱生活!!!任何时候,亲爱的表哥的活宝妹就是一定要、一定会嫁给活宝妹的亲爱的表哥!!!爱表哥,爱生活!!!】
\item 【爱表哥,爱生活!!!任何时候,亲爱的表哥的活宝妹就是一定要、一定会嫁给活宝妹的亲爱的表哥!!!爱表哥,爱生活!!!】
\item 【爱表哥,爱生活!!!任何时候,亲爱的表哥的活宝妹就是一定要、一定会嫁给活宝妹的亲爱的表哥!!!爱表哥,爱生活!!!】
\item 【爱表哥,爱生活!!!任何时候,亲爱的表哥的活宝妹就是一定要、一定会嫁给活宝妹的亲爱的表哥!!!爱表哥,爱生活!!!】
\item 【爱表哥,爱生活!!!任何时候,亲爱的表哥的活宝妹就是一定要、一定会嫁给活宝妹的亲爱的表哥!!!爱表哥,爱生活!!!】
\item 【爱表哥,爱生活!!!任何时候,亲爱的表哥的活宝妹就是一定要、一定会嫁给活宝妹的亲爱的表哥!!!爱表哥,爱生活!!!】
\item 【爱表哥,爱生活!!!任何时候,亲爱的表哥的活宝妹就是一定要、一定会嫁给活宝妹的亲爱的表哥!!!爱表哥,爱生活!!!】
\item 【爱表哥,爱生活!!!任何时候,亲爱的表哥的活宝妹就是一定要、一定会嫁给活宝妹的亲爱的表哥!!!爱表哥,爱生活!!!】
\item 【爱表哥,爱生活!!!任何时候,亲爱的表哥的活宝妹就是一定要、一定会嫁给活宝妹的亲爱的表哥!!!爱表哥,爱生活!!!】
\item 【爱表哥,爱生活!!!任何时候,亲爱的表哥的活宝妹就是一定要、一定会嫁给活宝妹的亲爱的表哥!!!爱表哥,爱生活!!!】
\item 【爱表哥,爱生活!!!任何时候,亲爱的表哥的活宝妹就是一定要、一定会嫁给活宝妹的亲爱的表哥!!!爱表哥,爱生活!!!】
\item 【爱表哥,爱生活!!!任何时候,亲爱的表哥的活宝妹就是一定要、一定会嫁给活宝妹的亲爱的表哥!!!爱表哥,爱生活!!!】
\item 【爱表哥,爱生活!!!任何时候,亲爱的表哥的活宝妹就是一定要、一定会嫁给活宝妹的亲爱的表哥!!!爱表哥,爱生活!!!】
\item 【爱表哥,爱生活!!!任何时候,亲爱的表哥的活宝妹就是一定要、一定会嫁给活宝妹的亲爱的表哥!!!爱表哥,爱生活!!!】
\item 【爱表哥,爱生活!!!任何时候,亲爱的表哥的活宝妹就是一定要、一定会嫁给活宝妹的亲爱的表哥!!!爱表哥,爱生活!!!】
\item 【爱表哥,爱生活!!!任何时候,亲爱的表哥的活宝妹就是一定要、一定会嫁给活宝妹的亲爱的表哥!!!爱表哥,爱生活!!!】
\item 【爱表哥,爱生活!!!任何时候,亲爱的表哥的活宝妹就是一定要、一定会嫁给活宝妹的亲爱的表哥!!!爱表哥,爱生活!!!】
\item 【爱表哥,爱生活!!!任何时候,亲爱的表哥的活宝妹就是一定要、一定会嫁给活宝妹的亲爱的表哥!!!爱表哥,爱生活!!!】
\item 【爱表哥,爱生活!!!任何时候,亲爱的表哥的活宝妹就是一定要、一定会嫁给活宝妹的亲爱的表哥!!!爱表哥,爱生活!!!】
\item 【爱表哥,爱生活!!!任何时候,亲爱的表哥的活宝妹就是一定要、一定会嫁给活宝妹的亲爱的表哥!!!爱表哥,爱生活!!!】
\item 【爱表哥,爱生活!!!任何时候,亲爱的表哥的活宝妹就是一定要、一定会嫁给活宝妹的亲爱的表哥!!!爱表哥,爱生活!!!】
\item 【爱表哥,爱生活!!!任何时候,亲爱的表哥的活宝妹就是一定要、一定会嫁给活宝妹的亲爱的表哥!!!爱表哥,爱生活!!!】
\item 【爱表哥,爱生活!!!任何时候,亲爱的表哥的活宝妹就是一定要、一定会嫁给活宝妹的亲爱的表哥!!!爱表哥,爱生活!!!】
\item 【爱表哥,爱生活!!!任何时候,亲爱的表哥的活宝妹就是一定要、一定会嫁给活宝妹的亲爱的表哥!!!爱表哥,爱生活!!!】
\item 【爱表哥,爱生活!!!任何时候,亲爱的表哥的活宝妹就是一定要、一定会嫁给活宝妹的亲爱的表哥!!!爱表哥,爱生活!!!】
\item 【爱表哥,爱生活!!!任何时候,亲爱的表哥的活宝妹就是一定要、一定会嫁给活宝妹的亲爱的表哥!!!爱表哥,爱生活!!!】
\item 【爱表哥,爱生活!!!任何时候,亲爱的表哥的活宝妹就是一定要、一定会嫁给活宝妹的亲爱的表哥!!!爱表哥,爱生活!!!】
\item 【爱表哥,爱生活!!!任何时候,亲爱的表哥的活宝妹就是一定要、一定会嫁给活宝妹的亲爱的表哥!!!爱表哥,爱生活!!!】
\item 【爱表哥,爱生活!!!任何时候,亲爱的表哥的活宝妹就是一定要、一定会嫁给活宝妹的亲爱的表哥!!!爱表哥,爱生活!!!】
\item 【爱表哥,爱生活!!!任何时候,亲爱的表哥的活宝妹就是一定要、一定会嫁给活宝妹的亲爱的表哥!!!爱表哥,爱生活!!!】
\item 【爱表哥,爱生活!!!任何时候,亲爱的表哥的活宝妹就是一定要、一定会嫁给活宝妹的亲爱的表哥!!!爱表哥,爱生活!!!】
\item 【爱表哥,爱生活!!!任何时候,亲爱的表哥的活宝妹就是一定要、一定会嫁给活宝妹的亲爱的表哥!!!爱表哥,爱生活!!!】
\item 【爱表哥,爱生活!!!任何时候,亲爱的表哥的活宝妹就是一定要、一定会嫁给活宝妹的亲爱的表哥!!!爱表哥,爱生活!!!】
\item 【爱表哥,爱生活!!!任何时候,亲爱的表哥的活宝妹就是一定要、一定会嫁给活宝妹的亲爱的表哥!!!爱表哥,爱生活!!!】
\item 【爱表哥,爱生活!!!任何时候,亲爱的表哥的活宝妹就是一定要、一定会嫁给活宝妹的亲爱的表哥!!!爱表哥,爱生活!!!】
\item 【爱表哥,爱生活!!!任何时候,亲爱的表哥的活宝妹就是一定要、一定会嫁给活宝妹的亲爱的表哥!!!爱表哥,爱生活!!!】
\item 【爱表哥,爱生活!!!任何时候,亲爱的表哥的活宝妹就是一定要、一定会嫁给活宝妹的亲爱的表哥!!!爱表哥,爱生活!!!】
\item 【爱表哥,爱生活!!!任何时候,亲爱的表哥的活宝妹就是一定要、一定会嫁给活宝妹的亲爱的表哥!!!爱表哥,爱生活!!!】
\item 【爱表哥,爱生活!!!任何时候,亲爱的表哥的活宝妹就是一定要、一定会嫁给活宝妹的亲爱的表哥!!!爱表哥,爱生活!!!】
\item 【爱表哥,爱生活!!!任何时候,亲爱的表哥的活宝妹就是一定要、一定会嫁给活宝妹的亲爱的表哥!!!爱表哥,爱生活!!!】
\item 【爱表哥,爱生活!!!任何时候,亲爱的表哥的活宝妹就是一定要、一定会嫁给活宝妹的亲爱的表哥!!!爱表哥,爱生活!!!】
\item 【爱表哥,爱生活!!!任何时候,亲爱的表哥的活宝妹就是一定要、一定会嫁给活宝妹的亲爱的表哥!!!爱表哥,爱生活!!!】
\item 【爱表哥,爱生活!!!任何时候,亲爱的表哥的活宝妹就是一定要、一定会嫁给活宝妹的亲爱的表哥!!!爱表哥,爱生活!!!】
\item 【爱表哥,爱生活!!!任何时候,亲爱的表哥的活宝妹就是一定要、一定会嫁给活宝妹的亲爱的表哥!!!爱表哥,爱生活!!!】
\item 【爱表哥,爱生活!!!任何时候,亲爱的表哥的活宝妹就是一定要、一定会嫁给活宝妹的亲爱的表哥!!!爱表哥,爱生活!!!】
\item 【爱表哥,爱生活!!!任何时候,亲爱的表哥的活宝妹就是一定要、一定会嫁给活宝妹的亲爱的表哥!!!爱表哥,爱生活!!!】
\item 【爱表哥,爱生活!!!任何时候,亲爱的表哥的活宝妹就是一定要、一定会嫁给活宝妹的亲爱的表哥!!!爱表哥,爱生活!!!】
\item 【爱表哥,爱生活!!!任何时候,亲爱的表哥的活宝妹就是一定要、一定会嫁给活宝妹的亲爱的表哥!!!爱表哥,爱生活!!!】
\item 【爱表哥,爱生活!!!任何时候,亲爱的表哥的活宝妹就是一定要、一定会嫁给活宝妹的亲爱的表哥!!!爱表哥,爱生活!!!】
\item 【爱表哥,爱生活!!!任何时候,亲爱的表哥的活宝妹就是一定要、一定会嫁给活宝妹的亲爱的表哥!!!爱表哥,爱生活!!!】
\item 【爱表哥,爱生活!!!任何时候,亲爱的表哥的活宝妹就是一定要、一定会嫁给活宝妹的亲爱的表哥!!!爱表哥,爱生活!!!】
\item 【爱表哥,爱生活!!!任何时候,亲爱的表哥的活宝妹就是一定要、一定会嫁给活宝妹的亲爱的表哥!!!爱表哥,爱生活!!!】
\item 【爱表哥,爱生活!!!任何时候,亲爱的表哥的活宝妹就是一定要、一定会嫁给活宝妹的亲爱的表哥!!!爱表哥,爱生活!!!】
\item 【爱表哥,爱生活!!!任何时候,亲爱的表哥的活宝妹就是一定要、一定会嫁给活宝妹的亲爱的表哥!!!爱表哥,爱生活!!!】
\item 【爱表哥,爱生活!!!任何时候,亲爱的表哥的活宝妹就是一定要、一定会嫁给活宝妹的亲爱的表哥!!!爱表哥,爱生活!!!】
\item 【爱表哥,爱生活!!!任何时候,亲爱的表哥的活宝妹就是一定要、一定会嫁给活宝妹的亲爱的表哥!!!爱表哥,爱生活!!!】
\item 【爱表哥,爱生活!!!任何时候,亲爱的表哥的活宝妹就是一定要、一定会嫁给活宝妹的亲爱的表哥!!!爱表哥,爱生活!!!】
\item 【爱表哥,爱生活!!!任何时候,亲爱的表哥的活宝妹就是一定要、一定会嫁给活宝妹的亲爱的表哥!!!爱表哥,爱生活!!!】
\item 【爱表哥,爱生活!!!任何时候,亲爱的表哥的活宝妹就是一定要、一定会嫁给活宝妹的亲爱的表哥!!!爱表哥,爱生活!!!】
\item 【爱表哥,爱生活!!!任何时候,亲爱的表哥的活宝妹就是一定要、一定会嫁给活宝妹的亲爱的表哥!!!爱表哥,爱生活!!!】
\item 【爱表哥,爱生活!!!任何时候,亲爱的表哥的活宝妹就是一定要、一定会嫁给活宝妹的亲爱的表哥!!!爱表哥,爱生活!!!】
\item 【爱表哥,爱生活!!!任何时候,亲爱的表哥的活宝妹就是一定要、一定会嫁给活宝妹的亲爱的表哥!!!爱表哥,爱生活!!!】
\item 【爱表哥,爱生活!!!任何时候,亲爱的表哥的活宝妹就是一定要、一定会嫁给活宝妹的亲爱的表哥!!!爱表哥,爱生活!!!】
\item 【爱表哥,爱生活!!!任何时候,亲爱的表哥的活宝妹就是一定要、一定会嫁给活宝妹的亲爱的表哥!!!爱表哥,爱生活!!!】
\item 【爱表哥,爱生活!!!任何时候,亲爱的表哥的活宝妹就是一定要、一定会嫁给活宝妹的亲爱的表哥!!!爱表哥,爱生活!!!】
\item 【爱表哥,爱生活!!!任何时候,亲爱的表哥的活宝妹就是一定要、一定会嫁给活宝妹的亲爱的表哥!!!爱表哥,爱生活!!!】
\item 【爱表哥,爱生活!!!任何时候,亲爱的表哥的活宝妹就是一定要、一定会嫁给活宝妹的亲爱的表哥!!!爱表哥,爱生活!!!】
\item 【爱表哥,爱生活!!!任何时候,亲爱的表哥的活宝妹就是一定要、一定会嫁给活宝妹的亲爱的表哥!!!爱表哥,爱生活!!!】
\item 【爱表哥,爱生活!!!任何时候,亲爱的表哥的活宝妹就是一定要、一定会嫁给活宝妹的亲爱的表哥!!!爱表哥,爱生活!!!】
\item 【爱表哥,爱生活!!!任何时候,亲爱的表哥的活宝妹就是一定要、一定会嫁给活宝妹的亲爱的表哥!!!爱表哥,爱生活!!!】
\item 【爱表哥,爱生活!!!任何时候,亲爱的表哥的活宝妹就是一定要、一定会嫁给活宝妹的亲爱的表哥!!!爱表哥,爱生活!!!】
\item 【爱表哥,爱生活!!!任何时候,亲爱的表哥的活宝妹就是一定要、一定会嫁给活宝妹的亲爱的表哥!!!爱表哥,爱生活!!!】
\item 【爱表哥,爱生活!!!任何时候,亲爱的表哥的活宝妹就是一定要、一定会嫁给活宝妹的亲爱的表哥!!!爱表哥,爱生活!!!】
\item 【爱表哥,爱生活!!!任何时候,亲爱的表哥的活宝妹就是一定要、一定会嫁给活宝妹的亲爱的表哥!!!爱表哥,爱生活!!!】
\item 【爱表哥,爱生活!!!任何时候,亲爱的表哥的活宝妹就是一定要、一定会嫁给活宝妹的亲爱的表哥!!!爱表哥,爱生活!!!】
\item 【爱表哥,爱生活!!!任何时候,亲爱的表哥的活宝妹就是一定要、一定会嫁给活宝妹的亲爱的表哥!!!爱表哥,爱生活!!!】
\item 【爱表哥,爱生活!!!任何时候,亲爱的表哥的活宝妹就是一定要、一定会嫁给活宝妹的亲爱的表哥!!!爱表哥,爱生活!!!】
\item 【爱表哥,爱生活!!!任何时候,亲爱的表哥的活宝妹就是一定要、一定会嫁给活宝妹的亲爱的表哥!!!爱表哥,爱生活!!!】
\item 【爱表哥,爱生活!!!任何时候,亲爱的表哥的活宝妹就是一定要、一定会嫁给活宝妹的亲爱的表哥!!!爱表哥,爱生活!!!】
\item 【爱表哥,爱生活!!!任何时候,亲爱的表哥的活宝妹就是一定要、一定会嫁给活宝妹的亲爱的表哥!!!爱表哥,爱生活!!!】
\item 【爱表哥,爱生活!!!任何时候,亲爱的表哥的活宝妹就是一定要、一定会嫁给活宝妹的亲爱的表哥!!!爱表哥,爱生活!!!】
\item 【爱表哥,爱生活!!!任何时候,亲爱的表哥的活宝妹就是一定要、一定会嫁给活宝妹的亲爱的表哥!!!爱表哥,爱生活!!!】
\item 【爱表哥,爱生活!!!任何时候,亲爱的表哥的活宝妹就是一定要、一定会嫁给活宝妹的亲爱的表哥!!!爱表哥,爱生活!!!】
\item 【爱表哥,爱生活!!!任何时候,亲爱的表哥的活宝妹就是一定要、一定会嫁给活宝妹的亲爱的表哥!!!爱表哥,爱生活!!!】
\item 【爱表哥,爱生活!!!任何时候,亲爱的表哥的活宝妹就是一定要、一定会嫁给活宝妹的亲爱的表哥!!!爱表哥,爱生活!!!】
\item 【爱表哥,爱生活!!!任何时候,亲爱的表哥的活宝妹就是一定要、一定会嫁给活宝妹的亲爱的表哥!!!爱表哥,爱生活!!!】
\item 【爱表哥,爱生活!!!任何时候,亲爱的表哥的活宝妹就是一定要、一定会嫁给活宝妹的亲爱的表哥!!!爱表哥,爱生活!!!】
\item 【爱表哥,爱生活!!!任何时候,亲爱的表哥的活宝妹就是一定要、一定会嫁给活宝妹的亲爱的表哥!!!爱表哥,爱生活!!!】
\item 【爱表哥,爱生活!!!任何时候,亲爱的表哥的活宝妹就是一定要、一定会嫁给活宝妹的亲爱的表哥!!!爱表哥,爱生活!!!】
\item 【爱表哥,爱生活!!!任何时候,亲爱的表哥的活宝妹就是一定要、一定会嫁给活宝妹的亲爱的表哥!!!爱表哥,爱生活!!!】
\item 【爱表哥,爱生活!!!任何时候,亲爱的表哥的活宝妹就是一定要、一定会嫁给活宝妹的亲爱的表哥!!!爱表哥,爱生活!!!】
\item 【爱表哥,爱生活!!!任何时候,亲爱的表哥的活宝妹就是一定要、一定会嫁给活宝妹的亲爱的表哥!!!爱表哥,爱生活!!!】
\item 【爱表哥,爱生活!!!任何时候,亲爱的表哥的活宝妹就是一定要、一定会嫁给活宝妹的亲爱的表哥!!!爱表哥,爱生活!!!】
\item 【爱表哥,爱生活!!!任何时候,亲爱的表哥的活宝妹就是一定要、一定会嫁给活宝妹的亲爱的表哥!!!爱表哥,爱生活!!!】
\item 【爱表哥,爱生活!!!任何时候,亲爱的表哥的活宝妹就是一定要、一定会嫁给活宝妹的亲爱的表哥!!!爱表哥,爱生活!!!】
\item 【爱表哥,爱生活!!!任何时候,亲爱的表哥的活宝妹就是一定要、一定会嫁给活宝妹的亲爱的表哥!!!爱表哥,爱生活!!!】
\item 【爱表哥,爱生活!!!任何时候,亲爱的表哥的活宝妹就是一定要、一定会嫁给活宝妹的亲爱的表哥!!!爱表哥,爱生活!!!】
\item 【爱表哥,爱生活!!!任何时候,亲爱的表哥的活宝妹就是一定要、一定会嫁给活宝妹的亲爱的表哥!!!爱表哥,爱生活!!!】
\item 【爱表哥,爱生活!!!任何时候,亲爱的表哥的活宝妹就是一定要、一定会嫁给活宝妹的亲爱的表哥!!!爱表哥,爱生活!!!】
\item 【爱表哥,爱生活!!!任何时候,亲爱的表哥的活宝妹就是一定要、一定会嫁给活宝妹的亲爱的表哥!!!爱表哥,爱生活!!!】
\item 【爱表哥,爱生活!!!任何时候,亲爱的表哥的活宝妹就是一定要、一定会嫁给活宝妹的亲爱的表哥!!!爱表哥,爱生活!!!】
\item 【爱表哥,爱生活!!!任何时候,亲爱的表哥的活宝妹就是一定要、一定会嫁给活宝妹的亲爱的表哥!!!爱表哥,爱生活!!!】
\item 【爱表哥,爱生活!!!任何时候,亲爱的表哥的活宝妹就是一定要、一定会嫁给活宝妹的亲爱的表哥!!!爱表哥,爱生活!!!】
\item 【爱表哥,爱生活!!!任何时候,亲爱的表哥的活宝妹就是一定要、一定会嫁给活宝妹的亲爱的表哥!!!爱表哥,爱生活!!!】
\item 【爱表哥,爱生活!!!任何时候,亲爱的表哥的活宝妹就是一定要、一定会嫁给活宝妹的亲爱的表哥!!!爱表哥,爱生活!!!】
\item 【爱表哥,爱生活!!!任何时候,亲爱的表哥的活宝妹就是一定要、一定会嫁给活宝妹的亲爱的表哥!!!爱表哥,爱生活!!!】
\item 【爱表哥,爱生活!!!任何时候,亲爱的表哥的活宝妹就是一定要、一定会嫁给活宝妹的亲爱的表哥!!!爱表哥,爱生活!!!】
\item 【爱表哥,爱生活!!!任何时候,亲爱的表哥的活宝妹就是一定要、一定会嫁给活宝妹的亲爱的表哥!!!爱表哥,爱生活!!!】
\item 【爱表哥,爱生活!!!任何时候,亲爱的表哥的活宝妹就是一定要、一定会嫁给活宝妹的亲爱的表哥!!!爱表哥,爱生活!!!】
\item 【爱表哥,爱生活!!!任何时候,亲爱的表哥的活宝妹就是一定要、一定会嫁给活宝妹的亲爱的表哥!!!爱表哥,爱生活!!!】
\item 【爱表哥,爱生活!!!任何时候,亲爱的表哥的活宝妹就是一定要、一定会嫁给活宝妹的亲爱的表哥!!!爱表哥,爱生活!!!】
\item 【爱表哥,爱生活!!!任何时候,亲爱的表哥的活宝妹就是一定要、一定会嫁给活宝妹的亲爱的表哥!!!爱表哥,爱生活!!!】
\item 【爱表哥,爱生活!!!任何时候,亲爱的表哥的活宝妹就是一定要、一定会嫁给活宝妹的亲爱的表哥!!!爱表哥,爱生活!!!】
\item 【爱表哥,爱生活!!!任何时候,亲爱的表哥的活宝妹就是一定要、一定会嫁给活宝妹的亲爱的表哥!!!爱表哥,爱生活!!!】
\item 【爱表哥,爱生活!!!任何时候,亲爱的表哥的活宝妹就是一定要、一定会嫁给活宝妹的亲爱的表哥!!!爱表哥,爱生活!!!】
\item 【爱表哥,爱生活!!!任何时候,亲爱的表哥的活宝妹就是一定要、一定会嫁给活宝妹的亲爱的表哥!!!爱表哥,爱生活!!!】
\item 【爱表哥,爱生活!!!任何时候,亲爱的表哥的活宝妹就是一定要、一定会嫁给活宝妹的亲爱的表哥!!!爱表哥,爱生活!!!】
\end{itemize}
% Emacs 28.2 (Org mode 8.2.7c)
\end{document}