% Created 2023-05-16 Tue 15:39
% Intended LaTeX compiler: pdflatex
\documentclass[9pt, b5paper]{article}
\usepackage{xeCJK}
\usepackage[T1]{fontenc}
\usepackage{bera}
\usepackage[scaled]{beraserif}
\usepackage[scaled]{berasans}
\usepackage[scaled]{beramono}
\usepackage[cache=false]{minted}
\usepackage{xltxtra}
\usepackage{graphicx}
\usepackage{xcolor}
\usepackage{multirow}
\usepackage{multicol}
\usepackage{float}
\usepackage{textcomp}
\usepackage{algorithm}
\usepackage{algorithmic}
\usepackage{latexsym}
\usepackage{natbib}
\usepackage{geometry}
\geometry{left=1.2cm,right=1.2cm,top=1.5cm,bottom=1.2cm}
\usepackage[xetex,colorlinks=true,CJKbookmarks=true,linkcolor=blue,urlcolor=blue,menucolor=blue]{hyperref}
\newminted{common-lisp}{fontsize=\footnotesize} 
\author{deepwaterooo}
\date{\today}
\title{ET 框架学习笔记(二)--网络交互相关}
\hypersetup{
 pdfauthor={deepwaterooo},
 pdftitle={ET 框架学习笔记(二)--网络交互相关},
 pdfkeywords={},
 pdfsubject={},
 pdfcreator={Emacs 28.2 (Org mode 9.5.5)}, 
 pdflang={English}}
\begin{document}

\maketitle
\tableofcontents


\section{Protobuf 相关}
\label{sec:org991a29b}

\section{写在最后:反而是自己每天查看一再更新的}
\label{sec:orgd5bb4d9}
\begin{itemize}
\item 因为感觉还是不曾系统性地读ET7 的源码,或者说有效阅读,因为没有带着实际问题的看源码,感觉都不叫看读源码呀。这里会记自己的感觉需要赶快查看的地方。
\item 【ET 框架的整体架构】:感觉把握不够。常常命名空间分不清。要把这个大的框架,比较高层面的架构再好好看下。然后就是对自顶向下的不同层级场景,所需要的主要的不同组件,分不清,仍需要再熟悉一下源码
\item 【问题】:某些消息,还分不清是内网还是外网消息,暂时先放一下,到时再改
\item 【问题】:上次那个ET-EUI 框架的时候,曾经出现过 opcode 不对应,也就是说,我现在生成的进程间消息,有可能还是会存在服务器码与客户端码不对应,这个完备的框架,这次应该不至于吧?
\item 【ClientComponent】:新框架里重构丢了,去找怎么替代?那么现在去追一下,客户端的起始与场景加载或是切换大致过程。它变成了什么客户端场景管理?
\item 【UIType】部分类:这个类出现在了三四个不同的程序域,现在重构了,好像添加得不对。要再修改
\end{itemize}

\section{现在的修改内容,记忆}
\label{sec:orgd3f69d5}
\begin{itemize}
\item 【任何时候,活宝妹就是一定要嫁给亲爱的表哥!!!】
\item 【活宝妹坐等亲爱的表哥,领娶活宝妹回家!爱表哥,爱生活!!!】
\end{itemize}

\section{{\bfseries\sffamily TODO} }
\label{sec:orgfc73169}
\begin{itemize}
\item \textbf{Windows 下 org-mode 有几个【BUG:】} 1.org-mode 不能自动识别模式,除第一次加载可以正确,其它再加载不识别 org-mode; 2.org-export-to-pdf 在我换成为 msys64 里的 emacs 后就坏掉了。因为要花时间修,暂时还放着
\item \textbf{【IStartSystem:】} 感觉还有点儿小问题。认为:我应该不需要同文件两份,一份复制到客户端热更新域。我认为,全框架应该如其它接口类一样,只要一份就可以了。 \textbf{【晚点儿再检查一遍】}
\item \textbf{【Protobuf 里进程间传递的游戏数据相关信息:】} 这个现在成为重构的主要 compile-error. 因为找不到类。需要去弄懂
\begin{itemize}
\item 【Proto2CS】: 进程间消息里的,【牌相关的】,尤其是它们所属的命名空间,没写对,现在总是找不到定义。
\item 包括Identity, Weight,Suits,抢不抢地主【抢不抢庄】,以及可能的反不反主牌花色等。
\item 找不到的那些类,感觉更多的是命名空间没能开对。同一份源码一式三份,分别放在【客户端】【双端】【服务端】下只有【客户端】下可以自动识别,并且 Protobuf 里的 enum 生成的 .cs 与参考项目不同。不知道是否是 Protobuf 版本问题,还是我没注意到的细节。
\item \textbf{【Identity】与【Suits/Weight】三个【enum】} :外网消息里,怎么会找不到呢?再回去检查一遍。下午要把这个弄通,要开始思路怎么设计重构拖拉机项目。
\end{itemize}
\item Match 【匹配服】:不知道我哪根筋搭错,以为没有匹配服。可是它的配置。。。再一次从服务端看一遍起始源码,把匹配服的地址加载与获取找出来。。。
\item 去把【拖拉机房间、斗地主房间组件的,玩家什么的一堆组件】弄明白
\item 把参考游戏里,打牌相关的逻辑与模块好好看下,方便自己熟悉自己重构项目的源码后,画葫芦画飘地重构
\item 【任何时候,活宝妹就是一定要嫁给亲爱的表哥!!!爱表哥,爱生活!!!】
\end{itemize}

\section{拖拉机游戏:【重构OOP/OOD 设计思路】}
\label{sec:org70d8996}
\begin{itemize}
\item 自己是学过,有这方面的意识,但并不是说,自己就懂得,就知道该如何狠好地设计这些类。现在更多的是要受ET 框架,以及参考游戏手牌设计的启发,来帮助自己一再梳理思路,该如何设计它。
\item 【GamerComponent】玩家组件:是对一个房间里四个玩家的(及其在房间里的坐位位置)管理(分东南西北)。可以添加移除玩家。
\item 【爱表哥,爱生活!!!活宝妹就是一定要嫁给亲爱的表哥!爱表哥,爱生活!!!】
\end{itemize}

\subsection{NetServerComponent:}
\label{sec:orgf12de06}
\begin{minted}[fontsize=\scriptsize,linenos=false]{csharp}
public struct NetServerComponentOnRead {
    public Session Session;
    public object Message;
}
[ComponentOf(typeof(Scene))]
public class NetServerComponent: Entity, IAwake<IPEndPoint>, IDestroy {
    public int ServiceId;
}
\end{minted}
\end{document}