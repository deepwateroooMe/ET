% Created 2023-05-17 Wed 16:57
% Intended LaTeX compiler: pdflatex
\documentclass[9pt, b5paper]{article}
\usepackage{xeCJK}
\usepackage[T1]{fontenc}
\usepackage{bera}
\usepackage[scaled]{beraserif}
\usepackage[scaled]{berasans}
\usepackage[scaled]{beramono}
\usepackage[cache=false]{minted}
\usepackage{xltxtra}
\usepackage{graphicx}
\usepackage{xcolor}
\usepackage{multirow}
\usepackage{multicol}
\usepackage{float}
\usepackage{textcomp}
\usepackage{algorithm}
\usepackage{algorithmic}
\usepackage{latexsym}
\usepackage{natbib}
\usepackage{geometry}
\geometry{left=1.2cm,right=1.2cm,top=1.5cm,bottom=1.2cm}
\usepackage[xetex,colorlinks=true,CJKbookmarks=true,linkcolor=blue,urlcolor=blue,menucolor=blue]{hyperref}
\newminted{common-lisp}{fontsize=\footnotesize} 
\author{deepwaterooo}
\date{\today}
\title{ET 框架学习笔记(二)--网络交互相关}
\hypersetup{
 pdfauthor={deepwaterooo},
 pdftitle={ET 框架学习笔记(二)--网络交互相关},
 pdfkeywords={},
 pdfsubject={},
 pdfcreator={Emacs 28.2 (Org mode 9.5.5)}, 
 pdflang={English}}
\begin{document}

\maketitle
\tableofcontents


\section{ET7 数据库相关【服务端】}
\label{sec:org6ca4215}
\begin{itemize}
\item 这个数据库系统,连个添加使用的范例也没有。。。就两个组件,一个管理类。什么也没留下。。
\item 这里不急着整理。现框架 \textbf{DB 放在服务端的Model} 里。它的管理体系成为管理各个不同区服的数据库 DBComponent。
\item 因为找不到任何参考使用的例子。我觉得需要搜索一下。在理解了参考项目数据库模块之后,根据搜索,决定是使用原参考项目总服务器代理系,还是这种相对改装了的管理区服系统?
\end{itemize}
\subsection{IDBCollection: 主要是方便写两个不同的数据库(好像是GeekServer 里两个数据库)。反正方便扩展吧}
\label{sec:orgf0b560b}
\begin{minted}[fontsize=\scriptsize,linenos=false]{csharp}
public interface IDBCollection {}
\end{minted}
\subsection{DBComponent: 带生成系。可以查表,查询数据}
\label{sec:orga03cdc7}
\begin{minted}[fontsize=\scriptsize,linenos=false]{java}
[ChildOf(typeof(DBManagerComponent))] // 用来缓存数据
public class DBComponent: Entity, IAwake<string, string, int>, IDestroy {
    public const int TaskCount = 32;
    public MongoClient mongoClient;
    public IMongoDatabase database;
}
\end{minted}
\subsection{DBManagerComponent: 有上面的 DBComponent 数组。数组长度固定吗?}
\label{sec:orge4c9bf9}
\begin{minted}[fontsize=\scriptsize,linenos=false]{java}
public class DBManagerComponent: Entity, IAwake, IDestroy {
    [StaticField]
    public static DBManagerComponent Instance;
    public DBComponent[] DBComponents = new DBComponent[IdGenerater.MaxZone]; // 没事吃饱了撑得,占一大堆空地
}
\end{minted}
\subsection{DBManagerComponentSystem: 主是要查询某个区服的数据库,从数组里}
\label{sec:org8902fbe}
\begin{minted}[fontsize=\scriptsize,linenos=false]{java}
[FriendOf(typeof(DBManagerComponent))]
public static class DBManagerComponentSystem {
    [ObjectSystem]
    public class DBManagerComponentAwakeSystem: AwakeSystem<DBManagerComponent> {
        protected override void Awake(DBManagerComponent self) {
            DBManagerComponent.Instance = self;
        }
    }
    [ObjectSystem]
    public class DBManagerComponentDestroySystem: DestroySystem<DBManagerComponent> {
        protected override void Destroy(DBManagerComponent self) {
            DBManagerComponent.Instance = null;
        }
    }
    public static DBComponent GetZoneDB(this DBManagerComponent self, int zone) {
        DBComponent dbComponent = self.DBComponents[zone];
        if (dbComponent != null) 
            return dbComponent;
        StartZoneConfig startZoneConfig = StartZoneConfigCategory.Instance.Get(zone);
        if (startZoneConfig.DBConnection == "") 
            throw new Exception($"zone: {zone} not found mongo connect string");
        dbComponent = self.AddChild<DBComponent, string, string, int>(startZoneConfig.DBConnection, startZoneConfig.DBName, zone);
        self.DBComponents[zone] = dbComponent;
        return dbComponent;
    }
}
\end{minted}
\subsection{DBProxyComponent: 【参考项目】里的。有生成系。}
\label{sec:orgdf9f6f2}
\begin{minted}[fontsize=\scriptsize,linenos=false]{java}
// 用来与数据库操作代理
public class DBProxyComponent: Component {
    public IPEndPoint dbAddress;
}
\end{minted}

\section{StartConfigComponent: 找【各种服】的起始初始化地址}
\label{sec:org529da8f}
\begin{itemize}
\item 这些组群服务器的起始被全部重构了,重构成配置单例了
\item 这里是,昨天之前理这部分的时候,理得不够透彻。去重构一个服的初始化与配置,就需要理清、找到这个服所关联的所有的起始配置相关。昨天之前做得还不够好。
\item 昨天晚上稍微理解得再深一点儿。应该也是,总结自己理一个相对大型框架的方法总结吧。
\item 今天早上的脑袋还是相对比较清醒,就把这块儿对自己来说相对不够熟悉相对难点儿的,再看一遍,希望有更多的收获!
\end{itemize}
\subsection{ConfigSingleton<T>: ProtoObject, ISingleton}
\label{sec:org1cdd8fe}
\begin{minted}[fontsize=\scriptsize,linenos=false]{java}
public abstract class ConfigSingleton<T>: ProtoObject, ISingleton where T: ConfigSingleton<T>, new() {
        [StaticField]
        private static T instance;
        public static T Instance {
            get {
                return instance ??= ConfigComponent.Instance.LoadOneConfig(typeof (T)) as T;
            }
        }
        void ISingleton.Register() {
            if (instance != null) {
                throw new Exception($"singleton register twice! {typeof (T).Name}");
            }
            instance = (T)this;
        }
        void ISingleton.Destroy() {
            T t = instance;
            instance = null;
            t.Dispose();
        }
        bool ISingleton.IsDisposed() {
            throw new NotImplementedException();
        }
        public override void AfterEndInit() { }
        public virtual void Dispose() { }
    }
\end{minted}
\subsection{SceneFactory 里可以给【匹配服】添加组件}
\label{sec:orge8ea462}
\begin{minted}[fontsize=\scriptsize,linenos=false]{java}
public static class SceneFactory {
    public static async ETTask<Scene> CreateServerScene(Entity parent, long id, long instanceId, int zone, string name, SceneType sceneType, StartSceneConfig startSceneConfig = null) {
        await ETTask.CompletedTask;
        Scene scene = EntitySceneFactory.CreateScene(id, instanceId, zone, sceneType, name, parent);
        scene.AddComponent<MailBoxComponent, MailboxType>(MailboxType.UnOrderMessageDispatcher);
        switch (scene.SceneType) {
        case SceneType.Router:
            scene.AddComponent<RouterComponent, IPEndPoint, string>(startSceneConfig.OuterIPPort, startSceneConfig.StartProcessConfig.InnerIP);
            break;
        case SceneType.RouterManager: // 正式发布请用CDN代替RouterManager
            // 云服务器在防火墙那里做端口映射
            scene.AddComponent<HttpComponent, string>($"http:// *:{startSceneConfig.OuterPort}/");
            break;
        case SceneType.Realm:
            scene.AddComponent<NetServerComponent, IPEndPoint>(startSceneConfig.InnerIPOutPort);
            break;
        case SceneType.Match: // <<<<<<<<<<<<<<<<<<<< 这里是,我可以添加【匹配服】相关功能组件的地方。【参考项目原原码】感觉被我弄丢了
            break;
        case SceneType.Gate:
            scene.AddComponent<NetServerComponent, IPEndPoint>(startSceneConfig.InnerIPOutPort);
            scene.AddComponent<PlayerComponent>();
            scene.AddComponent<GateSessionKeyComponent>();
            break;
        case SceneType.Map:
            scene.AddComponent<UnitComponent>();
            scene.AddComponent<AOIManagerComponent>();
            break;
        case SceneType.Location:
            scene.AddComponent<LocationComponent>();
            break;
//...
        }
        return scene;
    }
}
\end{minted}
\subsection{RouterAddressComponent: 路由器组件}
\label{sec:org1b0523a}
\begin{minted}[fontsize=\scriptsize,linenos=false]{java}
[ComponentOf(typeof(Scene))]
public class RouterAddressComponent: Entity, IAwake<string, int> {
    public IPAddress RouterManagerIPAddress { get; set; }
    public string RouterManagerHost;
    public int RouterManagerPort;
    public HttpGetRouterResponse Info;
    public int RouterIndex;
}
\end{minted}
\subsection{RouterAddressComponentSystem: 路由器的生成系}
\label{sec:orgc2d657d}
\begin{minted}[fontsize=\scriptsize,linenos=false]{java}
[FriendOf(typeof(RouterAddressComponent))]
public static class RouterAddressComponentSystem {
    public class RouterAddressComponentAwakeSystem: AwakeSystem<RouterAddressComponent, string, int> {
        protected override void Awake(RouterAddressComponent self, string address, int port) {
            self.RouterManagerHost = address;
            self.RouterManagerPort = port;
        }
    }
    public static async ETTask Init(this RouterAddressComponent self) {
        self.RouterManagerIPAddress = NetworkHelper.GetHostAddress(self.RouterManagerHost);
        await self.GetAllRouter();
    }
    private static async ETTask GetAllRouter(this RouterAddressComponent self) {
        string url = $"http:// {self.RouterManagerHost}:{self.RouterManagerPort}/get_router?v={RandomGenerator.RandUInt32()}";
        Log.Debug($"start get router info: {url}");
        string routerInfo = await HttpClientHelper.Get(url);
        Log.Debug($"recv router info: {routerInfo}");
        HttpGetRouterResponse httpGetRouterResponse = JsonHelper.FromJson<HttpGetRouterResponse>(routerInfo);
        self.Info = httpGetRouterResponse;
        Log.Debug($"start get router info finish: {JsonHelper.ToJson(httpGetRouterResponse)}");
        // 打乱顺序
        RandomGenerator.BreakRank(self.Info.Routers);
        self.WaitTenMinGetAllRouter().Coroutine();
    }
    // 等10分钟再获取一次
    public static async ETTask WaitTenMinGetAllRouter(this RouterAddressComponent self) {
        await TimerComponent.Instance.WaitAsync(5 * 60 * 1000);
        if (self.IsDisposed) 
            return;
        await self.GetAllRouter();
    }
    public static IPEndPoint GetAddress(this RouterAddressComponent self) {
        if (self.Info.Routers.Count == 0) 
            return null;
        string address = self.Info.Routers[self.RouterIndex++ % self.Info.Routers.Count];
        string[] ss = address.Split(':');
        IPAddress ipAddress = IPAddress.Parse(ss[0]);
        if (self.RouterManagerIPAddress.AddressFamily == AddressFamily.InterNetworkV6) { 
            ipAddress = ipAddress.MapToIPv6();
        }
        return new IPEndPoint(ipAddress, int.Parse(ss[1]));
    }
    public static IPEndPoint GetRealmAddress(this RouterAddressComponent self, string account) { // <<<<<<<<<<<<<<<<<<<< 照葫芦画飘,扩展方法 
        int v = account.Mode(self.Info.Realms.Count);
        string address = self.Info.Realms[v];
        string[] ss = address.Split(':');
        IPAddress ipAddress = IPAddress.Parse(ss[0]);
        // if (self.IPAddress.AddressFamily == AddressFamily.InterNetworkV6) 
        //    ipAddress = ipAddress.MapToIPv6();
        return new IPEndPoint(ipAddress, int.Parse(ss[1]));
    }
}
\end{minted}

\subsection{RouterHelper: 路由器帮助类,向路由器注册、申请?}
\label{sec:org4dc0e8c}
\begin{minted}[fontsize=\scriptsize,linenos=false]{java}
public static class RouterHelper {
    // 注册router
    public static async ETTask<Session> CreateRouterSession(Scene clientScene, IPEndPoint address) {
        (uint recvLocalConn, IPEndPoint routerAddress) = await GetRouterAddress(clientScene, address, 0, 0);
        if (recvLocalConn == 0) 
            throw new Exception($"get router fail: {clientScene.Id} {address}");
        Log.Info($"get router: {recvLocalConn} {routerAddress}");
        Session routerSession = clientScene.GetComponent<NetClientComponent>().Create(routerAddress, address, recvLocalConn);
        routerSession.AddComponent<PingComponent>();
        routerSession.AddComponent<RouterCheckComponent>();
        return routerSession;
    }
    public static async ETTask<(uint, IPEndPoint)> GetRouterAddress(Scene clientScene, IPEndPoint address, uint localConn, uint remoteConn) {
        Log.Info($"start get router address: {clientScene.Id} {address} {localConn} {remoteConn}");
        // return (RandomHelper.RandUInt32(), address);
        RouterAddressComponent routerAddressComponent = clientScene.GetComponent<RouterAddressComponent>();
        IPEndPoint routerInfo = routerAddressComponent.GetAddress();
        uint recvLocalConn = await Connect(routerInfo, address, localConn, remoteConn);
        Log.Info($"finish get router address: {clientScene.Id} {address} {localConn} {remoteConn} {recvLocalConn} {routerInfo}");
        return (recvLocalConn, routerInfo);
    }
    // 向router申请
    private static async ETTask<uint> Connect(IPEndPoint routerAddress, IPEndPoint realAddress, uint localConn, uint remoteConn) {
        uint connectId = RandomGenerator.RandUInt32();
        using Socket socket = new Socket(routerAddress.AddressFamily, SocketType.Dgram, ProtocolType.Udp);
        int count = 20;
        byte[] sendCache = new byte[512];
        byte[] recvCache = new byte[512];
        uint synFlag = localConn == 0? KcpProtocalType.RouterSYN : KcpProtocalType.RouterReconnectSYN;
        sendCache.WriteTo(0, synFlag);
        sendCache.WriteTo(1, localConn);
        sendCache.WriteTo(5, remoteConn);
        sendCache.WriteTo(9, connectId);
        byte[] addressBytes = realAddress.ToString().ToByteArray();
        Array.Copy(addressBytes, 0, sendCache, 13, addressBytes.Length);
        Log.Info($"router connect: {connectId} {localConn} {remoteConn} {routerAddress} {realAddress}");

        EndPoint recvIPEndPoint = new IPEndPoint(IPAddress.Any, 0);
        long lastSendTimer = 0;
        while (true) {
            long timeNow = TimeHelper.ClientFrameTime();
            if (timeNow - lastSendTimer > 300) {
                if (--count < 0) {
                    Log.Error($"router connect timeout fail! {localConn} {remoteConn} {routerAddress} {realAddress}");
                    return 0;
                }
                lastSendTimer = timeNow;
                // 发送
                socket.SendTo(sendCache, 0, addressBytes.Length + 13, SocketFlags.None, routerAddress);
            }
            await TimerComponent.Instance.WaitFrameAsync();
            // 接收
            if (socket.Available > 0) {
                int messageLength = socket.ReceiveFrom(recvCache, ref recvIPEndPoint);
                if (messageLength != 9) {
                    Log.Error($"router connect error1: {connectId} {messageLength} {localConn} {remoteConn} {routerAddress} {realAddress}");
                    continue;
                }
                byte flag = recvCache[0];
                if (flag != KcpProtocalType.RouterReconnectACK && flag != KcpProtocalType.RouterACK) {
                    Log.Error($"router connect error2: {connectId} {synFlag} {flag} {localConn} {remoteConn} {routerAddress} {realAddress}");
                    continue;
                }
                uint recvRemoteConn = BitConverter.ToUInt32(recvCache, 1);
                uint recvLocalConn = BitConverter.ToUInt32(recvCache, 5);
                Log.Info($"router connect finish: {connectId} {recvRemoteConn} {recvLocalConn} {localConn} {remoteConn} {routerAddress} {realAddress}");
                return recvLocalConn;
            }
        }
    }
}
\end{minted}

\subsection{StartProcessConfigCategory : ConfigSingleton<StartProcessConfigCategory>, IMerge: 【任何时候,活宝妹就是一定要嫁给亲爱的表哥!!!】}
\label{sec:org8af9e88}
\begin{minted}[fontsize=\scriptsize,linenos=false]{java}
[ProtoContract]
[Config]
public partial class StartProcessConfigCategory : ConfigSingleton<StartProcessConfigCategory>, IMerge {
    [ProtoIgnore]
    [BsonIgnore]
    private Dictionary<int, StartProcessConfig> dict = new Dictionary<int, StartProcessConfig>(); // 管理字典
    [BsonElement]
    [ProtoMember(1)]
    private List<StartProcessConfig> list = new List<StartProcessConfig>();
    public void Merge(object o) {
        StartProcessConfigCategory s = o as StartProcessConfigCategory;
        this.list.AddRange(s.list);
    }
    [ProtoAfterDeserialization]        
    public void ProtoEndInit() {
        foreach (StartProcessConfig config in list) {
            config.AfterEndInit();
            this.dict.Add(config.Id, config);
        }
        this.list.Clear();
        this.AfterEndInit();
    }
    public StartProcessConfig Get(int id) {
        this.dict.TryGetValue(id, out StartProcessConfig item);
        if (item == null) {
            throw new Exception($"配置找不到,配置表名: {nameof (StartProcessConfig)},配置id: {id}");
        }
        return item;
    }
    public bool Contain(int id) {
        return this.dict.ContainsKey(id);
    }
    public Dictionary<int, StartProcessConfig> GetAll() {
        return this.dict;
    }
    public StartProcessConfig GetOne() {
        if (this.dict == null || this.dict.Count <= 0) {
            return null;
        }
        return this.dict.Values.GetEnumerator().Current;
    }
}
[ProtoContract]
public partial class StartProcessConfig: ProtoObject, IConfig {
    [ProtoMember(1)]
    public int Id { get; set; }
    [ProtoMember(2)]
    public int MachineId { get; set; }
    [ProtoMember(3)]
    public int InnerPort { get; set; }
}
\end{minted}
\subsection{StartSceneConfig: ISupportInitialize 【各种服-配置,场景配置】}
\label{sec:org3be4504}
\begin{minted}[fontsize=\scriptsize,linenos=false]{csharp}
public partial class StartSceneConfig: ISupportInitialize {
    public long InstanceId;
    public SceneType Type; // 场景类型

    public StartProcessConfig StartProcessConfig {
        get {
            return StartProcessConfigCategory.Instance.Get(this.Process);
        }
    }
    public StartZoneConfig StartZoneConfig {
        get {
            return StartZoneConfigCategory.Instance.Get(this.Zone);
        }
    }
    // 内网地址外网端口,通过防火墙映射端口过来
    private IPEndPoint innerIPOutPort;
    public IPEndPoint InnerIPOutPort {
        get {
            if (innerIPOutPort == null) {
                this.innerIPOutPort = NetworkHelper.ToIPEndPoint($"{this.StartProcessConfig.InnerIP}:{this.OuterPort}");
            }
            return this.innerIPOutPort;
        }
    }
    // 外网地址外网端口
    private IPEndPoint outerIPPort;
    public IPEndPoint OuterIPPort {
        get {
            if (this.outerIPPort == null) {
                this.outerIPPort = NetworkHelper.ToIPEndPoint($"{this.StartProcessConfig.OuterIP}:{this.OuterPort}");
            }
            return this.outerIPPort;
        }
    }
    public override void AfterEndInit() {
        this.Type = EnumHelper.FromString<SceneType>(this.SceneType);
        InstanceIdStruct instanceIdStruct = new InstanceIdStruct(this.Process, (uint) this.Id);
        this.InstanceId = instanceIdStruct.ToLong();
    }
}
\end{minted}
\subsection{StartSceneConfigCategory : 【Matchs!】ConfigSingleton<StartSceneConfigCategory>, IMerge}
\label{sec:orgdedf08b}
\begin{itemize}
\item 读里面的登录服,会知道它是如何管理登录服的(就是后面的例子,当它要拿登录服的地址的时候),它们是区服,就是分各个小区管理。如果集群是这个样子,大概匹配服也就是一样分小区管理了。
\item 那么这个配置管理里,因为我要用匹配服与地图服,也要对至少是匹配服进行管理。那么,我在申请匹配的时候,网关服才能拿到匹配服的地址。
\item 只在【服务端】存在。但是在双端模式、与服务端模式下,每种端有两个文件来定义这个类。。一个在【ProtoContract】里,可能可以进程间消息传递?一个在 ConfigPartial 文件夹里
\item 上面的文件重复,还不是很懂。【重构】:因为我现在还比较喜欢使用Unity 下自带的双端模式,可是暂时只改【双端模式 ClientServer】下的文件,另一个专职服务端可能晚点儿再补上去。不用昨天晚上一样每个文件都改。
\end{itemize}
\begin{minted}[fontsize=\scriptsize,linenos=false]{csharp}
// 配置文件处理,或是服务器启动相关类,以前都没仔细读过
public partial class StartSceneConfigCategory {
    public MultiMap<int, StartSceneConfig> Gates = new MultiMap<int, StartSceneConfig>();
    public MultiMap<int, StartSceneConfig> ProcessScenes = new MultiMap<int, StartSceneConfig>();
    public Dictionary<long, Dictionary<string, StartSceneConfig>> ClientScenesByName = new Dictionary<long, Dictionary<string, StartSceneConfig>>();
    public StartSceneConfig LocationConfig;
    public List<StartSceneConfig> Realms = new List<StartSceneConfig>();
    public List<StartSceneConfig> Matchs = new List<StartSceneConfig>(); // <<<<<<<<<<<<<<<<<<<< 添加管理
    public List<StartSceneConfig> Routers = new List<StartSceneConfig>();
    public List<StartSceneConfig> Robots = new List<StartSceneConfig>();
    public StartSceneConfig BenchmarkServer;

    public List<StartSceneConfig> GetByProcess(int process) {
        return this.ProcessScenes[process];
    }
    public StartSceneConfig GetBySceneName(int zone, string name) {
        return this.ClientScenesByName[zone][name];
    }
    public override void AfterEndInit() {
        foreach (StartSceneConfig startSceneConfig in this.GetAll().Values) {
            this.ProcessScenes.Add(startSceneConfig.Process, startSceneConfig);
                
            if (!this.ClientScenesByName.ContainsKey(startSceneConfig.Zone)) {
                this.ClientScenesByName.Add(startSceneConfig.Zone, new Dictionary<string, StartSceneConfig>());
            }
            this.ClientScenesByName[startSceneConfig.Zone].Add(startSceneConfig.Name, startSceneConfig);
                
            switch (startSceneConfig.Type) {
            case SceneType.Realm:
                this.Realms.Add(startSceneConfig);
                break;
            case SceneType.Gate:
                this.Gates.Add(startSceneConfig.Zone, startSceneConfig);
                break;
            case SceneType.Match:                  // <<<<<<<<<<<<<<<<<<<< 自己加的
                this.Matchs.Add(startSceneConfig); // <<<<<<<<<<<<<<<<<<<< 
                break;
            case SceneType.Location:
                this.LocationConfig = startSceneConfig;
                break;
            case SceneType.Robot:
                this.Robots.Add(startSceneConfig);
                break;
            case SceneType.Router:
                this.Routers.Add(startSceneConfig);
                break;
            case SceneType.BenchmarkServer:
                this.BenchmarkServer = startSceneConfig;
                break;
            }
        }
    }
}
\end{minted}
\subsection{HttpGetRouterResponse: 这个 ProtoBuf 的消息类型}
\label{sec:orga96f7cd}
\begin{itemize}
\item 框架里,有个专用的路由器管理器场景(服),对路由器,或说各种服的地址进行管理
\item 主要是方便,一个路由器管理组件,来自顶向下地获取,各小区所有路由器地址的?想来当组件要拿地址时,每个小区分服都把自己的地址以消息的形式传回去的?
\end{itemize}
\begin{minted}[fontsize=\scriptsize,linenos=false]{java}
[Message(OuterMessage.HttpGetRouterResponse)]
[ProtoContract]
public partial class HttpGetRouterResponse: ProtoObject {
    [ProtoMember(1)]
    public List<string> Realms { get; set; }
    [ProtoMember(2)]
    public List<string> Routers { get; set; }
}
message HttpGetRouterResponse { // 这里,是 Outer proto 里的消息定义
	repeated string Realms = 1;
	repeated string Routers = 2;
	repeated string Matchs = 3;// 这行是我需要添加,和生成消息的
}
\end{minted}
\subsection{HttpGetRouterHandler : IHttpHandler: 获取各路由器的地址}
\label{sec:org1067867}
\begin{itemize}
\item 【匹配服】:因为我想拿这个服的地址,也需要这个帮助类里作相应的修改
\item StartSceneConfigCategory.Instance: 不明白这个实例是存放在哪里,因为可以 proto 消息进程间传递,那么可以试找,哪里调用这个帮助类拿东西?
\item 这个模块:现在还是理解不透。需要某个上午,把所有 RouterComponent 组件及其相关,再理一遍。
\begin{minted}[fontsize=\scriptsize,linenos=false]{csharp}
[HttpHandler(SceneType.RouterManager, "/get_router")]
public class HttpGetRouterHandler : IHttpHandler {
    public async ETTask Handle(Entity domain, HttpListenerContext context) {
        HttpGetRouterResponse response = new HttpGetRouterResponse();
        response.Realms = new List<string>();
        response.Matchs = new List<string>();// 匹配服链表  // <<<<<<<<<<<<<<<<<<<< 
        response.Routers = new List<string>();
        // 是去StartSceneConfigCategory 这里拿的:因为它可以 proto 消息里、进程间传递,这里还不是狠懂,这个东西存放在哪里?
        foreach (StartSceneConfig startSceneConfig in StartSceneConfigCategory.Instance.Realms) {
            response.Realms.Add(startSceneConfig.InnerIPOutPort.ToString());
        }
        foreach (StartSceneConfig startSceneConfig in StartSceneConfigCategory.Instance.Matchs) {
            response.Matchs.Add(startSceneConfig.InnerIPOutPort.ToString());
        }
        foreach (StartSceneConfig startSceneConfig in StartSceneConfigCategory.Instance.Routers) {
            response.Routers.Add($"{startSceneConfig.StartProcessConfig.OuterIP}:{startSceneConfig.OuterPort}");
        }
        HttpHelper.Response(context, response);
        await ETTask.CompletedTask;
    }
}
\end{minted}
\end{itemize}
\subsection{HttpHandler 标签系:标签自带场景类型}
\label{sec:org7090ba6}
\begin{minted}[fontsize=\scriptsize,linenos=false]{csharp}
public class HttpHandlerAttribute: BaseAttribute {
    public SceneType SceneType { get; }
    public string Path { get; }
    public HttpHandlerAttribute(SceneType sceneType, string path) {
        this.SceneType = sceneType;
        this.Path = path;
    }
}
\end{minted}
\subsection{LoginHelper: 登录服的获取地址的方式来获取匹配服的地址了。全框架只有这一个黄金案例}
\label{sec:orgdb3c067}
\begin{itemize}
\item 这个是用户登录前,还没能与网关服建立起任何关系,可能会不得不绕得复杂一点儿
\item 这个地方,有点儿奇怪的是:RouterAddressComponent 这个组件,找不到添加的地方。它可能有点儿特殊,就是它不走常规 AddComponent<>() 的方法添加,所以找不到添加,而只有获取。
\end{itemize}
\begin{minted}[fontsize=\scriptsize,linenos=false]{java}
public static class LoginHelper {
    public static async ETTask Login(Scene clientScene, string account, string password) {
        try {
            // 创建一个ETModel层的Session
            clientScene.RemoveComponent<RouterAddressComponent>();
            // 获取路由跟realmDispatcher地址
            RouterAddressComponent routerAddressComponent = clientScene.GetComponent<RouterAddressComponent>();
            if (routerAddressComponent == null) {
                routerAddressComponent = clientScene.AddComponent<RouterAddressComponent, string, int>(ConstValue.RouterHttpHost, ConstValue.RouterHttpPort);
                await routerAddressComponent.Init();
                clientScene.AddComponent<NetClientComponent, AddressFamily>(routerAddressComponent.RouterManagerIPAddress.AddressFamily);
            }
            IPEndPoint realmAddress = routerAddressComponent.GetRealmAddress(account); // <<<<<<<<<<<<<<<<<<<< 这里就是说,我必须去组件里扩展方法
            R2C_Login r2CLogin;
            using (Session session = await RouterHelper.CreateRouterSession(clientScene, realmAddress)) {
                r2CLogin = (R2C_Login) await session.Call(new C2R_Login() { Account = account, Password = password });
            }
            // 创建一个gate Session,并且保存到SessionComponent中: 与网关服的会话框。主要负责用户下线后会话框的自动移除销毁
            Session gateSession = await RouterHelper.CreateRouterSession(clientScene, NetworkHelper.ToIPEndPoint(r2CLogin.Address));
            clientScene.AddComponent<SessionComponent>().Session = gateSession;
            G2C_LoginGate g2CLoginGate = (G2C_LoginGate)await gateSession.Call(
                new C2G_LoginGate() { Key = r2CLogin.Key, GateId = r2CLogin.GateId});
            Log.Debug("登陆gate成功!");
            await EventSystem.Instance.PublishAsync(clientScene, new EventType.LoginFinish());
        }
        catch (Exception e) {
            Log.Error(e);
        }
    } 
}
\end{minted}
**

\section{服务器的功能概述:各服务器的作用(这个不是ET7 版本的,以前的)}
\label{sec:org51a5ec1}
\begin{itemize}
\item Manager:连接客户端的外网和连接内部服务器的内网,对服务器进程进行管理,自动检测和启动服务器进程。加载有内网组件NetInnerComponent,外网组件NetOuterComponent,服务器进程管理组件。自动启动突然停止运行的服务器,保证此服务器管理的其它服务器崩溃后能及时自动启动运行。
\item Realm:对Actor消息进行管理(添加、移除、分发等),连接内网和外网,对内网服务器进程进行操作,随机分配Gate服务器地址。内网组件NetInnerComponent,外网组件NetOuterComponent,Gate服务器随机分发组件。客户端登录时连接的第一个服务器,也可称为登录服务器。
\item Gate:对玩家进行管理,对Actor消息进行管理(添加、移除、分发等),连接内网和外网,对内网服务器进程进行操作,随机分配Gate服务器地址,对Actor消息进程进行管理,对玩家ID登录后的Key进行管理。加载有玩家管理组件PlayerComponent,管理登陆时联网的Key组件GateSessionKeyComponent。
\item Location:连接内网,服务器进程状态集中管理(Actor消息IP管理服务器)。加载有内网组件NetInnerComponent,服务器消息处理状态存储组件LocationComponent。对客户端的登录信息进行验证和客户端登录后连接的服务器,登录后通过此服务器进行消息互动,也可称为验证服务器。
\item Map:连接内网,对ActorMessage消息进行管理(添加、移除、分发等),对场景内现在活动物体存储管理,对内网服务器进程进行操作,对Actor消息进程进行管理,对Actor消息进行管理(添加、移除、分发等),服务器帧率管理。服务器帧率管理组件ServerFrameComponent。
\item AllServer:将以上服务器功能集中合并成一个服务器。另外增加DB连接组件DBComponent
\item Benchmark:连接内网和测试服务器承受力。加载有内网组件NetInnerComponent,服务器承受力测试组件BenchmarkComponent。
\end{itemize}

\section{Protobuf 相关,【Protobuf 里进程间传递的游戏数据相关信息:两个思路】}
\label{sec:org99e151e}
\begin{itemize}
\item 【一、】查找 enum 可能可以用系统平台下的 protoc 来代为生成,效果差不多。只起现 Proto2CS.cs 编译的补充作用。
\item 【二、】Card 类下的两个 enum 变量,在ILRuntime 热更新库下,还是需要帮它连一下的。用的是 HybridCLR
\item 【三、】查找 protoc 命令下,如何C\# 索引 Unity 第三方库。
\item 【Windows 下的 Protobuf 编译环境】:配置好,只是作为与ET 框架的Proto2CS.cs 所指挥的编译结果,作一个对比,两者应该效果是一样的,或是基本一样的,除了自定义里没有处理 enum.
\item Windows 下的命令行,就是用 protoc 来编译,可以参考如下. (这是 .cs 源码下的)
\begin{minted}[fontsize=\scriptsize,linenos=false]{csharp}
CommandRun($"protoc.exe", $"--csharp_out=\"./{outputPath}\" --proto_path=\"{protoPath}\" {protoName}");
\end{minted}
\item 现在的问题是, \textbf{Protobuf消息里面居然是有 unity 第三方库的索引} 。
\item 直接把 enum 生成的那三个 .cs 类分别复制进双端,服务器端与客户端。包括Card 类。那些编译错误会去天边。哈哈哈,除了一个Card 的两个变量之外(CardSuits, CardWeight)。
\item 【热更新库】:现在剩下的问题,就成为,判定是用了哪个热更新的库,ILRuntime, 还是 HybridCLR, 如果帮它连那两个变量。好像接的是 HybridCLR. 这个库是我之前还不曾真正用过的。
\begin{itemize}
\item 相比于ET6,彻底剔除了ILRuntime,使得代码简洁了不少,并且比较稳定
\end{itemize}
\end{itemize}

\section{写在最后:反而是自己每天查看一再更新的}
\label{sec:orgf0cbd66}
\begin{itemize}
\item 因为感觉还是不曾系统性地读ET7 的源码,或者说有效阅读,因为没有带着实际问题的看源码,感觉都不叫看读源码呀。这里会记自己的感觉需要赶快查看的地方。
\item 【ET 框架的整体架构】:感觉把握不够。常常命名空间分不清。要把这个大的框架,比较高层面的架构再好好看下。然后就是对自顶向下的不同层级场景,所需要的主要的不同组件,分不清,仍需要再熟悉一下源码
\item 【问题】:某些消息,还分不清是内网还是外网消息,暂时先放一下,到时再改
\item 【问题】:上次那个ET-EUI 框架的时候,曾经出现过 opcode 不对应,也就是说,我现在生成的进程间消息,有可能还是会存在服务器码与客户端码不对应,这个完备的框架,这次应该不至于吧?
\item 【ClientComponent】:新框架里重构丢了,去找怎么替代?那么现在去追一下,客户端的起始与场景加载或是切换大致过程。它变成了什么客户端场景管理?
\item 【UIType】部分类:这个类出现在了三四个不同的程序域,现在重构了,好像添加得不对。要再修改
\end{itemize}
\section{现在的修改内容:}
\label{sec:org24784ef}
\begin{itemize}
\item \textbf{【Windows 下 proto2cs 消息转化】} :上面拿匹配服地址,一个路由器相关消息加了参数,要重新生成一下。这个 mac 系统下不难,改天有时间也可以把 mac 下的运行环境配置好,这个功能模块就不用一定要切换电脑了。【手动做了一下】
\item \textbf{【UILobbyComponent 可以测试】} :这个大厅组件,Unity 里预设简单,可以试运行一下,看是否完全消除这个UI 组件的报错,这个屏的控件能否显示出来?还是错出得早,这个屏就出不来已经报错了?
\begin{itemize}
\item 【客户端】的逻辑是处理好了,编译全过后可以测试
\item 【服务端】:处理用户请求匹配房间的逻辑,仍在处理:C2G\_StartMatch\_ReqHandler. 必须记笔记。不记会忘。
\end{itemize}

\item \textbf{【TractorRoomComponent】} :因为是多组件嵌套,可以合并多组件为同一个组件;另早上看得一知半解的一个【ChildOf】标签,可以帮助组件套用吗?再找找理解消化一下
\item 【房间组件】:几个现存的 working-on 的问题:
\begin{itemize}
\item 多组件嵌套:手工合并为一个组件。彻底理解确认后,会合并
\item 【服务端】:处理用户请求匹配房间的逻辑,仍在处理
\begin{itemize}
\item 【数据库模块的整合】:网关服在转发请求匹配时,验证会话框有效后,验证用户身份时,需要去【用户数据库】拿用户数据。ET7 留了个DBManagerComponent, 还没能整合出这个模块
\item 【匹配服地址】网关服的处理逻辑里,验证完用户合格后,为代为转发消息到匹配服,但需要拿匹配服的地址。ET7 重构里,还没能改出这部分。服务器系统配置初始化时,可以链表管理各小构匹配服,再去拿相关匹配服的地址。ET7 框架里的路由器系统,自己还没有弄懂。
\end{itemize}
\end{itemize}
\item 【活宝妹坐等亲爱的表哥,领娶活宝妹回家!爱表哥,爱生活!!!】
\end{itemize}
\section{{\bfseries\sffamily TODO} 其它的:部分完成,或是待完成的大的功能版块,列举}
\label{sec:orgb369890}
\begin{itemize}
\item emacs 那天我弄了好久,把C-; ISpell 原定绑定的功能解除,重新绑定为自己喜欢的 expand-region. 今天第二次再弄,看一下几分钟能够解决完问题?我的这个破烂记性呀。。。【爱表哥,爱生活!!!任何时候,活宝妹就是一定要嫁给亲爱的表哥!!!】mingw64 lisp/textmode/flyspell.el 键的重新绑定。这下记住了。还好,花得不是太久。有以前的笔记 
\begin{itemize}
\item Windows 10 平台下,C-; 是绑定到了 ISpell 下的某个功能,可是现在这个破 emacs 老报错,连查是绑定给哪个功能,过程报错都被阻止了。。。
\end{itemize}
\item \textbf{【IStartSystem:】} 感觉还有点儿小问题。认为:我应该不需要同文件两份,一份复制到客户端热更新域。我认为,全框架应该如其它接口类一样,只要一份就可以了。 \textbf{【晚点儿再检查一遍】}
\item 如果这个一时半会儿解决不好,就把重构的设计思路再理一理。同时尽量去改重构的ET 框架里的编译错误。
\item 【Tractor】原 windows-form 项目,源码需要读懂,理解透彻,方便重构。
\item 去把【拖拉机房间、斗地主房间组件的,玩家什么的一堆组件】弄明白
\item 【任何时候,活宝妹就是一定要嫁给亲爱的表哥!!!爱表哥,爱生活!!!】
\end{itemize}
\section{拖拉机游戏:【重构OOP/OOD 设计思路】}
\label{sec:orge09c633}
\begin{itemize}
\item 自己是学过,有这方面的意识,但并不是说,自己就懂得,就知道该如何狠好地设计这些类。现在更多的是要受ET 框架,以及参考游戏手牌设计的启发,来帮助自己一再梳理思路,该如何设计它。
\item ET7 重构里,各组件都该是自己设计重构原项目的类的设计的必要起点。可以根据这些来系统设计重构。【活宝妹就是一定要嫁给亲爱的表哥!!!】
\item 【GamerComponent】玩家组件:是对一个房间里四个玩家的(及其在房间里的坐位位置)管理(分东南西北)。可以添加移除玩家。
\item 【Gamer】:每一个玩家
\item 【拖拉机游戏房间】:
\item 【爱表哥,爱生活!!!活宝妹就是一定要嫁给亲爱的表哥!爱表哥,爱生活!!!】
\end{itemize}
\end{document}