% Created 2023-06-27 Tue 12:34
\documentclass[9pt, b5paper]{article}
\usepackage{xeCJK}
\usepackage[T1]{fontenc}
\usepackage{bera}
\usepackage[scaled]{beraserif}
\usepackage[scaled]{berasans}
\usepackage[scaled]{beramono}
\usepackage[cache=false]{minted}
\usepackage{xltxtra}
\usepackage{graphicx}
\usepackage{xcolor}
\usepackage{multirow}
\usepackage{multicol}
\usepackage{float}
\usepackage{textcomp}
\usepackage{algorithm}
\usepackage{algorithmic}
\usepackage{latexsym}
\usepackage{natbib}
\usepackage{geometry}
\geometry{left=1.2cm,right=1.2cm,top=1.5cm,bottom=1.2cm}
\usepackage[xetex,colorlinks=true,CJKbookmarks=true,linkcolor=blue,urlcolor=blue,menucolor=blue]{hyperref}
\newminted{common-lisp}{fontsize=\footnotesize} 
\author{deepwaterooo}
\date{\today}
\title{ET 框架学习笔记(三)--网络交互相关}
\hypersetup{
  pdfkeywords={},
  pdfsubject={},
  pdfcreator={Emacs 28.2 (Org mode 8.2.7c)}}
\begin{document}

\maketitle
\tableofcontents


\section{ET7 数据库相关【服务端】}
\label{sec-1}
\begin{itemize}
\item 这个数据库系统,连个添加使用的范例也没有。。。就两个组件,一个管理类。什么也没留下。。
\item 这里不急着整理。现框架 \textbf{DB 放在服务端的Model} 里。它的管理体系成为管理各个不同区服的数据库 DBComponent。
\item 因为找不到任何参考使用的例子。我觉得需要搜索一下。在理解了参考项目数据库模块之后,根据搜索,决定是使用原参考项目总服务器代理系,还是这种相对改装了的管理区服系统?
\end{itemize}
\subsection{IDBCollection: 主要是方便写两个不同的数据库(好像是GeekServer 里两个数据库)。反正方便扩展吧}
\label{sec-1-1}
\begin{minted}[fontsize=\scriptsize,linenos=false]{csharp}
public interface IDBCollection {}
\end{minted}
\subsection{DBComponent: 带生成系。可以查表,查询数据}
\label{sec-1-2}
\begin{minted}[fontsize=\scriptsize,linenos=false]{csharp}
[ChildOf(typeof(DBManagerComponent))] // 用来缓存数据
public class DBComponent: Entity, IAwake<string, string, int>, IDestroy {
    public const int TaskCount = 32;
    public MongoClient mongoClient;
    public IMongoDatabase database;
}
\end{minted}
\subsection{DBManagerComponent: 有上面的 DBComponent 数组。数组长度固定吗?}
\label{sec-1-3}
\begin{minted}[fontsize=\scriptsize,linenos=false]{csharp}
public class DBManagerComponent: Entity, IAwake, IDestroy {
    [StaticField]
    public static DBManagerComponent Instance;
    public DBComponent[] DBComponents = new DBComponent[IdGenerater.MaxZone]; // 没事吃饱了撑得,占一大堆空地
}
\end{minted}
\subsection{DBManagerComponentSystem: 主是要查询某个区服的数据库,从数组里}
\label{sec-1-4}
\begin{minted}[fontsize=\scriptsize,linenos=false]{csharp}
[FriendOf(typeof(DBManagerComponent))]
public static class DBManagerComponentSystem {
    [ObjectSystem]
    public class DBManagerComponentAwakeSystem: AwakeSystem<DBManagerComponent> {
        protected override void Awake(DBManagerComponent self) {
            DBManagerComponent.Instance = self;
        }
    }
    [ObjectSystem]
    public class DBManagerComponentDestroySystem: DestroySystem<DBManagerComponent> {
        protected override void Destroy(DBManagerComponent self) {
            DBManagerComponent.Instance = null;
        }
    }
    public static DBComponent GetZoneDB(this DBManagerComponent self, int zone) {
        DBComponent dbComponent = self.DBComponents[zone];
        if (dbComponent != null)// 如果已经管理配置好,直接返回  
            return dbComponent;
        StartZoneConfig startZoneConfig = StartZoneConfigCategory.Instance.Get(zone);
        if (startZoneConfig.DBConnection == "")// 小区域里如果没有匹配或是出错,抛异常 
            throw new Exception($"zone: {zone} not found mongo connect string");
// 把这个小区域里的数据库配置好,加入系统管理,并返回 
        dbComponent = self.AddChild<DBComponent, string, string, int>(startZoneConfig.DBConnection, startZoneConfig.DBName, zone);
        self.DBComponents[zone] = dbComponent;
        return dbComponent;
    }
}
\end{minted}
\subsection{DBProxyComponent: 【参考项目】里的。有生成系。}
\label{sec-1-5}
\begin{minted}[fontsize=\scriptsize,linenos=false]{csharp}
// 用来与数据库操作代理
public class DBProxyComponent: Component {
    public IPEndPoint dbAddress;
}
\end{minted}

\subsection{StartZoneConfigCategory: 单例区服配置管理类}
\label{sec-1-6}
\begin{itemize}
\item 主要还是要把整个框架系统性的都弄懂了
\begin{minted}[fontsize=\scriptsize,linenos=false]{csharp}
[ProtoContract]
[Config]
public partial class StartZoneConfigCategory : ConfigSingleton<StartZoneConfigCategory>, IMerge {
    [ProtoIgnore]
    [BsonIgnore]
    private Dictionary<int, StartZoneConfig> dict = new Dictionary<int, StartZoneConfig>();
    [BsonElement]
    [ProtoMember(1)]
    private List<StartZoneConfig> list = new List<StartZoneConfig>();
    public void Merge(object o) {
        StartZoneConfigCategory s = o as StartZoneConfigCategory;
        this.list.AddRange(s.list);
    }
    [ProtoAfterDeserialization]        
    public void ProtoEndInit() {
        foreach (StartZoneConfig config in list) {
            config.AfterEndInit();
            this.dict.Add(config.Id, config);
        }
        this.list.Clear();
        this.AfterEndInit();
    }
    public StartZoneConfig Get(int id) {
        this.dict.TryGetValue(id, out StartZoneConfig item);
        if (item == null) 
            throw new Exception($"配置找不到,配置表名: {nameof (StartZoneConfig)},配置id: {id}");
        return item;
    }
    public bool Contain(int id) {
        return this.dict.ContainsKey(id);
    }
    public Dictionary<int, StartZoneConfig> GetAll() {
        return this.dict;
    }
    public StartZoneConfig GetOne() {
        if (this.dict == null || this.dict.Count <= 0) 
            return null;
        return this.dict.Values.GetEnumerator().Current;
    }
}
[ProtoContract]
public partial class StartZoneConfig: ProtoObject, IConfig {// 小区配置 
    [ProtoMember(1)]
    public int Id { get; set; }
    // 数据库地址
    [ProtoMember(2)]
    public string DBConnection { get; set; }
    // 数据库名
    [ProtoMember(3)]
    public string DBName { get; set; }
}
\end{minted}
\end{itemize}


\section{网关服:客户端信息发送的直接代理,中转站,组件分析}
\label{sec-2}
\begin{itemize}
\item SceneFactory: 【初始化】时,带如下几个组件
\end{itemize}
\begin{minted}[fontsize=\scriptsize,linenos=false]{csharp}
public static class SceneFactory {
    public static async ETTask<Scene> CreateServerScene(Entity parent, long id, long instanceId, int zone, string name, SceneType sceneType, StartSceneConfig startSceneConfig = null) {
        await ETTask.CompletedTask;
        Scene scene = EntitySceneFactory.CreateScene(id, instanceId, zone, sceneType, name, parent);
        // 任何场景:无序消息分发器,可接收消息,队列处理;发呢?
        scene.AddComponent<MailBoxComponent, MailboxType>(MailboxType.UnOrderMessageDispatcher); // 重构?应该是对进程间消息发收的浓缩与提练

        switch (scene.SceneType) {
            case SceneType.Router:
                scene.AddComponent<RouterComponent, IPEndPoint, string>(startSceneConfig.OuterIPPort, startSceneConfig.StartProcessConfig.InnerIP);
                break;
            case SceneType.RouterManager: // 正式发布请用CDN代替RouterManager
                // 云服务器在防火墙那里做端口映射
                scene.AddComponent<HttpComponent, string>($"http:// *:{startSceneConfig.OuterPort}/");
                break;
            // // case SceneType.Realm: // 注册登录服:
            // //     scene.AddComponent<NetServerComponent, IPEndPoint>(startSceneConfig.InnerIPOutPort);
            // //     break;
            case SceneType.Gate:
                scene.AddComponent<NetServerComponent, IPEndPoint>(startSceneConfig.InnerIPOutPort);
                scene.AddComponent<PlayerComponent>();
                scene.AddComponent<GateSessionKeyComponent>();
                break; // ...
\end{minted}
\subsection{NetServerComponent:}
\label{sec-2-1}
\begin{minted}[fontsize=\scriptsize,linenos=false]{csharp}
public struct NetServerComponentOnRead {
    public Session Session;
    public object Message;
}
[ComponentOf(typeof(Scene))]
public class NetServerComponent: Entity, IAwake<IPEndPoint>, IDestroy {
    public int ServiceId;
}
\end{minted}


\section{服务器的功能概述:各服务器的作用(这个不是ET7 版本的,以前的)}
\label{sec-3}
\begin{itemize}
\item Manager:连接客户端的外网和连接内部服务器的内网,对服务器进程进行管理,自动检测和启动服务器进程。加载有内网组件NetInnerComponent,外网组件NetOuterComponent,服务器进程管理组件。自动启动突然停止运行的服务器,保证此服务器管理的其它服务器崩溃后能及时自动启动运行。
\item Realm:对Actor消息进行管理(添加、移除、分发等),连接内网和外网,对内网服务器进程进行操作,随机分配Gate服务器地址。内网组件NetInnerComponent,外网组件NetOuterComponent,Gate服务器随机分发组件。客户端登录时连接的第一个服务器,也可称为登录服务器。
\item Gate:对玩家进行管理,对Actor消息进行管理(添加、移除、分发等),连接内网和外网,对内网服务器进程进行操作,随机分配Gate服务器地址,对Actor消息进程进行管理,对玩家ID登录后的Key进行管理。加载有玩家管理组件PlayerComponent,管理登陆时联网的Key组件GateSessionKeyComponent。
\item Location:连接内网,服务器进程状态集中管理(Actor消息IP管理服务器)。加载有内网组件NetInnerComponent,服务器消息处理状态存储组件LocationComponent。对客户端的登录信息进行验证和客户端登录后连接的服务器,登录后通过此服务器进行消息互动,也可称为验证服务器。
\item Map:连接内网,对ActorMessage消息进行管理(添加、移除、分发等),对场景内现在活动物体存储管理,对内网服务器进程进行操作,对Actor消息进程进行管理,对Actor消息进行管理(添加、移除、分发等),服务器帧率管理。服务器帧率管理组件ServerFrameComponent。
\item AllServer:将以上服务器功能集中合并成一个服务器。另外增加DB连接组件DBComponent
\item Benchmark:连接内网和测试服务器承受力。加载有内网组件NetInnerComponent,服务器承受力测试组件BenchmarkComponent。
\end{itemize}


\section{Session 会话框相关}
\label{sec-4}
\begin{itemize}
\item 当需要连的时候,比如网关服与匹配服,新的框架里连接时容易出现困难,找不到组件,或是用不对组件,或是组件用得不对,端没能分清楚。理解不够。
\item 就是说,这个新的ET7 框架下,服务端的这些,事件机制的,没弄明白没弄透彻。
\end{itemize}


\section{Root 客户端根场景管理以及必要的组件:【爱表哥,爱生活!!!任何时候,活宝妹就是一定要嫁给亲爱的表哥!!爱表哥,爱生活!!!】}
\label{sec-5}
\begin{itemize}
\item 昨天晚上,今天上午把这个根场景下的必要的组件,疑难点大致又看了一遍过了一遍。以后再有什么不懂,或是理解一点儿新的,再添加。【爱表哥,爱生活!!!任何时候,活宝妹就是一定要嫁给亲爱的表哥!!!】
\item 把这个客户端的根场景相关的管理组件整理一下。在系统启动起来的时候,公用组件以及客户端组件的时候,分别有添加一些必要的组件。
\item 这里,当把根场景 Root.Instance.Scene 下添加的组件整理好,就看见,几乎所有客户端必要需要的组件这里都添加了,那么刚才几分钟前,我想要自己添加一个的SceneType.AllServer 里模仿参考项目想要添加的。这里需要想一下:两个都需要吗,还是SceneType.AllServer 因为这个根场景下都加了,我自己画蛇添足的全服就可以不要了?
\end{itemize}
\subsection{Root.cs}
\label{sec-5-1}
\begin{minted}[fontsize=\scriptsize,linenos=false]{csharp}
// 管理根部的Scene: 这个根部,是全局视图的根节点
public class Root: Singleton<Root>, ISingletonAwake { // 单例类,自觉醒
    // 管理所有的Entity: 
    private readonly Dictionary<long, Entity> allEntities = new();
    public Scene Scene { get; private set; }
    public void Awake() {
        this.Scene = EntitySceneFactory.CreateScene(0, SceneType.Process, "Process");
    }
    public override void Dispose() {
        this.Scene.Dispose();
    }
    public void Add(Entity entity) {
        this.allEntities.Add(entity.InstanceId, entity);
    }
        
    public void Remove(long instanceId) {
        this.allEntities.Remove(instanceId);
    }
    public Entity Get(long instanceId) {
        Entity component = null;
        this.allEntities.TryGetValue(instanceId, out component);
        return component;
    }
        
    public override string ToString() {
        StringBuilder sb = new();
        HashSet<Type> noParent = new HashSet<Type>();
        Dictionary<Type, int> typeCount = new Dictionary<Type, int>();
        HashSet<Type> noDomain = new HashSet<Type>();
        foreach (var kv in this.allEntities) {
            Type type = kv.Value.GetType();
            if (kv.Value.Parent == null) {
                noParent.Add(type);
            }
            if (kv.Value.Domain == null) {
                noDomain.Add(type);
            }
            if (typeCount.ContainsKey(type)) {
                typeCount[type]++;
            }
            else {
                typeCount[type] = 1;
            }
        }
        sb.AppendLine("not set parent type: ");
        foreach (Type type in noParent) {
            sb.AppendLine($"\t{type.Name}");
        }
        sb.AppendLine("not set domain type: ");
        foreach (Type type in noDomain) {
            sb.AppendLine($"\t{type.Name}");
        }
        IOrderedEnumerable<KeyValuePair<Type, int>> orderByDescending = typeCount.OrderByDescending(s => s.Value);
        sb.AppendLine("Entity Count: ");
        foreach (var kv in orderByDescending) {
            if (kv.Value == 1) {
                continue;
            }
            sb.AppendLine($"\t{kv.Key.Name}: {kv.Value}");
        }
        return sb.ToString();
    }
}
\end{minted}
\subsection{EntryEvent1\_InitShare.cs: Root 根场景添加组件}
\label{sec-5-2}
\begin{itemize}
\item 这里是双端共享组件启动的时候,也就是说,Root.Instance.Scene 并不仅仅只是客户端场景,也是服务端场景。
\end{itemize}
\begin{minted}[fontsize=\scriptsize,linenos=false]{csharp}
// 公用的相关组件的初始化:
[Event(SceneType.Process)]
public class EntryEvent1_InitShare: AEvent<EventType.EntryEvent1> {

    protected override async ETTask Run(Scene scene, EventType.EntryEvent1 args) {
        Root.Instance.Scene.AddComponent<NetThreadComponent>();
        Root.Instance.Scene.AddComponent<OpcodeTypeComponent>();
        Root.Instance.Scene.AddComponent<MessageDispatcherComponent>();
        Root.Instance.Scene.AddComponent<NumericWatcherComponent>();
        Root.Instance.Scene.AddComponent<AIDispatcherComponent>();
        Root.Instance.Scene.AddComponent<ClientSceneManagerComponent>();
        await ETTask.CompletedTask;
    }
}
\end{minted}
\subsection{EntryEvent2\_InitServer: 服务端启动的时候添加的组件}
\label{sec-5-3}
\begin{minted}[fontsize=\scriptsize,linenos=false]{csharp}
[Event(SceneType.Process)]
public class EntryEvent2_InitServer: AEvent<ET.EventType.EntryEvent2> {
    protected override async ETTask Run(Scene scene, ET.EventType.EntryEvent2 args) {
        // 发送普通actor消息
        Root.Instance.Scene.AddComponent<ActorMessageSenderComponent>();
        // 发送location actor消息
        Root.Instance.Scene.AddComponent<ActorLocationSenderComponent>();
        // 访问location server的组件
        Root.Instance.Scene.AddComponent<LocationProxyComponent>();
        Root.Instance.Scene.AddComponent<ActorMessageDispatcherComponent>();
        Root.Instance.Scene.AddComponent<ServerSceneManagerComponent>();
        Root.Instance.Scene.AddComponent<RobotCaseComponent>();
        Root.Instance.Scene.AddComponent<NavmeshComponent>();
        // 【添加组件】:这里,还可以再添加一些游戏必要【根组件】,如果可以在服务器启动的时候添加的话。会影响服务器启动性能

        StartProcessConfig processConfig = StartProcessConfigCategory.Instance.Get(Options.Instance.Process);
        switch (Options.Instance.AppType) {
        case AppType.Server: {
            Root.Instance.Scene.AddComponent<NetInnerComponent, IPEndPoint>(processConfig.InnerIPPort);
            var processScenes = StartSceneConfigCategory.Instance.GetByProcess(Options.Instance.Process);
            foreach (StartSceneConfig startConfig in processScenes) {
                await SceneFactory.CreateServerScene(ServerSceneManagerComponent.Instance, startConfig.Id, startConfig.InstanceId, startConfig.Zone, startConfig.Name, startConfig.Type, startConfig);
            }
            break;
        }
        case AppType.Watcher: {
            StartMachineConfig startMachineConfig = WatcherHelper.GetThisMachineConfig();
            WatcherComponent watcherComponent = Root.Instance.Scene.AddComponent<WatcherComponent>();
            watcherComponent.Start(Options.Instance.CreateScenes);
            Root.Instance.Scene.AddComponent<NetInnerComponent, IPEndPoint>(NetworkHelper.ToIPEndPoint($"{startMachineConfig.InnerIP}:{startMachineConfig.WatcherPort}"));
            break;
        }
        case AppType.GameTool:
            break;
        }
        if (Options.Instance.Console == 1) {
            Root.Instance.Scene.AddComponent<ConsoleComponent>();
        }
    }
}
\end{minted}


\section{ETTask 和 ETVoid: 第三方库的ETTask, 参考ET-EUI 框架}
\label{sec-6}
\begin{itemize}
\item 特异包装:主要是实际了异步调用的流式写法。它方法定义的内部,是封装有协程异步状态机的?IAsyncStateMachine. 当要运行协程的下一步,也是调用和运行。NET 库里的 IAsyncStateMachine.moveNext()
\item .NET 还提供了 AsyncMethodBuilder 的 type trait 来让你自己实现这个状态机和你自己的 Task 类型,因此你可以最大程度发挥想象来编写你想控制的一切。ETTask|ETVoid 就是使用底层的这些方法来封装的结果。async/await 是一个纯编译器特性。
\item 这个框架里ET7 里,就有相关模块 \textbf{【具体说是,两个实体类,实际定义了两种不同返回值ETTask-ETVoid 的协程编译生成方法】} ,能够实现对这个包装的自动编译成协程的编译逻辑方法定义。理解上,感觉像是ET7 框架里,为了这个流式写法,定义了必要的标签系,和相关的协程生成方法,来帮助这个第三方库实现异步调用的流式写法。
\item 上面的,写得把自己都写昏了。就是ET7 框架是如何实现异步调用的流式写法的呢?它把异步调用封装成协程。面对ET7 框架里广泛用到的ETTask|ETVoid 两类稍带个性化异步任务,如同 ETTask 和 ETVoid 是框架自己的封装一样,这个框架,也使用 .NET 里的 IAsyncStateMachine 等底层接口API 等,自定义了异步协程任务的生成方法。
\item 这类方法里,都封装有一个ETTask, 因为自定义封装在这些自定义类里,就对可能会用到的操作提供了必要的API, 比如设置异常,拿取任务等等。
\item 上面的自定义方法生成器:有三类,分别是 AsyncETVoidMethodBuilder, AsyncETTaskMethodBuilder 和 AsyncETTaskCompletedMethodBuilder
\item 感觉因为这两大返回类型,我没有能看懂看透,所以上面一个部分的消息处理,两个函数Handle() 和 Run() 的返回类型,以及参数被我改得乱七八糟,是不应该的。
\item 磨刀不误砍柴工,我应该投入时间把这个第三方库一样的包装理解透彻,然后再去弄懂上面一个部分,再去改那些编译错误。
\item \textbf{【ET-EUI】里:} 原本类的定义什么的,也是一样的,那就是主要去看,他是怎么使用ETVoid, 为什么它使用ETVoid 不会报错,而我在ET7 里用就会。
\item \textbf{【多线程同步】} 关于多线程同步的理解:来自于网络: 
\begin{itemize}
\item ETTASK的由于没有开新线程,也没有使用线程池Task,所以肯定是在主线程运行的,那么游戏开始的SynchronizationContext.SetSynchronizationContext(OneThreadSynchronizationContext.Instance);这句代码有啥用呢?
\item 个人理解为, \textbf{在ET中虽然主逻辑是单线程的,但是与IO设备,比如从socket读取数据,或者从TCP,KCP获取网络数据得时候,是多线程的获取数据的,所以当数据到达时,为了保证是单线程,所以在获取数据的地方,以回调得方式,将回调方法扔到OneThreadSynchronizationContext中执行} ( \textbf{async设置了同步上下文是线程安全的} ,说的应该也是这个 OneThreadSynchronizationContext() 什么的相关的)
\item 白话多线程同步原理如下:下面的也是ET 框架中网络异步线程同步中干过的同步执行逻辑。那个类大概是 NetService.cs. 就是分主线程,异步线程,有队列,Update() 里同步的。
\begin{itemize}
\item ET是单线程的,所以不会管理线程
\item 跨线程都是把委托投递到一个队列,主线程不停从队列中取出委托执行
\item 你看看asynctool的代码,本质上就是把委托投递到主线程
\item 每帧取完队列中的所有委托,执行完
\end{itemize}
\item 这个细节,是自己第一个游戏里使用ET-EUI 作为服务端,非ET 框架的客户端与服务端连接时,自己曾经遇到过的。非ET 框架的客户端,是使用了一个其它的 UnityPlayer 里一个API 相关的第三方来同步异步线程的结果到主线程。所以这个细节还是印象深刻。
\end{itemize}
\item 首先要能把控得住多线程,才能谈性能。其次,et是服务端多进程,同样能利用多核。et是逻辑单线程,并不意味着只能单线程,你能把控得住,照样可以多线程,一般人是不行的。(这些,看不懂,感觉更像是避重就轻吹牛皮一样。。。)
\item \textbf{【ETTask-await 后面的执行线程:】}
\begin{itemize}
\item async await 如果用的Task, await后面的部分是不确定在哪个线程执行的,猫大以前4.0的做法就是把上下文抛到主线程,让主线程执行.
\item 如果用的是ETTask, await后面的部分是一定在主线程执行的. 就完全相当于写了个回调方法了
\item Task 实际上也是回调, 不过这个回调方法的执行原本可能不在主线程罢了
\end{itemize}
\item ETVoid是代替async void,意思是新开一个协程。ettask跟task一样。当然task不去await也相当于新开协程,但是编辑器会冒出提示,提示你await。所以新开协程最好用ETVoid。4.0用async void。使用场景,自己写写就明白啦. 协程就是回调.
\item \textbf{无GC ETTask},其实是利用对象池,注意,必须小心使用,不懂不要乱用。无GC 的原理同自己写第一个游戏,使用资源池是一样的,就是说,当一个ETTask 使用完毕,不再使用的时候,不是要GC 来回收,而是程序的逻辑自己管理,回收到对象池管理器,对于应用程序来说,就是不释放,自己管理它的再使用。不释放就不会引起GC 回收,所以叫无GC.
\begin{itemize}
\item 请不要随便使用ETTask的对象池,除非你完全搞懂了ETTask!!!
\item 假如开启了池,await之后不能再操作ETTask,否则可能操作到再次从池中分配出来的ETTask,产生灾难性的后果。(自己的理解, await 之后,再操作ETTask, 那么操作的极有可能是【当 boolean fromPool = true】从对象池新取出的一个异步任务,不是预期行为,当然就会引起一片混乱。。。可是,框架里仍然有狠多对异步任务 SetResult() 的地方,尤其是各种服的消息处理器处理逻辑里。什么情境下可以安全地使用SetResult(), 需要自己去搞明白)
\item SetResult的时候请现将tcs置空,避免多次对同一个ETTask SetResult. (这里,对一个异步任务,设置结果 SetResult(), 可能会设置多次吗?)
\item 大部分情况可以用objectwait代替ettask,推荐使用,绝对不会出问题
\end{itemize}
\item 这里因为弄不明白,他们建议的学习方法是:
\begin{itemize}
\item ettask还要啥教程?
\item 要搞懂ettask 下载一个jetbrain peek工具,反编译下看下生成的代码就行了。
\item 参考Timercomponent,看懂就全明白了
\item 看网上的文章看十年也不会明白,自己写一下timercomponet啥都懂了
\item 接下来,自己尝试理解这部分的方法应该是:给VS 2022 安装第三方插件库 ILSpy, 然后借用插件把编译码自己弄出来,插日志,作任何可以帮助自己理解的东西来理解这部分。 \textbf{【先给VS安装一个插件ILSpy,这样更容易反编译代码进行查看,另外要注意反编译async和await的时候,要把C\#代码版本改为4.0哦。】} 前在,这是网上提示的反编译方法。这个,改天再接着看,先再事理理解一点儿别的。今天一定更新一下。明天出行,没时间看和更新。
\end{itemize}
\item 【爱表哥,爱生活!!!活宝妹就是一定要嫁给亲爱的表哥!!!】
\end{itemize}
\subsection{IAsyncStateMachine}
\label{sec-6-1}
\begin{itemize}
\item 异步方法中,每遇见一个 await 调用,都会生成一个异步状态类,这个异步状态类会实现这个接口
\end{itemize}
\begin{minted}[fontsize=\scriptsize,linenos=false]{csharp}
namespace System.Runtime.CompilerServices {
    public interface IAsyncStateMachine {
        void MoveNext();
        void SetStateMachine(IAsyncStateMachine stateMachine);
    }
}
\end{minted}
\subsection{enum AwaiterStatus: IAwaiter.cs 文件里. 理解为异步任务的现执行进展状态}
\label{sec-6-2}
\begin{itemize}
\item 现框架里,扩展 IAwaiter, 自定义的现框架 ETTask 所可能有的三种不同状态。
\begin{minted}[fontsize=\scriptsize,linenos=false]{csharp}
public enum AwaiterStatus: byte {
    // The operation has not yet completed.
    Pending = 0,// 这个用在判断语句里比较多,主要用它来判断:异步任务是否已经完成 
    // The operation completed successfully.
    Succeeded = 1,
    // The operation completed with an error.
    Faulted = 2,
}
\end{minted}
\end{itemize}
\subsection{ETTaskCompleted: 已经完成了的异步任务。比较特殊:可以简单进行写结果?等等的必要回收工作,就可以返回异步任务对象池回收再利用?}
\label{sec-6-3}
\begin{itemize}
\item 因为我把 AsyncMethodBuilder 理解成为:异步任务的协程编译器编译逻辑。
\item 所以这个类就是定义,异步任务协程中的一个特殊状态:异步任务结束了,但是还没能写结果时的IAsyncStateMachine|IAwater 的一个最为特殊的状态。它可以用作可能需要写结果时的一个准备,但也可能不需要再写结果?在框架里用得狠多。所以它狠轻量,可以快速写结果或是快速回收到对象池复用。
\item 因为它是协程异步状态机中的一个相对特殊状态,本质上是异步状态机中的一个极特殊的状态,也提供了必要的API, 比如写结果。
\begin{minted}[fontsize=\scriptsize,linenos=false]{csharp}
[AsyncMethodBuilder(typeof (AsyncETTaskCompletedMethodBuilder))]
public struct ETTaskCompleted: ICriticalNotifyCompletion {
    [DebuggerHidden]
// 能不能理解为,已经结束了的异步任务ETTaskCompleted, 也是协程异步状态机中的一个状态,是IAwaker 的实体类实现状态,返回这个当前已经结束了的状态IAwater..
    public ETTaskCompleted GetAwaiter() {
        return this;
    }
    [DebuggerHidden]
    public bool IsCompleted => true;
    [DebuggerHidden]
    public void GetResult() {
    }
// 就是说:下面的两个回调函数,可以帮  助把异步任务的执行结果给返回回去
    [DebuggerHidden]
    public void OnCompleted(Action continuation) {              
    }
    [DebuggerHidden]
    public void UnsafeOnCompleted(Action continuation) {  
    }
}
\end{minted}
\end{itemize}
\subsection{struct ETVoid: ICriticalNotifyCompletion. 这里涉及协程的分阶段的执行相关逻辑的生成方法自动化相关的标签}
\label{sec-6-4}
\begin{minted}[fontsize=\scriptsize,linenos=false]{csharp}
[AsyncMethodBuilder(typeof (AsyncETVoidMethodBuilder))]// 【异步方法生成标签】:是。NET CompilerService里的属性标签。自动生成协程相关方法的标签?今天晚上也可以细看一下
internal struct ETVoid: ICriticalNotifyCompletion {
    [DebuggerHidden]
        public void Coroutine() { }
    [DebuggerHidden]
        public bool IsCompleted => true;
    [DebuggerHidden]
        public void OnCompleted(Action continuation) { }
    [DebuggerHidden]
        public void UnsafeOnCompleted(Action continuation) { }
}
\end{minted}
\subsection{ETTask: ICriticalNotifyCompletion:}
\label{sec-6-5}
\begin{itemize}
\item 这个类的定义比较大,分普通类,和泛型类。我的笔记需要记在同一个地方。今天早上这个类又记错地方记到ET-EUI 上去了
\begin{minted}[fontsize=\scriptsize,linenos=false]{csharp}
[AsyncMethodBuilder(typeof (ETAsyncTaskMethodBuilder))]
public class ETTask: ICriticalNotifyCompletion {
    public static Action<Exception> ExceptionHandler;// 异常回调
    public static ETTaskCompleted CompletedTask {    // 异步任务结束后的封装
        get {
            return new ETTaskCompleted();
        }
    }
    private static readonly Queue<ETTask> queue = new Queue<ETTask>();// 异步任务对象池
    // 请不要随便使用ETTask的对象池,除非你完全搞懂了ETTask!!!
    // 假如开启了池,await之后不能再操作ETTask,否则可能操作到再次从池中分配出来的ETTask,产生灾难性的后果
    // SetResult的时候请现将tcs置空,避免多次对同一个ETTask SetResult
    public static ETTask Create(bool fromPool = false) {
        if (!fromPool) 
            return new ETTask();
        if (queue.Count == 0) 
            return new ETTask() {fromPool = true};    
        return queue.Dequeue();
    }
    private void Recycle() { // 涉及ETTask 无GC 的逻辑实现:
        if (!this.fromPool)  // 因为不返回对象池,所以会GC
            return; // 原则:只有从池里取出来的,才返回池
        this.state = AwaiterStatus.Pending;// 【没明白:】回收时还设置为 Pending, 什么时候写的当前结果?应该是在回收前
        this.callback = null;
        if (queue.Count > 1000)  // 因为对象池中,异步任务数目已达 1000, 不再回收,也会产生 GC
            return;
        queue.Enqueue(this); // 真正无GC, 因为回收到对象池,队列里去了
    }
    private bool fromPool;
    private AwaiterStatus state;
    private object callback; // Action or ExceptionDispatchInfo
    private ETTask() {  }
    // 【不明白下面两个方法】:不知道这两个方法,绕来绕去,在干什么?
    [DebuggerHidden] // 下面,旦凡带 async 关键字的方法,都是异步方法,编译器编译 async 方法时,会自动生成方法所对应的Coroutine() 方法?
    private async ETVoid InnerCoroutine() { // 这里,怎么就可以用 ETVoid 了呢? private 内部异步方法
        await this; // 【不明白】:每次看见 await 后面接一个单词,就不知道是在等什么?等待这个ETTask 异步任务类初始化完成?
    }
    [DebuggerHidden]
    public void Coroutine() { // 公用无返回,非异步方法。它调用了类内部私有的异步方法 InnerCoroutine() 
        InnerCoroutine().Coroutine();// 这里因为理解不透,总感觉同上面的方法,返回 this, 又调用了自己本方法 Coroutine() 一样。。
    }
    [DebuggerHidden]
    public ETTask GetAwaiter() {
        return this;
    }
    public bool IsCompleted {
        [DebuggerHidden]
        get {
            return this.state != AwaiterStatus.Pending; // 只要不是 Pending 状态,就是异步任务执行结束
        }
    }
    [DebuggerHidden]
    public void UnsafeOnCompleted(Action action) {
        if (this.state != AwaiterStatus.Pending) { // 如果当前异步任务执行结束,就触发非空回调
            action?.Invoke();
            return;
        }
        this.callback = action; // 任务还没有结束,就纪录回调备用
    }
    [DebuggerHidden]
    public void OnCompleted(Action action) {
        this.UnsafeOnCompleted(action);
    }
    [DebuggerHidden]
    public void GetResult() {
        switch (this.state) {
            case AwaiterStatus.Succeeded:
                this.Recycle();
                break;
            case AwaiterStatus.Faulted:
                ExceptionDispatchInfo c = this.callback as ExceptionDispatchInfo;
                this.callback = null;
                this.Recycle();
                c?.Throw();
                break;
            default:
                throw new NotSupportedException("ETTask does not allow call GetResult directly when task not completed. Please use 'await'.");
        }
    }
    [DebuggerHidden]
    public void SetResult() {
        if (this.state != AwaiterStatus.Pending) {
            throw new InvalidOperationException("TaskT_TransitionToFinal_AlreadyCompleted");
        }
        this.state = AwaiterStatus.Succeeded;
        Action c = this.callback as Action;
        this.callback = null;
        c?.Invoke();
    }
    [MethodImpl(MethodImplOptions.AggressiveInlining)]
    [DebuggerHidden]
    public void SetException(Exception e) {
        if (this.state != AwaiterStatus.Pending) {
            throw new InvalidOperationException("TaskT_TransitionToFinal_AlreadyCompleted");
        }
        this.state = AwaiterStatus.Faulted;
        Action c = this.callback as Action;
        this.callback = ExceptionDispatchInfo.Capture(e);
        c?.Invoke();
    }
}
[AsyncMethodBuilder(typeof (ETAsyncTaskMethodBuilder<>))]
public class ETTask<T>: ICriticalNotifyCompletion {
    private static readonly Queue<ETTask<T>> queue = new Queue<ETTask<T>>();
    // 请不要随便使用ETTask的对象池,除非你完全搞懂了ETTask!!!
    // 假如开启了池,await之后不能再操作ETTask,否则可能操作到再次从池中分配出来的ETTask,产生灾难性的后果
    // SetResult的时候请现将tcs置空,避免多次对同一个ETTask SetResult
    public static ETTask<T> Create(bool fromPool = false) {
        if (!fromPool) 
            return new ETTask<T>();
        if (queue.Count == 0) 
            return new ETTask<T>() { fromPool = true };    
        return queue.Dequeue();
    }
    private void Recycle() {
        if (!this.fromPool) 
            return;
        this.callback = null;
        this.value = default;
        this.state = AwaiterStatus.Pending;
        // 太多了
        if (queue.Count > 1000) 
            return;
        queue.Enqueue(this);
    }
    private bool fromPool;
    private AwaiterStatus state;
    private T value;
    private object callback; // Action or ExceptionDispatchInfo
    private ETTask() {
    }
    [DebuggerHidden]
    private async ETVoid InnerCoroutine() {
        await this;
    }
    [DebuggerHidden]
    public void Coroutine() {
        InnerCoroutine().Coroutine();
    }
    [DebuggerHidden]
    public ETTask<T> GetAwaiter() {
        return this;
    }
    [DebuggerHidden]
    public T GetResult() {
        switch (this.state) {
        case AwaiterStatus.Succeeded:
            T v = this.value;
            this.Recycle();
            return v;
        case AwaiterStatus.Faulted:
            ExceptionDispatchInfo c = this.callback as ExceptionDispatchInfo;
            this.callback = null;
            this.Recycle();
            c?.Throw();
            return default;
        default:
            throw new NotSupportedException("ETask does not allow call GetResult directly when task not completed. Please use 'await'.");
        }
    }
    public bool IsCompleted {
        [DebuggerHidden]
        get {
            return state != AwaiterStatus.Pending;
        }
    } 
    [DebuggerHidden]
    public void UnsafeOnCompleted(Action action) {
        if (this.state != AwaiterStatus.Pending) {
            action?.Invoke();
            return;
        }
        this.callback = action;
    }
    [DebuggerHidden]
    public void OnCompleted(Action action) {
        this.UnsafeOnCompleted(action);
    }
    [DebuggerHidden]
    public void SetResult(T result) {
        if (this.state != AwaiterStatus.Pending) {
            throw new InvalidOperationException("TaskT_TransitionToFinal_AlreadyCompleted");
        }
        this.state = AwaiterStatus.Succeeded;
        this.value = result;
        Action c = this.callback as Action;
        this.callback = null;
        c?.Invoke();
    }
    [DebuggerHidden]
    public void SetException(Exception e) {
        if (this.state != AwaiterStatus.Pending) {
            throw new InvalidOperationException("TaskT_TransitionToFinal_AlreadyCompleted");
        }
        this.state = AwaiterStatus.Faulted;
        Action c = this.callback as Action;
        this.callback = ExceptionDispatchInfo.Capture(e);
        c?.Invoke();
    }
}
\end{minted}
\end{itemize}
\subsection{ETCancellationToken: 管理所有的取消?回调:因为可能不止一个取消回调,所以 HashSet 管理}
\label{sec-6-6}
\begin{minted}[fontsize=\scriptsize,linenos=false]{csharp}
public class ETCancellationToken {// 管理所有的【取消】回调:因为可能不止一个取消回调,所以 HashSet 管理 
    private HashSet<Action> actions = new HashSet<Action>();
    public void Add(Action callback) {
        // 如果action是null,绝对不能添加,要抛异常,说明有协程泄漏
        // 【不喜欢这个注释,看不懂,感觉它吓唬人的。。】
        this.actions.Add(callback);
    }
    public void Remove(Action callback) {
        this.actions?.Remove(callback);
    }
    public bool IsDispose() {
        return this.actions == null;
    }
    public void Cancel() {
        if (this.actions == null) {
            return;
        }
        this.Invoke();
    }
    private void Invoke() {
        HashSet<Action> runActions = this.actions;
        this.actions = null;
        try {
            foreach (Action action in runActions) {
                action.Invoke();
            }
        }
        catch (Exception e) {
            ETTask.ExceptionHandler.Invoke(e);
        }
    }
}
\end{minted}
\subsection{ETTaskHelper: 有个类中类 CoroutineBlocker 看不懂}
\label{sec-6-7}
\begin{minted}[fontsize=\scriptsize,linenos=false]{csharp}
public static class ETTaskHelper {
    public static bool IsCancel(this ETCancellationToken self) {
        if (self == null) 
            return false;
        return self.IsDispose();
    }
    // 【看不懂】:感觉理解这个类有难度
    private class CoroutineBlocker {
        private int count; // 不知道,这个变量记的是什么?
        private ETTask tcs;
        public CoroutineBlocker(int count) {
            this.count = count;
        }
        public async ETTask RunSubCoroutineAsync(ETTask task) {
            try {
                await task;
            }
            finally {
                --this.count;
                if (this.count <= 0 && this.tcs != null) { // 写结果?
                    ETTask t = this.tcs;
                    this.tcs = null;
                    t.SetResult();
                }
            }
        }
        public async ETTask WaitAsync() {
            if (this.count <= 0) 
                return;
            this.tcs = ETTask.Create(true);
            await tcs;
        }
    }
    public static async ETTask WaitAny(List<ETTask> tasks) {
        if (tasks.Count == 0) 
            return;
        CoroutineBlocker coroutineBlocker = new CoroutineBlocker(1);
        foreach (ETTask task in tasks) {
            coroutineBlocker.RunSubCoroutineAsync(task).Coroutine();
        }
        await coroutineBlocker.WaitAsync();
    }
    public static async ETTask WaitAny(ETTask[] tasks) {
        if (tasks.Length == 0) 
            return;
        CoroutineBlocker coroutineBlocker = new CoroutineBlocker(1);
        foreach (ETTask task in tasks) {
            coroutineBlocker.RunSubCoroutineAsync(task).Coroutine();
        }
        await coroutineBlocker.WaitAsync();
    }
    public static async ETTask WaitAll(ETTask[] tasks) {
        if (tasks.Length == 0) 
            return;
        CoroutineBlocker coroutineBlocker = new CoroutineBlocker(tasks.Length);
        foreach (ETTask task in tasks) {
            coroutineBlocker.RunSubCoroutineAsync(task).Coroutine();
        }
        await coroutineBlocker.WaitAsync();
    }
    public static async ETTask WaitAll(List<ETTask> tasks) {
        if (tasks.Count == 0) 
            return;
        CoroutineBlocker coroutineBlocker = new CoroutineBlocker(tasks.Count);
        foreach (ETTask task in tasks) {
            coroutineBlocker.RunSubCoroutineAsync(task).Coroutine();
        }
        await coroutineBlocker.WaitAsync();
    }
}
\end{minted}
\subsection{ETAsyncTaskMethodBuilder: 同样是换汤不换药的两个部分:普通类与泛型类}
\label{sec-6-8}
\begin{minted}[fontsize=\scriptsize,linenos=false]{csharp}
public struct ETAsyncTaskMethodBuilder {
    private ETTask tcs;
    // 1. Static Create method.
    [DebuggerHidden]
    public static ETAsyncTaskMethodBuilder Create() {
        ETAsyncTaskMethodBuilder builder = new ETAsyncTaskMethodBuilder() { tcs = ETTask.Create(true) };
        return builder;
    }
    // 2. TaskLike Task property.
    [DebuggerHidden]
    public ETTask Task => this.tcs;
    // 3. SetException
    [DebuggerHidden]
    public void SetException(Exception exception) {
        this.tcs.SetException(exception);
    }
    // 4. SetResult
    [DebuggerHidden]
    public void SetResult() {
        this.tcs.SetResult();
    }
    // 5. AwaitOnCompleted
    [DebuggerHidden]
    public void AwaitOnCompleted<TAwaiter, TStateMachine>(ref TAwaiter awaiter, ref TStateMachine stateMachine) where TAwaiter : INotifyCompletion where TStateMachine : IAsyncStateMachine {
        awaiter.OnCompleted(stateMachine.MoveNext);
    }
    // 6. AwaitUnsafeOnCompleted
    [DebuggerHidden]
    [SecuritySafeCritical]
    public void AwaitUnsafeOnCompleted<TAwaiter, TStateMachine>(ref TAwaiter awaiter, ref TStateMachine stateMachine) where TAwaiter : ICriticalNotifyCompletion where TStateMachine : IAsyncStateMachine {
        awaiter.OnCompleted(stateMachine.MoveNext);
    }
    // 7. Start
    [DebuggerHidden]
    public void Start<TStateMachine>(ref TStateMachine stateMachine) where TStateMachine : IAsyncStateMachine {
        stateMachine.MoveNext();
    }
    // 8. SetStateMachine
    [DebuggerHidden]
    public void SetStateMachine(IAsyncStateMachine stateMachine) {
    }
}
public struct ETAsyncTaskMethodBuilder<T> {
    private ETTask<T> tcs;
    // 1. Static Create method.
    [DebuggerHidden]
    public static ETAsyncTaskMethodBuilder<T> Create() {
        ETAsyncTaskMethodBuilder<T> builder = new ETAsyncTaskMethodBuilder<T>() { tcs = ETTask<T>.Create(true) };
        return builder;
    }
    // 2. TaskLike Task property.
    [DebuggerHidden]
    public ETTask<T> Task => this.tcs;
    // 3. SetException
    [DebuggerHidden]
    public void SetException(Exception exception) {
        this.tcs.SetException(exception);
    }
    // 4. SetResult
    [DebuggerHidden]
    public void SetResult(T ret) {
        this.tcs.SetResult(ret);
    }
    // 5. AwaitOnCompleted
    [DebuggerHidden]
    public void AwaitOnCompleted<TAwaiter, TStateMachine>(ref TAwaiter awaiter, ref TStateMachine stateMachine) where TAwaiter : INotifyCompletion where TStateMachine : IAsyncStateMachine {
        awaiter.OnCompleted(stateMachine.MoveNext);
    }
    // 6. AwaitUnsafeOnCompleted
    [DebuggerHidden]
    [SecuritySafeCritical]
    public void AwaitUnsafeOnCompleted<TAwaiter, TStateMachine>(ref TAwaiter awaiter, ref TStateMachine stateMachine) where TAwaiter : ICriticalNotifyCompletion where TStateMachine : IAsyncStateMachine {
        awaiter.OnCompleted(stateMachine.MoveNext);
    }
    // 7. Start
    [DebuggerHidden]
    public void Start<TStateMachine>(ref TStateMachine stateMachine) where TStateMachine : IAsyncStateMachine {
        stateMachine.MoveNext();
    }
    // 8. SetStateMachine
    [DebuggerHidden]
    public void SetStateMachine(IAsyncStateMachine stateMachine) {
    }
}
\end{minted}
\subsection{AsyncETTaskCompletedMethodBuilder:}
\label{sec-6-9}
\begin{minted}[fontsize=\scriptsize,linenos=false]{csharp}
public struct AsyncETTaskCompletedMethodBuilder {
    // 1. Static Create method.
    [DebuggerHidden]
    public static AsyncETTaskCompletedMethodBuilder Create() {
        AsyncETTaskCompletedMethodBuilder builder = new AsyncETTaskCompletedMethodBuilder();
        return builder;
    }
    // 2. TaskLike Task property(void)
    public ETTaskCompleted Task => default;
    // 3. SetException
    [DebuggerHidden]
    public void SetException(Exception e) {
        ETTask.ExceptionHandler.Invoke(e);
    }
    // 4. SetResult
    [DebuggerHidden]
    public void SetResult() { // do nothing
    }
    // 5. AwaitOnCompleted
    [DebuggerHidden]
    public void AwaitOnCompleted<TAwaiter, TStateMachine>(ref TAwaiter awaiter, ref TStateMachine stateMachine) where TAwaiter : INotifyCompletion where TStateMachine : IAsyncStateMachine {
        awaiter.OnCompleted(stateMachine.MoveNext);
    }
    // 6. AwaitUnsafeOnCompleted
    [DebuggerHidden]
    [SecuritySafeCritical]
    public void AwaitUnsafeOnCompleted<TAwaiter, TStateMachine>(ref TAwaiter awaiter, ref TStateMachine stateMachine) where TAwaiter : ICriticalNotifyCompletion where TStateMachine : IAsyncStateMachine {
        awaiter.UnsafeOnCompleted(stateMachine.MoveNext);
    }
    // 7. Start
    [DebuggerHidden]
    public void Start<TStateMachine>(ref TStateMachine stateMachine) where TStateMachine : IAsyncStateMachine {
        stateMachine.MoveNext();
    }
    // 8. SetStateMachine
    [DebuggerHidden]
    public void SetStateMachine(IAsyncStateMachine stateMachine) {
    }
}
\end{minted}
\subsection{AsyncETVoidMethodBuilder: 定义的是 async ETVoid 的编译方法 ?}
\label{sec-6-10}
\begin{minted}[fontsize=\scriptsize,linenos=false]{csharp}
// 异步ETVoid 内部生成方法:
internal struct AsyncETVoidMethodBuilder {
    // 1. Static Create method.
    [DebuggerHidden]
    public static AsyncETVoidMethodBuilder Create() {
        AsyncETVoidMethodBuilder builder = new AsyncETVoidMethodBuilder();
        return builder;
    }
    // 2. TaskLike Task property(void)
    [DebuggerHidden]
    public ETVoid Task => default;
    // 3. SetException
    [DebuggerHidden]
    public void SetException(Exception e) {
        ETTask.ExceptionHandler.Invoke(e);
    }
    // 4. SetResult
    [DebuggerHidden]
    public void SetResult() {
        // do nothing: 因为它实际的返回值是 void 
    }
    // 5. AwaitOnCompleted
    [DebuggerHidden]
    public void AwaitOnCompleted<TAwaiter, TStateMachine>(ref TAwaiter awaiter, ref TStateMachine stateMachine) where TAwaiter : INotifyCompletion where TStateMachine : IAsyncStateMachine {
        awaiter.OnCompleted(stateMachine.MoveNext);
    }
    // 6. AwaitUnsafeOnCompleted
    [DebuggerHidden]
    [SecuritySafeCritical]
    public void AwaitUnsafeOnCompleted<TAwaiter, TStateMachine>(ref TAwaiter awaiter, ref TStateMachine stateMachine) where TAwaiter : ICriticalNotifyCompletion where TStateMachine : IAsyncStateMachine {
        awaiter.UnsafeOnCompleted(stateMachine.MoveNext);
    }
    // 7. Start
    [DebuggerHidden]
    public void Start<TStateMachine>(ref TStateMachine stateMachine) where TStateMachine : IAsyncStateMachine {
        stateMachine.MoveNext();
    }
    // 8. SetStateMachine
    [DebuggerHidden]
    public void SetStateMachine(IAsyncStateMachine stateMachine) {
    }
}
\end{minted}
\subsection{ICriticalNotifyCompletion:}
\label{sec-6-11}
\begin{minted}[fontsize=\scriptsize,linenos=false]{csharp}
namespace System.Runtime.CompilerServices {
// 接口类:提供了一个,任务完成后的回调接口
    public interface ICriticalNotifyCompletion : INotifyCompletion {
        [SecurityCritical]
        void UnsafeOnCompleted(Action continuation);
    }
}
\end{minted}
\subsection{AsyncMethodBuilderAttribute:.NET 系统的标签}
\label{sec-6-12}
\begin{itemize}
\item 自己先前没能理解,为什么标记了【AsyncMethodBuilder(typeof(className))】就能标记某个类的协程生成方法
\item 是因为这个系统标签,它申明了 AttributeUsage 属性,申明了适用类型,可以是(AttributeTargets.Class | AttributeTargets.Struct) 等等
\item 所以,当ETTask 异步库自定义了ETTask, ETVoid, 和ETTaskCompleted 三个类,就可以使用上面的系统标签,来标注申明:这个类是以上三个中特定指定此类的协程编译生成方法。
\begin{minted}[fontsize=\scriptsize,linenos=false]{csharp}
[AttributeUsage(AttributeTargets.Class | AttributeTargets.Struct | AttributeTargets.Enum | AttributeTargets.Interface | AttributeTargets.Delegate, Inherited = false, AllowMultiple = false)]
public sealed class AsyncMethodBuilderAttribute : Attribute {
    public AsyncMethodBuilderAttribute(Type builderType);
    public Type BuilderType { get; }
}// 【任何时候,活宝妹就是一定要嫁给亲爱的表哥!!!爱表哥,爱生活!!!】
\end{minted}
\end{itemize}


\section{Actor 消息相关}
\label{sec-7}
\begin{itemize}
\item 下面套一点儿网络上的总结,可以帮助自己从比较高的层次上来理解和区分细节。
\item ET中,正常的网络消息需要建立一个session链接来发送,这类消息对应的proto需要由IMessage,IResponse,IRequest来修饰。(这是最常规,感觉最容易理解的)
\item 另外还有一种消息机制,称为 \textbf{【Actor机制】} ,挂载了MailBoxComponent的实体会成为一个actor. 而向Actor发送消息可以根据实体的instanceId来发送,不需要自己建立session链接,这类消息在proto中会打上IActorRequest, IActorResponse, IActorMessage的注释,标识为Actor消息。这种机制极大简化了服务器间向Actor发送消息的逻辑,使得实体间通信更加灵活方便。
\item 上面的,自己去想明白,挂载了MailBoxComponent的组件实体,知道对方实体的 instanceId, 背后的封装原理,仍然是对方实体 instanceId 之类的生成得比较聪明,自带自家进程 id, 让MailBoxCompoent 能够方便拿到发向收消息的进程?忘记了,好像是这样的。就是本质上仍是第一种,但封装得狠受用户弱弱程序员方便实用。。。
\item 但有的时候实体需要在服务器间传递(这一块儿还没有涉入,可以简单理解为玩家 me 从加州地图,重入到了亲爱的表哥身边的地图,不嫁给亲爱的表哥就永远不再离开。 me 大概可以理解为从一个地图服搬家转移重入到了另一个地图服, me 所属的进程可能已经变了),每次传递都会实例化一个新的,其instanceId也会变,但实体的id始终不会变,所以为了应对实体传递的问题,增加了proto需要修饰为IActorLocationRequest, IActorLocationResponse, IActorLocationMessage的消息【这一块儿仍不懂,改天再捡】,它可以根据实体Id来发送消息,不受实体在服务器间传递的影响,很好的解决了上面的问题。
\end{itemize}
\subsection{ActorMessageSender: 知道对方的instanceId,使用这个类发actor消息}
\label{sec-7-1}
\begin{itemize}
\item Tcs 成员变量:精华在这里:因为内部自带一个IActorResponse 的异步任务成员变量,可以帮助实现异步消息的自动回复
\item 正是因为内部成员自带一个异步任务,所以会多一个成员变量,就是标记是否要抛异常。这是异步任务成员变量带来的
\begin{minted}[fontsize=\scriptsize,linenos=false]{csharp}
public readonly struct ActorMessageSender {
    public long ActorId { get; }
    public long CreateTime { get; } // 最近接收或者发送消息的时间
    public IActorRequest Request { get; }      // 结构体,也自动封装了,发送的消息
    public bool NeedException { get; }         // 这上下三行:就帮助实现,返回消息的自动回复的结构包装
    public ETTask<IActorResponse> Tcs { get; } // <<<<<<<<<<<<<<<<<<<< 精华在这里:因为内部自带一个IActorResponse 的异步任务成员变量,可以帮助实现异步消息的自动回复
    public ActorMessageSender(long actorId, IActorRequest iActorRequest, ETTask<IActorResponse> tcs, bool needException) { // tv ... 
        this.ActorId = actorId;
        this.Request = iActorRequest;
        this.CreateTime = TimeHelper.ServerNow();
        this.Tcs = tcs;
        this.NeedException = needException;
    }
}
\end{minted}
\end{itemize}
\subsection{ActorMessageSenderComponent: 这个组件里有个计时器自动计时的超时时段、特定超时类型的超时时长成员变量,背后有套计时器管理组件,自动检测消息的发送超时。}
\label{sec-7-2}
\begin{itemize}
\item 超时时间:这个组件有计时器自动计时和超时激活的逻辑,这里定义了这个组件类型的超时时长,在ActorMessageSenderComponentSystem.cs 文件的 \textbf{【Invoke(TimerInvokeType.ActorMessageSenderChecker)】} 标注的ActorMessageSenderChecker 里会用到,检测超时与否
\item \textbf{【组件里消息自动超时Timer 的计时器机制】} :
\begin{itemize}
\item long TimeoutCheckTimer 是个重复闹钟
\item \textbf{【TimerComponent】} :是框架里的单例类,那么应该是,框架里所有的 Timer 定时计时器,应该是由这个单例管理类统一管理。那么这个组件应该能够负责相关逻辑。
\begin{minted}[fontsize=\scriptsize,linenos=false]{csharp}
[ComponentOf(typeof(Scene))]
public class ActorMessageSenderComponent: Entity, IAwake, IDestroy {
// 超时时间:这个组件有计时器自动计时和超时激活的逻辑,这里定义了这个组件类型的超时时长,在【Invoke(TimerInvokeType.ActorMessageSenderChecker)】标注的ActorMessageSenderChecker 里会用到,检测超时与否
    public const long TIMEOUT_TIME = 40 * 1000;
    public static ActorMessageSenderComponent Instance { get; set; }
    public int RpcId;
    public readonly SortedDictionary<int, ActorMessageSender> requestCallback = new SortedDictionary<int, ActorMessageSender>();
// 这个 long: 是重复闹钟的闹钟实例ID, 用来区分任何其它闹钟的
    public long TimeoutCheckTimer; 
    public List<int> TimeoutActorMessageSenders = new List<int>(); // 这桢更新里:待发送给的(接收者rpcId)接收者链表
}
\end{minted}
\end{itemize}
\end{itemize}
\subsection{ActorMessageSenderComponentSystem: 这个类底层封装比较多,功能模块因为是服务器端不太敦悉,多看几遍}
\label{sec-7-3}
\begin{itemize}
\item 这个类,可以看见ET7 框架更为系统化、消息机制的更为往底层或说更进一步的封装,就是今天下午看见的,以前的 handle() 或是 run() 方法,或回调实例 Action<T> reply, 现在的封装里,这些什么创建回复实例之类的,全部封装到了管理器或是帮助类
\item 如果发向同一个进程,则直接处理,不需要通过网络层。内网组件处理内网消息:这个分支可以再跟一下源码,理解一下重构的事件机制流程
\item 这个生成系,前半部分的计时器消息超时检测,看懂了;后半部分,还没看懂连能。今天上午能连多少连多少
\item 后半部分:是消息发送组件的相对底层逻辑。上层逻辑连通内外网消息,消息处理器,和读到消息发布事件后的触发调用等几个类。要把它们的连通流通原理弄懂。
\begin{minted}[fontsize=\scriptsize,linenos=false]{csharp}
[FriendOf(typeof(ActorMessageSenderComponent))]
public static class ActorMessageSenderComponentSystem {
    // 它自带个计时器,就是说,当服务器繁忙处理不过来,它就极有可能会自动超时,若是超时了,就返回个超时消息回去发送者告知一下,必要时它可以重发。而不超时,就正常基本流程处理了.那么,它就是一个服务端超负载下的自动减压逻辑
    [Invoke(TimerInvokeType.ActorMessageSenderChecker)] // 另一个新标签,激活系: 它标记说,这个激活系类,是 XXX 类型;紧跟着,就定义这个 XXX 类型的激活系类
    public class ActorMessageSenderChecker: ATimer<ActorMessageSenderComponent> {
        protected override void Run(ActorMessageSenderComponent self) { // 申明方法的接口是:ATimer<T> 抽象实现类,它实现了 AInvokeHandler<TimerCallback>
            try {
                self.Check(); // 调用组件自己的方法
             } catch (Exception e) {
                Log.Error($"move timer error: {self.Id}\n{e}");
            }
        }
    }
    [ObjectSystem]
    public class ActorMessageSenderComponentAwakeSystem: AwakeSystem<ActorMessageSenderComponent> {
// 【组件重复闹钟的设置】:实现组件内,消息的自动计时,超时触发Invoke 标签,调用相关逻辑来检测超时消息
        protected override void Awake(ActorMessageSenderComponent self) {
            ActorMessageSenderComponent.Instance = self;
// 这个重复闹钟,是消息自动计时超时过滤器的上下文连接桥梁
// 它注册的回调 TimerInvokeType.ActorMessageSenderChecker, 会每个消息超时的时候,都会回来调用 checker 的 Run()==>Check() 方法
// 应该是重复闹钟每秒重复一次,就每秒检查一次,调用一次Check() 方法来检查超时?是过滤器会给服务器减压;但这里的自动检测会把压分在各消息发送组件服务器上
// 这个重复间隔 1 秒钟的时间间隔,它计 1 秒钟开始,重复的逻辑是重复闹钟处理
            self.TimeoutCheckTimer = TimerComponent.Instance.NewRepeatedTimer(1000, TimerInvokeType.ActorMessageSenderChecker, self);
        }
    }//...
// Run() 方法:通过同步异常到ETTask, 通过ETTask 封装的抛异常方式抛出两类异常并返回;和对正常非异常返回消息,同步结果到ETTask, ETTask() 用触发调用注册过的非空回调
// 传进来的参数:是一个IActorResponse 实例,是有最小预处理(初始化了最基本成员变量:异常类型)、【写了个半好】的结果(异常)。结果还没同步到异步任务,待写;返回消息,待发送
    private static void Run(ActorMessageSender self, IActorResponse response) { 
        // 对于每个超时了的消息:超时错误码都是:ErrorCore.ERR_ActorTimeout, 所以会从发送消息超时异常里抛出异常,不用发送错误码【消息】回去,是抛异常
        if (response.Error == ErrorCore.ERR_ActorTimeout) { // 写:发送消息超时异常。因为同步到异步任务 ETTask 里,所以异步任务模块 ETTask会自动抛出异常
            self.Tcs.SetException(new Exception($"Rpc error: request, 注意Actor消息超时,请注意查看是否死锁或者没有reply: actorId: {self.ActorId} {self.Request}, response: {response}"));
            return;
        }
// 这个Run() 方法,并不是只有 Check() 【发送消息超时异常】一个方法调用。什么情况下的调用,会走到下面的分支?文件尾,有正常消息同步结果到ETTask 的调用 
// ActorMessageSenderComponent 一个组件,一次只执行一个(返回)消息发送任务,成员变量永远只管当前任务,
// 也是因为Actor 机制是并行的,一个使者一次只能发一个消息 ...
// 【组件管理器的执行频率, Run() 方法的调用频率】:要是消息太多,发不完怎么办呢?去搜索下面调用 Run() 方法的正常结果消息的调用处理频率。。。
        if (self.NeedException && ErrorCore.IsRpcNeedThrowException(response.Error)) { // 若是有异常(判断条件:消息要抛异常否?是否真有异常?),就先抛异常
            self.Tcs.SetException(new Exception($"Rpc error: actorId: {self.ActorId} request: {self.Request}, response: {response}"));
            return;
        }
        self.Tcs.SetResult(response); // 【写结果】:将【写了个半好】的消息,写进同步到异步任务的结果里;把异步任务的状态设置为完成;并触发必要的非空回调到发送者
        // 上面【异步任务 ETTask.SetResult()】,会调用注册过的一个回调,所以ETTask 封装,设置结果这一步,会自动触发调用注册过的一个回调(如果没有设置回调,因为空,就不会调用)
        // ETTask.SetResult() 异步任务写结果了,非空回调是会调用。非空回调是什么,是把返回消息发回去吗?不是。因为有独立的发送逻辑。
        // 再去想 IMHandler: 它是消息处理器。问题就变成是,当返回消息写好了,写好了一个完整的可以发送、待发送的消息,谁来处理的?有某个更底层的封装会调用这个类的发送逻辑。去把这个更底层的封装找出来,就是框架封装里,调用这个生成类Send() 方法的地方。
        // 这个服,这个自带计时器减压装配装置自带的消息处理器逻辑会处理?不是这个。减压装置,有发送消息超时,只触发最小检测,并抛发送消息超时异常给发送者告知,不写任何结果消息 
    }
    private static void Check(this ActorMessageSenderComponent self) {
        long timeNow = TimeHelper.ServerNow();
        foreach ((int key, ActorMessageSender value) in self.requestCallback) {
            // 因为是顺序发送的,所以,检测到第一个不超时的就退出
            // 超时触发的激活逻辑:是有至少一个超时的消息,才会【激活触发检测】;而检测到第一个不超时的,就退出下面的循环。
            if (timeNow < value.CreateTime + ActorMessageSenderComponent.TIMEOUT_TIME) 
                break;
            self.TimeoutActorMessageSenders.Add(key);
        }
// 超时触发的激活逻辑:是有至少一个超时的消息,才会【激活触发检测】;而检测到第一个不超时的,就退出上面的循环。
// 检测到第一个不超时的,理论上说,一旦有一个超时消息就会触发超时检测,但实际使用上,可能存在当检测逻辑被触发走到这里,实际中存在两个或是再多一点儿的超时消息?
        foreach (int rpcId in self.TimeoutActorMessageSenders) { // 一一遍历【超时了的消息】 :
            ActorMessageSender actorMessageSender = self.requestCallback[rpcId];
            self.requestCallback.Remove(rpcId);
            try { // ActorHelper.CreateResponse() 框架系统性的封装:也是通过对消息的发送类型与对应的回复类型的管理,使用帮助类,自动根据类型统一创建回复消息的实例
                // 对于每个超时了的消息:超时错误码都是:ErrorCore.ERR_ActorTimeout. 也就是,是个异常消息的回复消息实例生成帮助类
                IActorResponse response = ActorHelper.CreateResponse(actorMessageSender.Request, ErrorCore.ERR_ActorTimeout);
                Run(actorMessageSender, response); // 猜测:方法逻辑是,把回复消息发送给对应的接收消息的 rpcId
            } catch (Exception e) {
                Log.Error(e.ToString());
            }
        }
        self.TimeoutActorMessageSenders.Clear();
    }

    public static void Send(this ActorMessageSenderComponent self, long actorId, IMessage message) { // 发消息:这个方法,发所有类型的消息,最基接口
        if (actorId == 0) 
            throw new Exception($"actor id is 0: {message}");
        ProcessActorId processActorId = new(actorId);
        // 这里做了优化,如果发向同一个进程,则直接处理,不需要通过网络层
        if (processActorId.Process == Options.Instance.Process) { // 没看懂:这里怎么就说,消息是发向同一进程的了?
            NetInnerComponent.Instance.HandleMessage(actorId, message); // 原理清楚:本进程消息,直接交由本进程内网组件处理
            return;
        }
        Session session = NetInnerComponent.Instance.Get(processActorId.Process); // 非本进程消息,去走网络层
        session.Send(processActorId.ActorId, message);
    }
    public static int GetRpcId(this ActorMessageSenderComponent self) {
        return ++self.RpcId;
    }
    // 这个方法:只对当前进程的发送要求IActorResponse 的消息,封装自家进程的 rpcId, 也就是标明本进程发的消息,来自其它进程的返回消息,到时发到本进程。是特殊使用
    public static async ETTask<IActorResponse> Call(
        this ActorMessageSenderComponent self,
        long actorId,
        IActorRequest request,
        bool needException = true
        ) {
        request.RpcId = self.GetRpcId(); // 封装本进程的 rpcId 
        if (actorId == 0) throw new Exception($"actor id is 0: {request}");
        return await self.Call(actorId, request.RpcId, request, needException);
    }
    // 【艰森诲涩难懂!!】是更底层的实现细节,它封装帮助实现ET7 里消息超时自动过滤抛异常、返回消息的底层封装自动回复、封装了异步任务和必要成员变量来实现这些辅助过滤器等功能 
    public static async ETTask<IActorResponse> Call( // 跨进程发请求消息(要求回复):返回跨进程异步调用结果。是 await 关键字调用,用在异步方法里
        this ActorMessageSenderComponent self,
        long actorId,
        int rpcId,
        IActorRequest iActorRequest,
        bool needException = true
        ) {
        if (actorId == 0) 
            throw new Exception($"actor id is 0: {iActorRequest}");
// 对象池里:取一个异步任务。用这个异步作务实例,去创建下面的消息发送器实例。这里的 IActorResponse T 应该只是一个索引。因为前面看见系统扫描标签系创建返回实例,套到这个索引
        var tcs = ETTask<IActorResponse>.Create(true);
        // 下面,封装好消息发送器,交由消息发送组件管理;交由其管理,就自带消息发送计时超时过滤机制,实现服务器超负荷时的自动分压减压处理。一旦超时自动报废。。。
        self.requestCallback.Add(rpcId, new ActorMessageSender(actorId, iActorRequest, tcs, needException)); 
        self.Send(actorId, iActorRequest); // 把请求消息发出去:所有消息,都调用这个 
        long beginTime = TimeHelper.ServerFrameTime();
// 自己想一下的话:异步消息发出去,某个服会处理,有返回消息的话,这个服处理后会返回一个返回消息。
// 那么下面一行,不是等待创建 Create() 异步任务(同步方法狠快),而是等待这个处理发送消息的服,处理并返回来返回消息(是说,那个服,把处理结果同步到异步任务)
// 不是等异步任务的创建完成(同步方法狠快),实际是等处理发送消息的服,处理完并写好返回消息,同步到异步任务。
// 那个ETTask 里的回调 callback,是怎么回调的?这里Tcs 没有设置任何回调。ETTask 里所谓回调,是执行异步状态机的下一步,没有实际应用层面的回调意义
// 或说把返回消息的内容填好,【应该还没发回到消息发送者???】返回消息填好了,ETTask 异步任务的结果同步到位了,底层会自动发回来
// 【异步任务结果是怎么回来的?】是前面看过的IMHandler 的底层封装(AMRpcHandler 的抽象逻辑里)发送回来的。ET7 IMHandler 不是重构实现了返回消息的自动发送回复给发送者吗?再去看一遍。
        IActorResponse response = await tcs;  // 等待消息处理服处理完,写好同步好结果到异步任务、异步任务执行完成,状态为 Succeed
        long endTime = TimeHelper.ServerFrameTime();
        long costTime = endTime - beginTime;
        if (costTime > 200) 
            Log.Warning($"actor rpc time > 200: {costTime} {iActorRequest}");
        return response; // 返回:异步网络调用的结果
    }
// 【组件管理器的执行频率, Run() 方法的调用频率】:要是消息太多,发不完怎么办呢?去搜索下面调用 Run() 方法的正常结果消息的调用处理频率。。。
// 【ActorHandleHelper 帮助类】:老是调用这里的方法,要去查那个文件。【本质:内网消息处理器的处理逻辑,一旦是返回消息,就会调用 ActorHandleHelper, 会调用这个方法来处理返回消息】        
// 下面方法:处理IActorResponse 消息,也就是,发回复消息给收消息的人XX, 那么谁发,怎么发,就是这个方法的定义
    // 当是处理【同一进程的消息】:拿到的消息发送器就是当前组件自己,那么只要把结果同步到当前组件的Tcs 异步任务结果里,异步任务结果就会自动触发调用注册过的回调。全部流程结束
    public static void HandleIActorResponse(this ActorMessageSenderComponent self, IActorResponse response) {
        ActorMessageSender actorMessageSender;
// 下面取、实例化 ActorMessageSender 来看,感觉收消息的 rpcId, 与消息发送者 ActorMessageSender 成一一对应关系。上面的Call() 方法里,创建实例化消息发送者就是这么创始垢 
        if (!self.requestCallback.TryGetValue(response.RpcId, out actorMessageSender)) // 这里取不到,是说,这个返回消息的发送已经被处理了?
            return;
        self.requestCallback.Remove(response.RpcId); // 这个有序字典,就成为实时更新:随时添加,随时删除
        Run(actorMessageSender, response); // <<<<<<<<<<<<<<<<<<<< 
    }
}
\end{minted}
\item 几个类弄懂: ActorHandleHelper, 以及再上面的,NetInnerComponentOnReadEvent 事件发布等,上层调用的几座桥连通了,才算把整个流程弄懂了。
\item 现在不懂的就变成为:更为底层的,Session 会话框,socket 层的机制。但是因为它们更为底层,亲爱的表哥的活宝妹,现在把有限的精力投入支理解这个框架,适配自己的游戏比较重要。其它不太重要,或是更为底层的,改天有必要的时候再捡再看。【爱表哥,爱生活!!!任何时候,活宝妹就是一定要嫁给亲爱的表哥!!!爱表哥,爱生活!!!】
\end{itemize}
\subsection{LocationProxyComponent: 这个代理,什么情况下会用到?}
\label{sec-7-4}
\begin{itemize}
\item 就是有个启动类管理 StartSceneConfigCategory 类,它会分门别类地管理一些什么网关、注册登录服,地址服之类的东西。然后从这个里面拿位置服务器地址?大概意思是这样。
\item 这个类先前仔细读过。还记得小伙伴搬家吗?有的小伙伴搬得狠慢,要花狠久,它搬家过程中就要上锁。大致是这类位置转移管理,位置添加、更新等相关管理操作。
\begin{minted}[fontsize=\scriptsize,linenos=false]{csharp}
[ComponentOf(typeof(Scene))]
public class LocationProxyComponent: Entity, IAwake, IDestroy {
    [StaticField]
    public static LocationProxyComponent Instance;
}
\end{minted}
\end{itemize}
\subsection{LocationProxyComponentSystem}
\label{sec-7-5}
\begin{minted}[fontsize=\scriptsize,linenos=false]{csharp}
// [ObjectSystem] awake() etc
public static class LocationProxyComponentSystem {
    private static long GetLocationSceneId(long key) {
        return StartSceneConfigCategory.Instance.LocationConfig.InstanceId;
    }
    public static async ETTask Add(this LocationProxyComponent self, long key, long instanceId) {
        await ActorMessageSenderComponent.Instance
            .Call(GetLocationSceneId(key),
                  new ObjectAddRequest() { Key = key, InstanceId = instanceId });
    }
    public static async ETTask Lock(this LocationProxyComponent self, long key, long instanceId, int time = 60000) {
        await ActorMessageSenderComponent.Instance
            .Call(GetLocationSceneId(key),
                  new ObjectLockRequest() { Key = key, InstanceId = instanceId, Time = time });
    }
    public static async ETTask UnLock(this LocationProxyComponent self, long key, long oldInstanceId, long instanceId) {
        await ActorMessageSenderComponent.Instance
            .Call(GetLocationSceneId(key),
                  new ObjectUnLockRequest() { Key = key, OldInstanceId = oldInstanceId, InstanceId = instanceId });
    }
    public static async ETTask Remove(this LocationProxyComponent self, long key) {
        await ActorMessageSenderComponent.Instance
            .Call(GetLocationSceneId(key),
                  new ObjectRemoveRequest() { Key = key });
    }
    public static async ETTask<long> Get(this LocationProxyComponent self, long key) {
        if (key == 0) 
            throw new Exception($"get location key 0");
        // location server配置到共享区,一个大战区可以配置N多个location server,这里暂时为1
        ObjectGetResponse response = (ObjectGetResponse) await ActorMessageSenderComponent.Instance
            .Call(GetLocationSceneId(key),
                new ObjectGetRequest() { Key = key });
        return response.InstanceId;
    }
    public static async ETTask AddLocation(this Entity self) {
        await LocationProxyComponent.Instance.Add(self.Id, self.InstanceId);
    }
    public static async ETTask RemoveLocation(this Entity self) {
        await LocationProxyComponent.Instance.Remove(self.Id);
    }
}
\end{minted}
\subsection{ActorLocationSender: 知道对方的Id,使用这个类发actor消息}
\label{sec-7-6}
\begin{minted}[fontsize=\scriptsize,linenos=false]{csharp}
[ChildOf(typeof(ActorLocationSenderComponent))]
public class ActorLocationSender: Entity, IAwake, IDestroy {
    public long ActorId;
    public long LastSendOrRecvTime; // 最近接收或者发送消息的时间
    public int Error;
}
\end{minted}
\subsection{ActorLocationSenderComponent: 位置发送组件}
\label{sec-7-7}
\begin{minted}[fontsize=\scriptsize,linenos=false]{csharp}
 [ComponentOf(typeof(Scene))]
 public class ActorLocationSenderComponent: Entity, IAwake, IDestroy {
     public const long TIMEOUT_TIME = 60 * 1000;
     public static ActorLocationSenderComponent Instance { get; set; }
     public long CheckTimer;
 }
\end{minted}
\subsection{ActorLocationSenderComponentSystem: 这个类,也要明天上午再看一下}
\label{sec-7-8}
\begin{minted}[fontsize=\scriptsize,linenos=false]{csharp}
[Invoke(TimerInvokeType.ActorLocationSenderChecker)]
public class ActorLocationSenderChecker: ATimer<ActorLocationSenderComponent> {
    protected override void Run(ActorLocationSenderComponent self) {
        try {
            self.Check();
        }
        catch (Exception e) {
            Log.Error($"move timer error: {self.Id}\n{e}");
        }
    }
}
// [ObjectSystem] // ...
[FriendOf(typeof(ActorLocationSenderComponent))]
[FriendOf(typeof(ActorLocationSender))]
public static class ActorLocationSenderComponentSystem {
    public static void Check(this ActorLocationSenderComponent self) {
        using (ListComponent<long> list = ListComponent<long>.Create()) {
            long timeNow = TimeHelper.ServerNow();
            foreach ((long key, Entity value) in self.Children) {
                ActorLocationSender actorLocationMessageSender = (ActorLocationSender) value;
                if (timeNow > actorLocationMessageSender.LastSendOrRecvTime + ActorLocationSenderComponent.TIMEOUT_TIME) 
                    list.Add(key);
            }
            foreach (long id in list) {
                self.Remove(id);
            }
        }
    }
    private static ActorLocationSender GetOrCreate(this ActorLocationSenderComponent self, long id) {
        if (id == 0) 
            throw new Exception($"actor id is 0");
        if (self.Children.TryGetValue(id, out Entity actorLocationSender)) {
            return (ActorLocationSender) actorLocationSender;
        }
        actorLocationSender = self.AddChildWithId<ActorLocationSender>(id);
        return (ActorLocationSender) actorLocationSender;
    }
    private static void Remove(this ActorLocationSenderComponent self, long id) {
        if (!self.Children.TryGetValue(id, out Entity actorMessageSender)) 
            return;
        actorMessageSender.Dispose();
    }
    public static void Send(this ActorLocationSenderComponent self, long entityId, IActorRequest message) {
        self.Call(entityId, message).Coroutine();
    }
    public static async ETTask<IActorResponse> Call(this ActorLocationSenderComponent self, long entityId, IActorRequest iActorRequest) {
        ActorLocationSender actorLocationSender = self.GetOrCreate(entityId);
        // 先序列化好
        int rpcId = ActorMessageSenderComponent.Instance.GetRpcId();
        iActorRequest.RpcId = rpcId;
        long actorLocationSenderInstanceId = actorLocationSender.InstanceId;
        using (await CoroutineLockComponent.Instance.Wait(CoroutineLockType.ActorLocationSender, entityId)) {
            if (actorLocationSender.InstanceId != actorLocationSenderInstanceId) 
                throw new RpcException(ErrorCore.ERR_ActorTimeout, $"{iActorRequest}");
            // 队列中没处理的消息返回跟上个消息一样的报错
            if (actorLocationSender.Error == ErrorCore.ERR_NotFoundActor) 
                return ActorHelper.CreateResponse(iActorRequest, actorLocationSender.Error);
            try {
                return await self.CallInner(actorLocationSender, rpcId, iActorRequest);
            }
            catch (RpcException) {
                self.Remove(actorLocationSender.Id);
                throw;
            }
            catch (Exception e) {
                self.Remove(actorLocationSender.Id);
                throw new Exception($"{iActorRequest}", e);
            }
        }
    }
    private static async ETTask<IActorResponse> CallInner(this ActorLocationSenderComponent self, ActorLocationSender actorLocationSender, int rpcId, IActorRequest iActorRequest) {
        int failTimes = 0;
        long instanceId = actorLocationSender.InstanceId;
        actorLocationSender.LastSendOrRecvTime = TimeHelper.ServerNow();
        while (true) {
            if (actorLocationSender.ActorId == 0) {
                actorLocationSender.ActorId = await LocationProxyComponent.Instance.Get(actorLocationSender.Id);
                if (actorLocationSender.InstanceId != instanceId) 
                    throw new RpcException(ErrorCore.ERR_ActorLocationSenderTimeout2, $"{iActorRequest}");
            }
            if (actorLocationSender.ActorId == 0) {
                actorLocationSender.Error = ErrorCore.ERR_NotFoundActor;
                return ActorHelper.CreateResponse(iActorRequest, ErrorCore.ERR_NotFoundActor);
            }
            IActorResponse response = await ActorMessageSenderComponent.Instance.Call(actorLocationSender.ActorId, rpcId, iActorRequest, false);
            if (actorLocationSender.InstanceId != instanceId) 
                throw new RpcException(ErrorCore.ERR_ActorLocationSenderTimeout3, $"{iActorRequest}");
            switch (response.Error) {
                case ErrorCore.ERR_NotFoundActor: {
                    // 如果没找到Actor,重试
                    ++failTimes;
                    if (failTimes > 20) {
                        Log.Debug($"actor send message fail, actorid: {actorLocationSender.Id}");
                        actorLocationSender.Error = ErrorCore.ERR_NotFoundActor;
                        // 这里不能删除actor,要让后面等待发送的消息也返回ERR_NotFoundActor,直到超时删除
                        return response;
                    }
                    // 等待0.5s再发送
                    await TimerComponent.Instance.WaitAsync(500);
                    if (actorLocationSender.InstanceId != instanceId)
                        throw new RpcException(ErrorCore.ERR_ActorLocationSenderTimeout4, $"{iActorRequest}");
                    actorLocationSender.ActorId = 0;
                    continue;
                }
                case ErrorCore.ERR_ActorTimeout: 
                    throw new RpcException(response.Error, $"{iActorRequest}");
            }
            if (ErrorCore.IsRpcNeedThrowException(response.Error)) {
                throw new RpcException(response.Error, $"Message: {response.Message} Request: {iActorRequest}");
            }
            return response;
        }
    }
}
\end{minted}
\subsection{ActorHelper: 帮助创建IActorResponse 回复消息。狠简单}
\label{sec-7-9}
\begin{minted}[fontsize=\scriptsize,linenos=false]{csharp}
public static class ActorHelper {
    public static IActorResponse CreateResponse(IActorRequest iActorRequest, int error) {
        Type responseType = OpcodeTypeComponent.Instance.GetResponseType(iActorRequest.GetType());
        IActorResponse response = (IActorResponse)Activator.CreateInstance(responseType);
        response.Error = error;
        response.RpcId = iActorRequest.RpcId;
        return response;
    }
}
\end{minted}
\subsection{Actor 消息处理器:基本原理}
\label{sec-7-10}
\begin{itemize}
\item 消息到达MailboxComponent,MailboxComponent是有类型的,不同的类型邮箱可以做不同的处理。目前有两种邮箱类型GateSession跟MessageDispatcher。
\begin{itemize}
\item GateSession邮箱在收到消息的时候会立即转发给客户端。Actor 消息是指来自于服务端的消息(一定是来自于服务端的消息?Actor 一定是进程间,来自于其它服务端的?)。网关服是小区下所有用户的接收消息的代理。所以,网关服一旦收到服务端的返回消息,作为小区下所有用户的代理,就直接转发相应用户。【亲爱的表哥,永远是活宝妹的代理!任何时候,亲爱的表哥的活宝妹就是一定要嫁给亲爱的表哥!!爱表哥,爱生活!!!】
\item MessageDispatcher类型会再次对Actor消息进行分发到具体的Handler处理,默认的MailboxComponent类型是MessageDispatcher。
\end{itemize}
\end{itemize}
\subsection{MailboxType}
\label{sec-7-11}
\begin{minted}[fontsize=\scriptsize,linenos=false]{csharp}
public enum MailboxType {
    MessageDispatcher, // 消息分发器
    UnOrderMessageDispatcher,// 无序分发
    GateSession,// 网关?
}
\end{minted}

\subsection{ActorMessageDispatcherInfo | ActorMessageDispatcherComponent}
\label{sec-7-12}
\begin{minted}[fontsize=\scriptsize,linenos=false]{csharp}
public class ActorMessageDispatcherInfo {
    public SceneType SceneType { get; }
    public IMActorHandler IMActorHandler { get; }
    public ActorMessageDispatcherInfo(SceneType sceneType, IMActorHandler imActorHandler) {
        this.SceneType = sceneType;
        this.IMActorHandler = imActorHandler;
    }
}
[ComponentOf(typeof(Scene))] // Actor消息分发组件
public class ActorMessageDispatcherComponent: Entity, IAwake, IDestroy, ILoad {
    [StaticField]
    public static ActorMessageDispatcherComponent Instance;
    public readonly Dictionary<Type, List<ActorMessageDispatcherInfo>> ActorMessageHandlers = new();
}
\end{minted}
\subsection{ActorMessageDispatcherComponentHelper: 帮助类}
\label{sec-7-13}
\begin{itemize}
\item Actor消息分发组件:对于管理器里的,对同一发送消息类型,不同场景下不同处理器的链表管理,多看几遍
\item 这里,对于同一发送消息类型, 是会、是可能存在【从不同的场景类型中返回,带不同的消息处理器】 以致于必须得链表管理同一发送消息类型的不同可能处理情况。
\begin{minted}[fontsize=\scriptsize,linenos=false]{csharp}
[FriendOf(typeof(ActorMessageDispatcherComponent))] // Actor消息分发组件:对于管理器里的,对同一发送消息类型,不同场景下不同处理器的链表管理,多看几遍
public static class ActorMessageDispatcherComponentHelper {// Awake() Load() Destroy() 省略掉了
    private static void Load(this ActorMessageDispatcherComponent self) { // 加载:程序域回载的时候
        self.ActorMessageHandlers.Clear(); // 清空字典 
        var types = EventSystem.Instance.GetTypes(typeof (ActorMessageHandlerAttribute)); // 扫描程序域里的特定消息处理器标签 
        foreach (Type type in types) {
            object obj = Activator.CreateInstance(type); // 加载时:框架封装,自动创建【消息处理器】实例
            IMActorHandler imHandler = obj as IMActorHandler;
            if (imHandler == null) {
                throw new Exception($"message handler not inherit IMActorHandler abstract class: {obj.GetType().FullName}");
            }
            object[] attrs = type.GetCustomAttributes(typeof(ActorMessageHandlerAttribute), false);
            foreach (object attr in attrs) {
                ActorMessageHandlerAttribute actorMessageHandlerAttribute = attr as ActorMessageHandlerAttribute;
                Type messageType = imHandler.GetRequestType(); // 因为消息处理接口的封装:可以拿到发送类型
                Type handleResponseType = imHandler.GetResponseType();// 因为消息处理接口的封装:可以拿到返回消息的类型
                if (handleResponseType != null) {
                    Type responseType = OpcodeTypeComponent.Instance.GetResponseType(messageType);
                    if (handleResponseType != responseType) {
                        throw new Exception($"message handler response type error: {messageType.FullName}");
                    }
                }
                // 将必要的消息【发送类型】【返回类型】存起来,统一管理,备用
                // 这里,对于同一发送消息类型, 是会、是可能存在【从不同的场景类型中返回,带不同的消息处理器】 以致于必须得链表管理
                // 这里,感觉因为想不到、从概念上也地无法理解,可能会存在的适应情况、上下文场景,所以这里的链表管理同一发送消息类型,理解起来还有点儿困难
                ActorMessageDispatcherInfo actorMessageDispatcherInfo = new(actorMessageHandlerAttribute.SceneType, imHandler);
                self.RegisterHandler(messageType, actorMessageDispatcherInfo); // 存在本管理组件,所管理的字典里
            }
        }
    }
    private static void RegisterHandler(this ActorMessageDispatcherComponent self, Type type, ActorMessageDispatcherInfo handler) {
        // 这里,对于同一发送消息类型, 是会、是可能存在【从不同的场景类型中返回,带不同的消息处理器】 以致于必须得链表管理
        // 这里,感觉因为想不到、从概念上也地无法理解,可能会存在的适应情况、上下文场景,所以这里的链表管理同一发送消息类型,理解起来还有点儿困难
        if (!self.ActorMessageHandlers.ContainsKey(type)) 
            self.ActorMessageHandlers.Add(type, new List<ActorMessageDispatcherInfo>());
        self.ActorMessageHandlers[type].Add(handler);
    }
    public static async ETTask Handle(this ActorMessageDispatcherComponent self, Entity entity, int fromProcess, object message) {
        List<ActorMessageDispatcherInfo> list;
        if (!self.ActorMessageHandlers.TryGetValue(message.GetType(), out list)) // 根据消息的发送类型,来取所有可能的处理器包装链表 
            throw new Exception($"not found message handler: {message}");
        SceneType sceneType = entity.DomainScene().SceneType; // 定位:当前消息的场景类型
        foreach (ActorMessageDispatcherInfo actorMessageDispatcherInfo in list) { // 遍历:这个发送消息类型,所有存在注册过的消息处理器封装
            if (actorMessageDispatcherInfo.SceneType != sceneType)  // 场景不符就跳过
                continue;
            // 定位:是当前特定场景下的消息处理器,那么,就调用这个处理器,要它去干事。【爱表哥,爱生活!!!任何时候,活宝妹就是一定要嫁给亲爱的表哥!!!】
            await actorMessageDispatcherInfo.IMActorHandler.Handle(entity, fromProcess, message);   
        }
    }
}
\end{minted}
\end{itemize}
\subsection{ActorMessageHandlerAttribute 标签系: 去找几个典型标签看看}
\label{sec-7-14}
\begin{minted}[fontsize=\scriptsize,linenos=false]{csharp}
public class ActorMessageHandlerAttribute: BaseAttribute {
    public SceneType SceneType { get; }
    public ActorMessageHandlerAttribute(SceneType sceneType) {
        this.SceneType = sceneType;
    }
}
\end{minted}
\subsection{[ActorMessageHandler(SceneType.Gate)] 标签使用举例}
\label{sec-7-15}
\begin{itemize}
\item 是以前框架中或是参考项目中的例子。标签使用申明说,这是【网关服】上的一个Actor 消息处理器定义类。
\begin{minted}[fontsize=\scriptsize,linenos=false]{csharp}
[ActorMessageHandler(SceneType.Gate)]
public class Actor_MatchSucess_NttHandler : AMActorHandler<User, Actor_MatchSucess_Ntt> {
    protected override void Run(User user, Actor_MatchSucess_Ntt message) {
        user.IsMatching = false;
        user.ActorID = message.GamerID;
        Log.Info($"玩家{user.UserID}匹配成功");
    }
}
\end{minted}
\end{itemize}
\subsection{MailBoxComponent: 挂上这个组件表示该Entity是一个Actor,接收的消息将会队列处理}
\label{sec-7-16}
\begin{minted}[fontsize=\scriptsize,linenos=false]{csharp}
// 挂上这个组件表示该Entity是一个Actor,接收的消息将会队列处理
[ComponentOf]
public class MailBoxComponent: Entity, IAwake, IAwake<MailboxType> {
    // Mailbox的类型
    public MailboxType MailboxType { get; set; }
}
\end{minted}
\subsection{【服务端】ActorHandleHelper 帮助类:连接上下层的中间层桥梁}
\label{sec-7-17}
\begin{itemize}
\item 读了ActorMessageSenderComponentSystem.cs 的具体的消息内容处理、发送,以及计时器消息的超时自动抛超时错误码过滤等底层逻辑处理,
\item 读上下面的顶层的 NetInnerComponentOnReadEvent.cs 的顶层某个某些服,读到消息后的消息处理逻辑
\item 知道,当前帮助类,就是衔接上面的两条顶层调用,与底层具体处理逻辑的桥,把框架上中下层连接连通起来。
\item 分析这个类,应该可以理解底层不同逻辑方法的前后调用关系,消息处理的逻辑模块先后顺序,以及必要的可能的调用频率,或调用上下文情境等。明天上午再看一下
\item 是谁调用这个帮助类? \textbf{IMHandler类的某些继承类} 。我目前仍只总结和清楚了两个抽象继承类,但还不曾熟悉任何实现子类,要去弄那些,顺便把位置相关的也弄懂了
\item 上面 \textbf{【ActorMessageSenderComponentSystem.cs】的使用情境} :有个 \textbf{【服务端热更新的帮助】类MessageHelper.cs}, 发Actor 消息,与ActorLocation 位置消息,也会都是调用 ActorMessageSenderComponentSystem.cs 里定义的底层逻辑。 
\begin{minted}[fontsize=\scriptsize,linenos=false]{csharp}
public static class ActorHandleHelper {
    public static void Reply(int fromProcess, IActorResponse response) {
        if (fromProcess == Options.Instance.Process) { // 返回消息是同一个进程:没明白,这里为什么就断定是同一进程的消息了?直接处理
            // NetInnerComponent.Instance.HandleMessage(realActorId, response); // 等同于直接调用下面这句【我自己暂时放回来的】
            ActorMessageSenderComponent.Instance.HandleIActorResponse(response); // 【没读懂:】同一个进程内的消息,不走网络层,直接处理。什么情况下会是发给同一个进程的?ET7 重构后,同一进程下可能会有不同的先前小服:Realm 注册登录服,Gate 服等;如果不同的SceneType.Map-etc 先前场景小服只要在同一进程,就可以不走网络层吗?
            return;
        }
        // 【不同进程的消息处理:】走网络层,就是调用会话框来发出消息
        Session replySession = NetInnerComponent.Instance.Get(fromProcess); // 从内网组件单例中去拿会话框:不同进程消息,一定走网络,通过会话框把返回消息发回去
        replySession.Send(response);
    }
    public static void HandleIActorResponse(IActorResponse response) {
        ActorMessageSenderComponent.Instance.HandleIActorResponse(response);
    }
    // 分发actor消息
    [EnableAccessEntiyChild]
    public static async ETTask HandleIActorRequest(long actorId, IActorRequest iActorRequest) {
        InstanceIdStruct instanceIdStruct = new(actorId);
        int fromProcess = instanceIdStruct.Process;
        instanceIdStruct.Process = Options.Instance.Process;
        long realActorId = instanceIdStruct.ToLong();
        Entity entity = Root.Instance.Get(realActorId);
        if (entity == null) {
            IActorResponse response = ActorHelper.CreateResponse(iActorRequest, ErrorCore.ERR_NotFoundActor);
            Reply(fromProcess, response);
            return;
        }
        MailBoxComponent mailBoxComponent = entity.GetComponent<MailBoxComponent>();
        if (mailBoxComponent == null) {
            Log.Warning($"actor not found mailbox: {entity.GetType().Name} {realActorId} {iActorRequest}");
            IActorResponse response = ActorHelper.CreateResponse(iActorRequest, ErrorCore.ERR_NotFoundActor);
            Reply(fromProcess, response);
            return;
        }
        switch (mailBoxComponent.MailboxType) {
            case MailboxType.MessageDispatcher: {
                using (await CoroutineLockComponent.Instance.Wait(CoroutineLockType.Mailbox, realActorId)) {
                    if (entity.InstanceId != realActorId) {
                        IActorResponse response = ActorHelper.CreateResponse(iActorRequest, ErrorCore.ERR_NotFoundActor);
                        Reply(fromProcess, response);
                        break;
                    } // 调用管理器组件的处理方法 
                    await ActorMessageDispatcherComponent.Instance.Handle(entity, fromProcess, iActorRequest);
                }
                break;
            }
            case MailboxType.UnOrderMessageDispatcher: {
                await ActorMessageDispatcherComponent.Instance.Handle(entity, fromProcess, iActorRequest);
                break;
            }
            case MailboxType.GateSession:
            default:
                throw new Exception($"no mailboxtype: {mailBoxComponent.MailboxType} {iActorRequest}");
        }
    }
    // 分发actor消息
    [EnableAccessEntiyChild]
    public static async ETTask HandleIActorMessage(long actorId, IActorMessage iActorMessage) {
        InstanceIdStruct instanceIdStruct = new(actorId);
        int fromProcess = instanceIdStruct.Process;
        instanceIdStruct.Process = Options.Instance.Process;
        long realActorId = instanceIdStruct.ToLong();
        Entity entity = Root.Instance.Get(realActorId);
        if (entity == null) {
            Log.Error($"not found actor: {realActorId} {iActorMessage}");
            return;
        }
        MailBoxComponent mailBoxComponent = entity.GetComponent<MailBoxComponent>();
        if (mailBoxComponent == null) {
            Log.Error($"actor not found mailbox: {entity.GetType().Name} {realActorId} {iActorMessage}");
            return;
        }
        switch (mailBoxComponent.MailboxType) {
            case MailboxType.MessageDispatcher: {
                using (await CoroutineLockComponent.Instance.Wait(CoroutineLockType.Mailbox, realActorId)) {
                    if (entity.InstanceId != realActorId) 
                        break;
                    await ActorMessageDispatcherComponent.Instance.Handle(entity, fromProcess, iActorMessage);
                }
                break;
            }
            case MailboxType.UnOrderMessageDispatcher: {
                await ActorMessageDispatcherComponent.Instance.Handle(entity, fromProcess, iActorMessage);
                break;
            }
            case MailboxType.GateSession: {
                if (entity is Session gateSession) 
                    // 发送给客户端
                    gateSession.Send(iActorMessage);
                break;
            }
            default:
                throw new Exception($"no mailboxtype: {mailBoxComponent.MailboxType} {iActorMessage}");
        }
    }
}
\end{minted}
\end{itemize}
\subsection{NetInnerComponentOnReadEvent:}
\label{sec-7-18}
\begin{itemize}
\item 框架相对顶层的:某个某些服,读到消息后,发布读到消息事件后,触发的消息处理逻辑
\item 这个,应该是服务端发布读事件后,触发的订阅者处理读到消息的回调逻辑:分消息类型,进行不同的处理
\end{itemize}
\begin{minted}[fontsize=\scriptsize,linenos=false]{csharp}
// 这个,应该是服务端发布读事件后,触发的订阅者处理读到消息的回调逻辑:分消息类型,进行不同的处理
[Event(SceneType.Process)]
public class NetInnerComponentOnReadEvent: AEvent<NetInnerComponentOnRead> {
    protected override async ETTask Run(Scene scene, NetInnerComponentOnRead args) {
        try {
            long actorId = args.ActorId;
            object message = args.Message;
            // 收到actor消息,放入actor队列
            switch (message) { // 分不同的消息类型,借助 ActorHandleHelper 帮助类,对消息进行处理。既处理【请求消息】,也处理【返回消息】,还【普通消息】
                case IActorResponse iActorResponse: {
                    ActorHandleHelper.HandleIActorResponse(iActorResponse);
                    break;
                }
                case IActorRequest iActorRequest: {
                    await ActorHandleHelper.HandleIActorRequest(actorId, iActorRequest);
                    break;
                }
                case IActorMessage iActorMessage: {
                    await ActorHandleHelper.HandleIActorMessage(actorId, iActorMessage);
                    break;
                }
            }
        }
        catch (Exception e) {
            Log.Error($"InnerMessageDispatcher error: {args.Message.GetType().Name}\n{e}");
        }
        await ETTask.CompletedTask;
    }
}
\end{minted}

\section{StartConfigComponent: 找【各种服】的起始初始化地址}
\label{sec-8}
\begin{itemize}
\item 【服务端】是自己最不熟悉的模块。活宝妹可以啃下安卓,可以写游戏,学习和熟悉游戏引擎都没问题。学习上或是运动上,活宝妹喜欢像运行员一样可以挑战运动极限。活宝妹只有一样,那就是活宝妹就是一定要嫁给亲爱的表哥!!!
\item 【服务端、各服务器的配置、启动初始化】:是这个模块想要总结的内容。这个模块,因为框架重构里所接入的【路由器系统】的整合(感觉起来,就是通过网络,一台台服务端的服务器起来,一台台起来的服务器都向某个路由服,如同各客户端实时向位置服更新客户端的位置信息般,各小服专职服都向路由服上班打卡?要把这些看明白),让活宝妹理解起这个模块来显得相对困难,大概明天上午一上午的时间,都会花在这个模块上。
\item 同步,需要把所涉及的为方便服务端各服务器初始化而定义的各接口,实现类,以及用法弄明白。
\end{itemize}
\subsection{模块里所用到的几个。NET 里的接口, 以及自定义的框架底层辅助体系类等}
\label{sec-8-1}
\subsubsection{ISupportInitialize: 【初始化】的支持接口,就是提供了【初始化之前】【初始化之后】的回调,两个API}
\label{sec-8-1-1}
\begin{minted}[fontsize=\scriptsize,linenos=false]{csharp}
namespace System.ComponentModel {
    public interface ISupportInitialize {
        void BeginInit();
        void EndInit();
    }
}
\end{minted}
\subsubsection{IInvoke: 抽象类会在事件系统 EventSystem.cs 中被用到}
\label{sec-8-1-2}
\begin{minted}[fontsize=\scriptsize,linenos=false]{csharp}
public interface IInvoke {
    Type Type { get; }
}
public abstract class AInvokeHandler<A>: IInvoke where A: struct {
    public Type Type {
        get {
            return typeof (A);
        }
    }
    public abstract void Handle(A a);
}
public abstract class AInvokeHandler<A, T>: IInvoke where A: struct {
    public Type Type {
        get {
            return typeof (A);
        }
    }
    public abstract T Handle(A a);
}
\end{minted}
\subsubsection{ISingleton 单例类接口:框架最底层,有狠多必要的单例类包装,统一实现这个单例接口,就是抽象提纯到框架最底层封装}
\label{sec-8-1-3}
\begin{minted}[fontsize=\scriptsize,linenos=false]{csharp}
public interface ISingleton: IDisposable {
    void Register();
    void Destroy();
    bool IsDisposed();
}
public abstract class Singleton<T>: ISingleton where T: Singleton<T>, new() {
    private bool isDisposed;
    [StaticField]
    private static T instance;
    public static T Instance {
        get {
            return instance;
        }
    }
    void ISingleton.Register() {
        if (instance != null) 
            throw new Exception($"singleton register twice! {typeof (T).Name}");
        instance = (T)this;
    }
    void ISingleton.Destroy() {
        if (this.isDisposed) 
            return;
        this.isDisposed = true;
        instance.Dispose();
        instance = null;
    }
    bool ISingleton.IsDisposed() {
        return this.isDisposed;
    }
    public virtual void Dispose() {
    }
}
\end{minted}
\subsection{ProtoObject: 继承自上面的系统接口,定义必要的回调抽象API}
\label{sec-8-2}
\begin{minted}[fontsize=\scriptsize,linenos=false]{csharp}
public abstract class ProtoObject: Object, ISupportInitialize {
    public object Clone() { // 【进程间可传递的消息】:为什么这里的复制过程,是先序列化,再反序列化?
        byte[] bytes = SerializeHelper.Serialize(this);
        return SerializeHelper.Deserialize(this.GetType(), bytes, 0, bytes.Length);
    }
    public virtual void BeginInit() {
    }
    public virtual void EndInit() {
    }
    public virtual void AfterEndInit() { // 这个回调,与上一个 EndInit() 区别是?
    }
}
\end{minted}
\subsection{ConfigAttribute: 抓这个出去,想去理解,当扫标签属性时,扫到的应该是形如阅读过的【Invoke(TimerInvokeType.ActorMessegaeSenderChecker)】之类的AttributeTargets.Class 特定类的类型?}
\label{sec-8-3}
\begin{itemize}
\item 结合接下来 ConfigLoader.cs 中的源码,多想几遍或查一下,确认自己理解正确了。
\item 框架里有狠多【Config】标签。这里我在看【Invoke|(TimerInvokeType.ActorMessegaeSenderChecker)】相关时,好像把Invoke 与Config 弄混了。它们是两个不同的标签。
\item 还需要把【Config】标签独立出来,再读一下
\end{itemize}
\begin{minted}[fontsize=\scriptsize,linenos=false]{csharp}
[AttributeUsage(AttributeTargets.Class)]
public class ConfigAttribute: BaseAttribute {
}
\end{minted}
\subsection{ConfigLoader.cs: 【服务端】是理解接下来部分的基础。【客户端】有不同逻辑。所以要把两边的都看一下}
\label{sec-8-4}
\begin{minted}[fontsize=\scriptsize,linenos=false]{csharp}
[Invoke] // 激活系: 这个激活系是同属ET 强大的事件系统的一个标签和回调逻辑,处理两种类型: GetAllConfigBytes 和 GetOneConfigBytes
public class GetAllConfigBytes: AInvokeHandler<ConfigComponent.GetAllConfigBytes, Dictionary<Type, byte[]>> {
    public override Dictionary<Type, byte[]> Handle(ConfigComponent.GetAllConfigBytes args) {
        Dictionary<Type, byte[]> output = new Dictionary<Type, byte[]>();
        List<string> startConfigs = new List<string>() {
            "StartMachineConfigCategory",  // 涉及底层配置的几个单例类,为什么这四个单例类类型重要: Machine, Process 进程、Scene 场景, Zone 区
            "StartProcessConfigCategory", 
            "StartSceneConfigCategory", 
            "StartZoneConfigCategory",
        };
// 类型:这里,扫的是所有【Invoke】标签(好像不对),还是说如【Invoke(TimerInvokeType.ActorMessegaeSenderChecker)】之类的Invoke 标签的类型属性?去看一下方法定义
        HashSet<Type> configTypes = EventSystem.Instance.GetTypes(typeof (ConfigAttribute)); // 【Config】标签
        foreach (Type configType in configTypes) {
            string configFilePath;
            if (startConfigs.Contains(configType.Name)) { // 就是,配置文件的地址可能不一样
                configFilePath = $"../Config/Excel/s/{Options.Instance.StartConfig}/{configType.Name}.bytes";    
            }
            else {
                configFilePath = $"../Config/Excel/s/{configType.Name}.bytes";
            }
            output[configType] = File.ReadAllBytes(configFilePath);
        }
        return output;
    }
}
[Invoke]
public class GetOneConfigBytes: AInvokeHandler<ConfigComponent.GetOneConfigBytes, byte[]> {
    public override byte[] Handle(ConfigComponent.GetOneConfigBytes args) {
        // 【Invoke 回调逻辑】:从框架特定位置,读取特定属性条款的配置,返回字节数组
        byte[] configBytes = File.ReadAllBytes($"../Config/{args.ConfigName}.bytes");
        return configBytes;
    }
}
\end{minted}
\subsection{ConfigComponent 组件:单例类。底层组件,负责服务端配置相关管理?}
\label{sec-8-5}
\begin{itemize}
\item 这个底层组件的内部,涉及ET 标签事件系统的扫描【Config】标签,并Invoke 相关(服务端的配置与启动?)这里花点儿时间,再进去把ET 事件系统中各小服服务端根据(excel? 等) 配置文件来加载和启动服务端(或是服务端的必要配置)的原理弄懂
\item 框架事件系统里,有对各种不同标签的处理逻辑。Invoke 同理。程序域加载时,它扫描和管理框架里的所有必要相关标签,同Invoke 标签同样有字典(套字典)纪录管理不同参数类型(args)的字典,字典里不同类型(type) 的激活处理器。对于特定的参数类型,type 类型,如果能够找到激活处理器,就会触发调用此激活回调,来作相应的处理。
\end{itemize}
\begin{minted}[fontsize=\scriptsize,linenos=false]{csharp}
public T Invoke<A, T>(int type, A args) where A: struct {
    // 先试着去拿,框架里这个【特定 args 类型】的所有标签申明过的 invokeHandlers
    if (!this.allInvokes.TryGetValue(typeof(A), out var invokeHandlers)) {
        throw new Exception($"Invoke error: {typeof(A).Name}");
    }
    // 再试着去拿,【特定类型 type】的 invokeHandler 处理器
    if (!invokeHandlers.TryGetValue(type, out var invokeHandler)) {
        throw new Exception($"Invoke error: {typeof(A).Name} {type}");
    }
    var aInvokeHandler = invokeHandler as AInvokeHandler<A, T>;
    if (aInvokeHandler == null) {
        throw new Exception($"Invoke error, not AInvokeHandler: {typeof(T).Name} {type}");
    }
    return aInvokeHandler.Handle(args); // 调用【Invoke】标签的相应处理回调逻辑
}
public void Invoke<A>(A args) where A: struct {
    Invoke(0, args);
}
public T Invoke<A, T>(A args) where A: struct {
    return Invoke<A, T>(0, args);
}
\end{minted}
\begin{itemize}
\item 框架最底层的封装原理如此。这里,更多的是需要去找当前配置系,激活处理器的具体实现逻辑(在ConfigLoader.cs 文件里,两个回调类类型),来理解这个初始化加载模块。
\item 感觉今天上午把目前看到的这些,读得还算比较透彻。【亲爱的表哥,活宝妹一定要嫁的亲爱的表哥!!!任何时候,亲爱的表哥的活宝妹就是一定要嫁给亲爱的表哥!!爱表哥,爱生活!!!】
\end{itemize}
\begin{minted}[fontsize=\scriptsize,linenos=false]{csharp}
// Config组件会扫描所有的有【ConfigAttribute? Invoke|()】标签的配置,加载进来. 不是说Config 组件就【Config】标签标签的!
public class ConfigComponent: Singleton<ConfigComponent> {
    public struct GetAllConfigBytes {  }
    public struct GetOneConfigBytes {
        public string ConfigName;// 只是用一个字符串来区分不同配置 
    }
    private readonly Dictionary<Type, ISingleton> allConfig = new Dictionary<Type, ISingleton>();
    public override void Dispose() {
        foreach (var kv in this.allConfig) {
            kv.Value.Destroy();
        }
    }
    public object LoadOneConfig(Type configType) {
        this.allConfig.TryGetValue(configType, out ISingleton oneConfig);// oneConfig:这里算是自定义变量的【申明与赋值】?
        if (oneConfig != null) {
            oneConfig.Destroy();
        } 
        // 跟进Invoke: 去看一下框架里事件系统,找到具体的激活回调逻辑定义类:ConfigLoader.cs, 去查看里面对 GetOneConfigBytes 类型的激活触发逻辑
        byte[] oneConfigBytes = EventSystem.Instance.Invoke<GetOneConfigBytes, byte[]>(new GetOneConfigBytes() {ConfigName = configType.FullName});
        object category = SerializeHelper.Deserialize(configType, oneConfigBytes, 0, oneConfigBytes.Length);
        ISingleton singleton = category as ISingleton;
        singleton.Register(); // 【单例类初始化】:如果已经初始化过,会抛异常;单例类只初始化一次
        this.allConfig[configType] = singleton; // 底层:管理类单例类,不同类型,各有一个。框架里就有上面看过的四大单例类
        return category;
    }
    public void Load() { // 【加载】:系统加载,程序域加载 
        this.allConfig.Clear(); // 清空
        // 【原理】:借助框架强大事件系统,扫描域里【Invoke|()】标签(2 种);根据参数类型,调用触发激活逻辑,到服务端特定路径特定文件中去读取所有相关配置,并返回字典
        Dictionary<Type, byte[]> configBytes = EventSystem.Instance.Invoke<GetAllConfigBytes, Dictionary<Type, byte[]>>(new GetAllConfigBytes());
        foreach (Type type in configBytes.Keys) {
            byte[] oneConfigBytes = configBytes[type];
            this.LoadOneInThread(type, oneConfigBytes);
        }
    }
    public async ETTask LoadAsync() { // 哪里会调用这个方法?Entry.cs 服务端起来的时候,会调用此底层组件,加载各单例管理类。细看一下这里服务端启动初始化逻辑
        this.allConfig.Clear();
        Dictionary<Type, byte[]> configBytes = EventSystem.Instance.Invoke<GetAllConfigBytes, Dictionary<Type, byte[]>>(new GetAllConfigBytes());
        using ListComponent<Task> listTasks = ListComponent<Task>.Create();
        foreach (Type type in configBytes.Keys) {
            byte[] oneConfigBytes = configBytes[type];
// 四大单例管理类(Machine,Process,Scene,Zone):每个单例类,开一个任务线路去完成?好像是这样的。
// 不明白为什么必须管理那四个,多不同场景可以位于同一进程,一台机器可以多核多进程?区区区。。。不明白
            Task task = Task.Run(() => LoadOneInThread(type, oneConfigBytes)); 
            listTasks.Add(task);
        }
        await Task.WhenAll(listTasks.ToArray());
    }

    private void LoadOneInThread(Type configType, byte[] oneConfigBytes) {
        object category = SerializeHelper.Deserialize(configType, oneConfigBytes, 0, oneConfigBytes.Length);
        lock (this) {
            ISingleton singleton = category as ISingleton;
            singleton.Register(); // 注册单例类:就是启动初始化一个单例类吧,框架里 Invoke 配置相关,有四大单例类
            this.allConfig[configType] = singleton;
        }
    }
}
\end{minted}
\subsection{ConfigSingleton<T>: ProtoObject, ISingleton}
\label{sec-8-6}
\begin{minted}[fontsize=\scriptsize,linenos=false]{java}
public abstract class ConfigSingleton<T>: ProtoObject, ISingleton where T: ConfigSingleton<T>, new() {
        [StaticField]
        private static T instance;
        public static T Instance {
            get {
                return instance ??= ConfigComponent.Instance.LoadOneConfig(typeof (T)) as T;
            }
        }
        void ISingleton.Register() {
            if (instance != null) {
                throw new Exception($"singleton register twice! {typeof (T).Name}");
            }
            instance = (T)this;
        }
        void ISingleton.Destroy() {
            T t = instance;
            instance = null;
            t.Dispose();
        }
        bool ISingleton.IsDisposed() {
            throw new NotImplementedException();
        }
        public override void AfterEndInit() { }
        public virtual void Dispose() { }
    }
\end{minted}
\subsection{SceneFactory 里可以给【匹配服】添加组件}
\label{sec-8-7}
\begin{minted}[fontsize=\scriptsize,linenos=false]{java}
public static class SceneFactory {
    public static async ETTask<Scene> CreateServerScene(Entity parent, long id, long instanceId, int zone, string name, SceneType sceneType, StartSceneConfig startSceneConfig = null) {
        await ETTask.CompletedTask;
        Scene scene = EntitySceneFactory.CreateScene(id, instanceId, zone, sceneType, name, parent);
        scene.AddComponent<MailBoxComponent, MailboxType>(MailboxType.UnOrderMessageDispatcher);
        switch (scene.SceneType) {
        case SceneType.Router:
            scene.AddComponent<RouterComponent, IPEndPoint, string>(startSceneConfig.OuterIPPort, startSceneConfig.StartProcessConfig.InnerIP);
            break;
        case SceneType.RouterManager: // 正式发布请用CDN代替RouterManager
            // 云服务器在防火墙那里做端口映射
            scene.AddComponent<HttpComponent, string>($"http:// *:{startSceneConfig.OuterPort}/");
            break;
        case SceneType.Realm:
            scene.AddComponent<NetServerComponent, IPEndPoint>(startSceneConfig.InnerIPOutPort);
            break;
        case SceneType.Match: // <<<<<<<<<<<<<<<<<<<< 这里是,我可以添加【匹配服】相关功能组件的地方。【参考项目原原码】感觉被我弄丢了
            break;
        case SceneType.Gate:
            scene.AddComponent<NetServerComponent, IPEndPoint>(startSceneConfig.InnerIPOutPort);
            scene.AddComponent<PlayerComponent>();
            scene.AddComponent<GateSessionKeyComponent>();
            break;
        case SceneType.Map:
            scene.AddComponent<UnitComponent>();
            scene.AddComponent<AOIManagerComponent>();
            break;
        case SceneType.Location:
            scene.AddComponent<LocationComponent>();
            break;
//...
        }
        return scene;
    }
}
\end{minted}
\subsection{RouterAddressComponent: 路由器组件}
\label{sec-8-8}
\begin{minted}[fontsize=\scriptsize,linenos=false]{java}
[ComponentOf(typeof(Scene))]
public class RouterAddressComponent: Entity, IAwake<string, int> {
    public IPAddress RouterManagerIPAddress { get; set; }
    public string RouterManagerHost;
    public int RouterManagerPort;
    public HttpGetRouterResponse Info;
    public int RouterIndex;
}
\end{minted}
\subsection{RouterAddressComponentSystem: 路由器的生成系}
\label{sec-8-9}
\begin{minted}[fontsize=\scriptsize,linenos=false]{java}
[FriendOf(typeof(RouterAddressComponent))]
public static class RouterAddressComponentSystem {
    public class RouterAddressComponentAwakeSystem: AwakeSystem<RouterAddressComponent, string, int> {
        protected override void Awake(RouterAddressComponent self, string address, int port) {
            self.RouterManagerHost = address;
            self.RouterManagerPort = port;
        }
    }
    public static async ETTask Init(this RouterAddressComponent self) {
        self.RouterManagerIPAddress = NetworkHelper.GetHostAddress(self.RouterManagerHost);
        await self.GetAllRouter();
    }
    private static async ETTask GetAllRouter(this RouterAddressComponent self) {
        string url = $"http:// {self.RouterManagerHost}:{self.RouterManagerPort}/get_router?v={RandomGenerator.RandUInt32()}";
        Log.Debug($"start get router info: {url}");
        string routerInfo = await HttpClientHelper.Get(url);
        Log.Debug($"recv router info: {routerInfo}");
        HttpGetRouterResponse httpGetRouterResponse = JsonHelper.FromJson<HttpGetRouterResponse>(routerInfo);
        self.Info = httpGetRouterResponse;
        Log.Debug($"start get router info finish: {JsonHelper.ToJson(httpGetRouterResponse)}");
        // 打乱顺序
        RandomGenerator.BreakRank(self.Info.Routers);
        self.WaitTenMinGetAllRouter().Coroutine();
    }
    // 等10分钟再获取一次
    public static async ETTask WaitTenMinGetAllRouter(this RouterAddressComponent self) {
        await TimerComponent.Instance.WaitAsync(5 * 60 * 1000);
        if (self.IsDisposed) 
            return;
        await self.GetAllRouter();
    }
    public static IPEndPoint GetAddress(this RouterAddressComponent self) {
        if (self.Info.Routers.Count == 0) 
            return null;
        string address = self.Info.Routers[self.RouterIndex++ % self.Info.Routers.Count];
        string[] ss = address.Split(':');
        IPAddress ipAddress = IPAddress.Parse(ss[0]);
        if (self.RouterManagerIPAddress.AddressFamily == AddressFamily.InterNetworkV6) { 
            ipAddress = ipAddress.MapToIPv6();
        }
        return new IPEndPoint(ipAddress, int.Parse(ss[1]));
    }
    public static IPEndPoint GetRealmAddress(this RouterAddressComponent self, string account) { // <<<<<<<<<<<<<<<<<<<< 照葫芦画飘,扩展方法 
        int v = account.Mode(self.Info.Realms.Count);
        string address = self.Info.Realms[v];
        string[] ss = address.Split(':');
        IPAddress ipAddress = IPAddress.Parse(ss[0]);
        // if (self.IPAddress.AddressFamily == AddressFamily.InterNetworkV6) 
        //    ipAddress = ipAddress.MapToIPv6();
        return new IPEndPoint(ipAddress, int.Parse(ss[1]));
    }
}
\end{minted}

\subsection{RouterHelper: 路由器帮助类,向路由器注册、申请?}
\label{sec-8-10}
\begin{minted}[fontsize=\scriptsize,linenos=false]{java}
public static class RouterHelper {
    // 注册router
    public static async ETTask<Session> CreateRouterSession(Scene clientScene, IPEndPoint address) {
        (uint recvLocalConn, IPEndPoint routerAddress) = await GetRouterAddress(clientScene, address, 0, 0);
        if (recvLocalConn == 0) 
            throw new Exception($"get router fail: {clientScene.Id} {address}");
        Log.Info($"get router: {recvLocalConn} {routerAddress}");
        Session routerSession = clientScene.GetComponent<NetClientComponent>().Create(routerAddress, address, recvLocalConn);
        routerSession.AddComponent<PingComponent>();
        routerSession.AddComponent<RouterCheckComponent>();
        return routerSession;
    }
    public static async ETTask<(uint, IPEndPoint)> GetRouterAddress(Scene clientScene, IPEndPoint address, uint localConn, uint remoteConn) {
        Log.Info($"start get router address: {clientScene.Id} {address} {localConn} {remoteConn}");
        // return (RandomHelper.RandUInt32(), address);
        RouterAddressComponent routerAddressComponent = clientScene.GetComponent<RouterAddressComponent>();
        IPEndPoint routerInfo = routerAddressComponent.GetAddress();
        uint recvLocalConn = await Connect(routerInfo, address, localConn, remoteConn);
        Log.Info($"finish get router address: {clientScene.Id} {address} {localConn} {remoteConn} {recvLocalConn} {routerInfo}");
        return (recvLocalConn, routerInfo);
    }
    // 向router申请
    private static async ETTask<uint> Connect(IPEndPoint routerAddress, IPEndPoint realAddress, uint localConn, uint remoteConn) {
        uint connectId = RandomGenerator.RandUInt32();
        using Socket socket = new Socket(routerAddress.AddressFamily, SocketType.Dgram, ProtocolType.Udp);
        int count = 20;
        byte[] sendCache = new byte[512];
        byte[] recvCache = new byte[512];
        uint synFlag = localConn == 0? KcpProtocalType.RouterSYN : KcpProtocalType.RouterReconnectSYN;
        sendCache.WriteTo(0, synFlag);
        sendCache.WriteTo(1, localConn);
        sendCache.WriteTo(5, remoteConn);
        sendCache.WriteTo(9, connectId);
        byte[] addressBytes = realAddress.ToString().ToByteArray();
        Array.Copy(addressBytes, 0, sendCache, 13, addressBytes.Length);
        Log.Info($"router connect: {connectId} {localConn} {remoteConn} {routerAddress} {realAddress}");

        EndPoint recvIPEndPoint = new IPEndPoint(IPAddress.Any, 0);
        long lastSendTimer = 0;
        while (true) {
            long timeNow = TimeHelper.ClientFrameTime();
            if (timeNow - lastSendTimer > 300) {
                if (--count < 0) {
                    Log.Error($"router connect timeout fail! {localConn} {remoteConn} {routerAddress} {realAddress}");
                    return 0;
                }
                lastSendTimer = timeNow;
                // 发送
                socket.SendTo(sendCache, 0, addressBytes.Length + 13, SocketFlags.None, routerAddress);
            }
            await TimerComponent.Instance.WaitFrameAsync();
            // 接收
            if (socket.Available > 0) {
                int messageLength = socket.ReceiveFrom(recvCache, ref recvIPEndPoint);
                if (messageLength != 9) {
                    Log.Error($"router connect error1: {connectId} {messageLength} {localConn} {remoteConn} {routerAddress} {realAddress}");
                    continue;
                }
                byte flag = recvCache[0];
                if (flag != KcpProtocalType.RouterReconnectACK && flag != KcpProtocalType.RouterACK) {
                    Log.Error($"router connect error2: {connectId} {synFlag} {flag} {localConn} {remoteConn} {routerAddress} {realAddress}");
                    continue;
                }
                uint recvRemoteConn = BitConverter.ToUInt32(recvCache, 1);
                uint recvLocalConn = BitConverter.ToUInt32(recvCache, 5);
                Log.Info($"router connect finish: {connectId} {recvRemoteConn} {recvLocalConn} {localConn} {remoteConn} {routerAddress} {realAddress}");
                return recvLocalConn;
            }
        }
    }
}
\end{minted}

\subsection{StartSceneConfig: ISupportInitialize 【各种服-配置,场景配置】}
\label{sec-8-11}
\begin{minted}[fontsize=\scriptsize,linenos=false]{csharp}
public partial class StartSceneConfig: ISupportInitialize {
    public long InstanceId;
    public SceneType Type; // 场景类型

    public StartProcessConfig StartProcessConfig {
        get {
            return StartProcessConfigCategory.Instance.Get(this.Process);
        }
    }
    public StartZoneConfig StartZoneConfig {
        get {
            return StartZoneConfigCategory.Instance.Get(this.Zone);
        }
    }
    // 内网地址外网端口,通过防火墙映射端口过来
    private IPEndPoint innerIPOutPort;
    public IPEndPoint InnerIPOutPort {
        get {
            if (innerIPOutPort == null) {
                this.innerIPOutPort = NetworkHelper.ToIPEndPoint($"{this.StartProcessConfig.InnerIP}:{this.OuterPort}");
            }
            return this.innerIPOutPort;
        }
    }
    // 外网地址外网端口
    private IPEndPoint outerIPPort;
    public IPEndPoint OuterIPPort {
        get {
            if (this.outerIPPort == null) {
                this.outerIPPort = NetworkHelper.ToIPEndPoint($"{this.StartProcessConfig.OuterIP}:{this.OuterPort}");
            }
            return this.outerIPPort;
        }
    }
    public override void AfterEndInit() {
        this.Type = EnumHelper.FromString<SceneType>(this.SceneType);
        InstanceIdStruct instanceIdStruct = new InstanceIdStruct(this.Process, (uint) this.Id);
        this.InstanceId = instanceIdStruct.ToLong();
    }
}
\end{minted}
\subsection{StartProcessConfigCategory : ConfigSingleton<StartProcessConfigCategory>, IMerge: 【任何时候,活宝妹就是一定要嫁给亲爱的表哥!!!】}
\label{sec-8-12}
\begin{minted}[fontsize=\scriptsize,linenos=false]{java}
[ProtoContract]
[Config]
public partial class StartProcessConfigCategory : ConfigSingleton<StartProcessConfigCategory>, IMerge {
    [ProtoIgnore]
    [BsonIgnore]
    private Dictionary<int, StartProcessConfig> dict = new Dictionary<int, StartProcessConfig>(); // 管理字典
    [BsonElement]
    [ProtoMember(1)]
    private List<StartProcessConfig> list = new List<StartProcessConfig>();
    public void Merge(object o) {
        StartProcessConfigCategory s = o as StartProcessConfigCategory;
        this.list.AddRange(s.list);
    }
    [ProtoAfterDeserialization]        
    public void ProtoEndInit() {
        foreach (StartProcessConfig config in list) {
            config.AfterEndInit();
            this.dict.Add(config.Id, config);
        }
        this.list.Clear();
        this.AfterEndInit();
    }
    public StartProcessConfig Get(int id) {
        this.dict.TryGetValue(id, out StartProcessConfig item);
        if (item == null) {
            throw new Exception($"配置找不到,配置表名: {nameof (StartProcessConfig)},配置id: {id}");
        }
        return item;
    }
    public bool Contain(int id) {
        return this.dict.ContainsKey(id);
    }
    public Dictionary<int, StartProcessConfig> GetAll() {
        return this.dict;
    }
    public StartProcessConfig GetOne() {
        if (this.dict == null || this.dict.Count <= 0) {
            return null;
        }
        return this.dict.Values.GetEnumerator().Current;
    }
}
[ProtoContract]
public partial class StartProcessConfig: ProtoObject, IConfig {
    [ProtoMember(1)]
    public int Id { get; set; }
    [ProtoMember(2)]
    public int MachineId { get; set; }
    [ProtoMember(3)]
    public int InnerPort { get; set; }
}
\end{minted}
\subsection{StartSceneConfigCategory : 【Matchs!】ConfigSingleton<StartSceneConfigCategory>, IMerge}
\label{sec-8-13}
\begin{itemize}
\item 为什么这个类,会是写了两遍呢?有什么不同?要想一想。总结不该是,通过写,和一再阅读复习和思考,把原本不曾摆放一起的内容摆一起,帮助自己想透彻背后的原理吗?
\item 读里面的登录服,会知道它是如何管理登录服的(就是后面的例子,当它要拿登录服的地址的时候),它们是区服,就是分各个小区管理。如果集群是这个样子,大概匹配服也就是一样分小区管理了。
\item 那么这个配置管理里,因为我要用匹配服与地图服,也要对至少是匹配服进行管理。那么,我在申请匹配的时候,网关服才能拿到匹配服的地址。
\item 只在【服务端】存在。但是在双端模式、与服务端模式下,每种端有两个文件来定义这个类。。一个在【ProtoContract】里,可能可以进程间消息传递?一个在 ConfigPartial 文件夹里
\item 上面的文件重复,还不是很懂。【重构】:因为我现在还比较喜欢使用Unity 下自带的双端模式,可是暂时只改【双端模式 ClientServer】下的文件,另一个专职服务端可能晚点儿再补上去。不用昨天晚上一样每个文件都改。
\end{itemize}
\begin{minted}[fontsize=\scriptsize,linenos=false]{csharp}
// 配置文件处理,或是服务器启动相关类,以前都没仔细读过
public partial class StartSceneConfigCategory {
    public MultiMap<int, StartSceneConfig> Gates = new MultiMap<int, StartSceneConfig>();
    public MultiMap<int, StartSceneConfig> ProcessScenes = new MultiMap<int, StartSceneConfig>();
    public Dictionary<long, Dictionary<string, StartSceneConfig>> ClientScenesByName = new Dictionary<long, Dictionary<string, StartSceneConfig>>();
    public StartSceneConfig LocationConfig;
    public List<StartSceneConfig> Realms = new List<StartSceneConfig>();
    public List<StartSceneConfig> Matchs = new List<StartSceneConfig>(); // <<<<<<<<<<<<<<<<<<<< 添加管理
    public List<StartSceneConfig> Routers = new List<StartSceneConfig>();
    public List<StartSceneConfig> Robots = new List<StartSceneConfig>();
    public StartSceneConfig BenchmarkServer;

    public List<StartSceneConfig> GetByProcess(int process) {
        return this.ProcessScenes[process];
    }
    public StartSceneConfig GetBySceneName(int zone, string name) {
        return this.ClientScenesByName[zone][name];
    }
    public override void AfterEndInit() {
        foreach (StartSceneConfig startSceneConfig in this.GetAll().Values) {
            this.ProcessScenes.Add(startSceneConfig.Process, startSceneConfig);
                
            if (!this.ClientScenesByName.ContainsKey(startSceneConfig.Zone)) {
                this.ClientScenesByName.Add(startSceneConfig.Zone, new Dictionary<string, StartSceneConfig>());
            }
            this.ClientScenesByName[startSceneConfig.Zone].Add(startSceneConfig.Name, startSceneConfig);
                
            switch (startSceneConfig.Type) {
            case SceneType.Realm:
                this.Realms.Add(startSceneConfig);
                break;
            case SceneType.Gate:
                this.Gates.Add(startSceneConfig.Zone, startSceneConfig);
                break;
            case SceneType.Match:                  // <<<<<<<<<<<<<<<<<<<< 自己加的
                this.Matchs.Add(startSceneConfig); // <<<<<<<<<<<<<<<<<<<< 
                break;
            case SceneType.Location:
                this.LocationConfig = startSceneConfig;
                break;
            case SceneType.Robot:
                this.Robots.Add(startSceneConfig);
                break;
            case SceneType.Router:
                this.Routers.Add(startSceneConfig);
                break;
            case SceneType.BenchmarkServer:
                this.BenchmarkServer = startSceneConfig;
                break;
            }
        }
    }
}
\end{minted}
\subsection{HttpGetRouterResponse: 这个 ProtoBuf 的消息类型}
\label{sec-8-14}
\begin{itemize}
\item 框架里,有个专用的路由器管理器场景(服),对路由器,或说各种服的地址进行管理
\item 主要是方便,一个路由器管理组件,来自顶向下地获取,各小区所有路由器地址的?想来当组件要拿地址时,每个小区分服都把自己的地址以消息的形式传回去的?
\end{itemize}
\begin{minted}[fontsize=\scriptsize,linenos=false]{java}
[Message(OuterMessage.HttpGetRouterResponse)]
[ProtoContract]
public partial class HttpGetRouterResponse: ProtoObject {
    [ProtoMember(1)]
    public List<string> Realms { get; set; }
    [ProtoMember(2)]
    public List<string> Routers { get; set; }
}
message HttpGetRouterResponse { // 这里,是 Outer proto 里的消息定义
	repeated string Realms = 1;
	repeated string Routers = 2;
	repeated string Matchs = 3;// 这行是我需要添加,和生成消息的
}
\end{minted}
\subsection{HttpGetRouterHandler : IHttpHandler: 获取各路由器的地址}
\label{sec-8-15}
\begin{itemize}
\item 【匹配服】:因为我想拿这个服的地址,也需要这个帮助类里作相应的修改
\item StartSceneConfigCategory.Instance: 不明白这个实例是存放在哪里,因为可以 proto 消息进程间传递,那么可以试找,哪里调用这个帮助类拿东西?
\item 这个模块:现在还是理解不透。需要某个上午,把所有 RouterComponent 组件及其相关,再理一遍。
\begin{minted}[fontsize=\scriptsize,linenos=false]{csharp}
[HttpHandler(SceneType.RouterManager, "/get_router")]
public class HttpGetRouterHandler : IHttpHandler {
    public async ETTask Handle(Entity domain, HttpListenerContext context) {
        HttpGetRouterResponse response = new HttpGetRouterResponse();
        response.Realms = new List<string>();
        response.Matchs = new List<string>();// 匹配服链表  // <<<<<<<<<<<<<<<<<<<< 
        response.Routers = new List<string>();
        // 是去StartSceneConfigCategory 这里拿的:因为它可以 proto 消息里、进程间传递,这里还不是狠懂,这个东西存放在哪里?
        foreach (StartSceneConfig startSceneConfig in StartSceneConfigCategory.Instance.Realms) {
            response.Realms.Add(startSceneConfig.InnerIPOutPort.ToString());
        }
        foreach (StartSceneConfig startSceneConfig in StartSceneConfigCategory.Instance.Matchs) {
            response.Matchs.Add(startSceneConfig.InnerIPOutPort.ToString());
        }
        foreach (StartSceneConfig startSceneConfig in StartSceneConfigCategory.Instance.Routers) {
            response.Routers.Add($"{startSceneConfig.StartProcessConfig.OuterIP}:{startSceneConfig.OuterPort}");
        }
        HttpHelper.Response(context, response);
        await ETTask.CompletedTask;
    }
}
\end{minted}
\end{itemize}
\subsection{HttpHandler 标签系:标签自带场景类型}
\label{sec-8-16}
\begin{minted}[fontsize=\scriptsize,linenos=false]{csharp}
public class HttpHandlerAttribute: BaseAttribute {
    public SceneType SceneType { get; }
    public string Path { get; }
    public HttpHandlerAttribute(SceneType sceneType, string path) {
        this.SceneType = sceneType;
        this.Path = path;
    }
}
\end{minted}
\subsection{LoginHelper: 登录服的获取地址的方式来获取匹配服的地址了。全框架只有这一个黄金案例}
\label{sec-8-17}
\begin{itemize}
\item 这个是用户登录前,还没能与网关服建立起任何关系,可能会不得不绕得复杂一点儿】:它就是用户登录前、登录时,若是客户端场景还没有这个组件,就添加一下,没什么奇怪的。
\end{itemize}
\begin{minted}[fontsize=\scriptsize,linenos=false]{java}
public static class LoginHelper {
    public static async ETTask Login(Scene clientScene, string account, string password) {
        try {
            // 创建一个ETModel层的Session
            clientScene.RemoveComponent<RouterAddressComponent>();
            // 获取路由跟realmDispatcher地址
            RouterAddressComponent routerAddressComponent = clientScene.GetComponent<RouterAddressComponent>();
            if (routerAddressComponent == null) {
                routerAddressComponent = clientScene.AddComponent<RouterAddressComponent, string, int>(ConstValue.RouterHttpHost, ConstValue.RouterHttpPort);
                await routerAddressComponent.Init();
                clientScene.AddComponent<NetClientComponent, AddressFamily>(routerAddressComponent.RouterManagerIPAddress.AddressFamily);
            }
            IPEndPoint realmAddress = routerAddressComponent.GetRealmAddress(account); // <<<<<<<<<<<<<<<<<<<< 这里就是说,我必须去组件里扩展方法
            R2C_Login r2CLogin;
            using (Session session = await RouterHelper.CreateRouterSession(clientScene, realmAddress)) {
                r2CLogin = (R2C_Login) await session.Call(new C2R_Login() { Account = account, Password = password });
            }
            // 创建一个gate Session,并且保存到SessionComponent中: 与网关服的会话框。主要负责用户下线后会话框的自动移除销毁
            Session gateSession = await RouterHelper.CreateRouterSession(clientScene, NetworkHelper.ToIPEndPoint(r2CLogin.Address));
            clientScene.AddComponent<SessionComponent>().Session = gateSession;
            G2C_LoginGate g2CLoginGate = (G2C_LoginGate)await gateSession.Call(
                new C2G_LoginGate() { Key = r2CLogin.Key, GateId = r2CLogin.GateId});
            Log.Debug("登陆gate成功!");
            await EventSystem.Instance.PublishAsync(clientScene, new EventType.LoginFinish());
        }
        catch (Exception e) {
            Log.Error(e);
        }
    } 
}
\end{minted}
\subsection{GateSessionKeyComponent:}
\label{sec-8-18}
\begin{minted}[fontsize=\scriptsize,linenos=false]{csharp}
[ComponentOf(typeof(Scene))]
public class GateSessionKeyComponent : Entity, IAwake {
    public readonly Dictionary<long, string> sessionKey = new Dictionary<long, string>();
}
\end{minted}
% Emacs 28.2 (Org mode 8.2.7c)
\end{document}