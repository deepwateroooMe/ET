% Created 2023-05-15 Mon 12:10
\documentclass[9pt, b5paper]{article}
\usepackage{xeCJK}
\usepackage[T1]{fontenc}
\usepackage{bera}
\usepackage[scaled]{beraserif}
\usepackage[scaled]{berasans}
\usepackage[scaled]{beramono}
\usepackage[cache=false]{minted}
\usepackage{xltxtra}
\usepackage{graphicx}
\usepackage{xcolor}
\usepackage{multirow}
\usepackage{multicol}
\usepackage{float}
\usepackage{textcomp}
\usepackage{algorithm}
\usepackage{algorithmic}
\usepackage{latexsym}
\usepackage{natbib}
\usepackage{geometry}
\geometry{left=1.2cm,right=1.2cm,top=1.5cm,bottom=1.2cm}
\usepackage[xetex,colorlinks=true,CJKbookmarks=true,linkcolor=blue,urlcolor=blue,menucolor=blue]{hyperref}
\newminted{common-lisp}{fontsize=\footnotesize} 
\author{deepwaterooo}
\date{\today}
\title{ET 框架学习笔记(二)--网络交互相关}
\hypersetup{
  pdfkeywords={},
  pdfsubject={},
  pdfcreator={Emacs 28.2 (Org mode 8.2.7c)}}
\begin{document}

\maketitle
\tableofcontents


\section{IAwake 接口类系统,IStart 重构丢了}
\label{sec-1}
\begin{itemize}
\item 感觉还比较直接,就是帮助搭建热更新域与Unity 常规工程域生命周期回调的桥,搭桥连线,连能就可以了。应该可以扩散出个IStart 接口类
\end{itemize}
\subsection{IMessage,IRequest,IResponse: 进程内?消息类}
\label{sec-1-1}
\begin{minted}[fontsize=\scriptsize,linenos=false]{csharp}
public interface IMessage {}
public interface IRequest: IMessage {
    int RpcId { get; set; }
}
public interface IResponse: IMessage {
    int Error { get; set; }
    string Message { get; set; }
    int RpcId { get; set; }
}
\end{minted}
\subsection{IActorMessage,IActorRequest,IActorResponse: 进程间的?消息类}
\label{sec-1-2}
\begin{minted}[fontsize=\scriptsize,linenos=false]{csharp}
// 不需要返回消息
public interface IActorMessage: IMessage {}
public interface IActorRequest: IRequest {}
public interface IActorResponse: IResponse {}
\end{minted}
\subsection{IActorLocationMessage: 进程间的位置消息相关}
\label{sec-1-3}
\begin{minted}[fontsize=\scriptsize,linenos=false]{csharp}
public interface IActorLocationMessage: IActorRequest {}
public interface IActorLocationRequest: IActorRequest {}
public interface IActorLocationResponse: IActorResponse {}
\end{minted}
\subsection{IMHandler,IMActorHandler: 消息处理器口类【傻傻分不清楚】}
\label{sec-1-4}
\begin{minted}[fontsize=\scriptsize,linenos=false]{csharp}
public interface IMHandler { // 同进程内的
    void Handle(Session session, object message);
    Type GetMessageType();
    Type GetResponseType();
}
public interface IMActorHandler { // 进程间的?
    // ETTask Handle(Entity entity, int fromProcess, object actorMessage);
    void Handle(Entity entity, int fromProcess, object actorMessage); // 自已改成这样的
    Type GetRequestType();
    Type GetResponseType();
}
\end{minted}
\subsection{ILoad,ISystemType: 加载系}
\label{sec-1-5}
\begin{minted}[fontsize=\scriptsize,linenos=false]{csharp}
public interface ISystemType {
    Type Type();
    Type SystemType();
    InstanceQueueIndex GetInstanceQueueIndex();
}

public interface ILoad {
}
public interface ILoadSystem: ISystemType {
    void Run(Entity o);
}
[ObjectSystem]
public abstract class LoadSystem<T> : ILoadSystem where T: Entity, ILoad {
    void ILoadSystem.Run(Entity o) {
        this.Load((T)o);
    }
    Type ISystemType.Type() {
        return typeof(T);
    }
    Type ISystemType.SystemType() {
        return typeof(ILoadSystem);
    }
    InstanceQueueIndex ISystemType.GetInstanceQueueIndex() {
        return InstanceQueueIndex.Load;
    }
    protected abstract void Load(T self);
}
\end{minted}
\subsection{IAwake: 最多可以带四个参数}
\label{sec-1-6}
\begin{minted}[fontsize=\scriptsize,linenos=false]{csharp}
 public interface IAwake {}
 public interface IAwake<A> {}
 public interface IAwake<A, B> {}
 public interface IAwake<A, B, C> {}
 public interface IAwake<A, B, C, D> {}
\end{minted}
\subsection{IStartSystem,StartSystem<T>: 自己加的。【还有问题】系统找不到}
\label{sec-1-7}
\begin{minted}[fontsize=\scriptsize,linenos=false]{csharp}
public interface IStart { }
public interface IStartSystem : ISystemType {
    void Run(Entity o);
}
[ObjectSystem]
public abstract class StartSystem<T> : IStartSystem where T: Entity, IStart {
    public void IStartSystem.Run(Entity o) {
        this.Start((T)o);
    }
    public Type ISystemType.Type() {
        return typeof(T);
    }
    public Type ISystemType.SystemType() {
        return typeof(IStartSystem);
    }
    InstanceQueueIndex ISystemType.GetInstanceQueueIndex() { // 这里没看懂在干什么,大概还有个地方,我得去改
        return InstanceQueueIndex.Start; 
    }
    public abstract void Start(T self);
}
// 整合进了系统:InstanceQueueIndex
public enum InstanceQueueIndex {
    None = -1,
    Start, // 需要把这个回调加入框架统筹管理里去 
    Update,
    LateUpdate,
    Load,
    Max,
}
\end{minted}
\begin{itemize}
\item 参考项目:除了原文件放在ET 域。也【复制了一份到客户端的热更新域里】。可是感觉不应该。因为其它所有的回调都不用复制就可以用。我哪里可能还是没能设置对
\item 改天再检查一下。但是否,对于非系统框架扩展接口,不得不这样?仍然感觉不应该,因为系统框架里其它的生命周期回调函数都不需要复制。改天再做。
\end{itemize}
\subsection{IUpdate:}
\label{sec-1-8}
\begin{minted}[fontsize=\scriptsize,linenos=false]{csharp}
public interface IUpdate {
}
public interface IUpdateSystem: ISystemType {
    void Run(Entity o);
}
[ObjectSystem]
public abstract class UpdateSystem<T> : IUpdateSystem where T: Entity, IUpdate {
    void IUpdateSystem.Run(Entity o) {
        this.Update((T)o);
    }
    Type ISystemType.Type() {
        return typeof(T);
    }
    Type ISystemType.SystemType() {
        return typeof(IUpdateSystem);
    }
    InstanceQueueIndex ISystemType.GetInstanceQueueIndex() {
        return InstanceQueueIndex.Update;
    }
    protected abstract void Update(T self);
}
\end{minted}
\subsection{ILateUpdate: 好像是用于物理引擎,或是相机什么的更新,生命周期回调}
\label{sec-1-9}
\begin{minted}[fontsize=\scriptsize,linenos=false]{csharp}
public interface ILateUpdate {
}
public interface ILateUpdateSystem: ISystemType {
    void Run(Entity o);
}
[ObjectSystem]
public abstract class LateUpdateSystem<T> : ILateUpdateSystem where T: Entity, ILateUpdate {
    void ILateUpdateSystem.Run(Entity o) {
        this.LateUpdate((T)o);
    }
    Type ISystemType.Type() {
        return typeof(T);
    }
    Type ISystemType.SystemType() {
        return typeof(ILateUpdateSystem);
    }
    InstanceQueueIndex ISystemType.GetInstanceQueueIndex() {
        return InstanceQueueIndex.LateUpdate;
    }
    protected abstract void LateUpdate(T self);
}
\end{minted}
\subsection{ISingletonAwake|Update|LateUpdate: Singleton 生命周期回调}
\label{sec-1-10}
\begin{minted}[fontsize=\scriptsize,linenos=false]{csharp}
public interface ISingletonAwake {
    void Awake();
}
public interface ISingletonUpdate {
    void Update();
}
public interface ISingletonLateUpdate {
    void LateUpdate();
}
\end{minted}
\subsection{ISingleton,Singleton<T>: 单例}
\label{sec-1-11}
\begin{minted}[fontsize=\scriptsize,linenos=false]{csharp}
public interface ISingleton: IDisposable {
    void Register();
    void Destroy();
    bool IsDisposed();
}
public abstract class Singleton<T>: ISingleton where T: Singleton<T>, new() {
    private bool isDisposed;
    [StaticField]
    private static T instance;
    public static T Instance {
        get {
            return instance;
        }
    }
    void ISingleton.Register() {
        if (instance != null) {
            throw new Exception($"singleton register twice! {typeof (T).Name}");
        }
        instance = (T)this;
    }
    void ISingleton.Destroy() {
        if (this.isDisposed) {
            return;
        }
        this.isDisposed = true;

        instance.Dispose();
        instance = null;
    }
    bool ISingleton.IsDisposed() {
        return this.isDisposed;
    }
    public virtual void Dispose() {
    }
}
\end{minted}
\subsection{IDestroy,IDestroySystem,DestroySystem<T>: 销毁系}
\label{sec-1-12}
\begin{minted}[fontsize=\scriptsize,linenos=false]{csharp}
public interface IDestroy {
}
public interface IDestroySystem: ISystemType {
    void Run(Entity o);
}
[ObjectSystem]
public abstract class DestroySystem<T> : IDestroySystem where T: Entity, IDestroy {
    void IDestroySystem.Run(Entity o) {
        this.Destroy((T)o);
    }
    Type ISystemType.SystemType() {
        return typeof(IDestroySystem);
    }
    InstanceQueueIndex ISystemType.GetInstanceQueueIndex() {
        return InstanceQueueIndex.None;
    }
    Type ISystemType.Type() {
        return typeof(T);
    }
    protected abstract void Destroy(T self);
}
\end{minted}
\subsection{IEvent,AEvent<A>: 事件}
\label{sec-1-13}
\begin{minted}[fontsize=\scriptsize,linenos=false]{csharp}
public interface IEvent {
    Type Type { get; }
}
public abstract class AEvent<A>: IEvent where A: struct {
    public Type Type {
        get {
            return typeof (A);
        }
    }
    protected abstract ETTask Run(Scene scene, A a);
    public async ETTask Handle(Scene scene, A a) {
        try {
            await Run(scene, a);
        }
        catch (Exception e) {
            Log.Error(e);
        }
    }
}
\end{minted}
\subsection{IAddComponent: 添加组件系}
\label{sec-1-14}
\begin{minted}[fontsize=\scriptsize,linenos=false]{csharp}
 public interface IAddComponent { }
 public interface IAddComponentSystem: ISystemType {
     void Run(Entity o, Entity component);
 }
 [ObjectSystem]
 public abstract class AddComponentSystem<T> : IAddComponentSystem where T: Entity, IAddComponent {
     void IAddComponentSystem.Run(Entity o, Entity component) {
         this.AddComponent((T)o, component);
     }
     Type ISystemType.SystemType() {
         return typeof(IAddComponentSystem);
     }
     InstanceQueueIndex ISystemType.GetInstanceQueueIndex() {
         return InstanceQueueIndex.None;
     }
     Type ISystemType.Type() {
         return typeof(T);
     }
     protected abstract void AddComponent(T self, Entity component);
 }
\end{minted}
\subsection{IGetComponent: 获取组件系。【这里没有看明白】:再去找细节  // <<<<<<<<<<<<<<<<<<<<}
\label{sec-1-15}
\begin{minted}[fontsize=\scriptsize,linenos=false]{csharp}
 // GetComponentSystem有巨大作用,比如每次保存Unit的数据不需要所有组件都保存,只需要保存Unit变化过的组件
 // 是否变化可以通过判断该组件是否GetComponent,Get了就记录该组件【这里没有看明白】:再去找细节  // <<<<<<<<<<<<<<<<<<<< 
 // 这样可以只保存Unit变化过的组件
 // 再比如传送也可以做此类优化
 public interface IGetComponent {
 }
 public interface IGetComponentSystem: ISystemType {
     void Run(Entity o, Entity component);
 }
 [ObjectSystem]
 public abstract class GetComponentSystem<T> : IGetComponentSystem where T: Entity, IGetComponent {
     void IGetComponentSystem.Run(Entity o, Entity component) {
         this.GetComponent((T)o, component);
     }
     Type ISystemType.SystemType() {
         return typeof(IGetComponentSystem);
     }
     InstanceQueueIndex ISystemType.GetInstanceQueueIndex() {
         return InstanceQueueIndex.None;
     }
     Type ISystemType.Type() {
         return typeof(T);
     }
     protected abstract void GetComponent(T self, Entity component);
 }
\end{minted}
\subsection{ISerializeToEntity,IDeserialize,IDeserializeSystem,DeserializeSystem<T>: 序列化,反序列化}
\label{sec-1-16}
\begin{minted}[fontsize=\scriptsize,linenos=false]{csharp}
public interface ISerializeToEntity {
}
public interface IDeserialize {
}
public interface IDeserializeSystem: ISystemType {
    void Run(Entity o);
}
// 反序列化后执行的System
[ObjectSystem]
public abstract class DeserializeSystem<T> : IDeserializeSystem where T: Entity, IDeserialize {
    void IDeserializeSystem.Run(Entity o) {
        this.Deserialize((T)o);
    }
    Type ISystemType.SystemType() {
        return typeof(IDeserializeSystem);
    }
    InstanceQueueIndex ISystemType.GetInstanceQueueIndex() {
        return InstanceQueueIndex.None;
    }
    Type ISystemType.Type() {
        return typeof(T);
    }
    protected abstract void Deserialize(T self);
}
\end{minted}
\subsection{IInvoke,AInvokeHandler<A>,AInvokeHandler<A, T>: 激活类}
\label{sec-1-17}
\begin{minted}[fontsize=\scriptsize,linenos=false]{csharp}
public interface IInvoke {
    Type Type { get; }
}
public abstract class AInvokeHandler<A>: IInvoke where A: struct {
    public Type Type {
        get {
            return typeof (A);
        }
    }
    public abstract void Handle(A a);
}
public abstract class AInvokeHandler<A, T>: IInvoke where A: struct {
    public Type Type {
        get {
            return typeof (A);
        }
    }
    public abstract T Handle(A a);
}
\end{minted}
\subsection{ProtoBuf 相关:IExtensible,IExtension,IProtoOutput<TOutput>,IMeasuredProtoOutput<TOutput>,MeasureState<T>: 看不懂}
\label{sec-1-18}
\subsubsection{IExtensible}
\label{sec-1-18-1}
\begin{minted}[fontsize=\scriptsize,linenos=false]{csharp}
// Indicates that the implementing type has support for protocol-buffer
// <see cref="IExtension">extensions</see>.
// <remarks>Can be implemented by deriving from Extensible.</remarks>
public interface IExtensible {
    // Retrieves the <see cref="IExtension">extension</see> object for the current
    // instance, optionally creating it if it does not already exist.
    // <param name="createIfMissing">Should a new extension object be
    // created if it does not already exist?</param>
    // <returns>The extension object if it exists (or was created), or null
    // if the extension object does not exist or is not available.</returns>
    // <remarks>The <c>createIfMissing</c> argument is false during serialization,
    // and true during deserialization upon encountering unexpected fields.</remarks>
    IExtension GetExtensionObject(bool createIfMissing);
}
\end{minted}
\subsubsection{IExtension}
\label{sec-1-18-2}
\begin{minted}[fontsize=\scriptsize,linenos=false]{csharp}
// Provides addition capability for supporting unexpected fields during
// protocol-buffer serialization/deserialization. This allows for loss-less
// round-trip/merge, even when the data is not fully understood.
public interface IExtension {
    // Requests a stream into which any unexpected fields can be persisted.
    // <returns>A new stream suitable for storing data.</returns>
    Stream BeginAppend();
    // Indicates that all unexpected fields have now been stored. The
    // implementing class is responsible for closing the stream. If
    // "commit" is not true the data may be discarded.
    // <param name="stream">The stream originally obtained by BeginAppend.</param>
    // <param name="commit">True if the append operation completed successfully.</param>
    void EndAppend(Stream stream, bool commit);
    // Requests a stream of the unexpected fields previously stored.
    // <returns>A prepared stream of the unexpected fields.</returns>
    Stream BeginQuery();
    // Indicates that all unexpected fields have now been read. The
    // implementing class is responsible for closing the stream.
    // <param name="stream">The stream originally obtained by BeginQuery.</param>
    void EndQuery(Stream stream);
    // Requests the length of the raw binary stream; this is used
    // when serializing sub-entities to indicate the expected size.
    // <returns>The length of the binary stream representing unexpected data.</returns>
    int GetLength();
}
// Provides the ability to remove all existing extension data
public interface IExtensionResettable : IExtension {
    void Reset();
}
\end{minted}
\subsubsection{IProtoOutput<TOutput>,IMeasuredProtoOutput<TOutput>,MeasureState<T>: 看得头大}
\label{sec-1-18-3}
\begin{minted}[fontsize=\scriptsize,linenos=false]{csharp}
// Represents the ability to serialize values to an output of type <typeparamref name="TOutput"/>
public interface IProtoOutput<TOutput> {
    // Serialize the provided value
    void Serialize<T>(TOutput destination, T value, object userState = null);
}
// Represents the ability to serialize values to an output of type <typeparamref name="TOutput"/>
// with pre-computation of the length
public interface IMeasuredProtoOutput<TOutput> : IProtoOutput<TOutput> {
    // Measure the length of a value in advance of serialization
    MeasureState<T> Measure<T>(T value, object userState = null);
    // Serialize the previously measured value
    void Serialize<T>(MeasureState<T> measured, TOutput destination);
}
// Represents the outcome of computing the length of an object; since this may have required computing lengths
// for multiple objects, some metadata is retained so that a subsequent serialize operation using
// this instance can re-use the previously calculated lengths. If the object state changes between the
// measure and serialize operations, the behavior is undefined.
public struct MeasureState<T> : IDisposable {
// note: * does not actually implement this API;
// it only advertises it for 3.* capability/feature-testing, i.e.
// callers can check whether a model implements
// IMeasuredProtoOutput<Foo>, and *work from that*
    public void Dispose() => throw new NotImplementedException();
    public long Length => throw new NotImplementedException();
}
\end{minted}

\section{【拖拉机游戏房间】组件: 分析}
\label{sec-2}
\subsection{TractorRoomEvent: 拖拉机房间,【待修改完成】}
\label{sec-2-1}
\begin{minted}[fontsize=\scriptsize,linenos=false]{csharp}
// UI 系统的事件机制:定义,如何创建拖拉机游戏房间【TODO:】UNITY 里是需要制作相应预设的
[UIEvent(UIType.TractorRoom)]
public class TractorRoomEvent: AUIEvent {
    public override async ETTask<UI> OnCreate(UIComponent uiComponent, UILayer uiLayer) {
        await ETTask.CompletedTask;
        await uiComponent.DomainScene().GetComponent<ResourcesLoaderComponent>().LoadAsync(UIType.TractorRoom.StringToAB());

        GameObject bundleGameObject = (GameObject) ResourcesComponent.Instance.GetAsset(UIType.TractorRoom.StringToAB(), UIType.TractorRoom);
        GameObject room = UnityEngine.Object.Instantiate(bundleGameObject, UIEventComponent.Instance.GetLayer((int)uiLayer));
        UI ui = uiComponent.AddChild<UI, string, GameObject>(UIType.TractorRoom, room);
        // 【拖拉机游戏房间】:它可能由好几个不同的组件组成,这里要添加的不止一个
        ui.AddComponent<GamerComponent>(); // 玩家组件:这个控件带个UI 小面板,要怎么添加呢?
        ui.AddComponent<TractorRoomComponent>(); // <<<<<<<<<<<<<<<<<<<< 房间组件:合成组件系统,自带【互动组件】
        return ui;
    }
    public override void OnRemove(UIComponent uiComponent) {
        ResourcesComponent.Instance.UnloadBundle(UIType.TractorRoom.StringToAB());
    }
}
\end{minted}
\subsection{GamerComponent: 玩家【管理类组件】,是对房间里四个玩家的管理。}
\label{sec-2-2}
\begin{itemize}
\item 【GamerComponent】玩家组件:是对一个房间里四个玩家的(及其在房间里的坐位位置)管理(分东南西北)。可以添加移除玩家。
\begin{minted}[fontsize=\scriptsize,linenos=false]{csharp}
// 组件:是提供给房间用,用来管理游戏中每个房间里的最多三个当前玩家
public class GamerComponent : Entity, IAwake { // 它也有【生成系】
    private readonly Dictionary<long, int> seats = new Dictionary<long, int>();
    private readonly Gamer[] gamers = new Gamer[4]; 
    public Gamer LocalGamer { get; set; } // 提供给房间组件用的:就是当前玩家。。。
    // 添加玩家
    public void Add(Gamer gamer, int seatIndex) {
        gamers[seatIndex] = gamer;
        seats[gamer.UserID] = seatIndex;
    }
    // 获取玩家
    public Gamer Get(long id) {
        int seatIndex = GetGamerSeat(id);
        if (seatIndex >= 0) 
            return gamers[seatIndex];
        return null;
    }
    // 获取所有玩家
    public Gamer[] GetAll() {
        return gamers;
    }
    // 获取玩家座位索引
    public int GetGamerSeat(long id) {
        int seatIndex;
        if (seats.TryGetValue(id, out seatIndex)) 
            return seatIndex;
        return -1;
    }
    // 移除玩家并返回
    public Gamer Remove(long id) {
        int seatIndex = GetGamerSeat(id);
        if (seatIndex >= 0) {
            Gamer gamer = gamers[seatIndex];
            gamers[seatIndex] = null;
            seats.Remove(id);
            return gamer;
        }
        return null;
    }
    public override void Dispose() {
        if (this.IsDisposed) 
            return;
        base.Dispose();
        this.LocalGamer = null;
        this.seats.Clear();
        for (int i = 0; i < this.gamers.Length; i++) 
            if (gamers[i] != null) {
                gamers[i].Dispose();
                gamers[i] = null;
            }
    }
}
\end{minted}
\end{itemize}
\subsection{Gamer: 【服务端】一个玩家个例。对应这个玩家的相关信息}
\label{sec-2-3}
\begin{minted}[fontsize=\scriptsize,linenos=false]{csharp}
// 房间玩家对象
public sealed class Gamer : Entity, IAwake<long> {
    // 用户ID(唯一)
    public long UserID { get; private set; }
    // 玩家GateActorID
    public long PlayerID { get; set; }
    // 玩家所在房间ID
    public long RoomID { get; set; }
    // 是否准备
    public bool IsReady { get; set; }
    // 是否离线
    public bool isOffline { get; set; }
    public void Awake(long id) {
        this.UserID = id;
    }
    public override void Dispose() {
        if (this.IsDisposed) return;
        base.Dispose();
        this.UserID = 0;
        this.PlayerID = 0;
        this.RoomID = 0;
        this.IsReady = false;
        this.isOffline = false;
    }
}
\end{minted}
\subsection{Gamer: 【客户端】一个玩家个例。它说只要一点儿信息就行}
\label{sec-2-4}
\begin{itemize}
\item 传进程间消息的时候,也只传这两个关键参数。
\begin{minted}[fontsize=\scriptsize,linenos=false]{csharp}
public sealed class Gamer : Entity { // 玩家对象
    // 玩家唯一ID
    public long UserID { get; set; }
    // 是否准备
    public bool IsReady { get; set; }
    public override void Dispose() {
        if (this.IsDisposed) return;
        base.Dispose();
        this.UserID = 0;
        this.IsReady = false;
    }
}
\end{minted}
\end{itemize}
\subsection{GamerUIComponent: 【客户端】玩家UI 组件:每个玩家背个小面板,来显示必要信息(钱,抢不抢庄,反过的主等)}
\label{sec-2-5}
\begin{minted}[fontsize=\scriptsize,linenos=false]{csharp}
public class GamerUIComponent : Entity, IStart { // 玩家UI组件
    public GameObject Panel { get; private set; } // UI面板
    // 玩家昵称
    public string NickName { get { return name.text; } }
    private Image headPhoto;
    private Text prompt;
    private Text name;
    private Text money;
    public void Start() {
        if (this.GetParent<Gamer>().IsReady) 
            SetReady();
    }
    // 重置面板
    public void ResetPanel() {
        ResetPrompt();
        this.headPhoto.gameObject.SetActive(false);
        this.name.text = "空位";
        this.money.text = "";
        this.Panel = null;
        this.prompt = null;
        this.name = null;
        this.money = null;
        this.headPhoto = null;
    }
    // 设置面板
    public void SetPanel(GameObject panel) {
        this.Panel = panel;
        // 绑定关联
        this.prompt = this.Panel.Get<GameObject>("Prompt").GetComponent<Text>();
        this.name = this.Panel.Get<GameObject>("Name").GetComponent<Text>();
        this.money = this.Panel.Get<GameObject>("Money").GetComponent<Text>();
        this.headPhoto = this.Panel.Get<GameObject>("HeadPhoto").GetComponent<Image>();
        UpdatePanel();
    }
    // 更新面板
    public void UpdatePanel() {
        if (this.Panel != null) {
            SetUserInfo();
            headPhoto.gameObject.SetActive(false);
        }
    }
    // 设置玩家身份
    public void SetIdentity(Identity identity) {
        if (identity == Identity.None) return;
        string spriteName = $"Identity_{Enum.GetName(typeof(Identity), identity)}";
        Sprite headSprite = CardHelper.GetCardSprite(spriteName);
        headPhoto.sprite = headSprite;
        headPhoto.gameObject.SetActive(true);
    }
    // 玩家准备
    public void SetReady() {
        prompt.text = "准备!";
    }
    // 出牌错误
    public void SetPlayCardsError() {
        prompt.text = "您出的牌不符合规则!";
    }
    // 玩家不出
    public void SetDiscard() {
        prompt.text = "不出";
    }
    // 打2 时,玩家抢不抢庄:或者去想,玩家要不要反主牌花色
    public void SetGrab(GrabLandlordState state) {
        switch (state) {
        case GrabLandlordState.Not:
            break;
        case GrabLandlordState.Grab:
            prompt.text = "抢地主";
            break;
        case GrabLandlordState.UnGrab:
            prompt.text = "不抢";
            break;
        }
    }
    public void ResetPrompt() { // 重置提示
        prompt.text = "";
    }
    public void GameStart() { // 游戏开始
        ResetPrompt();
    }
    private async void SetUserInfo() { // 设置用户信息
        G2C_GetUserInfo_Ack g2C_GetUserInfo_Ack = await SessionComponent.Instance.Session.Call(new C2G_GetUserInfo_Req() { UserID = this.GetParent<Gamer>().UserID }) as G2C_GetUserInfo_Ack;
        if (this.Panel != null) {
            name.text = g2C_GetUserInfo_Ack.NickName;
            money.text = g2C_GetUserInfo_Ack.Money.ToString();
        }
    }
    public override void Dispose() {
        if (this.IsDisposed) return;
        base.Dispose();
        ResetPanel(); // 重置玩家UI
    }
}
\end{minted}
\subsection{Protobuf 里面的消息与参考}
\label{sec-2-6}
\begin{itemize}
\item 这里把 Protobuf 里面可以传的游戏相关也整理一下。
\begin{minted}[fontsize=\scriptsize,linenos=false]{csharp}
message GamerInfo {
    int64 UserID = 1;
    bool IsReady = 2;
}
message GamerScore {
    int64 UserID = 1;
    int64 Score = 2;
}
message GamerState {
    int64 UserID = 1;
    ET.Server.Identity UserIdentity = 2; // 命名空间的问题
	GrabLandlordState State = 3;
}
message GamerCardNum { // IMessage
    int64 UserID = 1;
    int32 Num = 2;
}
message Actor_GamerGrabLandlordSelect_Ntt { // IActorMessage 参考去想:抢庄,与反主牌花色,如何写消息 
    int32 RpcId = 90;
    int64 ActorId = 94;
    int64 UserID = 1;
    bool IsGrab = 2;
}
\end{minted}
\end{itemize}
\subsection{TractorRoomComponent: 游戏房间,自带其它组件,当有嵌套时,如何才能系统化地、工厂化地、UI 上的事件驱动地,生成这个组件呢?}
\label{sec-2-7}
\begin{minted}[fontsize=\scriptsize,linenos=false]{csharp}
public class TractorRoomComponent : Entity, IAwake {
    private TractorInteractionComponent interaction; // 嵌套组件:互动组件
    private Text multiples;
    public readonly GameObject[] GamersPanel = new GameObject[4];
    public bool Matching { get; set; }
    public TractorInteractionComponent Interaction { // 组件里套组件,要如何事件机制触发生成?
        get {
            if (interaction == null) {
                UI uiRoom = this.GetParent<UI>();
                UI uiInteraction = TractorInteractionFactory.Create(UIType.TractorInteraction, uiRoom);
                interaction = uiInteraction.GetComponent<TractorInteractionComponent>();
            }
            return interaction;
        }
    }
\end{minted}
\subsection{TractorInteractionComponent: 感觉是视图UI 上的一堆调控,逻辑控制}
\label{sec-2-8}
\begin{itemize}
\item 上下这一两个组件里,除了 ProtoBuf 消息里传递的类找不到,没有其它错误
\item 【嵌套】:是这里的难点。其它都可以一个触发一个地由事件发布触发订阅者的回调,可是当一个组件内存在嵌套,又是系统化【内部组件生成完成后,外部组件才生成完成】生成,我是要把这两个组件合并成一个吗?还是说,我不得不把它折成粒度更小的UI 上的事件驱动机制,以符合系统框架?要去所源码弄透。
\begin{minted}[fontsize=\scriptsize,linenos=false]{csharp}
// 【互动组件】:一堆的视图控件管理 
public class TractorInteractionComponent : Entity, IAwake { // 多个按钮:有些暂时是隐藏的
    private Button playButton;
    private Button promptButton;
    private Button discardButton;
    private Button grabButton;
    private Button disgrabButton;
    private Button changeGameModeButton;
    private List<Card> currentSelectCards = new List<Card>();

    public bool isTrusteeship { get; set; }
    public bool IsFirst { get; set; }
\end{minted}
\end{itemize}










\section{Net 网络交互相关:只在【服务端】用}
\label{sec-3}
\subsection{NetInnerComponent: 【服务端】对不同进程的处理组件。是服务器的组件}
\label{sec-3-1}
\begin{minted}[fontsize=\scriptsize,linenos=false]{csharp}
namespace ET.Server {
    // 【服务器】:对不同进程的一些处理
    public struct ProcessActorId {
        public int Process;
        public long ActorId;
        public ProcessActorId(long actorId) {
            InstanceIdStruct instanceIdStruct = new InstanceIdStruct(actorId);
            this.Process = instanceIdStruct.Process;
            instanceIdStruct.Process = Options.Instance.Process;
            this.ActorId = instanceIdStruct.ToLong();
        }
    }
    
    public struct NetInnerComponentOnRead {
        public long ActorId;
        public object Message;
    }
    
    [ComponentOf(typeof(Scene))]
    public class NetInnerComponent: Entity, IAwake<IPEndPoint>, IAwake, IDestroy {
        public int ServiceId;
        
        public NetworkProtocol InnerProtocol = NetworkProtocol.KCP;
        [StaticField]
        public static NetInnerComponent Instance;
    }
}
\end{minted}
\subsection{NetInnerComponentSystem: 生成系}
\label{sec-3-2}
\begin{minted}[fontsize=\scriptsize,linenos=false]{csharp}
[FriendOf(typeof(NetInnerComponent))]
public static class NetInnerComponentSystem {
    [ObjectSystem]
    public class NetInnerComponentAwakeSystem: AwakeSystem<NetInnerComponent> {
        protected override void Awake(NetInnerComponent self) {
            NetInnerComponent.Instance = self;
            switch (self.InnerProtocol) {
                case NetworkProtocol.TCP: {
                    self.ServiceId = NetServices.Instance.AddService(new TService(AddressFamily.InterNetwork, ServiceType.Inner));
                    break;
                }
                case NetworkProtocol.KCP: {
                    self.ServiceId = NetServices.Instance.AddService(new KService(AddressFamily.InterNetwork, ServiceType.Inner));
                    break;
                }
            }
            NetServices.Instance.RegisterReadCallback(self.ServiceId, self.OnRead);
            NetServices.Instance.RegisterErrorCallback(self.ServiceId, self.OnError);
        }
    }
    [ObjectSystem]
    public class NetInnerComponentAwake1System: AwakeSystem<NetInnerComponent, IPEndPoint> {
        protected override void Awake(NetInnerComponent self, IPEndPoint address) {
            NetInnerComponent.Instance = self;
            switch (self.InnerProtocol) {
                case NetworkProtocol.TCP: {
                    self.ServiceId = NetServices.Instance.AddService(new TService(address, ServiceType.Inner));
                    break;
                }
                case NetworkProtocol.KCP: {
                    self.ServiceId = NetServices.Instance.AddService(new KService(address, ServiceType.Inner));
                    break;
                }
            }
            NetServices.Instance.RegisterAcceptCallback(self.ServiceId, self.OnAccept);
            NetServices.Instance.RegisterReadCallback(self.ServiceId, self.OnRead);
            NetServices.Instance.RegisterErrorCallback(self.ServiceId, self.OnError);
        }
    }
    [ObjectSystem]
    public class NetInnerComponentDestroySystem: DestroySystem<NetInnerComponent> {
        protected override void Destroy(NetInnerComponent self) {
            NetServices.Instance.RemoveService(self.ServiceId);
        }
    }
    private static void OnRead(this NetInnerComponent self, long channelId, long actorId, object message) {
        Session session = self.GetChild<Session>(channelId);
        if (session == null) 
            return;
        session.LastRecvTime = TimeHelper.ClientFrameTime();
        self.HandleMessage(actorId, message);
    }
    public static void HandleMessage(this NetInnerComponent self, long actorId, object message) {
        EventSystem.Instance.Publish(Root.Instance.Scene, new NetInnerComponentOnRead() { ActorId = actorId, Message = message });
    }
    private static void OnError(this NetInnerComponent self, long channelId, int error) {
        Session session = self.GetChild<Session>(channelId);
        if (session == null) 
            return;
        session.Error = error;
        session.Dispose();
    }
    // 这个channelId是由CreateAcceptChannelId生成的
    private static void OnAccept(this NetInnerComponent self, long channelId, IPEndPoint ipEndPoint) {
        Session session = self.AddChildWithId<Session, int>(channelId, self.ServiceId);
        session.RemoteAddress = ipEndPoint;
        // session.AddComponent<SessionIdleCheckerComponent, int, int, int>(NetThreadComponent.checkInteral, NetThreadComponent.recvMaxIdleTime, NetThreadComponent.sendMaxIdleTime);
    }
    private static Session CreateInner(this NetInnerComponent self, long channelId, IPEndPoint ipEndPoint) {
        Session session = self.AddChildWithId<Session, int>(channelId, self.ServiceId);
        session.RemoteAddress = ipEndPoint;
        NetServices.Instance.CreateChannel(self.ServiceId, channelId, ipEndPoint);
        // session.AddComponent<InnerPingComponent>();
        // session.AddComponent<SessionIdleCheckerComponent, int, int, int>(NetThreadComponent.checkInteral, NetThreadComponent.recvMaxIdleTime, NetThreadComponent.sendMaxIdleTime);
        return session;
    }
    // 内网actor session,channelId是进程号
    public static Session Get(this NetInnerComponent self, long channelId) {
        Session session = self.GetChild<Session>(channelId);
        if (session != null) 
            return session;
        IPEndPoint ipEndPoint = StartProcessConfigCategory.Instance.Get((int) channelId).InnerIPPort;
        session = self.CreateInner(channelId, ipEndPoint);
        return session;
    }
}
\end{minted}
\section{消息处理器: AMActorHandler<E, Message> 继承类的返回类型,全改成了 void}
\label{sec-4}
\subsection{AMActorHandler<E, Message>: 基类的抽象方法 Run 的返回类型被固定死了,报了狠多错}
\label{sec-4-1}
\begin{itemize}
\item 这样,可以把所有自己继承类的报错去掉。可是因为还没能理解透彻,不知道先前的ETVoid 是为什么,现在会不会产生什么其它意外的错。作个记号。
\begin{minted}[fontsize=\scriptsize,linenos=false]{csharp}
[EnableClass]
public abstract class AMActorHandler<E, Message>: IMActorHandler where E : Entity where Message : class, IActorMessage {

    // protected abstract ETTask Run(E entity, Message message);  // <<<<<<<<<<<<<<<<<<<< 
    protected abstract void Run(E entity, Message message);  // 可以改成是自己想要的,返回类型,因为只有自已的继承类在使用,不影响其它 

    public async ETTask Handle(Entity entity, int fromProcess, object actorMessage) {
        if (actorMessage is not Message msg) {
            Log.Error($"消息类型转换错误: {actorMessage.GetType().FullName} to {typeof (Message).Name}");
            return;
        }
        if (entity is not E e) {
            Log.Error($"Actor类型转换错误: {entity.GetType().Name} to {typeof (E).Name} --{typeof (Message).Name}");
            return;
        }
        await this.Run(e, msg);
    }
    public Type GetRequestType() {
        if (typeof (IActorLocationMessage).IsAssignableFrom(typeof (Message))) {
            Log.Error($"message is IActorLocationMessage but handler is AMActorHandler: {typeof (Message)}");
        }
        return typeof (Message);
    }
    public Type GetResponseType() {
        return null;
    }
}
\end{minted}
\end{itemize}
\subsection{IMActorHandler: 接口类的定义,同样要改}
\label{sec-4-2}
\begin{minted}[fontsize=\scriptsize,linenos=false]{csharp}
public interface IMActorHandler {
    // ETTask Handle(Entity entity, int fromProcess, object actorMessage);
    void Handle(Entity entity, int fromProcess, object actorMessage); // 自已改成这样的
    Type GetRequestType();
    Type GetResponseType();
}
\end{minted}


\section{写在最后:反而是自己每天查看一再更新的}
\label{sec-5}
\begin{itemize}
\item 因为感觉还是不曾系统性地读ET7 的源码,或者说有效阅读,因为没有带着实际问题的看源码,感觉都不叫看读源码呀。这里会记自己的感觉需要赶快查看的地方。
\item 【ET 框架的整体架构】:感觉把握不够。常常命名空间分不清。要把这个大的框架,比较高层面的架构再好好看下。然后就是对自顶向下的不同层级场景,所需要的主要的不同组件,分不清,仍需要再熟悉一下源码
\item 【问题】:某些消息,还分不清是内网还是外网消息,暂时先放一下,到时再改
\item 【问题】:上次那个ET-EUI 框架的时候,曾经出现过 opcode 不对应,也就是说,我现在生成的进程间消息,有可能还是会存在服务器码与客户端码不对应,这个完备的框架,这次应该不至于吧?
\item 【ClientComponent】:新框架里重构丢了,去找怎么替代?那么现在去追一下,客户端的起始与场景加载或是切换大致过程。它变成了什么客户端场景管理?
\item 【UIType】部分类:这个类出现在了三四个不同的程序域,现在重构了,好像添加得不对。要再修改
\end{itemize}

\section{现在的修改内容,记忆}
\label{sec-6}
\begin{itemize}
\item 【任何时候,活宝妹就是一定要嫁给亲爱的表哥!!!】
\item 【活宝妹坐等亲爱的表哥,领娶活宝妹回家!爱表哥,爱生活!!!】
\end{itemize}
\section{{\bfseries\sffamily TODO} 今天晚上把几个消息抓全了,免得一堆的报错}
\label{sec-7}
\begin{itemize}
\item \textbf{【IStartSystem:】} 感觉还有点儿小问题。认为:我应该不需要同文件两份,一份复制到客户端热更新域。我认为,全框架应该如其它接口类一样,只要一份就可以了。 \textbf{【晚点儿再检查一遍】}
\item 【Proto2CS】: 进程间消息里的,【牌相关的】,尤其是它们所属的命名空间,没写对,现在总是找不到定义。
\item \textbf{【Protobuf 里进程间传递的游戏数据相关信息:】} 这个现在成为重构的主要 compile-error. 因为找不到类。需要去弄懂
\begin{itemize}
\item 包括Identity, Weight,Suits,抢不抢地主【抢不抢庄】,以及可能的反不反主牌花色等。
\item 找不到的那些类,我可能【内网】【外网】消息没分对。
\item 【Identity】与【Card】:外网消息里,怎么会找不到呢?再回去检查一遍。下午要把这个弄通,要开始思路怎么设计重构拖拉机项目。
\end{itemize}
\item 去把【拖拉机房间、斗地主房间组件的,玩家什么的一堆组件】弄明白
\item 把参考游戏里,打牌相关的逻辑与模块好好看下,方便自己熟悉自己重构项目的源码后,画葫芦画飘地重构
\item 【任何时候,活宝妹就是一定要嫁给亲爱的表哥!!!爱表哥,爱生活!!!】
\end{itemize}
\section{拖拉机游戏:【重构OOP/OOD 设计思路】}
\label{sec-8}
\begin{itemize}
\item 自己是学过,有这方面的意识,但并不是说,自己就懂得,就知道该如何狠好地设计这些类。现在更多的是要受ET 框架,以及参考游戏手牌设计的启发,来帮助自己一再梳理思路,该如何设计它。
\item 【GamerComponent】玩家组件:是对一个房间里四个玩家的(及其在房间里的坐位位置)管理(分东南西北)。可以添加移除玩家。
\item 
\end{itemize}
% Emacs 28.2 (Org mode 8.2.7c)
\end{document}