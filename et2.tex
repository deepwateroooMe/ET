% Created 2023-05-15 Mon 07:54
\documentclass[9pt, b5paper]{article}
\usepackage{xeCJK}
\usepackage[T1]{fontenc}
\usepackage{bera}
\usepackage[scaled]{beraserif}
\usepackage[scaled]{berasans}
\usepackage[scaled]{beramono}
\usepackage[cache=false]{minted}
\usepackage{xltxtra}
\usepackage{graphicx}
\usepackage{xcolor}
\usepackage{multirow}
\usepackage{multicol}
\usepackage{float}
\usepackage{textcomp}
\usepackage{algorithm}
\usepackage{algorithmic}
\usepackage{latexsym}
\usepackage{natbib}
\usepackage{geometry}
\geometry{left=1.2cm,right=1.2cm,top=1.5cm,bottom=1.2cm}
\usepackage[xetex,colorlinks=true,CJKbookmarks=true,linkcolor=blue,urlcolor=blue,menucolor=blue]{hyperref}
\newminted{common-lisp}{fontsize=\footnotesize} 
\author{deepwaterooo}
\date{\today}
\title{ET 框架学习笔记(二)--网络交互相关}
\hypersetup{
  pdfkeywords={},
  pdfsubject={},
  pdfcreator={Emacs 28.2 (Org mode 8.2.7c)}}
\begin{document}

\maketitle
\tableofcontents


\section{Net 网络交互相关}
\label{sec-1}
\subsection{NetInnerComponent: 【服务端】对不同进程的处理组件。是服务器的组件}
\label{sec-1-1}
\begin{minted}[fontsize=\scriptsize,linenos=false]{csharp}
namespace ET.Server {
    // 【服务器】:对不同进程的一些处理
    public struct ProcessActorId {
        public int Process;
        public long ActorId;
        public ProcessActorId(long actorId) {
            InstanceIdStruct instanceIdStruct = new InstanceIdStruct(actorId);
            this.Process = instanceIdStruct.Process;
            instanceIdStruct.Process = Options.Instance.Process;
            this.ActorId = instanceIdStruct.ToLong();
        }
    }
    
    public struct NetInnerComponentOnRead {
        public long ActorId;
        public object Message;
    }
    
    [ComponentOf(typeof(Scene))]
    public class NetInnerComponent: Entity, IAwake<IPEndPoint>, IAwake, IDestroy {
        public int ServiceId;
        
        public NetworkProtocol InnerProtocol = NetworkProtocol.KCP;
        [StaticField]
        public static NetInnerComponent Instance;
    }
}
\end{minted}
\subsection{NetInnerComponentSystem: 生成系}
\label{sec-1-2}
\begin{minted}[fontsize=\scriptsize,linenos=false]{csharp}
[FriendOf(typeof(NetInnerComponent))]
public static class NetInnerComponentSystem {
    [ObjectSystem]
    public class NetInnerComponentAwakeSystem: AwakeSystem<NetInnerComponent> {
        protected override void Awake(NetInnerComponent self) {
            NetInnerComponent.Instance = self;
            switch (self.InnerProtocol) {
                case NetworkProtocol.TCP: {
                    self.ServiceId = NetServices.Instance.AddService(new TService(AddressFamily.InterNetwork, ServiceType.Inner));
                    break;
                }
                case NetworkProtocol.KCP: {
                    self.ServiceId = NetServices.Instance.AddService(new KService(AddressFamily.InterNetwork, ServiceType.Inner));
                    break;
                }
            }
            NetServices.Instance.RegisterReadCallback(self.ServiceId, self.OnRead);
            NetServices.Instance.RegisterErrorCallback(self.ServiceId, self.OnError);
        }
    }
    [ObjectSystem]
    public class NetInnerComponentAwake1System: AwakeSystem<NetInnerComponent, IPEndPoint> {
        protected override void Awake(NetInnerComponent self, IPEndPoint address) {
            NetInnerComponent.Instance = self;
            switch (self.InnerProtocol) {
                case NetworkProtocol.TCP: {
                    self.ServiceId = NetServices.Instance.AddService(new TService(address, ServiceType.Inner));
                    break;
                }
                case NetworkProtocol.KCP: {
                    self.ServiceId = NetServices.Instance.AddService(new KService(address, ServiceType.Inner));
                    break;
                }
            }
            NetServices.Instance.RegisterAcceptCallback(self.ServiceId, self.OnAccept);
            NetServices.Instance.RegisterReadCallback(self.ServiceId, self.OnRead);
            NetServices.Instance.RegisterErrorCallback(self.ServiceId, self.OnError);
        }
    }
    [ObjectSystem]
    public class NetInnerComponentDestroySystem: DestroySystem<NetInnerComponent> {
        protected override void Destroy(NetInnerComponent self) {
            NetServices.Instance.RemoveService(self.ServiceId);
        }
    }
    private static void OnRead(this NetInnerComponent self, long channelId, long actorId, object message) {
        Session session = self.GetChild<Session>(channelId);
        if (session == null) 
            return;
        session.LastRecvTime = TimeHelper.ClientFrameTime();
        self.HandleMessage(actorId, message);
    }
    public static void HandleMessage(this NetInnerComponent self, long actorId, object message) {
        EventSystem.Instance.Publish(Root.Instance.Scene, new NetInnerComponentOnRead() { ActorId = actorId, Message = message });
    }
    private static void OnError(this NetInnerComponent self, long channelId, int error) {
        Session session = self.GetChild<Session>(channelId);
        if (session == null) 
            return;
        session.Error = error;
        session.Dispose();
    }
    // 这个channelId是由CreateAcceptChannelId生成的
    private static void OnAccept(this NetInnerComponent self, long channelId, IPEndPoint ipEndPoint) {
        Session session = self.AddChildWithId<Session, int>(channelId, self.ServiceId);
        session.RemoteAddress = ipEndPoint;
        // session.AddComponent<SessionIdleCheckerComponent, int, int, int>(NetThreadComponent.checkInteral, NetThreadComponent.recvMaxIdleTime, NetThreadComponent.sendMaxIdleTime);
    }
    private static Session CreateInner(this NetInnerComponent self, long channelId, IPEndPoint ipEndPoint) {
        Session session = self.AddChildWithId<Session, int>(channelId, self.ServiceId);
        session.RemoteAddress = ipEndPoint;
        NetServices.Instance.CreateChannel(self.ServiceId, channelId, ipEndPoint);
        // session.AddComponent<InnerPingComponent>();
        // session.AddComponent<SessionIdleCheckerComponent, int, int, int>(NetThreadComponent.checkInteral, NetThreadComponent.recvMaxIdleTime, NetThreadComponent.sendMaxIdleTime);
        return session;
    }
    // 内网actor session,channelId是进程号
    public static Session Get(this NetInnerComponent self, long channelId) {
        Session session = self.GetChild<Session>(channelId);
        if (session != null) 
            return session;
        IPEndPoint ipEndPoint = StartProcessConfigCategory.Instance.Get((int) channelId).InnerIPPort;
        session = self.CreateInner(channelId, ipEndPoint);
        return session;
    }
}
\end{minted}
% Emacs 28.2 (Org mode 8.2.7c)
\end{document}